\documentclass[11pt]{article}
\usepackage[physics]{personal_commands}
\usepackage[italian]{babel}

\title{\textbf{Note del corso di Fisica 1}}
\author{Gabriel Antonio Videtta}
\date{22 marzo 2023}

\begin{document}
	
	\maketitle
	
	\begin{center}
		\Large \textbf{Derivate parziali e integrali di linea}
	\end{center}

	\begin{definition}
		Data una funzione $f : X \to \RR$ con $X \subseteq \RR^n$ definita nelle variabili $x_1$, $x_2$, ..., $x_n$, si
		definisce la \textit{derivata parziale} di $f$ rispetto a $x_i$ come la derivata di $f$ rispetto a $x_i$
		mantenendo le altri variabili come costanti, e si indica con la notazione $\frac{\partial f}{\partial x_i}$.
	\end{definition}

	\begin{example}
		Sia $f : \RR^3 \to \RR$ tale che $f(x, y, z) = x^2y + z - xyz$. \\
		
		\li $\frac{\partial f}{\partial x} = 2xy - yz$, \\ \vskip 0.01in
		\li $\frac{\partial f}{\partial y} = x^2 - xz$, \\ \vskip 0.015in
		\li $\frac{\partial f}{\partial z} = 1-xy$.
	\end{example}
	
\end{document}
