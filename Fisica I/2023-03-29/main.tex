\documentclass[11pt]{article}
\usepackage[physics]{personal_commands}
\usepackage[italian]{babel}

\title{\textbf{Note del corso di Fisica 1}}
\author{Gabriel Antonio Videtta}
\date{\today}

\begin{document}
	
	\maketitle
	
	\begin{center}
		\Large \textbf{Esempi di forze conservative}
	\end{center}

	Un esempio notevole di forza conservativa è quello della
	forza elastica $\vec f = -k \vec r$. Sia infatti $\vec f = (f_x, f_y, f_z)$.
	Allora $L_{\gamma(A, B)} = \int_{\gamma(A, B)} \vec f \cdot d\vec r =
	\int_{x_A}^{x_B} f_x dx + \int_{y_A}^{y_B} f_y dy + \int_{z_A}^{z_B} f_z dz =
	-k (\int_{x_A}^{x_B} x dx + \int_{y_A}^{y_B} y dy + \int_{z_A}^{z_B} z dz) =
	-\frac{k}{2} (\norm{B}^2 - \norm{A}^2)$, ossia non dipende dalla traiettoria
	$\gamma$. Si ricava allora che $U(x) = \frac{k}{2} x ^2$, nel caso
	unidimensionale.
	
	%TODO: recuperare lezione.
	
	\begin{definition} (impulso di una forza)
		Si definisce \textbf{impulso di una forza} l'integrale
		$\vec I(t_1, t_2) = \int_{t_1}^{t_2} \vec F(t) dt$.
	\end{definition}

	Sia $\vec F = \sum_{i=1}^N \vec F_i$. Allora $\vec I(t_1, t_2) =
	\sum_{i=1}^N \vec I_i(t_1, t_2)$, dove $\vec I_i$ è calcolato su $\vec F_i$.
	
	\begin{theorem} (dell'impulso)
		Vale l'identità $\vec I(t_1, t_2) = \vec P(t_2) - \vec P(t_1) = \Delta \vec P$.
	\end{theorem}

	\begin{definition} (momento di un vettore applicato)
		Si definisce \textbf{momento di un vettore} $\vec v$ dal polo
		$\omega$ sul punto applicato $A$ con vettore $\vec r$ il
		vettore perpendicolare ad ambo i vettori $\vec r \times \vec v$.
	\end{definition}
\end{document}
