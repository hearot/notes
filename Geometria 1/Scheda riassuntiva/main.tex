\documentclass[10pt,landscape]{article}
\usepackage{amssymb,amsmath,amsthm,amsfonts}
\usepackage{multicol,multirow}
\usepackage{marvosym}
\usepackage{calc}
\usepackage{ifthen}
\usepackage[landscape]{geometry}
\usepackage[colorlinks=true,citecolor=blue,linkcolor=blue]{hyperref}
\usepackage{personal_commands}

\setlength{\extrarowheight}{0pt}

\ifthenelse{\lengthtest { \paperwidth = 11in}}
{ \geometry{top=.5in,left=.5in,right=.5in,bottom=.5in} }
{\ifthenelse{ \lengthtest{ \paperwidth = 297mm}}
	{\geometry{top=1cm,left=1cm,right=1cm,bottom=1cm} }
	{\geometry{top=1cm,left=1cm,right=1cm,bottom=1cm} }
}
%\pagestyle{empty}
\makeatletter
\renewcommand{\section}{\@startsection{section}{1}{0mm}%
	{-1ex plus -.5ex minus -.2ex}%
	{0.5ex plus .2ex}%x
	{\normalfont\large\bfseries}}
\renewcommand{\subsection}{\@startsection{subsection}{2}{0mm}%
	{-1explus -.5ex minus -.2ex}%
	{0.5ex plus .2ex}%
	{\normalfont\normalsize\bfseries}}
\renewcommand{\subsubsection}{\@startsection{subsubsection}{3}{0mm}%
	{-1ex plus -.5ex minus -.2ex}%
	{1ex plus .2ex}%
	{\normalfont\small\bfseries}}
\makeatother
\setcounter{secnumdepth}{0}
\setlength{\parindent}{0pt}
\setlength{\parskip}{0pt plus 0.5ex}
% -----------------------------------------------------------------------

\title{Schede riassuntive di Geometria 1}

\begin{document}
	
	\parskip=0.7ex
	
	\raggedright
	\footnotesize
	
	\begin{center}
		\Large{\textbf{Schede riassuntive di Geometria 1}} \\
	\end{center}
	\begin{multicols}{3}
		\setlength{\premulticols}{1pt}
		\setlength{\postmulticols}{1pt}
		\setlength{\multicolsep}{1pt}
		\setlength{\columnsep}{2pt}
		
		\subsection{Alcuni accenni alla geometria di \texorpdfstring{$\RR^3$}{R\^{}3}}
		
		Si definisce prodotto scalare la forma
		bilineare simmetrica unicamente determinata da $\innprod{\vec{e_i}, \vec{e_j}} = \delta_{ij}$. Vale la seguente identità: $\innprod{(x, y, z), (x', y', z')} = xx' + yy' + zz'$.
		
		Inoltre $\innprod{\vec{a}, \vec{b}} = \card{\vec{a}} \card{\vec{b}} \cos(\theta)$, dove $\theta$ è l'angolo compreso tra i due vettori.
		Due vettori $\vec{a}$, $\vec{b}$ si dicono ortogonali
		se e solo se $\innprod{\vec{a}, \vec{b}} = 0$.
		
		Si definisce prodotto vettoriale la forma bilineare alternante
		da $\RR^3 \times \RR^3$
		in $\RR^3$ tale che $\vec{e_1} \times \vec{e_2} = \vec{e_3}$,
		$\vec{e_2} \times \vec{e_3} = \vec{e_1}$,
		$\vec{e_3} \times \vec{e_1} = \vec{e_2}$ e
		$\vec{e_i} \times \vec{e_i} = \vec{0}$. Dati due
		vettori $(x, y, z)$ e $(x', y', z')$, si può determinarne
		il prodotto vettoriale informalmente come:
		
		\[ \begin{vmatrix}
			\vec{e_1} & \vec{e_2} & \vec{e_3} \\
			x & y & z \\
			x' & y' & z'
		\end{vmatrix} . \]
		
		Vale l'identità $\card{\vec{a} \times \vec{b}} = \card{\vec{a}} \card{\vec{b}} \sin(\theta)$, dove $\theta$ è l'angolo con cui, ruotando di
		$\theta$ in senso antiorario $\vec{a}$, si ricade su $\vec{b}$.
		Due vettori $\vec{a}$, $\vec{b}$ si dicono paralleli se $\exists
		k \mid \vec{a} = k \vec{b}$, o equivalentemente se
		$\vec{a} \times \vec{b} = \vec{0}$. Altrettanto si può dire
		se $\innprod{\vec{a}, \vec{b}} = \card{\vec{a}} \card{\vec{b}}$ (i.e.
		$\cos(\theta) = 1 \implies \theta = 0$).
		
		Una retta in $\RR^3$ è un sottospazio affine della
		forma $\vec{v} + \Span(\vec{r})$. Analogamente
		un piano è della forma $\vec{v} + \Span(\vec{x}, \vec{y})$.
		
		Nella forma cartesiana, un piano è della forma $ax+by+cz=d$,
		dove $(a,b,c)$ è detta normale del piano. Una retta è
		l'intersezione di due piani, e dunque è un sistema lineare
		di due equazioni di un piano. Due piani sono perpendicolari
		fra loro se e solo se le loro normali sono ortogonali. Due
		piani sono paralleli se e solo se le loro normali sono parallele.
		Il vettore $\vec{r}$ che genera lo $\Span$ di una retta che è
		intersezione di due piani può essere computato come
		prodotto vettoriale delle normali dei due piani.
		
		Valgono le seguenti identità:
		
		\begin{itemize}
			\item $\vec{a} \times (\vec{b} \times \vec{c}) =
			\innprod{\vec{a}, \vec{c}}\,\vec{b} - \innprod{\vec{a}, \vec{b}}\,\vec{c}$ (\textit{identità di Lagrange}),
			\item $\vec{a} \times (\vec{b} \times \vec{c}) + \vec{b} \times (\vec{c} \times \vec{a}) + \vec{c} \times (\vec{a} \times \vec{b}) =
			\vec{0}$ (\textit{identità di Jacobi}).
		\end{itemize}
		
		Dati tre punti $\vec{a}$, $\vec{b}$, $\vec{c}$, il volume
		del parallelepipedo individuato da questi punti è:
		
		\[\card{\det\begin{pmatrix}\vec{a} \\ \vec{b} \\ \vec{c}\end{pmatrix}} =
		\card{\innprod{\vec{a}, \vec{b} \times \vec{c}}}.\]
		
		Tre punti sono complanari se e solo se il volume di tale parallelpipedo è nullo
		(infatti questo è equivalente a dire che almeno uno dei tre punti
		si scrive come combinazione lineare degli altri due).
		
		
		\subsection{Proprietà generali di uno spazio vettoriale}
		
		Uno spazio vettoriale $V$ su un campo $\KK$ soddisfa i seguenti
		assiomi:
		
		\begin{itemize}
			\item $(V, +)$ è un gruppo abeliano,
			\item il prodotto esterno da $\KK \times V$ in $V$ è
			associativo rispetto agli scalari (i.e. $a(b\vec{v}) = (ab)\vec{v}$),
			\item $1_{\KK} \cdot \vec{v} = \vec{v}$,
			\item il prodotto esterno è distributivo da ambo i
			lati (i.e. $(a+b)\vec{v} = a\vec{v} + b\vec{v}$ e
			$a(\vec{v} + \vec{w}) = a\vec{v} + a\vec{w}$.
		\end{itemize}
		
		Un insieme di vettori $I$ si dice linearmente indipendente se
		una qualsiasi combinazione lineare di un suo sottinsieme
		finito è nulla se e solo se i coefficienti dei vettori
		sono tutti nulli. Si dice linearmente dipendente in caso
		contrario.
		
		Un insieme di vettori $G$ si dice generatore di $V$ se ogni vettore
		di $V$ si può scrivere come combinazione lineare di un numero
		finito di elementi di $G$, ossia se $V = \Span(G)$.
		
		Una base è un insieme contemporaneamente linearmente indipendente
		e generatore di $V$. Equivalentemente una base è un insieme generatore
		minimale rispetto all'inclusione e un insieme linearmente indipendente
		massimale, sempre rispetto all'inclusione. Ogni spazio vettoriale,
		anche quelli non finitamente generati,
		ammettono una base. La dimensione della base è unica ed è il
		numero di elementi dell'insieme che è base.
		
		Dato un insieme linearmente indipendente $I$ in uno spazio di dimensione
		finita, tale insieme, data una base $\basis$, può essere esteso
		a una base $T$ che contiene $I$ e che è completato da
		elementi di $\basis$.
		
		Analogamente, dato un insieme generatore finito $G$, da esso
		si può estrarre sempre una base dello spazio.
		
		Uno spazio vettoriale fondato su un campo infinito
		con un insieme di vettori infinito non
		è mai unione finita di sottospazi propri. Un insieme linearmente
		indipendente di $V$ con esattamente $\dim V$ elementi è una
		base di $V$. Analogamente, un insieme generatore di $V$ con esattamente
		$\dim V$ elementi è una base di $V$.
		
		Sia $\basis = \{\vec{v_1}, \ldots, \vec{v_n}\}$ una base ordinata dello spazio vettoriale $V$.
		
		\begin{itemize}
			\item $\zerovecset$ e $V$ sono detti sottospazi banali,
			\item lo $\Span$ di $n$ vettori è il più piccolo sottospazio
			di $V$ contenenti tali vettori,
			\item $\Span(\basis) = V$,
			\item $\Span(\emptyset) = \zerovecset$,
			\item dato $X$ generatore di $V$, $X \setminus \{\vec{x_0}\}$
			genera $V \iff \vec{x_0} \in \Span(X \setminus \{\vec{x_0}\})$,
			\item $X \subseteq Y$ è un sottospazio di $Y \iff \Span(X) = X$,
			\item $\Span(X) \subseteq Y \iff X \subseteq Y$, se $Y$ è uno spazio,
			\item $\Span(\Span(A)) = \Span(A)$,
			\item se $I$ è un insieme linearmente indipendente di $V$,
			allora $\card{I} \leq \dim V$,
			\item se $G$ è un insieme generatore di $V$, allora
			$\card{G} \geq \dim V$,
			\item $[\vec{v}]_\basis$ è la rappresentazione
			di $\vec{v}$ nella base ordinata $\basis$, ed è
			un vettore di $\KK^n$ che alla coordinata $i$-esima
			associa il coefficiente di $\vec{v_i}$ nella combinazione
			lineare di $\vec{v}$ nella base $\basis$,
			\item la rappresentazione nella base $\basis$ è sempre
			unica ed esiste sempre (è quindi un isomorfismo tra $V$ e
			$\KK^n$),
			\item si definisce base canonica di $\KK^n$ la base
			$e = \{\vec{e_1}, \ldots, \vec{e_n}\}$, dove
			$\vec{e_i}$ è un vettore con tutte le coordinate nulle,
			eccetto per la $i$-esima, che è pari ad $1$ (pertanto
			$\dim \KK^n = n$),
			\item una base naturale di $M(m, n, \KK)$ è data
			da $\basis = \{E_{11}, E_{12}, \ldots, E_{1n}, \ldots, E_{mn}\}$,
			dove $E_{ij}$ è una matrice con tutti gli elementi nulli, eccetto
			quello nel posto $(i, j)$, che è pari ad $1$ (dunque
			$\dim M(m, n, \KK) = mn$),
			\item le matrici $A$ di taglia $n$ tali che $A^\top = A$ formano il
			sottospazio $\Sym(n, \KK)$ di $M(n, \KK)$, detto sottospazio delle matrici
			simmetriche, la cui base naturale è data da
			$\basis' = \{E_{ij} + E_{ji} \in \basis \mid i < j\} \cup
			\{E_{ij} \in \basis \mid i = j\}$, dove $\basis$ è la
			base naturale di $M(m, n, \KK)$ (dunque $\dim \Sym(n, \KK) = \frac{n(n+1)}{2}$),
			\item le matrici $A$ di taglia $n$ tali che $A^\top = -A$ formano il
			sottospazio $\Lambda(n, \KK)$ di $M(n, \KK)$, detto sottospazio delle matrici
			antisimmetriche, la cui base naturale è data da
			$\basis' = \{E_{ij} - E_{ji} \in \basis \mid i < j\}$, dove $\basis$ è la
			base naturale di $M(m, n, \KK)$ (dunque $\dim \Lambda(n, \KK) = \frac{n(n-1)}{2}$),
			\item poiché $\Sym(n, \KK) \cap \Lambda(n, \KK) = \zerovecset$ e
			$\dim \Sym(n, \KK) + \dim \Lambda(n, \KK) = \dim M(n, \KK)$,
			vale che $M(n, \KK) = \Sym(n, \KK) \oplus \Lambda(n, \KK)$,
			\item una base naturale di $\KK[x]$ è data da $\basis = \{x^n \mid
			n \in \NN \}$, mentre una di $\KK_t[x]$ è data da $\basis \cap
			\KK_t[x] = \{x^n \mid n \in \NN \land n \leq t\}$ (quindi
			$\dim \KK[x] = \infty$ e $\dim \KK_t[x] = t+1$),
			\item una base naturale di $\KK$ è $1_\KK = \{1_\KK\}$ (quindi
			$\dim \KK = 1$),
			\item un sottospazio di dimensione $1$ si definisce \textit{retta},
			uno di dimensione $2$ \textit{piano}, uno di dimensione $3$
			\textit{spazio}, e, infine, uno di dimensione $n-1$ un iperpiano,
			\item un iperpiano $\Pi$ è sempre rappresentabile da un'equazione cartesiana
			nelle coordinate della rappresentazione della base (infatti ogni
			iperpiano è il kernel di un funzionale $f \in \dual{V}$, e $M^\basis_{1_\KK}(f) \, [\vec{v}]_\basis = 0$ è l'equazione cartesiana; è sufficiente prendere una base di $\Pi$ e completarla
			a base di $V$ con un vettore $\vec{t}$, considerando infine
			$\Ker \dual{\vec{t}}$).
			
		\end{itemize}
		
		\subsection{Applicazioni lineari, somme dirette, quozienti e
			prodotti diretti}
		
		Un'applicazione da $V$ in $W$ si dice applicazione lineare
		se:
		
		\begin{itemize}
			\item $f(\vec{v} + \vec{w}) = f(\vec{v}) + f(\vec{w})$,
			\item $f(\alpha\vec{v}) = \alpha f(\vec{v})$.
		\end{itemize}
		
		Si definisce $\mathcal{L}(V, W) \subseteq W^V$ come lo spazio delle
		applicazioni lineari da $V$ a $W$. Si definisce
		$\End(V)$ come lo spazio degli endomorfismi di $V$, ossia
		delle applicazioni lineari da $V$ in $V$, dette anche
		operatori. Un'applicazione lineare si dice isomorfismo
		se è bigettiva. La composizione di funzioni è associativa.
		
		Dato un sottospazio $A$ di $V$, si definisce lo spazio
		quoziente $V/A$ come l'insieme quoziente $V/{\sim}$ della relazione
		di equivalenza $\vec{a} \sim \vec{b} \iff a-b \in A$ dotato
		dell'usuale somma e prodotto esterno. Si scrive $[\vec{v}]_A$
		come $\vec{v} + A$ e vale che $A = \vec{0} + A$. In particolare
		$\vec{v} + A = A \iff \vec{v} \in A$.
		
		Siano $f : V \to W$, $h : V \to W$, $g : W \to Z$ tre
		applicazioni lineari.
		$\basis_V$ e $\basis_W$ sono
		due basi rispettivamente di $V$ e $W$. In particolare
		sia $\basis_V = \{\vec{v_1}, \ldots, \vec{v_n}\}$. Si
		ricorda che $\rg(f) = \dim \Im f$. Siano $e$ ed $e'$ le
		basi canoniche rispettivamente di $\KK^n$ e $\KK^m$.
		
		\begin{itemize}
			\item $f(\vec{0}_V) = \vec{0}_W$,
			\item $\Ker f = f^{-1}(\vec{0}_W)$ è un sottospazio di $V$,
			\item $\Im f = f(V)$ è un sottospazio di $W$,
			\item $\Im f = \Span(f(\vec{v_1}), \ldots, f(\vec{v_n}))$,
			\item $f$ è iniettiva $\iff \Ker f = \zerovecset$,
			\item $V/\Ker f \cong \Im f$ (\textit{primo teorema d'isomorfismo}),
			\item $\dim \Ker f + \dim \Im f = \dim V$ (\textit{teorema del rango}, o formula delle dimensioni,
			valido se la dimensione di $V$ è finita),
			\item $g \circ f$ è un'applicazione lineare da $V$ in $Z$,
			\item la composizione di funzioni è associativa e distributiva
			da ambo i lati,
			\item $g \circ (\alpha f) = \alpha (g \circ f) = (\alpha g) \circ f$,
			se $\alpha \in \KK$,
			\item $\Ker f \subseteq \Ker (g \circ f)$,
			\item $\Im (g \circ f) \subseteq \Im g$,
			\item $\dim \Im (g \circ f) = \dim \Im \restr{g}{\Im f} =
			\dim \Im f - \dim \Ker \restr{g}{\Im f} = \dim \Im f -
			\dim (\Ker g \cap \Im f)$ (è sufficiente applicare la formula             delle dimensioni sulla composizione),
			\item $\dim \Im (g \circ f) \leq \min\{\dim \Im g, \dim \Im f\}$,
			\item $\dim \Ker (g \circ f) \leq \dim \Ker g + \dim \Ker f$ (è
			sufficiente applicare la formula delle dimensioni su
			$\restr{(g \circ f)}{\Ker (g \circ f)}$),
			\item $f$ iniettiva $\implies \dim V \leq \dim W$,
			\item $f$ surgettiva $\implies \dim V \geq \dim W$,
			\item $f$ isomorfismo $\implies \dim V = \dim W$,
			\item $g \circ f$ iniettiva $\implies f$ iniettiva,
			\item $g \circ f$ surgettiva $\implies g$ surgettiva,
			\item $f$ surgettiva $\implies \rg(g \circ f) = \rg(g)$,
			\item $g$ iniettiva $\implies \rg(g \circ f) = \rg(f)$,
			\item $M^{\basis_V}_{\basis_W}(f) = \begin{pmatrix} \; [f(\vec{v_1})]_{\basis_W} \, \mid \, \cdots \, \mid \, [f(\vec{v_n})]_{\basis_W} \; \end{pmatrix}$ è la matrice
			associata a $f$ sulle basi $\basis_V$, $\basis_W$,
			\item $M^V_W(f + h) = M^V_W(f) + M^V_W(h)$,
			\item $M^V_Z(g \circ f) = M^W_Z(g) M^V_W(f)$,
			\item data $A \in M(m, n, \KK)$, sia $f_A : \KK^n \to \KK^m$ tale
			che $f_A(\vec{x}) = A \vec{x}$, allora $M^{e}_{e'}(f_A) = A$,
			\item $f$ è completamente determinata dai suoi valori in una
			qualsiasi base di $V$ ($M^{\basis_V}_{\basis_W}$ è un isomorfismo
			tra $\mathcal{L}(V, W)$ e $M(\dim W, \dim V, \mathbb{K})$),
			\item $\dim \mathcal{L}(V, W) = \dim V \cdot \dim W$ (dall'isomorfismo
			di sopra),
			\item $[\,]^{-1}_{\basis_W} \circ M^{\basis_V}_{\basis_W}(f) \circ
			{[\,]_{\basis_V}} = f$,
			\item $[f(\vec{v})]_{\basis_W} = M^{\basis_V}_{\basis_W}(f) \cdot
			[\vec{v}]_{\basis_V}$,
			\item $\Im(f) = [\,]^{-1}_{\basis_W}\left(\Im M^{\basis_V}_{\basis_W}(f)\right)$
			\item $\rg(f) = \rg\left(M^{\basis_V}_{\basis_W}(f)\right)$,
			\item $\Ker(f) = [\,]^{-1}_{\basis_V}\left(\Ker M^{\basis_V}_{\basis_W}(f)\right)$,
			\item $\dim \Ker(f) = \dim \Ker M^{\basis_V}_{\basis_W}(f)$.
		\end{itemize}
		
		Siano $\basis_V'$, $\basis_W'$ altre due basi rispettivamente
		di $V$ e $W$. Allora vale il \textit{teorema del cambiamento
			di base}:
		
		\[ M^{\basis_V'}_{\basis_W'}(f) = M^{\basis_W}_{\basis_W'}(id_W) \,
		M^{\basis_V}_{\basis_W}(f) \, M^{\basis_V'}_{\basis_V}(id_V).\]
		
		Siano $A$ e $B$ due sottospazi di $V$. $\basis_A$ e $\basis_B$ sono
		due basi rispettivamente di $A$ e $B$.
		
		\begin{itemize}
			\item $A+B = \{\vec{a}+\vec{b} \in V \mid \vec{a} \in A, \vec{b} \in
			B\}$ è un sottospazio,
			\item $\dim (A+B) = \dim A + \dim B - \dim (A \cap B)$
			(\textit{formula di Grassmann}),
			\item $A$ e $B$ sono in somma diretta $\iff A \cap B = \zerovecset \iff$ ogni elemento di $A+B$ si scrive in modo unico come somma di
			$\vec{a} \in A$ e $\vec{b} \in B \iff \dim (A+B) = \dim A + \dim B$
			(in tal caso si scrive $A+B = A\oplus B$),
			\item $\dim V/A = \dim V - \dim A$ (è sufficiente applicare il
			teorema del rango alla proiezione al quoziente),
			\item $\dim V \times W = \dim V + \dim W$ ($\basis_V \times \{\vec{0}_W\} \cup \{\vec{0}_V\} \times \basis_W$ è una base
			di $V \times W$).
		\end{itemize}
		
		Si definisce \textit{immersione} da $V$ in $V \times W$
		l'applicazione lineare $i_V$ tale che $i_V(\vec{v}) = (\vec{v}, \vec{0})$.
		Si definisce \textit{proiezione} da $V \times W$ in $V$
		l'applicazione lineare $p_V$ tale che $p_V(\vec{v}, \vec{w}) = \vec{v}$.
		Analogamente si può fare con gli altri spazi del prodotto cartesiano.
		
		Si dice che $B$ è un supplementare di $A$ se $V = A \oplus B$ (ossia $\iff
		\dim A + \dim B = \dim V \, \land \, A \cap B = \zerovecset$). Il supplementare
		non è per forza unico. Per trovare un supplementare di $A$ è sufficiente
		completare $\basis_A$ ad una base $\basis$ di $V$ e considerare
		$B := \Span(\basis \setminus \basis_A)$.
		
		\subsubsection{Somma diretta di più sottospazi}
		
		Si dice che i sottospazi $W_1$, ..., $W_k$ di $V$ sono in somma
		diretta, e si scrive $W_1 + \ldots + W_k = W_1 \oplus \ldots \oplus W_k$,
		se la rappresentazione di un vettore della somma di questi sottospazi
		è unica, ossia se esistono unici $\ww 1 \in W_1$, ..., $\ww k \in W_k$ tali
		per cui $\w \in W_1 + \ldots + W_k$ si scrive come $\w = \ww 1 + \ldots + \ww k$. \\
		
		In generale, sono equivalenti i seguenti fatti:
		
		\begin{enumerate}[(i)]
			\itemsep0em
			\item $W_1$, ..., $W_k$ sono in somma diretta,
			\item Se esistono $\ww 1 \in W_1$, ..., $\ww k \in W_k$ tali per cui
			$\ww 1 + \ldots + \ww k = \vec 0$, allora $\ww 1 = \cdots = \ww k = \vec 0$ (è sufficiente considerare due scritture alternative e poi farne la differenza per dimostrare un'implicazione),
			\item Se $\basis_{W_1}$, ..., $\basis_{W_k}$ sono basi di $W_1$, ..., $W_k$,
			allora $\bigcup_{i=1}^k \basis_{W_i}$ è base di $W_1 + \ldots + W_k$ (è sufficiente considerare l'indipendenza lineare per dimostrare un'implicazione),
			\item $\dim (W_1 + \ldots + W_k) = \dim W_1 + \ldots + \dim W_k$ (si dimostra facilmente che è equivalente a (iii), e quindi che lo è alle altre proposizioni),
			\item $W_i \cap (W_1 + \ldots + W_{i-1}) = \zerovecset$ $\forall 2 \leq i \leq k$ (è sufficiente spezzare la somma in $(W_1 + \ldots + W_{i-1}) + W_i$ e ricondursi al caso di due sottospazi, mostrando in particolare, per induzione, l'equivalenza con (iv), da cui seguono le altre equivalenze),
			\item $W_i \cap (W_1 + \ldots + W_{i-1} + \widehat{W_i} + W_{i+1} + W_k) = \zerovecset$ $\forall 1 \leq i \leq k$, ossia $W_i$, intersecato con la somma
			dei restanti sottospazi, è di dimensione nulla (è facile ricondursi alla proposizione (v) per induzione).
		\end{enumerate}
		
		\subsection{Proprietà generali delle matrici}
		
		Si dice che una matrice $A \in M(n, \KK)$ è singolare se $\det(A) = 0$,
		o equivalentemente se non è invertibile. Compatibilmente, si
		dice che una matrice $A \in M(n, \KK)$ è non singolare se $\det(A) \neq
		0$, ossia se $A$ è invertibile.
		
		Si definisce la matrice trasposta di
		$A \in M(m, n, \KK)$, detta $A^\top$, in modo
		tale che $A_{ij} = A^\top_{ji}$.
		
		\begin{itemize}
			\item $(AB)^\top = B^\top A^\top$,
			\item $(A+B)^\top = A^\top + B^\top$,
			\item $(\lambda A)^\top = \lambda A^\top$,
			\item $(A^\top)^\top = A$,
			\item se $A$ è invertibile, $(A^\top)^{-1} = (A^{-1})^\top$,
			\item $ \begin{pmatrix}
				A
				& \rvline & B \\
				\hline
				C & \rvline &
				D
			\end{pmatrix}\begin{pmatrix}
				E
				& \rvline & F \\
				\hline
				G & \rvline &
				H
			\end{pmatrix}=\begin{pmatrix}
				AE+BG
				& \rvline & AF+BH  \\
				\hline
				CE+DG & \rvline &
				CF+DH
			\end{pmatrix}$.
		\end{itemize}
		
		Siano $A \in M(m, n, \KK)$ e $B \in M(n, m, \KK)$.
		
		Si definisce $\GL(n, \KK)$ come il gruppo delle matrici
		di taglia $n$ invertibili sulla moltiplicazione matriciale. Si definisce
		triangolare superiore una matrice i cui elementi al di sotto
		della diagonale sono nulli, mentre si definisce triangolare
		inferiore una matrice i cui elementi nulli sono quelli al di sopra
		della diagonale.
		
		Si definiscono
		\[ Z(M(n, \KK)) = \left\{ A \in M(n, \KK) \mid AB=BA \, \forall B \in M(n, \KK) \right\}, \]
		ossia l'insieme delle matrici che commutano con tutte le altre matrici, e
		\[ Z_{\GL}(M(n, \KK)) = \left\{ A \in M(n, \KK) \mid AB=BA \, \forall B \in \GL(n, \KK) \right\}, \]
		ovvero l'insieme delle matrici che commutano con tutte le matrici
		di $\GL(n, \KK)$.
		
		Si definisce $\tr \in M(m, \KK)^*$ come il funzionale che associa
		ad ogni matrice la somma degli elementi sulla sua diagonale. 
		
		\begin{itemize}
			\item $\tr(A^\top) = \tr(A)$,
			\item $\tr(AB) = \tr(BA)$,
			\item $Z(M(n, \KK)) = \Span(I_n)$,
			\item $Z_{\GL}(M(n, \KK)) = \Span(I_n)$.
		\end{itemize}
		
		Sia $A \in M(n, \KK)$. Sia $C_A \in \End(M(n, \KK))$ definito in modo
		tale che $C_A(B) = AB - BA$. Allora $\Ker C_A = M(n, \KK)
		\iff A \in \Span(I_n)$. Siano $I$ un insieme di $n^2$ indici
		distinti, allora l'insieme
		
		\[ T = \left\{ A^i \mid i \in I \right\} \]
		
		è linearmente dipendente (è sufficiente notare che
		se così non fosse, se $A \notin \Span(I_n)$,
		tale $T$ sarebbe base di $M(n, \KK)$, ma
		così $\Ker C_A = M(n, \KK) \implies A \in \Span(I_n)$,
		\Lightning{}, e che se $A \in \Span(I_n)$, $T$
		è chiaramente linearmente dipendente).
		
		In generale esiste sempre un polinomio $p(X) \in \KK[x]$
		di grado $n$ tale per cui $p(A) = 0$, dove un tale polinomio
		è per esempio il polinomio caratteristico di $p$, ossia $p(\lambda)=
		\det(\lambda I_n - A)$ (\textit{teorema di 
			Hamilton-Cayley}).
		
		\subsection{Rango di una matrice}
		
		Si definisce rango di una matrice $A$ il numero di colonne linearmente
		indipendenti di $A$. Siano $A$, $B \in M(m, n, \KK)$.
		
		\begin{itemize}
			\item $\rg(A) = \rg(A^\top)$ (i.e.~il rango è lo stesso se calcolato
			sulle righe invece che sulle colonne),
			\item $\rg(A) \leq \min\{m, n\}$ (come conseguenza dell'affermazione
			precedente),
			\item $\rg(A+B) \leq \rg(A) + \rg(B) \impliedby \Im (A+B) \subseteq
			\Im(A) + \Im(B)$,
			\item $\rg(A+B) = \rg(A) + \rg(B) \implies \Im(A+B) = \Im(A) \oplus \Im(B)$ (è sufficiente applicare la formula di Grassmann),
			\item $\rg(A)$ è il minimo numero di matrici di rango uno che
			sommate restituiscono $A$ (è sufficiente usare la proposizione
			precedente per dimostrare che devono essere almeno $\rg(A)$),
			\item $\rg(A)=1 \implies \exists B \in M(m, 1, \KK)$, $C \in M(1, n, \KK) \mid A=BC$ (infatti $A$ può scriversi come $\begin{pmatrix}[c|c|c]\alpha_1 A^i & \cdots & \alpha_n A^i \end{pmatrix}$ per un certo $i \leq n$ tale che $A^i \neq \vec{0}$).
		\end{itemize}
		
		Siano $A \in M(m, n, \KK)$, $B \in M(n, k, \KK)$ e $C \in M(k, t, \KK)$.
		
		\begin{itemize}
			\item $\rg(AB) \geq \rg(A) + \rg(B) - n$ (\textit{disuguaglianza
				di Sylvester} -- è sufficiente
			usare la formula delle dimensioni ristretta alla composizione
			$f_A \circ f_B$),
			\item $\rg(ABC) \geq \rg(AB) + \rg(BC) - \rg(B)$ (\textit{disuguaglianza di Frobenius}, di cui la proposizione
			precedente è un caso particolare con $B = I_n$ e $k=n$),
			\item $\rg(AB) = \rg(B) \impliedby \Ker A = \zerovecset$ (è
			sufficiente usare la formula delle dimensioni ristretta
			alla composizione $f_A \circ f_B$),
			\item $\rg(AB) = \rg(A) \impliedby f_B$ surgettiva (come sopra).
		\end{itemize}
		
		Sia $A \in M(n, \KK)$.
		
		\begin{itemize}
			\item se $A$ è antisimmetrica e il campo su cui si fonda
			lo spazio vettoriale non ha caratteristica $2$, allora
			$\rg(A)$ è pari,
			\item $\rg(A) = n \iff \dim \Ker A = 0 \iff \det(A) \neq 0 \iff A$ è invertibile,
		\end{itemize}
		
		\subsection{Sistemi lineari, algoritmo di eliminazione di Gauss ed 
			SD-equivalenza}
		
		Un sistema lineare di $m$ equazioni in $n$ variabili può essere
		rappresentato nella forma $A\vec{x} = B$, dove $A \in M(m, n, \KK)$,
		$\vec{x} \in \KK^n$ e $B \in \KK^m$. Un sistema lineare si
		dice omogeneo se $B = \vec{0}$. In tal caso l'insieme delle soluzioni del
		sistema coincide con $\Ker A = \Ker f_A$, dove $f_A : \KK^n \to \KK^m$ è 
		l'applicazione lineare indotta dalla matrice $A$. Le soluzioni
		di un sistema lineare sono raccolte nel sottospazio affine
		$\vec{s} + \Ker A$, dove $\vec{s}$ è una qualsiasi soluzione
		del sistema completo.
		
		\begin{itemize}
			\item $A\vec{x} = B$ ammette soluzione se e solo se
			$B \in \Span(A^1, \ldots, A^n) \iff \Span(A^1, \ldots, A^n, B) =
			\Span(A^1, \ldots, A^n) \iff \dim \Span(A^1, \ldots, A^n, B) =
			\dim \Span(A^1, \ldots, A^n) \iff
			\dim \Im (A \mid B) = \dim \Im A \iff \rg (A \mid B) = \rg (A)$
			(\textit{teorema di Rouché-Capelli}),
			\item $A\vec{x} = B$, se la ammette, ha un'unica soluzione
			se e solo se $\Ker A = \zerovecset \iff \rg A = n$.
		\end{itemize}
		
		Si definiscono tre operazioni sulle righe di una matrice $A$:
		
		\begin{enumerate}
			\item l'operazione di scambio di riga,
			\item l'operazione di moltiplicazione di una riga
			per uno scalare non nullo,
			\item la somma di un multiplo non nullo di una riga
			ad un'altra riga distinta.
		\end{enumerate}
		
		Queste operazioni non variano né $\Ker A$ né $\rg (A)$. Si possono effettuare le stesse medesime operazioni
		sulle colonne (variando tuttavia $\Ker A$, ma lasciando
		invariato $\Im A$ -- e quindi $\rg (A)$). L'algoritmo di eliminazione di Gauss
		procede nel seguente modo:
		
		\begin{enumerate}
			\item se $A$ ha una riga, l'algoritmo termina;
			\item altrimenti si prenda la prima riga di $A$ con il primo elemento
			non nullo e la si scambi con la prima riga di $A$ (in caso
			non esista, si proceda all'ultimo passo),
			\item per ogni riga di $A$ con primo elemento non nullo,
			esclusa la prima, si sottragga un multiplo della prima riga in modo
			tale che la riga risultante abbia il primo elemento nullo,
			\item si ripeta l'algoritmo considerando come matrice $A$ la
			matrice risultante dall'algoritmo senza la prima riga e la
			prima colonna (in caso tale matrice non possa esistere,
			l'algoritmo termina).
		\end{enumerate}
		
		Si definiscono \textit{pivot} di una matrice l'insieme dei primi
		elementi non nulli di ogni riga della matrice.
		Il rango della matrice iniziale $A$ è pari al numero di \textit{pivot}
		della matrice risultante dall'algoritmo di eliminazione di Gauss.
		Una matrice che processata dall'algoritmo di eliminazione di Gauss
		restituisce sé stessa è detta matrice a scala.
		
		Agendo solo attraverso
		operazioni per riga, l'algoritmo di eliminazione di Gauss non
		modifica $\Ker A$ (si può tuttavia integrare l'algoritmo con le
		operazioni per colonna, perdendo quest'ultimo beneficio).
		
		Agendo
		su una matrice a scala con operazioni per riga considerando
		la matrice riflessa (ossia dove l'elemento $(1, 1)$ e $(m, n)$ sono
		scambiati), si può ottenere una matrice a scala ridotta,
		ossia un matrice dove tutti i pivot sono $1$ e dove tutti
		gli elementi sulle colonne dei pivot, eccetto i pivot stessi,
		sono nulli.
		
		Si definisce:
		
		\[I^{m \times n}_r =
		\begin{pmatrix}
			I_r
			& \rvline & \bigzero \\
			\hline
			\bigzero & \rvline &
			\bigzero
		\end{pmatrix} \in M(m, n, \KK). \]
		
		Per ogni applicazione lineare $f : V \to W$, con $\dim V = n$ e
		$\dim W = m$ esistono due basi $\basis_V$, $\basis_W$ rispettivamente
		di $V$ e $W$ tale che $M^{\basis_V}_{\basis_W}(f) = I^{m \times n}_r$,
		dove $r=\rg(f)$ (è sufficiente completare con $I$ a base di $V$ una base
		di $\Ker f$ e poi prendere come base di $W$ il completamento di $f(I)$
		su una base di $W$).
		
		Si definisce SD-equivalenza la relazione d'equivalenza su
		$M(m, n, \KK)$ indotta dalla relazione $A \sim_{SD} B \iff \exists P \in
		\GL(m, \KK)$, $Q \in \GL(n, \KK) \mid A=PBQ$. L'invariante completo
		della SD-equivalenza è il rango: $\rg(A) = \rg(B) \iff A \sim_{SD} B$
		(infatti $\rg(A) = r \iff A \sim_{SD} I^{m \times n}_r$ -- è sufficiente
		applicare il cambio di base e sfruttare il fatto che esistono
		sicuramente due basi per cui $f_A$ ha $I^{m \times n}_r$ come
		matrice associata).
		
		Poiché $I^{m \times n}_r$ ha sempre rango $r$, l'insieme
		quoziente della SD-equivalenza su $M(m, n, \KK)$ è il seguente:
		
		\[ M(m, n, \KK)/{\sim_{SD}} = \left\{[\vec{0}], \left[I^{m \times n}_1\right], \ldots, \left[I^{m \times n}_{\min\{m, n\}}\right] \right\}, \]
		
		contenente esattamente $\min\{m, n\}$ elementi. L'unico elemento
		di $[\vec{0}]$ è $\vec{0}$ stesso.
		
		\subsubsection{La regola di Cramer}
		
		Qualora $m=n$ e $A$ fosse invertibile (i.e. $\det(A) \neq 0$),
		per calcolare il valore di $\vec{x}$ si può applicare
		la regola di Cramer.
		
		Si definisce:
		\[ A_i^* = \begin{pmatrix}
			A^1 & \cdots & A^i \to B & \cdots & A^n
		\end{pmatrix}, \]
		
		dove si sostituisce alla $i$-esima colonna di $A$ il vettore $B$. Allora
		vale la seguente relazione:
		
		\[ \vec{x} = \frac{1}{\det(A)} \begin{pmatrix}
			\det(A_1^*) \\ \vdots \\ \det(A_n^*)
		\end{pmatrix}. \]
		
		\subsection{L'inverso (generalizzato e non) di una matrice}
		
		Si definisce matrice dei cofattori di una matrice $A \in M(n, \KK)$ la
		seguente matrice:
		
		\[ \Cof A = \begin{pmatrix}
			\Cof_{1,1}(A) & \ldots & \Cof_{1,n}(A) \\
			\vdots & \ddots & \vdots \\ 
			\Cof_{n,1}(A) & \ldots & \Cof_{n,n}(A),
		\end{pmatrix}, \]
		
		dove, detta $A_{i,j}$ il minore di $A$ ottenuto eliminando
		la $i$-esima riga e la $j$-esima colonna, si definisce il cofattore (o
		complemento algebrico) nel seguente modo:
		
		\[ \Cof_{i,j}(A) = (-1)^{i+j} \det( A_{i, j}). \]
		
		Si definisce inoltre l'aggiunta classica:
		
		\[ \adj(A) = (\Cof A)^\top. \]
		
		Allora, se $A$ ammette un inverso (i.e. se $\det(A) \neq 0$),
		vale la seguente relazione:
		
		\[ A^{-1} = \frac{1}{\det(A)} \adj(A). \]
		
		\vskip 0.05in
		
		Quindi, per esempio, $A^{-1}$ è a coefficienti
		interi $\iff \det(A) = \pm 1$.
		
		Siano $A$, $B \in M(n, \KK)$.
		
		\begin{itemize}
			\item $\adj(AB) = \adj(B)\adj(A)$,
			\item $\adj(A^\top) = \adj(A)^\top$.
		\end{itemize}
		
		Si definisce inverso generalizzato di una matrice $A \in M(m, n, \KK)$
		una matrice $X \in M(n, m, \KK) \mid AXA=A$. Ogni matrice ammette
		un inverso generalizzato (è sufficiente considerare gli inversi
		generalizzati di $I^{m \times n}_r$ e la SD-equivalenza di $A$
		con $I^{m \times n}_r$, dove $\rg(A)=r$). Se $m=n$ ed $A$ è invertibile, allora
		$A^{-1}$ è l'unico inverso generalizzato di $A$. Gli inversi
		generalizzati di $I^{m \times n}_r$ sono della forma:
		
		\[X =
		\begin{pmatrix}
			I_r
			& \rvline & B \\
			\hline
			C & \rvline &
			D
		\end{pmatrix} \in M(m, n, \KK). \]
		
		\subsection{Endomorfismi e similitudine}
		
		Si definisce la similitudine tra matrici su $M(n, \KK)$ come la relazione
		di equivalenza determinata da $A \sim B \iff \exists P \in \GL(n, \KK)
		\mid A = PBP^{-1}$. $A \sim B \implies \rg(A)=\rg(B)$, $\tr(A)=\tr(B)$,
		$\det(A)=\det(B)$, $P_\lambda(A) = P_\lambda(B)$ (invarianti \textit{non completi} della similitudine).
		Vale inoltre che $A \sim B \iff A$ e $B$ hanno la stessa forma
		canonica di Jordan, a meno di permutazioni dei blocchi di Jordan
		(invariante \textit{completo} della similitudine). La matrice
		identità è l'unica matrice identica a sé stessa.
		
		Sia $p \in \End(V)$. Si dice che un endomorfismo è un automorfismo
		se è un isomorfismo. Gli automorfismi formano un sottospazio vettoriale
		di $\End(V)$ denotato con $\Aut(V)$ o $\GL(V)$. Siano $\basis$, $\basis'$ due qualsiasi
		basi di $V$.
		
		\begin{itemize}
			\item $p$ automorfismo $\iff p$ iniettivo $\iff p$ surgettivo (è
			sufficiente applicare la formula delle dimensioni),
			\item $M^\basis_{\basis'}(id_V) M^{\basis'}_\basis(id_V)
			= I_n$ (dunque entrambe le matrici sono invertibili e sono
			l'una l'inverso dell'altra),
			\item se $p$ è un automorfismo, $M^\basis_{\basis'}(p^{-1}) =
			M^{\basis'}_\basis(p)^{-1}$,
			\item $M^\basis_{\basis}(p) = \underbrace{M^{\basis'}_\basis (id_V)}_{P} \,
			M^{\basis'}_{\basis'}(p) \,
			\underbrace{M^{\basis}_{\basis'} (id_V)}_{P^{-1}}$ (ossia
			$M^\basis_{\basis}(p) \sim M^{\basis'}_{\basis'}(p)$).
		\end{itemize}
		
		$M^\basis_{\basis'}(id_V) M^{\basis'}_\basis(id_V)
		= I_n$. Dunque entrambe le matrici sono invertibili. Inoltre
		$M^\basis_\basis(id_V) = I_n$.
		
		Si definisce un analogo della similitudine anche per gli endomorfismi:
		due endomorfismi $f$, $g \in \End(V)$ si dicono coniugati se e solo se $\exists
		h \in \GL(V) \mid f = h \, g \, h\inv$. Il coniugio induce in particolare
		un'altra relazione di equivalenza. Due endomorfismi $f$ e $g$ sono coniugati
		se e solo se le loro matrici associate nella stessa base $\basis$ sono simili.
		
		\subsubsection{Duale, biduale e annullatore}
		
		Si definisce duale di uno spazio vettoriale $V$ lo
		spazio $\dual{V} = \mathcal{L}(V, \KK)$, i cui elementi
		sono detti funzionali. Analogamente
		il biduale è il duale del duale di $V$: $\bidual{V} = \dual{(\dual{V})} = \mathcal{L}(\dual{V}, \KK)$.
		
		Sia data una base $\basis = \{\vec{v_1}, \ldots, \vec{v_n}\}$ di
		uno spazio vettoriale $V$ di dimensione $n$. Allora $\dim \dual{V}
		= \dim \mathcal{L}(V, \KK) = \dim V \cdot \dim \KK = \dim V$. Si definisce
		il funzionale $\dual{\vec{v_i}}$ come l'applicazione lineare
		univocamente determinata dalla relazione:
		
		\[ \dual{\vec{v_i}}(\vec{v_j}) = \delta_{ij}. \]
		
		\vskip 0.05in
		
		Sia $\basis^* = \{\vec{v_1}^*, \ldots, \vec{v_n}^*\}$. Allora
		$\basis^*$ è una base di $\dual{V}$. Poiché $V$ e $\dual{V}$
		hanno la stesso dimensione, tali spazi sono isomorfi, sebbene
		non canonicamente. Ciononostante, $V$ e $\bidual{V}$ sono
		canonicamente isomorfi tramite l'isomorfismo:
		
		\[ \bidual{\varphi} : V \to \bidual{V}, \; \vec{v} \mapsto \restr{\val}{\dual{V}}, \]
		
		che associa ad ogni vettore $\vec{v}$ la funzione
		di valutazione in una funzionale in $\vec{v}$, ossia:
		
		\[ \restr{\val}{\dual{V}} : \dual{V} \to \KK, \; f \mapsto f(\vec{v}). \]
		
		Sia $U \subseteq V$ un sottospazio di $V$.
		Si definisce il sottospazio di $\mathcal{L}(V, W)$:
		
		\[ \Ann_{\mathcal{L}(V, W)}(U) = \left\{ f \in \mathcal{L}(V, W) \mid f(U) = \zerovecset \right\}. \]
		
		Se $V$ è a dimensione finita, la dimensione di
		$\Ann(U)$ è pari a $(\dim V - \dim U) \cdot \dim W$ (è sufficiente
		prendere una base di $U$, completarla a base di $V$ e
		notare che $f(U) = \zerovecset \iff$ ogni valutazione
		in $f$ degli elementi della base di $U$ è nullo $\iff$ la matrice
		associata di $f$ ha tutte colonne nulle in corrispondenza degli
		elementi della base di $U$).
		
		Si scrive semplicemente $\Ann(U)$ quando $W=\KK$ (ossia
		quando le funzioni sono funzionali di $V$). In tal
		caso $\dim \Ann(U) = \dim V - \dim U$.
		
		\begin{itemize}
			\item $\bidual{\varphi}(U) \subseteq \Ann(\Ann(U))$,
			\item se $V$ è a dimensione finita, $\bidual{\varphi}(U) = \bidual{U} = \Ann(\Ann(U))$ (è sufficiente
			applicare la formula delle dimensioni $\restr{\bidual{\varphi}}{U}$ e notare l'uguaglianza
			tra le due dimensioni),
			\item se $V$ è a dimensione finita e $W$ è un altro
			sottospazio di $V$,
			$U = W \iff \Ann(U) = \Ann(W)$ (è sufficiente
			considerare $\Ann(\Ann(U)) = \Ann(\Ann(W))$ e
			applicare la proposizione precedente, ricordandosi
			che $\bidual{\varphi}$ è un isomorfismo, ed è
			dunque iniettivo).
		\end{itemize}
		
		Si definisce l'applicazione trasposta $^\top$ da $\mathcal{L}(V, W)$ a
		$\mathcal{L}(\dual{W}, \dual{V})$ in modo tale che $f^\top(g)
		= g \circ f \in \dual{V}$. Siano $f$, $g \in \mathcal{L}(V,W)$ e
		sia $h \in \mathcal{L}(W,Z)$.
		
		\begin{itemize}
			\item $(f+g)^\top = f^\top + g^\top$,
			\item $(\lambda f)^\top = \lambda f^\top$,
			\item se $f$ è invertibile, $(f^{-1})^\top = (f^\top)^{-1}$,
			\item $(h \circ f)^\top = f^\top \circ h^\top$.
		\end{itemize}
		
		Siano $\basis_V$, $\basis_W$ due basi rispettivamente di $V$ e
		di $W$. Allora vale la seguente relazione:
		
		\[ M^{\basis_W^*}_{\basis_V^*}(f^\top) = M^{\basis_V}_{\basis_W}(f)^\top. \]
		
		\subsection{Applicazioni multilineari}
		
		Sia $f : V_1 \times \ldots \times V_n \to W$ un'applicazione, dove
		$V_i$ è uno spazio vettoriale $\forall i \leq n$, così come $W$. Tale
		applicazione si dice $n$-lineare ed appartiene allo spazio
		$\Mult(V_1 \times \ldots \times V_n, W)$, se è lineare in ogni sua coordinata, ossia se:
		
		\begin{itemize}
			\item $f(x_1, \ldots, x_i + y_i, \ldots, x_n) =
			f(x_1, \ldots, x_i, \ldots, x_n) + f(x_1, \ldots, y_i, \ldots, x_n)$,
			\item $f(x_1, \ldots, \alpha x_i, \ldots, x_n) = \alpha f(x_1, \ldots, x_i, \ldots, x_n)$.
		\end{itemize}
		
		Sia $W=\KK$, e siano tutti gli spazi $V_i$ fondati su tale campo: allora
		$\Mult(V_1 \times \ldots \times V_n, \KK)$ si scrive anche come $V_1^* \otimes \ldots \otimes V_n^*$, e tale spazio
		è detto prodotto tensoriale tra $V_1$, ..., $V_n$.
		Sia $V_i$ di dimensione finita $\forall i \leq n$. Siano $\basis_{V_i} = \left\{ \vec{v^{(i)}_1}, \ldots,  \vec{v^{(i)}_{k_i}} \right\}$ base
		di $V_i$, dove $k_i = \dim V_i$.
		Si definisce l'applicazione $n$-lineare $\dual{\vec{v^{(1)}_{j_1}}}
		\otimes \cdots \otimes \dual{\vec{v^{(n)}_{j_n}}}\in \Mult(V_1 \times \ldots
		\times V_n, \KK)$ univocamente determinata dalla relazione:
		
		\[ \dual{\vec{v^{(1)}_{j_1}}} \otimes \cdots \otimes \dual{\vec{v^{(n)}_{j_n}}}(\vec{w_1}, \ldots, \vec{w_n}) = \dual{\vec{v^{(1)}_{j_1}}}(\vec{w_1}) \cdots \dual{\vec{v^{(n)}_{j_n}}}(\vec{w_n}).   \]
		
		Si definisce l'insieme $\basis_{\otimes}$ nel seguente modo:
		
		\[ \basis_{\otimes} = \left\{ \dual{\vec{v^{(1)}_{j_1}}} \otimes \cdots \otimes \dual{\vec{v^{(n)}_{j_n}}} \mid 1 \leq j_1 \leq k_1, \, \ldots, \, 1 \leq j_n \leq k_n \right\}. \]
		
		Poiché ogni applicazione $n$-lineare è univocamente determinata
		dai valori che assume ogni combinazione degli elementi delle basi
		degli spazi $V_i$, vi è un isomorfismo tra $\Mult(V_1 \times \ldots
		\times V_n, \KK)$ e $\KK^{\basis_{V_1} \times \cdots \times \basis_{V_n}}$, che ha dimensione $\prod_{i=1}^n k_i = k$. Pertanto
		anche $\dim \Mult(V_1 \times \ldots \times V_n, \KK) = k$.
		
		Poiché $\basis_{\otimes}$ genera $\Mult(V_1 \times \ldots
		\times V_n, \KK)$ e i suoi elementi sono tanti quanto è la
		dimensione dello spazio, tale insieme è una base di $\Mult(V_1 \times 
		\ldots \times V_n, \KK)$.
		
		Se $V_i = V_1 = V$ $\forall i \leq n$, si dice che $\Mult(V^n, \KK)$
		è lo spazio delle forme $n$-lineari di $V$. 
		
		\subsubsection{Applicazioni multilineari simmetriche}
		
		Sia $V$ uno spazio di dimensione $n$. Una 
		forma $k$-lineare $f$ si dice simmetrica 
		ed appartiene allo spazio $\Sym^k(V)$ se:
		
		\[ f(\vec{x_1}, \ldots, \vec{x_k}) = f(\vec{x_{\sigma(1)}}, \ldots, \vec{x_{\sigma(k)}}), \quad \forall \sigma \in S_k. \]
		
		Poiché ogni applicazione $n$-lineare simmetrica è univocamente
		determinata dai valori che assume negli elementi della base
		disposti in modo non decrescente, $\dim \Sym^k(V) = \binom{n+k-1}{k}$.
		
		Sia $\basis = \{\vec{v_1}, \ldots, \vec{v_n}\}$ una base
		di $V$. Dato un insieme di indici non decrescente $I$,
		si definisce il prodotto simmetrico (o \textit{prodotto vee}) 
		$\dual{\vec{v_{i_1}}} \vee \cdots \vee \dual{\vec{v_{i_k}}}$
		tra elementi della base come la forma $k$-lineare simmetrica
		determinata dalla relazione:
		
		\[ \dual{\vec{v_{i_1}}} \vee \cdots \vee \dual{\vec{v_{i_k}}} = \sum_{\sigma \in S_k} \dual{\vec{v_{i_{\sigma(1)}}}} \otimes \cdots \otimes \dual{\vec{v_{i_{\sigma(k)}}}}. \]
		
		Si definisce l'insieme:
		
		\[\basis_{\Sym} = \left\{  \dual{\vec{v_{i_1}}} \vee \cdots \vee \dual{\vec{v_{i_k}}} \mid 1 \leq i_1 \leq \cdots \leq i_k \leq n \right\}. \]
		
		L'insieme $\basis_{\Sym}$ è sia generatore che linearmente
		indipendente su $\Sym^k(V)$, ed è dunque base. Allora
		$\dim \Sym^k(V) = \binom{n+k-1}{k}$.
		
		\subsubsection{Applicazioni multilineari alternanti}
		
		Sia $V$ uno spazio di dimensione $n$. Una forma
		$k$-lineare $f$ si dice alternante (o antisimmetrica)
		ed appartiene allo spazio $\Lambda^k(V)$ (talvolta scritto
		come $\operatorname{Alt}^k(V)$) se:
		
		\[ f(x_1, \ldots, x_k) = 0 \impliedby \exists \, i, j \leq k \mid x_i = x_j. \]
		
		\vskip 0.05in
		
		Questo implica che:
		
		\[ f(x_1, \ldots, x_k) = \sgn(\sigma) f(x_{\sigma(1)}, \ldots, x_{\sigma(n)}), \quad \forall \sigma \in S_k \]
		
		Se $k > n$, un argomento della base di $V$ si ripete sempre nel
		computo $f$ negli elementi della base, e quindi ogni alternante è
		pari a $\vec{0}$, ossia $\dim \Lambda^k(V) = 0$.
		
		Sia $\basis = \{\vec{v_1}, \ldots, \vec{v_n}\}$ una base
		di $V$. Dato un insieme di indici crescente $I$,
		si definisce il prodotto esterno (o \textit{prodotto wedge}) 
		$\dual{\vec{v_{i_1}}} \wedge \cdots \wedge \dual{\vec{v_{i_k}}}$
		tra elementi della base come la forma $k$-lineare alternante
		determinata dalla relazione:
		\[ \dual{\vec{v_{i_1}}} \wedge \cdots \wedge \dual{\vec{v_{i_k}}} = \sum_{\sigma \in S_k} \sgn(\sigma) \, \dual{\vec{v_{i_{\sigma(1)}}}} \otimes \cdots \otimes \dual{\vec{v_{i_{\sigma(k)}}}}. \]
		
		Si definisce l'insieme:
		
		\[\basis_{\Lambda} = \left\{  \dual{\vec{v_{i_1}}} \wedge \cdots \wedge \dual{\vec{v_{i_k}}} \mid 1 \leq i_1 < \cdots < i_k \leq n \right\}. \]
		
		L'insieme $\basis_{\Lambda}$ è sia generatore che linearmente
		indipendente su $\Lambda^k(V)$, ed è dunque base. Allora
		$\dim \Lambda^k(V) = \binom{n}{k}$. Riassumendo si può scrivere:
		
		\[\dim \Lambda^k(V) = \begin{cases} 0 & \text{se } k > n\,, \\ \binom{n}{k} & \text{altrimenti}. \end{cases}\]
		
		Quindi è quasi sempre vero che:
		
		\[ \underbrace{\dim \Sym^k(V)}_{= \, \binom{n+k-1}{k}} + \underbrace{\dim \Lambda^k(V)}_{\leq \, \binom{n}{k}} < \underbrace{\dim \Mult(V^k, \KK)}_{=\,n^k}, \]
		
		e dunque che $\Sym^k(V) + \Lambda^k(V) \neq \Mult(V^k, \KK)$.
		
		
		
		\subsection{Determinante di una matrice}
		
		Si definisce il determinante $\det$ di una matrice di taglia
		$n \times n$ come l'unica forma $n$-lineare alternante di $(\KK^n)^n$
		tale che $\det(\vec{e_1}, \ldots, \vec{e_n}) = 1$ (infatti
		$\dim \Lambda^n (V) = \binom{n}{n} = 1$, e quindi ogni forma
		alternante è multipla delle altre, eccetto per lo zero).
		
		Equivalentemente $\det = \dual{\vec{e_1}} \, \wedge \cdots \wedge \, \dual{\vec{e_n}}$.
		
		Siano $A$, $B \in M(n, \KK)$. Si scrive
		$\det(A)$ per indicare $\det(A_1, \ldots, A_n)$. Vale pertanto la
		seguente relazione:
		
		\[ \det(A) = \sum_{\sigma \in S_n} \sgn(\sigma) \, a_{1\sigma(1)} \cdots a_{n\sigma(n)}. \]
		
		\begin{itemize}
			\item $\det(I_n) = 1$,
			\item $\det \begin{pmatrix}
				a & b \\ c & d
			\end{pmatrix} = ad-bc$,
			\item $\det \begin{pmatrix}
				a & b & c \\ d & e & f \\ g & h & i
			\end{pmatrix} = a(ei-fh) - b(di-fg) + c(dh-eg)$,
			\item $\det(A) \neq 0 \iff A$ invertibile (ossia non singolare),
			\item $\det(\lambda A) = \lambda^n A$,
			\item $\det(A) = \det(A^\top)$ (è sufficiente applicare la definizione
			di $\det$ e manipolare algebricamente il risultato per evidenziare
			l'uguaglianza),
			\item se $A$ è antisimmetrica, $n$ è dispari e $\Char \KK \neq 2$,
			$\det(A) = \det(-A^\top) = (-1)^n \det(A^\top) = (-1)^n \det(A) = -\det(A) \implies \det(A) = 0$ (quindi ogni matrice antisimmetrica di taglia
			dispari non è invertibile),
			\item $\det(AB) = \det(A)\det(B)$ (\textit{teorema di Binet} -- è
			sufficiente considerare la forma $\frac{\det(AB)}{\det(B)}$ in
			funzione delle righe di $A$ e determinare che tale forma
			è alternante e che vale $1$ nell'identità, e che, per l'unicità
			del determinante, deve obbligatoriamente essere pari a
			$\det(A)$),
			\item se $A$ è invertibile, $\det(A^{-1}) = \det(A)^{-1}$,
			\item $\det   \begin{pmatrix}
				\lambda_{1} & & \\
				& \ddots & \\
				& & \lambda_{n}
			\end{pmatrix} = \det(\lambda_1 \vec{e_1}, \ldots, \lambda_n \vec{e_n}) = \prod_{i=1}^n \lambda_i$,
			\item se $A$ è triangolare superiore (o inferiore), allora $\det(A)$ è
			il prodotto degli elementi sulla sua diagonale principale,
			\item $\det(A_1, \ldots, A_n) = \sgn(\sigma) \det(A_{\sigma(1)}, \ldots, A_{\sigma(n)})$, $\forall \sigma \in S_n$ (infatti $\det$ è alternante),
			\item \setlength{\extrarowheight}{1.3pt}$\det \begin{pmatrix}
				A
				& \rvline & B \\
				\hline
				C & \rvline &
				D
			\end{pmatrix} = \det(AD-BC)$, se $C$ e $D$ commutano e $D$ è invertibile,
			\item $\det \begin{pmatrix}
				A
				& \rvline & B \\
				\hline
				0 & \rvline &
				C
			\end{pmatrix} = \det(A)\det(C)$\setlength{\extrarowheight}{0pt},
			\item se $A$ è nilpotente (ossia se $\exists k \mid A^k = 0$),
			$\det(A) = 0$,
			\item se $A$ è idempotente (ossia se $A^2 = A$), allora
			$\det(A) = 1$ o $\det(A) = 0$,
			\item se $A$ è ortogonale (ossia se $AA^\top = I_n$), allora
			$\det(A) = \pm 1$,
			\item se $A$ è un'involuzione (ossia se $A^2 = I_n$), allora
			$\det(A) = \pm 1$,
			\item se ogni minore di taglia $k$ di $A$ ha determinante nullo,
			allora tutti i minori di $A$ taglia maggiore o uguale a $k$ hanno
			determinante nullo (è una diretta applicazione dello sviluppo di Laplace).
		\end{itemize}
		
		Le operazioni del terzo tipo dell'algoritmo di eliminazione
		di Gauss (ossia l'aggiunta a una riga di un multiplo di un'altra
		riga -- a patto che le due righe siano distinte) non alterano il
		determinante della matrice iniziale, mentre lo scambio di righe
		ne inverte il segno (corrisponde a una trasposizione di $S_n$).
		L'operazione del secondo tipo (la moltiplicazione di una riga
		per uno scalare) altera il determinante moltiplicandolo per
		tale scalare.
		
		Inoltre, se $D$ è invertibile, vale la decomposizione di Schur:
		\setlength{\extrarowheight}{1.3pt}
		\begin{gather*}
			\begin{pmatrix}
				A
				& \rvline & B \\
				\hline
				C & \rvline &
				D
			\end{pmatrix} = \begin{pmatrix}
				I_k
				& \rvline & BD^{-1} \\
				\hline
				0 & \rvline &
				I_k
			\end{pmatrix}
			\begin{pmatrix}
				A-BD^{-1}C
				& \rvline & 0 \\
				\hline
				0 & \rvline &
				D
			\end{pmatrix} \\
			\begin{pmatrix}
				I_k
				& \rvline & 0 \\
				\hline
				D^{-1}C & \rvline &
				I_k
			\end{pmatrix},
		\end{gather*}
		\setlength{\extrarowheight}{0pt}
		
		dove $k \times k$ è la taglia di $A$. Pertanto vale
		la seguente relazione, sempre se $D$ è invertibile:
		
		\[ \det \begin{pmatrix}
			A
			& \rvline & B \\
			\hline
			C & \rvline &
			D
		\end{pmatrix} = \det(A-BD^{-1}C)\det(D). \]
		
		È possibile computare il determinante di $A$, scelta la riga $i$, mediante lo
		sviluppo di Laplace:
		\[ \det(A) = \sum_{j=1}^n a_{ij} \Cof_{i,j}(A) = \sum_{j=1}^n (-1)^{i+j} a_{ij} \det(A_{i,j}). \]
		
		Si definisce matrice di Vandermonde una matrice $A \in M(n, \KK)$ della
		forma:
		
		\[ A = \begin{pmatrix}
			1 & x_1 & x_1^2 & \dots & x_1^{n-1}\\
			1 & x_2 & x_2^2 & \dots & x_2^{n-1}\\
			\vdots & \vdots & \vdots & \ddots &\vdots \\
			1 & x_n & x_n^2 & \dots & x_n^{n-1}.
		\end{pmatrix} \]
		
		Vale allora che:
		
		\[ \det(A) = \prod_{1 \leq i < j \leq n} (x_j - x_i), \]
		
		verificabile notando che $\det(A)$ è di grado $\frac{n(n-1)}{2}$ e
		che ponendo $x_i = x_j$ per una coppia $(i, j)$, tale matrice
		ha due righe uguali, e quindi determinante nullo $\implies (x_j - x_i) \mid \det(A) \overbrace{\implies}^{\text{UFD}} \det(A) = \prod_{1 \leq i < j \leq n} (x_j - x_i) $.
		
		Pertanto una matrice di Vandermonde è invertibile se e solo se la sua
		seconda colonna contiene tutti scalari distinti nelle coordinate. Tale
		matrice risulta utile nello studio dell'interpolazione di Lagrange
		(ossia nella dimostrazione dell'unicità del polinomio di $n-1$ grado
		tale che $p(\alpha_i) = \beta_i$ per $i$ coppie ($\alpha_i$, $\beta_i$) con
		$\alpha_i$ tutti distinti).
		
		\subsubsection{Rango tramite il determinante degli orlati}
		
		Si dicono \textit{sottomatrici} della matrice $A \in M(m, n, \KK)$ tutte
		le matrici contenute in $A$, ossia le matrici $B$ che sono ottenibili da $A$
		mantenendo solo alcune sue righe e colonne. In generale, si scrive $A^{j_1, \ldots, j_s}_{i_1, \ldots, i_t}$ per indicare la sottomatrice ottenuta
		da $A$ mantenendo le colonne di indice $j_1$, ..., $j_s$ e le righe di
		indice $i_1$, ..., $i_t$. Quando è omesso l'indice delle colonne o l'indice
		delle righe, si sottintende di aver mantenuto o tutte le colonne o tutte le righe
		(e.g.~$A_{1,2}$ è la sottomatrice di $A$ ottenuta mantenendo tutte le colonne e
		le prime due righe). Si dice che $M$ è \textit{minore} di $A$ una sua sottomatrice quadrata. Si chiamano \textit{orlati} di un minore $M$ di taglia $k$ i minori di taglia $k+1$ di $A$ aventi $M$ come minore.
		
		\begin{itemize}
			\item se $B$ è una sottomatrice di $A$, allora $\rg(B) \leq \rg(A)$ (è sufficiente prendere un numero massimo di colonne linearmente indipendenti
			di $B$ e mostrare che le relative colonne in $A$ sono ancora linearmente indipendenti),
			\item $\rg(A) = \max\{\rg(B) \mid B \text{ sottomatrice di }\! A\}$ (è sufficiente utilizzare il precedente risultato; infatti $A$ è una sottomatrice di $A$),
			\item $\rg(A) = \max\{\rg(B) \mid B \text{ minore invertibile di }\! A\} = \max\{n \mid \text{esiste un minore di $A$ di taglia $n$ invertibile} \}$ (è sufficiente utilizzare la prima disuguaglianza e considerare un minore di $A$ composto dalle righe e le colonne linearmente indipendenti di $A$, che sono
			dello stesso numero, dal momento che il rango per righe è uguale al rango per colonne),
			\item $\rg(A)$ è il più piccolo naturale $n$ tale per cui, per ogni minore
			$M$ di $A$ di taglia maggiore di $n$, $\det(M) = 0$ (ossia $M$ è singolare; segue direttamente dal precedente risultato),
			\item $\rg(A)$ è il più piccolo naturale $n$ tale per cui, per ogni minore
			$M$ di $A$ di taglia $n+1$, $\det(M) = 0$ (ossia $M$ è singolare; segue dal precedente risultato a cui si combina lo sviluppo di Laplace del determinante -- se ogni minore di taglia $k$ ha determinante nullo, anche tutti i minori di
			taglia maggiore di $k$ hanno determinante nullo).
			\item esiste un minore $M$ di taglia $k$ di $A$ con $\det(M) \neq 0$ $\implies \rg(A) \geq k$ (deriva direttamente dall'ultimo risultato sul rango),
			\item per ogni minore $M$ di taglia $k$ di $A$ vale che $\det(M) = 0$
			$\implies \rg(A) < k$ (come sopra).
		\end{itemize}
		
		Si può facilitare lo studio del rango tramite il teorema di Kronecker (o degli orlati): $\rg(A)$ è il più piccolo naturale $n$ tale per cui esista un minore
		$M$ di taglia $k$ con $\det(M) \neq 0$ e per cui ogni suo orlato $O$ è tale
		per cui $\det(O) = 0$.
		
		Sia infatti, senza perdità di generalità, $M = A^{1,\ldots, k}_{1,\ldots,k}$ tale minore (altrimenti è sufficiente considerare una permutazione delle righe e
		delle colonne per ricadere in questo caso; tale permutazione è ammessa dall'algoritmo di Gauss). Si mostra che $A^j \in \Span(A^1, \ldots, A^k)$ $\forall j > k$. Si consideri ogni orlato $M_j$ di $M$ ottenuto scegliendo
		la $j$-esima colonna di $A$: per ipotesi $\det(M_j) = 0$, ed il rango è almeno
		$k$. Quindi $\rg(M_j) = k$; poiché le prime $k$ righe sono linearmente indipendenti, l'ultima riga aggiunta deve certamente appartenere al loro
		sottospazio generato. Quindi ogni riga di $A^{1,\ldots, k, j}$ appartiene
		al sottospazio $\Span(A_1, \ldots, A_k)$, da cui si deduce che $\rg(A^{1,\ldots, k, j}) = k$, e quindi che $\rg(A^{1,\ldots,k,j}) = k \implies A^j \in \Span(A^1, \ldots, A^k) \implies \rg(A) = k$.
		
		
		
		\subsection{Autovalori e diagonalizzabilità}
		
		Sia $f \in \End(V)$. Si dice che $\lambda \in \KK$ è un autovalore
		di $f$ se e solo se $\exists \vec{v} \neq \vec{0}$, $\vec{v} \in V$
		tale che $f(\vec{v}) = \lambda \vec{v}$, e in tal caso si dice
		che $\vec{v}$ è un autovettore relativo a $\lambda$. Un autovalore
		è tale se esiste una soluzione non nulla a $(f - \lambda \Idv) \vec{v} = \vec{0}$, ossia se e solo se:
		\[\det(f - \lambda \Idv) = 0. \]
		
		Questa relazione è ben definita dacché il determinante è invariante
		per qualsiasi cambio di base applicato ad una matrice associata
		di $f$. Si definisce allora $p_f(\lambda) = \det(f - \lambda \Idv)$,
		detto polinomio caratteristico di $f$, ancora invariante per
		matrici associate a $f$. Si denota inoltre con
		spettro di $f$ l'insieme $\Sp(f)$ degli autovalori di $f$ e con
		$V_\lambda = \Ker(f - \lambda \Idv)$ lo spazio degli autovettori
		relativo a $\lambda$, detto autospazio di $\lambda$.
		
		Si definisce la molteplicità algebrica $\mu_{a,f}(\lambda)$ di un autovalore
		$\lambda$ come la molteplicità che assume come radice del polinomio
		$p_f(\lambda)$. Si definisce la molteplicità geometrica
		$\mu_{g,f}(\lambda)$ di un autovalore $\lambda$ come la dimensione
		del suo autospazio $V_\lambda$. Quando è noto l'endomorfismo
		che si sta considerando si omette la dicitura $f$ nel pedice delle
		molteplicità.
		
		\begin{itemize}
			\item $p_f(\lambda)$ ha sempre grado $n = \dim V$,
			\item $p_f(\lambda)$ è sempre monico a meno del segno,
			\item il coefficiente di $\lambda^n$ è sempre $(-1)^n$,
			\item il coefficiente di $\lambda^{n-1}$ è $(-1)^{n+1} \tr(f)$,
			\item il termine noto di $p_f(\lambda)$ è $\det(f - 0 \cdot \Idv) = \det(f)$,
			\item poiché $p_f(\lambda)$ appartiene all'anello euclideo $\KK[\lambda]$, che è dunque un UFD, esso ammette al più
			$n$ radici,
			\item $\Sp(f)$ ha al più $n$ elementi, ossia esistono al massimo
			$n$ autovalori (dalla precedente considerazione),
			\item se $\KK = \CC$ e $\charpoly{f} \in \RR[\lambda]$, $\lambda \in
			\Sp(f) \iff \overline{\lambda} \in \Sp(f)$ (infatti $\lambda$ è
			soluzione di $\charpoly{f}$, e quindi anche $\overline{\lambda}$
			deve esserne radice, dacché i coefficienti di $\charpoly{f}$ sono
			in $\RR$),
			\item se $\KK$ è un campo algebricamente chiuso, $p_f(\lambda)$
			ammette sempre almeno un autovalore distinto (o esattamente
			$n$ se contati con molteplicità),
			\item $0 \in \Sp(f) \iff \dim \Ker f > 0 \iff \rg f < 0 \iff \det(f) = 0$,
			\item autovettori relativi ad autovalori distinti sono sempre
			linearmente indipendenti,
			\item dati $\lambda_1$, ..., $\lambda_k$ autovalori di $f$,
			gli spazi $V_{\lambda_1}$, ..., $V_{\lambda_k}$ sono sempre
			in somma diretta,
			\item $\sum_{i=1}^k \mu_a(\lambda_i)$ corrisponde al numero
			di fattori lineari di $p_f(\lambda)$,
			\item $\sum_{i=1}^k \mu_a(\lambda_i) = n \iff$ $p_f(\lambda)$
			è completamente fattorizzabile in $\KK[\lambda]$,
			\item vale sempre la disuguaglianza $n \geq \mu_a(\lambda) \geq
			\mu_g(\lambda) \geq 1$ (è sufficiente considerare una
			base di $V_\lambda$ estesa a base di $V$ e calcolarne il
			polinomio caratteristico sfruttando i blocchi della matrice
			associata, notando che $\mu_g(\lambda)$ deve forzatamente essere
			minore di $\mu_a(\lambda)$),
			\item vale sempre la disuguaglianza $n \geq \sum_{i=1}^k \mu_a(\lambda_i) \geq \sum_{i=1}^k \mu_g(\lambda_i)$,
			\item se $W \subseteq V$ è un sottospazio $f$-invariante,
			allora $\charpolyrestr{f}{W} \mid p_f(\lambda)$\footnote{Quando si lavora
				su degli endomorfismi, la notazione $\restr{f}{W}$ è impiegata per
				considerare $f$ ristretta a $W$ sia sul dominio che sul codominio.} (è sufficiente
			prendere una base di $W$ ed estenderla a base di $V$, considerando
			poi la matrice associata in tale base, che è a blocchi),
			\item se $W \subseteq V$ è un sottospazio $f$-invariante,
			ed estesa una base $\basis_W$ di $W$ ad una $\basis$ di $V$,
			detto $U = \Span(\basis \setminus \basis_W)$ il supplementare di $W$ che si ottiene da tale base $\basis$, vale
			che $\charpoly{f} = \charpolyrestr{f}{W} \cdot \charpoly{\hat{f}}$,
			dove $\hat{f} : V/W \to V/W$ è tale che $\hat{f}(\vec{u} + W) = f(\vec{u}) + W$ (come prima, è sufficiente considerare una matrice
			a blocchi),
			\item se $V = W \oplus U$, dove sia $W$ che $U$ sono $f$-invarianti,
			allora $\charpoly{f} = \charpolyrestr{f}{W} \cdot \charpolyrestr{f}{U}$ (la matrice associata in un'unione di basi
			di $W$ e $U$ è infatti diagonale a blocchi),
			\item se sia $W$ che $U$ sono $f$-invarianti, allora $f$ è diagonalizzabile
			se e solo se sia $\restr{f}{W}$ che $\restr{f}{U}$ lo sono.
		\end{itemize}
		
		Si dice che $f$ è diagonalizzabile se $V$ ammette una base per cui
		la matrice associata di $f$ è diagonale, o equivalentemente se,
		dati $\lambda_1$, ..., $\lambda_k$ autovalori di $f$, si verifica
		che:
		
		\[ V = V_{\lambda_1} \oplus \cdots \oplus V_{\lambda_k}. \]
		
		Ancora in modo equivalente si può dire che $f$ è diagonalizzabile
		se e solo se:
		
		\[ \begin{cases} \sum_{i=1}^k \mu_a(\lambda_i) = n, \\ \mu_g(\lambda_i) = \mu_a(\lambda_i) \; \forall 1 \leq i \leq k, \end{cases} \]
		
		ossia se il polinomio caratteristico è completamente fattorizzabile
		in $\KK[\lambda]$ (se non lo fosse, la somma diretta
		$V_{\lambda_1} \oplus \cdots \oplus V_{\lambda_k}$ avrebbe
		forzatamente dimensione minore di $V$, ed esisterebbero altri
		autovalori in un qualsiasi campo di spezzamento di $p_f(\lambda)$) e se $\sum_{i=1}^k \mu_g(\lambda_i) = n$. Tale condizione, in un
		campo algebricamente chiuso, si riduce a $\mu_g(\lambda_i) = \mu_a(\lambda_i)$, $\forall 1 \leq i \leq k$.
		
		Considerando la forma canonica di Jordan di $f$, si osserva anche
		che $f$ è diagonalizzabile se e solo se per ogni autovalore la
		massima taglia di un blocco di Jordan è esattamente $1$, ossia
		se il polinomio minimo di $f$ è un prodotto di fattori lineari
		distinti. Si può fare la stessa considerazione guardando al
		teorema di decomposizione primaria (gli indici di Fitting del
		sottospazio generalizzato sono esattamente le moltiplicità algebriche
		degli autovalori nel polinomio minimo).
		
		Data $f$ diagonalizzabile, la matrice diagonale $J$ a cui $f$ è
		associata è, dati gli autovalori $\lambda_1$, ..., $\lambda_k$,
		una matrice diagonale dove $\lambda_i$ compare sulla diagonale
		esattamente $\mu_g(\lambda_i)$ volte.
		
		Data $A \in M(n, \KK)$, $A$ è diagonalizzabile se e solo se $f_A$,
		l'applicazione indotta dalla matrice $A$, è diagonalizzabile,
		ossia se $A$ è simile ad una matrice diagonale $J$, computabile
		come prima. Si scrive in particolare $p_A(\lambda)$ per indicare
		$p_{f_A}(\lambda)$.
		
		Una matrice $P \in \GL(M(n, \KK))$
		tale che $A = P J P\inv$, è tale che $AP = PJ$: presa la $i$-esima
		colonna, allora, $AP^{(i)} = PJ^{(i)} = P^{(i)}$; ossia è sufficiente
		costruire una matrice $P$ dove l'$i$-esima colonna è un autovettore
		relativo all'autovalore presente in $J_{ii}$ linearmente indipendente
		con gli altri autovettori presenti in $P$ relativi allo stesso
		autovalore (esattamente nello stesso modo in cui si costruisce in
		generale tale $P$ con la forma canonica di Jordan).
		
		Se $A$ e $B$ sono diagonalizzabili, allora $A \sim B \iff p_A(\lambda) =
		p_B(\lambda)$ (infatti due matrici diagonali hanno lo stesso polinomio
		caratteristico se e solo se compaiono gli stessi identici autovalori).
		
		Se $f$ è diagonalizzabile, allora ogni spazio $W$ $f$-invariante di
		$V$ è tale che:
		\[ W = (W \cap V_{\lambda_1}) \oplus \cdots \oplus (W \cap V_{\lambda_k}), \]
		
		dove $\lambda_1$, ..., $\lambda_k$ sono gli autovalori distinti di
		$f$, e dunque $\restr{f}{W}$ è sempre diagonalizzabile, se $f$ lo è.
		
		Se $f$ è diagonalizzabile, anche $f^k$ lo è, per ogni $k \in \NN$. Se
		ogni vettore di $V$ è un autovettore di $f$, allora $f = \lambda \Id$,
		con $\lambda \in \KK$ (è sufficiente considerare l'eventuale esistenza di più
		autospazi e due vettori $\v$ e $\w$ di due autospazi distinti e considerare
		le due scritture possibili di $f(\v + \w)$).
		
		Si dice infine che $f$ è triangolabile (o triangolarizzabile) se $V$
		ammette una base per cui la matrice associata di $f$ è triangolare superiore
		(o inferiore, dal momento che è sufficiente riordinare dal basso la base
		per ottenere una matrice associata triangolare superiore). Vale in particolare
		che $f$ è triangolabile se e soltanto se $p_f(\lambda)$ è completamente
		riducibile in fattori lineari in $\KK$ (dunque, nel caso di $\KK$ algebricamente
		chiuso, $f$ è sempre triangolabile). Infatti, se $f$ è triangolabile, il polinomio
		caratteristico ha come radici esattamente gli elementi sulla diagonale della
		matrice associata di $f$ nella base $\basis$ in cui tale matrice è triangolare
		superiore (e dunque $p_f(\lambda)$ è riducibile in fattori lineari). Se invece $p_f(\lambda)$ è riducibile in fattori lineari, si può applicare il seguente
		algoritmo [...]
		
		\subsubsection{Diagonalizzabilità e triangolabilità simultanea}

		Due endomorfismi $f$, $g \in \End(V)$ diagonalizzabili si dicono simultaneamente diagonalizzabili se esiste una base $\basis$ di $V$
		tale per cui sia la matrice associata di $f$ in $\basis$ che quella
		di $g$ sono diagonali. Vale in particolare che $f$ e $g$ sono
		simultaneamente diagonalizzabili se e solo se $f \circ g = g \circ f$.
		Per trovare tale base è sufficiente, dati $\lambda_1$, ...,
		$\lambda_k$ autovalori di $f$, considerare $\restr{g}{V_{\lambda_i}}$
		$\forall 1 \leq i \leq k$ ($V_{\lambda_i}$ è infatti $g$-invariante,
		dacché, per $\vec{v} \in V_{\lambda_i}$, $f(g(\vec{v})) =
		g(f(\vec{v})) = g(\lambda_i \vec{v}) = \lambda_i g(\vec{v}) \implies
		g(\vec{v}) \in V_{\lambda_i}$), che, essendo una restrizione di
		un endomorfismo diagonalizzabile su un sottospazio invariante, è diagonalizzabile: presa allora
		una base di autovettori di $\restr{g}{V_{\lambda_i}}$, questi sono
		anche base di autovettori di $V_{\lambda_i}$; unendo tutti questi
		autovettori in un'unica base $\basis$ di $V$, si otterrà dunque
		che una base in cui le matrici associate di $f$ e $g$ sono diagonali.
		
		Analogamente due endomorfismi $f$, $g \in \End(V)$ triangolabili si dicono
		simultaneamente triangolabili se [...]

		\subsection{Prodotto scalare e congruenza}
		Si consideri una mappa $\varphi : V \times V \to \KK$. Si dice che
		$\varphi$ è un prodotto scalare (e quindi che $\varphi \in \PS(V)$, lo spazio dei prodotti scalari) se è una forma bilineare simmetrica.
		In particolare vale la seguente identità:
		
		\[ \varphi\left( \sum_{i=1}^s a_i \vv i, \sum_{j=1}^t b_j \ww j \right) =
		\sum_{i=1}^s  \sum_{j=1}^t a_i b_j \varphi(\vv i, \ww j). \]
		
		Se $\basis = \{ \vv 1, \ldots ,\vv n \}$ è una base di $V$, si definisce $M_\basis(\varphi) = (\varphi(\vv i, \vv j))_{i,j=1\mbox{--}n}$ come la matrice associata al prodotto scalare $\varphi$. In particolare,
		se $a_\varphi : V \to V^*$ è la mappa lineare che associa a $\v$ il funzionale $\varphi(\v, \cdot) \in V^*$
		tale che $\varphi(\v, \cdot)(\w) = \varphi(\v, \w)$. Si scrive $(V, \varphi)$ per indicare uno
		spazio vettoriale $V$ dotato del prodotto scalare $\varphi$.
		
		Si definisce prodotto scalare \textit{standard} il prodotto $\varphi$ tale che
		$\varphi(\v, \w) = [\v]_\basis^\top [\w]_\basis$.
		
		Si dice che due vettori $\v$, $\w \in V$ sono ortogonali tra loro, scritto come $\v \perp \w$, se
		$\varphi(\v, \w) = 0$. Dato $W$ sottospazio di $V$, si definisce $W^\perp$ come il sottospazio di $V$ dei vettori ortogonali a tutti i vettori di $W$. Si dice che $\varphi$ è non degenere se $V^\perp = \zerovecset$.
		Si scrive in particolare che $V^\perp = \Rad(\varphi)$.
		
		Si dice che $V = U \oplus^\perp W$ (ossia che $U$ e $W$ sono in somma diretta ortogonale) se $V = U \oplus W$ e $U \subseteq W^\perp$. Sia $i : W \to V$ tale che $\w \mapsto \w$. Si scrive $\restr{\varphi}{W}$ intendendo $\restr{\varphi}{W \times W}$.
		
		Ad ogni prodotto scalare si può associare una forma quadratica (e viceversa) $q : V \to \KK$ tale che
		$q(\v) = \varphi(\v, \v)$. Un vettore $\v \in V$ si dice isotropo se $q(\v) = 0$ (altrimenti si dice
		anisotropo). Si definisce il cono isotropo $\CI(\varphi)$ come l'insieme dei vettori isotropi di $V$.
		
		Se $\KK = \RR$, si dice che $\varphi$ è semidefinito positivo ($\varphi \geq 0$) se $q(\v) \geq 0$ $\forall \v \in V$, e che è semidefinito negativo ($\varphi \leq 0$) se $q(\v) \leq 0$ $\forall \v \in V$. Si dice
		che $\varphi$ è definito positivo ($\varphi > 0$) se $\varphi \geq 0$ e se $q(\v) = 0 \iff \v = \vec 0$,
		e che è definito negativo ($\varphi < 0$) se $\varphi \leq 0$ e se $q(\v) = 0 \iff \v = \vec 0$.
		
		Si dice che $\varphi$ è definito se è definito positivo o definito negativo. Analogamente $\varphi$
		è semidefinito se è semidefinito positivo o semidefinito negativo.
		
		\begin{itemize}
			\item $M_\basis(\varphi)$ è simmetrica,
			\item $\varphi(\v, \w) = [\v]_\basis^\top M_\basis(\varphi) [\w]_\basis$,
			\item $M_\basis(\varphi) = M^\basis_{\basis^*}(a_\varphi)$,
			\item $\Ker a_\varphi = V^\perp$,
			\item $\varphi$ è non degenere se e solo se $M_\basis(\varphi)$ è invertibile,
			\item $W^\perp = \Ker i^\top \circ a_\varphi$,
			\item $a_\varphi(W^\perp) = \Ann(W) \cap \Imm a_\varphi$,
			\item $\dim W + \dim W^\perp = \dim V + \dim (W \cap V^\perp)$ (da sopra),
			\item $V = W \oplus^\perp W^\perp$ se $\restr{\varphi}{W}$ è non degenere ($\iff W \cap W^\perp = \Rad(\restr{\varphi}{W}) = \zerovecset$),
			\item $(W^\perp)^\perp = W^\dperp = W + \Rad(\varphi) = W + V^\perp$,
			\item $(U + W)^\perp = U^\perp \cap W^\perp$,
			\item $(U \cap W)^\perp \supseteq U^\perp + W^\perp$,
			\item $(U \cap W)^\perp = U^\perp + W^\perp$, se $\varphi$ è non degenere,
			\item $\varphi$ è definito $\iff$ $\CI(\varphi) = \zerovecset$,
			\item $\varphi$ è semidefinito $\iff$ $\CI(\varphi) = V^\perp = \Rad(\varphi)$ (considera l'esistenza
			di due vettori $\v$, $\w \in V$ con forme quadratiche discordi, osserva che sono linearmente indipendenti
			e trova un $\lambda \in \KK$ tale per cui $\v + \lambda \w$ crea un assurdo).
		\end{itemize}
		
		Se $\basis'$ è un'altra base di $V$, vale il seguente \textit{teorema di cambiamento di base}:
		
		\[ M_{\basis'}(\varphi) = M_{\basis}^{\basis'}(\Idv)^\top \, M_\basis(\varphi) \, M_{\basis}^{\basis'}(\Idv). \]
		
		Si definisce relazione di congruenza la relazione di equivalenza $\cong$ (o $\equiv$) definita
		su $\Sym(n, \KK)$ nel seguente modo:
		
		\[ A \cong B \iff \exists P \in \GL(n, \KK) \mid A = P^\top B P. \]
		
		
		\begin{itemize}
			\item $A \cong B \implies \rg(A) = \rg(B)$ (il rango è invariante per congruenza; e dunque si può
			definire $\rg(\varphi)$ come il rango di una qualsiasi matrice associata a $\varphi$),
			\item $A \cong B \implies \det(A) \det(B) \geq 0$ (in $\KK = \RR$ il segno del determinante è invariante per congruenza),
			\item Due matrici associate a $\varphi$ in basi diverse sono congruenti per la formula
			di cambiamento di base.
		\end{itemize}
		
		Si definiscono i seguenti tre indici per $\KK = \RR$:
		
		\begin{itemize}
			\item $\iota_+ = \max\{ \dim W \mid W \subseteq V \E \restr{\varphi}{W} > 0 \}$,
			\item $\iota_- = \max\{ \dim W \mid W \subseteq V \E \restr{\varphi}{W} < 0 \}$,
			\item $\iota_0 = \dim V^\perp$,
		\end{itemize}
		
		e si definisce segnatura di $\varphi$ la terna $\sigma = (\iota_+, \iota_-, \iota_0)$.
		
		Si dice che una base $\basis$ di $V$ è ortogonale se i suoi vettori sono a due a due ortogonali (e
		quindi la matrice associata in tale base è diagonale). Se $\Char \KK \neq 2$, valgono i seguenti risultati:
		
		\begin{itemize}
			\item $\varphi(\v, \w) = \frac{q(\v + \w) - q(\v) - q(\w)}{2}$ (formula di polarizzazione; $\varphi$ è
			completamente determinata dalla sua forma quadratica),
			
			\item Esiste sempre una base ortogonale $\basis$ di $V$ (teorema di Lagrange; è sufficiente considerare
			l'esistenza di un vettore anisotropo $\w \in V$ ed osservare che $V = W \oplus^\perp W^\perp$, dove $W = \Span(V)$, concludendo per induzione; o in caso di non esistenza di tale $\w$, concludere per il
			risultato precedente),
			
			\item (se $\KK = \CC$) Esiste sempre una base ortogonale $\basis$ di $V$ tale che:
			
			\[ M_\basis(\varphi) = \Matrix{I_r & \rvline & 0 \\ \hline 0 & \rvline & 0\,}, \]
			
			\vskip 0.05in
			
			dove $r = \rg(\varphi)$ (teorema di Sylvester, caso complesso; si consideri una base ortogonale e se
			ne normalizzino i vettori anisotropi),
			
			\item Due matrici simmetriche con stesso rango allora non solo sono SD-equivalenti, ma sono
			anche congruenti,
			
			\item (se $\KK = \RR$) Esiste sempre una base ortogonale $\basis$ di $V$ tale che:
			
			\[ M_\basis(\varphi) = \Matrix{I_{\iota_+} & \rvline & 0 & \rvline & 0 \\ \hline 0 & \rvline & -I_{\iota_-} & \rvline & 0 \\ \hline 0 & \rvline & 0 & \rvline & 0\cdot I_{\iota_0} }. \]
			
			\vskip 0.05in
			
			Inoltre $\sigma$ è un invariante completo per la congruenza, e vale che, su una qualsiasi base ortogonale $\basis'$ di $V$, $\iota_+$ è esattamente il numero
			di vettori anisotropi di base con forma quadratica positiva, che $\iota_-$ è il numero di vettori con forma
			negativa e che $\iota_0$ è il numero di vettori isotropi (teorema di Sylvester, caso reale; si consideri
			una base ortogonale e se ne normalizzino i vettori anisotropi, facendo infine eventuali considerazioni
			dimensionali per dimostrare la seconda parte dell'enunciato),
			
			\item $\varphi > 0 \iff \sigma = (n, 0, 0)$ e $\varphi < 0 \iff \sigma = (0, n, 0)$,
			\item $\varphi \geq 0 \iff \sigma = (n - k, 0, k)$ e $\varphi \leq 0 \iff \sigma = (0, n - k, k)$,
			con $0 \leq k \leq n$ tale che $k = \dim V^\perp$,
			
			\item I vettori isotropi di una base ortogonale sono una base di $V^\perp$,
			
			\item $\rg(\varphi) = \iota_+ + \iota_-$,
			
			\item $n = \iota_+ + \iota_- + \iota_0$,
			
			\item Se $W$ è un sottospazio di $V$, $\iota_+(\varphi) \geq \iota_+(\restr{\varphi}{W})$ e
			$\iota_-(\varphi) \geq \iota_-(\restr{\varphi}{W})$,
			
			\item Se $V = U \oplus^\perp W$, $\sigma(\varphi) = \sigma(\restr{\varphi}{U}) + \sigma(\restr{\varphi}{W})$,
			
			\item Se $\KK = \RR$ e $A = M_\basis(\varphi)$, allora: 
			\[ \sigma = \textstyle \left( \sum_{\substack{\lambda \in \Sp(\varphi) \\ \lambda > 0}} \mu_a(\lambda), \; \sum_{\substack{\lambda \in \Sp(A) \\ \lambda < 0}} \mu_a(\lambda), \; \mu_0(\lambda) \right), \]			
			come conseguenza del teorema spettrale reale.
		\end{itemize}
		
		Si chiama matrice di Sylvester una matrice della forma vista nell'enunciato del teorema di Sylvester
		reale, e si dice che una base $\basis$ è una base di Sylvester se la matrice ad essa associata è di
		Sylvester. Per il teorema di Sylvester, tale base esiste sempre, e la matrice di Sylvester è unica per
		ogni prodotto scalare $\varphi$.
		
		\subsubsection{Algoritmo di ortogonalizzazione di Gram-Schmidt}
		
		Data una base $\basis$ di $V$, se $\abs{\CI(\varphi) \cap \basis} \leq 1$ (ossia se ogni vettore di
		$\basis$ è anisotropo o al più vi è un vettore isotropo, posto in fondo come $\vv n$), si può
		trovare una base ortogonale $\basis' = \{ \vv 1', \ldots, \vv n' \}$ a partire da $\basis$ tale che ne mantenga la stessa bandiera, ossia tale che:
		\[ \Span(\vv 1', \ldots, \vv i') = \Span(\vv 1, \ldots, \vv i) \forall 1 \leq i \leq n. \]
		
		Si definisce $C(\w, \v) = \frac{\varphi(\v, \w)}{\varphi(\w, \w)}$ come il coefficiente di Fourier
		di $\v$ rispetto a $\w$. L'algoritmo allora funziona nel seguente modo:
		
		\begin{enumerate}
			\item Si prenda in considerazione $\vv 1$ e si sottragga ad ogni altro vettore $\vv i$ della base il
			vettore $C(\vv 1, \vv i) \, \vv 1$,
			\item Si ripeta il processo considerando come $\basis$ tutti i vettori di $\basis$ con $\vv 1$ escluso,
			o si termini l'algoritmo una volta che è rimasto un solo vettore.
		\end{enumerate}

		\subsubsection{Metodo di Jacobi per il calcolo della segnatura}
		Sia $A = M_\basis(\varphi)$ una matrice associata a $\varphi$ nella base $\basis$.
		Sia $d_0 := 1$. Se $d_i = \det(A_{1, \ldots, i}^{1, \ldots, i})$ (è possibile anche
		prendere un'altra sequenza di minori, a patto che essi siano principali e che siano
		crescenti per inclusione) è diverso da zero
		per ogni $1 \leq i \leq n-1$, allora $\iota_+$ è il numero di permanenze di segno
		di $d_i$ (zero escluso), $\iota_-$ è il numero di variazioni di segno (zero escluso), e $\iota_0$ è $1$ se
		$d_n = 0$ o $0$ altrimenti.
		
		In generale, se $W$ è un sottospazio di $W'$, $W$ ha codimensione $1$ rispetto a $W'$ e $\det(M_{\basis_W}(\restr{\varphi}{W})) \neq 0$ per una base $\basis_W$ di $W$, allora la segnatura
		di $\restr{\varphi}{W'}$ è la stessa di $\restr{\varphi}{W}$, dove si aggiunge
		$1$ a $\iota_+$, se i determinanti $\det(M_{\basis_W}(\restr{\varphi}{W}))$ e $\det(M_{\basis_{W'}}(\restr{\varphi}{W}))$ (dove $\basis_{W'}$ è una base di $W'$) concordano di segno, $1$ a $\iota_-$, se
		sono discordi, o $1$ a $\iota_0$ se l'ultimo di questi due determinanti è nullo.
		
		Dal metodo di Jacobi si deduce il criterio di definitezza di Sylvester: $A$ è
		definita positiva se e solo se $d_i > 0$ $\forall 1 \leq i \leq n$; $A$ è
		definita negativa se e solo se $(-1)^i d_i > 0$ $\forall 1 \leq i \leq n$.
		
		\subsubsection{Sottospazi isotropi e indice di Witt}
		
		Si dice che un sottospazio $W$ di $V$ è isotropo se $\restr{\varphi}{W} = 0$, o
		equivalentemente se $W \subseteq W^\perp$ (i.e.~se $W \cap W^\perp = W$, e quindi
		se $\Rad(\restr{\varphi}{W}) = W$). Si definisce allora l'indice di Witt $W(\varphi)$ come
		la dimensione massima di un sottospazio isotropo di $V$.
		
		\begin{itemize}
			\item $V^\perp$ è un sottospazio isotropo,
			\item Se $W$ è isotropo, allora $\dim W \leq \frac{\dim V + \dim \Rad(\varphi)}{2}$,
			\item Se $W$ è isotropo e $\varphi$ è non degenere, allora $\dim W \leq \frac{1}{2} \dim V$,
			\item Se $\KK = \RR$, allora $W(\varphi) = \min\{ i_+, i_- \} + i_0$ (è sufficiente considerare
			una base di Sylvester e creare una nuova base i cui i vettori sono o isotropi o della forma $\vv i - \ww i$, dove $q(\vv i) = 1$ e $q(\ww i) = 1$, concludendo con discussioni dimensionali),
			\item Se $\varphi$ è definito, allora $W(\varphi) = 0$,
			\item Se $\varphi$ è semidefinito, allora $W(\varphi) = i_0$ (e $W = V^\perp$ è un sottospazio
			isotropo di tale dimensione).
		\end{itemize}
		
		\subsubsection{Isometrie tra spazi vettoriali}

		Due spazi vettoriali $(V, \varphi)$ e $(W, \psi)$ su $\KK$ si dicono isometrici tra loro se
		esiste un isomorfismo $f : V \to W$ tale che $\varphi(\vv 1, \vv 2) = \psi(f(\vv 1), f(\vv 2))$.

		Se $f$ è un isomorfismo tra $V$ e $W$, sono equivalenti le seguenti affermazioni:

		\begin{enumerate}[(i)]
			\item $(V, \varphi)$ e $(W, \psi)$ sono isometrici tra loro tramite $f$,
			\item $\forall \basis$ base di $V$, $M_\basis(\varphi) = M_{f(\basis)}(\psi)$,
			\item $\exists \basis$ base di $V$, $M_\basis(\varphi) = M_{f(\basis)}(\psi)$.
		\end{enumerate}
		
		Inoltre, $V$ e $W$ sono isometrici se e solo se hanno la stessa dimensione e le matrici associate
		a $\varphi$ e $\psi$ in due basi di $V$ e di $W$ sono congruenti (infatti, in tal caso, esistono due
		basi di $V$ e di $W$ che condividono la stessa matrice associata, ed è possibile associare ad uno
		ad uno gli elementi di queste basi).
		
		Pertanto, se $\basis_V$ e $\basis_W$ sono due basi di $V$ e di $W$, $\KK = \RR$ e $M_{\basis_V}(\varphi)$ e $M_{\basis_W}(\psi)$ condividono la stessa segnatura, allora $V$ e $W$ sono
		isometrici tra loro (come conseguenza del teorema di Sylvester reale).
		
		Analogamente, se $\KK = \CC$ e $M_{\basis_V}(\varphi)$ e $M_{\basis_W}(\psi)$ condividono lo stesso
		rango, allora $V$ e $W$ sono isometrici tra loro (come conseguenza stavolta del teorema di Sylvester
		complesso).
		
		\vfill
		\hrule
		~\\
		Gabriel Antonio Videtta, \url{https://poisson.phc.dm.unipi.it/~videtta/}
	\end{multicols}
	
\end{document}