\documentclass[11pt]{article}
\usepackage{personal_commands}
\usepackage[italian]{babel}

\title{\textbf{Note del corso di Geometria 1}}
\author{Gabriel Antonio Videtta}
\date{26 aprile 2023}

\begin{document}
	
	\maketitle
	
	\begin{center}
		\Large \textbf{Azioni di un gruppo e introduzione agli spazi affini}
	\end{center}

	
	\begin{note}
		Nel corso delle lezioni si è impiegata la notazione $g.x$ per indicare
		l'azione di un gruppo su un dato elemento $x \in X$. Tuttavia si è
		preferito utilizzare la notazione $g \cdot x$ nel corso del documento. \\
		
		Inoltre, con $G$ si indicherà un generico gruppo, e con $X$ un
		generico insieme, sul quale $G$ agisce, qualora non indicato diversamente.
	\end{note}

	\begin{definition} [azione di un gruppo su un insieme]
		Sia $G$ un gruppo e sia $X$ un insieme. Un'azione sinistra, comunemente detta solo \textbf{azione}, di $G$
		su $X$ è un'applicazione da $G \times X$ in $X$ tale
		che $(g, x) \mapsto g \cdot x$ e che:
		
		\begin{enumerate}[(i)]
			\item $e \cdot x = x$ $\forall x \in X$,
			\item $g \cdot (h \cdot x) = (gh) \cdot x$ $\forall x \in X$, $\forall g$, $h \in G$.
		\end{enumerate}
	\end{definition}

	\begin{remark}\nl
		\li Data un'azione di $G$ su $X$, si può definire un'applicazione
		$f_g : X \to X$ tale che, dato $g \in G$, $f_g(x) = g \cdot x$. \\
		\li Tale applicazione $f_g$ è bigettiva, dal momento che $f_{g\inv}$ è una sua
		inversa, sia destra che sinistra. Infatti $(f_g \circ f_{g\inv})(x) =
		g \cdot (g\inv \cdot x) = (g g\inv) \cdot x = e \cdot x = x$, e così il viceversa. 
	\end{remark}

	\begin{definition}
		L'azione di un gruppo $G$ su un insieme $X$ si dice \textbf{fedele} se
		l'omomorfismo $\varphi_G$ da $G$ in $S(X)$, ossia nel gruppo delle bigezioni su $G$, che
		associa $g$ a $f_g$ è iniettiva.
	\end{definition}
	
	\begin{remark}
		Si osserva che dire che un'azione di un gruppo è fedele è equivalente
		a dire che $\Ker \varphi_G = \{ e \}$, ossia che $f_g = \Id \iff g = e$.
	\end{remark}
	
	\begin{example}
		Si possono fare alcuni esempi di azioni classiche su alcuni gruppi.
		\begin{enumerate}[(i)]
			\item $S(X)$ agisce su $X$ in modo tale che $f \cdot x = f(x)$ $\forall f \in S(X), x \in X$.
			
			\item $G$ agisce su $G$ stesso tramite l'operazione del gruppo, ossia $g \cdot g' = gg'$ $\forall g$, $g' \in G$.
			
			\item Data un'azione sinistra di $G$ su $X$ tale che $(g, x) \mapsto g \cdot x$, si può definire
			naturalmente un'azione destra da $X \times G$ in $X$ in modo tale che $(x, g) \mapsto x \cdot g = g\inv \cdot x$.
			Infatti $x \cdot e = e\inv \cdot x = e \cdot x = x$, e $(x \cdot g) \cdot g' = (g\inv \cdot x) \cdot g' =
			{g'}\inv \cdot (g\inv \cdot x) = ({g'}\inv g\inv) \cdot x = (g g')\inv \cdot x = x \cdot (g g')$.
		\end{enumerate}
	\end{example}
	
	\begin{definition} [$G$-insieme]
		Se esiste un azione di $G$ su $X$, si dice che $X$ è un $G$\textit{-insieme}.
	\end{definition}
	
	\begin{definition} [orbita di $x$]
		Sia $\sim_G$ la relazione d'equivalenza tale che $x \sim_G y \defiff \exists g \in G \mid g \cdot x = y$.
		Allora le classi di equivalenza si dicono \textbf{orbite}, ed in particolare si indica l'orbita a cui
		appartiene un dato $x \in X$ come $\Orb_G(x) = O_x$ (o come $\Orb(x)$, quando $G$ è noto), ed è detta \textit{orbita di} $x$.
	\end{definition}
	
	\begin{example} Si possono individuare facilmente alcune orbite per alcune azioni classiche.
		\begin{enumerate}[(i)]
			\item Se $G = \GL(n, \KK)$ è il gruppo delle matrici invertibili su $\KK$ di taglia $n$ rispetto
			all'operazione di moltiplicazione matriciale, $G$ opera naturalmente su $M(n, \KK)$ tramite
			la similitudine, ossia $G$ agisce in modo tale che $P \cdot M = P M P\inv$ $\forall P \in \GL(n, \KK)$,
			$M \in M(n, \KK)$. In particolare, data $M \in M(n, \KK)$, $\Orb(M)$ coincide esattamente
			con la classe di similitudine di $M$.
			
			\item Se $G = \GL(n, \KK)$, $G$ opera naturalmente anche su $\Sym(n, \KK)$
			tramite la congruenza, ossia tramite la mappa $(P, A) \mapsto P^\top A P$. L'orbita $\Orb(A)$ è la classe di congruenza delle matrice simmetria $A \in \Sym(n, \KK)$. Analogamente si può costruire un'azione per le
			matrici hermitiane.
			
			\item Se $G = O_n$, il gruppo delle matrici ortogonali di taglia $n$ su $\KK$, $G$ opera su $\RR^n$ tramite la mappa $O \cdot \vec v \mapsto O \vec v$. L'orbita $\Orb(\vec v)$ è in particolare la sfera $n$-dimensionale di raggio $\norm{x}$.
		\end{enumerate}
	\end{example}

	\begin{definition} [stabilizzatore di $x$]
		Lo \textbf{stabilizzatore} di un punto $x \in X$ è l'insieme degli elementi di $G$ che
		agiscono su $x$ lasciandolo invariato, ossia lo stabilizzatore $\Stab_G(x)$ (scritto semplicemente come $\Stab(x)$ se $G$ è noto) è il sottogruppo
		di $G$ tale che:
		
		\[ \Stab_G(X) = \{g \in G \mid g \cdot x = x \}. \]
	\end{definition}

	\begin{example}
		Sia $H \subseteq G$ un sottogruppo di $G$ e sia $X = G/H$. Allora $X$ è un $G$-insieme tramite l'azione $g' \cdot (gH) = g'gH$. In particolare
		vale che $\Stab(gH) = gH$, e quindi che $\Stab(eH) = H$.
	\end{example}

	\begin{theorem} [di orbita-stabilizzatore]
		Sia $X$ un $G$-insieme e sia $x \in X$. Allora esiste un'applicazione
		bigettiva da $G/\Stab(x)$ a $\Orb(x)$.
	\end{theorem}

	\begin{proof}
		Sia $\tau$ l'applicazione da $G/\Stab(x)$ a $\Orb(x)$ tale
		che $\tau(g\Stab(x)) = g \cdot x$. Si dimostra innanzitutto che $\tau$ è
		ben definita. Sia infatti $g' = g s \in G$, con $g \in G$ e $s \in \Stab(x)$, allora $\tau(g' \Stab(x)) = g' \cdot x = g \cdot (s \cdot x) = g \cdot x = \tau(g \Stab(x))$, per cui $\tau$ è ben definita.  \\ 
		
		Chiaramente $\tau$ è surgettiva: sia infatti $y \in \Orb(x)$, allora
		$\exists g \in G \mid g \cdot x = y \implies \tau(g \Stab(x)) = g \cdot x = y$. Siano ora $g$, $g' \in G$ tali che $\tau(g \Stab(x)) = \tau(g' \Stab(x))$, allora $g \cdot x = g' \cdot x \implies (g' g\inv) \cdot x = x \implies g' g\inv \in \Stab(x)$. Pertanto $g \Stab(x) = g' \Stab(x)$, e $\tau$ è allora iniettiva, da cui la tesi.
	\end{proof}

	\begin{remark}\nl
		\li Come conseguenza del teorema di orbita-stabilizzatore,
		si osserva che $\abs{G/\Stab(x)} = \abs{\Orb(x)}$, se $\Orb(x)$ è
		finito, e quindi si conclude, per il teorema di Lagrange, che
		$\abs{G} = \abs{\Stab(x)} \abs{\Orb(x)}$. \\
		
		\li Il teorema di orbita-stabilizzatore implica il primo teorema di omomorfismo. Siano infatti
		$G$, $H$ due gruppi e sia $f$ un omomorfismo da $G$ in $H$. Si può allora costruire un azione
		di $G$ in $H$ in modo tale che $g \cdot h = f(g) h$ $\forall g \in G$, $h \in H$. Infatti
		$e_G \cdot h = f(e_G) h = e_H h = h$ e $g \cdot (g' \cdot h) = g \cdot (f(g') h) = f(g) f(g') h =
		f(g g') h = (g g') \cdot h$, $\forall g$, $g' \in G$, $h \in H$. \\
		
		Si osserva che $\Stab(e_H) = \Ker f$:
		infatti $\Stab(e_H) = \{ g \in G \mid g \cdot e_H = f(g) e_H = f(g) = e_H \} = \Ker f$. Inoltre,
		$\Orb(e_H) = \Im f$, dal momento che $\Orb(e_H) = \{ h \in H \mid \exists g \in G \tc g \cdot h = f(g) h = e_H \iff f(g) = h\inv \} = \{ h \in H \mid \exists g \in G \text{ t.c. } f(g) = h \} = \Im f$, dove
		si è usato che $h\inv \in \Im f \iff h \in \Im f$. \\
		
		Dal momento allora che $\Stab(e_H)$ è il kernel di $f$, vale che $\Stab(e_H) \nsg G$, e quindi
		che $G/\Stab(e_H)$ è un gruppo.
		Si verifica allora che l'applicazione $\tau$ costruita nella dimostrazione del teorema di orbita-stabilizzatore
		è un omomorfismo. Siano infatti $g \Stab(e_H)$, $g' \Stab(e_H) \in G/\Stab(e_H)$, allora
		$\tau(g \Stab(e_H) \, g' \Stab(e_H)) = \tau((g g') \Stab(e_H)) = (g g') \cdot e_H = f(g g') e_H =
		f(g) e_H f(g') e_H = \tau(g \Stab(e_H)) \, \tau(g' \Stab(e_H))$. \\
		
		Si conclude dunque, per il teorema di orbita-stabilizzatore, che $\tau$ è bigettiva, e dunque
		che $G/\Ker f = G/\Stab(e_H) \cong \Orb(e_H) = \Im f$, ossia si ottiene la tesi del primo
		teorema di omomorfismo.
	\end{remark}

	\begin{definition}
		Si dice che $G$ \textbf{opera liberamente} su $X$ se, dato $x \in X$, l'applicazione da $G$ in $X$ tale che
		$g \mapsto g \cdot x$ è iniettiva, ossia se $\Stab(x) = \{e\}$.
	\end{definition}
	
	\begin{definition}
		Si dice che $G$ \textbf{opera transitivamente} su $X$ se, dato $x \in X$, l'applicazione da $G$ in $X$ tale che $g \mapsto g \cdot x$ è surgettiva, ossia se $x \sim_G y$ $\forall x$, $y \in X$, cioè se c'è un'unica orbita, che coincide con $X$. In
		tal caso si dice che $X$ è un insieme \textbf{omogeneo} per l'azione di $G$ (o semplicemente che è
		un \textit{$G$-insieme omogeneo}).
	\end{definition}

	\begin{example} Si possono fare alcuni esempi classici di insiemi $X$
		omogenei per la propria azione.
		\begin{enumerate}[(i)]
			\item $O_n$ opera sulla sfera $n$-dimensione di $\RR^n$ transitivamente. In particolare, si può trovare un'analogia per lo stabilizzatore di una coordinata di un vettore $\v$ di $\RR^n$.
			Per esempio, se si vuole fissare il vettore $\e n$,
			$\forall O \in \Stab(\e n)$ deve valere che $O \e n = \e n$,
			ossia l'ultima colonna di $O$ deve essere esattamente
			$\e n$. Dal momento però che $O$ è ortogonale, le sue
			colonne devono formare una base ortonormale di $\RR^n$,
			e quindi tutta l'ultima riga di $O$, eccetto per il suo
			ultimo elemento, deve essere nulla. Allora $O$ deve
			essere della seguente forma:
			
			\[ O = \Matrix{& & & & & & \rvline & 0 \, \\  & & & & & & \rvline & \vdots \, \\ & & & \mbox{\normalfont\Large $A$} & & &  \rvline & \vdots \, \\ & & & & & & \rvline & \vdots \, \\ & & & & & & \rvline & \vdots \, \\ \hline & 0 & \cdots & \cdots & \cdots & \cdots & \rvline & 1 \,}, \]

			\vskip 0.05in			
			
			dove $A \in M(n-1, \RR)$. Affinché allora $O$ sia ortogonale,
			anche $A$ deve esserlo. Pertanto vi è una bigezione tra
			$\Stab(\e n)$ e $O_{n-1}$.
			
			\item Sia $\Gr_k(\RR^n) = \{ W \subseteq \RR^n \mid \dim W = k \}$, detto la Grassmanniana di $\RR^n$ di ordine $k$. $O_n$ opera transitivamente su $\Gr_K(\RR^n)$.
		\end{enumerate}
	\end{example}

	\begin{definition}
		Si dice che $G$ \textbf{opera in maniera semplicemente transitiva} su $X$
		se, dato $x \in X$, l'applicazione da $G$ in $X$ tale che $g \mapsto g \cdot x$ è una bigezione,
		ossia se $G$ opera transitivamente e liberamente.
	\end{definition}

	\begin{definition}
		Un insieme $X$ che subisce un'azione del gruppo $G$ che opera in maniera
		semplicemente transitiva è
		detto un \textbf{$G$-insieme omogeneo principale}.
	\end{definition}
	
	\begin{example}
		Se $X = G$ e l'azione considerata è quella naturale dell'operazione di $G$,
		tale azione opera in maniera semplicemente transitiva. Dato $x \in X$, si consideri infatti l'applicazione $\tau$
		da $G$ in $G$ tale che
		$g \mapsto g \cdot x = gx$. Si osserva che $\tau$ è surgettiva, dacché, dato $h \in G$,
		$h = h x\inv x = \tau(h x\inv)$. Inoltre $\tau$ è iniettiva, dal momento che, dati $g$, $g'$
		tali che $\tau(g) = \tau(g')$, allora $gx = g' x \implies g = g'$. Pertanto $\tau$ è bigettiva, e
		l'azione opera allora in maniera semplicemente transitiva.
	\end{example}
	
	\begin{remark}\nl
		\li Se $X$ è $G$-omogeneo principale, l'azione di $G$ su $X$ è fedele. Infatti, $f_g = \Id \implies g \cdot x = x$ $\forall x \in X$. Dal momento però che $X$ è $G$-omogeneo principale, $G$ opera liberamente su $X$,
		e quindi $\Stab(x) = \{e\}$ $\forall x \in X \implies g = e$. \\

		\li Se $X$ è $G$-omogeneo e $G$ è abeliano, allora $G$ agisce fedelmente su $X$ $\iff$ $X$ è $G$-omogeneo principale. \\
		
		Se $G$ agisce fedelmente su $X$, dato $x \in X$, si può considerare infatti $g \in \Stab(x) \implies g \cdot x = x$. Si osserva allora
		che $f_g = \Id$. Dato infatti $y \in X$, dacché $X$ è $G$-omogeneo, $\exists g' \in G \mid y = g' \cdot x$,
		da cui si ricava che $f_g(y) = g \cdot y = g \cdot (g' \cdot x) = (gg') \cdot x = (g'g) \cdot x = g' \cdot (g \cdot x) = g' \cdot x = y$, ossia proprio che $f_g = \Id$. Dal momento però che l'azione di $G$ su $X$ è fedele,
		$f_g = \Id \implies g = e$, ossia $\Stab(x) = \{e\}$ $\forall x \in X$, per cui si conclude che l'azione
		di $G$ opera in maniera semplicemente transitiva su $X$, e dunque che $X$ è $G$-omogeneo principale. \\
		
		Viceversa, se $X$ è $G$-omogeneo principale, $\Stab(x) = \{ e \}$ $\forall x \in X$. Allora, se $f_g = \Id$,
		per ogni $x \in X$ deve valere che $g \in \Stab(x) = \{ e \} \implies g = e$.
	\end{remark}
	
	\hr

	\begin{definition} [spazio affine]
		Sia $V$ uno spazio vettoriale su un campo $\KK$ qualsiasi.
		Allora uno spazio affine $E$ associato a $V$ è un qualunque
		$V$-insieme omogeneo principale\footnote{Per gruppo $V$ si intende il gruppo abeliano $(V, +)$.}.
		In particolare si indica l'azione di $V$ su $E$ $(\v, P) \mapsto \v \cdot P$ come $P + \v$ (o
		analogamente come $\v + P$). Inoltre, gli elementi di $E$ si
		diranno \textit{punti di} $E$.
	\end{definition}
	
	\begin{remark}
		Dal momento che $E$ è un $V$-insieme omogeneo principale, valgono le seguenti proprietà.
		
		\begin{enumerate}[(i)]
			\item Poiché $E$ è omogeneo, per ogni $P \in E$, $Q \in E$ esiste $\v \in V$ tale che $P + \v = Q$.
			Inoltre, dal momento che $V$ opera liberamente su $E$, tale $\v$ è unico, e si indica come
			$Q - P$ o come $\vvec{PQ}$.
			
			\item Vale l'identità $P + \vec 0 = P$, dal momento che $\vec 0$ è l'identità del gruppo $(V, +)$
			e l'applicazione $P + \v$ è un azione di $V$. Allo stesso modo, vale che $(P + \v) + \w = P + (\v + \w) =
			P + (\w + \v) = (P + \w) + \v$, pertanto si può scrivere, senza alcuna ambiguità, $P + \v + \w$.
			
			\item Fissato $O \in E$, l'applicazione da $V$ in $E$ tale che $\v \mapsto O + \v$ è una bigezione,
			dal momento che $V$ opera su $E$ in maniera semplicemente transitiva.
			
			\item Analogamente, fissato $O \in E$, l'applicazione $\tau$ da $E$ in $V$ tale che $P \mapsto P - O = \vvec{OP}$
			è una bigezione. Infatti $\tau$ è surgettiva: $\forall \v \in V$, $\tau(O + \v) = (O + \v) - O = \v$,
			coerentemente con le operazioni aritmetiche. Infine, $\tau$ è iniettiva: siano $P$, $Q \in E$ tali che
			$\tau(P) = \tau(Q)$, allora $P = O + (P - O) = O + \tau(P) = O + \tau(Q) = O + (Q - O) = Q$, per
			cui $\tau$ è bigettiva.
			
			\item Dati $P$, $Q \in E$, vale l'identità $P - Q = -(Q - P)$. Infatti $P = Q + (P-Q) = P + (Q-P) + (P-Q) =
			P + ((Q-P) + (P-Q))$. Allora, essendo l'azione di $V$ libera su $E$ (ovvero, come osservato prima,
			essendo $\vvec{PP}$ unicamente zero), $(Q-P) + (P-Q) = \vec 0 \implies P-Q = -(Q-P)$.
			
			\item Dati $P_1$, $P_2$, $P_3 \in E$, vale l'identità $(P_3 - P_2) + (P_2 - P_1) = P_3 - P_1$. Infatti
			$P_1 + (P_2 - P_1) + (P_3 - P_2) = P_2 + (P_3 - P_2) = P_3 \implies (P_2 - P_1) + (P_3 - P_2) = P_3 - P_1$.
		\end{enumerate}
	\end{remark}

	Siano adesso $P_1$, ..., $P_n$ punti di $E$. Dati $\lambda_1$, ..., $\lambda_n \in \KK$ e $O \in E$ si può allora individuare il punto $P = O + \sum_{i=1}^n \lambda_i (P_1 - O) \in E$.
	
	\begin{proposition}
		Dati $P_1$, ..., $P_n$ punti di $E$ e $\lambda_1$, ..., $\lambda_n \in \KK$, il punto
		$P(O) = O + \sum_{i=1}^n \lambda_i (P_i - O)$ rappresenta lo stesso identico punto al
		variare del punto $O$ se e solo se $\sum_{i=1}^n \lambda_i = 1$.
	\end{proposition}
	
	\begin{proof}
		Siano $O$, $O'$ due punti distinti di $E$. Allora $P(O) = P(O') \iff O + \sum_{i=1}^n \lambda_i (P_i - O) =
		O' + \sum_{i=1}^n \lambda_i (P_i - O') = O + (O' - O) + \sum_{i=1}^n \lambda_i (P_i - O') \iff \sum_{i=1}^n \lambda_i (P_i - O) = (O' - O) + \sum_{i=1}^n \lambda_i ((P_i - O) + (O - O'))$. Distribuendo
		la somma e utilizzando l'identità dell'\textit{Osservazione} (v), si ottiene allora che $P(O) = P(O') \iff
		\sum_{i=1}^n \lambda_i = 1$. 
	\end{proof}
	
	\begin{definition} [combinazione affine di punti]
		Un punto $P \in E$ è \textbf{combinazione affine} dei punti
		$P_1$, ..., $P_n$ se $\exists \lambda_1$, ..., $\lambda_n \in \KK$, $O \in E$ tali che $P = O + \sum_{i=1}^n \lambda_i (P_i - O)$ e che
		$\sum_{i=1}^n \lambda_i = 1$. Dal momento che per la precedente proposizione $P$ è invariante al variare di $O \in E$, si scriverà, senza alcuna ambiguità, che
		$P = \sum_{i=1}^n \lambda_i P_i$.
	\end{definition}
	
	\begin{definition} [sottospazio affine]
		Un sottoinsieme $D \subseteq E$ si dice \textbf{sottospazio affine} di $E$ se ogni combinazione
		affine di finiti termini di $D$ appartiene a $D$.
	\end{definition}
	
	\begin{definition} [sottospazio affine generato un insieme $S$]
		Dati $S \subseteq E$, si dice \textbf{sottospazio affine generato da $S$}
		l'insieme delle combinazioni affini di finiti termini dei punti di $S$, denotato con $\Aff(S)$.
	\end{definition}
	
	\begin{remark}\nl
		\li Come avviene per $\Span$ nel caso degli spazi vettoriali, dati $P_1$, ..., $P_n \in E$, si usa scrivere $\Aff(P_1, \ldots, P_n)$
		per indicare $\Aff(\{P_1, \ldots, P_n\})$. \\
		
		\li Si osserva che in effetti, dato $S \subseteq E$, $\Aff(S)$ è un sottospazio affine, ossia è
		chiuso per combinazioni affini dei propri punti. Siano infatti $P_1$, ..., $P_n$ punti di $\Aff(S)$
		e siano $\lambda_1$, ..., $\lambda_n \in \KK$ tali che $\sum_{i=1}^n \lambda_i = 1$. Si deve
		mostrare dunque che $\sum_{i=1}^n \lambda_i P_i \in \Aff(S)$. Dal momento che $P_i \in \Aff(S)$ esiste $k_i \in \NN^+$ tale per cui esistano
		$S_{i,1}$, ..., $S_{i,k_i} \in S$ e $\lambda_{i,1}$, ..., $\lambda_{i,k_i} \in \KK$ tali per cui
		$P_i = \sum_{j=1}^{k_i} \lambda_{i,j} S_{i,j}$ e $\sum_{j=1}^{k_i} \lambda_{i,j} = 1$. Allora
		$\sum_{i=1}^n \lambda_i P_i = \sum_{i=1}^n \lambda_i (\sum_{j=1}^{k_i} \lambda_{i,j} S_{i,j}) =
		\sum_{i=1}^n \sum_{j=1}^{k_i} \lambda_i \lambda_{i,j} S_{i,j}$. Inoltre $\sum_{i=1}^n \sum_{j=1}^{k_i} \lambda_i \lambda_{i,j} = \sum_{i=1}^n \lambda_i (\sum_{j=1}^{k_i}  \lambda_{i,j}) = \sum_{i=1}^n \lambda_i = 1$.
		Pertanto $\sum_{i=1}^n \lambda_i P_i$ è combinazione affine di elementi di $S$, e quindi $\sum_{i=1}^n \lambda_i P_i \in \Aff(S)$. \\
		
		\li Siano $P_1$, $P_2 \in E$. Allora il sottospazio affine $\Aff(P_1, P_2) = \{ \lambda_1 P_1 + \lambda_2 P_2 \mid \lambda_1 + \lambda_2 = 1, \lambda_1, \lambda_2 \in \KK \} = \{ (1-\lambda) P_1 + \lambda P_2 \mid \lambda \in \KK \} = \{ P_1 + \lambda (P_2 - P_1) \mid \lambda \in \KK \}$ è detto \textit{retta affine passante per $P_1$ e $P_2$}. Analogamente il sottospazio affine generato da tre elementi è detto \textit{piano affine}. \\
		
		\li Dato un insieme di punti $S \subseteq E$, $\Aff(S)$ è il più piccolo sottospazio affine, per inclusione,
		contenente $S$. Infatti, se $T$ è un sottospazio affine contenente $S$, per definizione $T$ deve
		contenere tutte le combinazioni affini di $S$, e quindi $\Aff(S)$.
	\end{remark}
\end{document}
