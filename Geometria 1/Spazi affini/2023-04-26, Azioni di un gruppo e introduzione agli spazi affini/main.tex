\documentclass[11pt]{article}
\usepackage{personal_commands}
\usepackage[italian]{babel}

\title{\textbf{Note del corso di Geometria 1}}
\author{Gabriel Antonio Videtta}
\date{26 aprile 2023}

\begin{document}
	
	\maketitle
	
	\begin{center}
		\Large \textbf{Azioni di un gruppo e introduzione agli spazi affini}
	\end{center}
	
	\wip
	
	\begin{note}
		Nel corso delle lezioni si è impiegata la notazione $g.x$ per indicare
		l'azione di un gruppo su un dato elemento $x \in X$. Tuttavia si è
		preferito indicare $g.x$ con $g \cdot x$ nel corso del documento. \\
		
		Inoltre, con $G$ si indicherà un generico gruppo, e con $X$ un
		generico insieme, sul quale $G$ agisce, qualora non indicato diversamente.
	\end{note}

	\begin{definition} [azione di un gruppo su un insieme]
		Sia $G$ un gruppo e sia $X$ un insieme. Un'azione sinistra, comunemente detta solo \textbf{azione}, di $G$
		su $X$ è un'applicazione da $G \times X$ in $X$ tale
		che $(g, x) \mapsto g \cdot x$ e che:
		
		\begin{enumerate}[(i)]
			\item $e \cdot x = x$ $\forall x \in X$,
			\item $g \cdot (h \cdot x) = (gh) \cdot x$ $\forall x \in X$, $\forall g$, $h \in G$.
		\end{enumerate}
	\end{definition}

	\begin{remark}\nl
		\li Data un'azione di $G$ su $X$, si può definire un'applicazione
		$f_g : X \to X$ tale che, dato $g \in G$, $f_g(x) = g \cdot x$. \\
		\li Tale applicazione $f_g$ è bigettiva, dal momento che $f_{g\inv}$ è una sua
		inversa, sia destra che sinistra. Infatti $(f_g \circ f_{g\inv})(x) =
		g \cdot (g\inv \cdot x) = (g g\inv) \cdot x = e \cdot x = x$, e così il viceversa. 
	\end{remark}

	\begin{definition}
		L'azione di un gruppo $G$ su un insieme $X$ si dice \textbf{fedele} se
		l'omomorfismo $\varphi_G$ da $G$ in $S(G)$, ossia nel gruppo delle bigezioni su $G$, che
		associa $g$ a $f_g$ è iniettiva.
	\end{definition}
	
	\begin{remark}
		Si osserva che dire che un'azione di un gruppo è fedele è equivalente
		a dire che $\Ker \varphi_G = \{ e \}$, ossia che $f_g = \Id \iff g = e$.
	\end{remark}
	
	\begin{example}
		Si possono fare alcuni esempi di azioni classiche su alcuni gruppi.
		\begin{enumerate}[(i)]
			\item $S(X)$ agisce su $X$ in modo tale che $f \cdot x = f(x)$ $\forall f \in S(X), x \in X$.
			
			\item $G$ agisce su $G$ stesso tramite l'operazione del gruppo, ossia $g \cdot g' = gg'$ $\forall g$, $g' \in G$.
			
			\item Data un'azione sinistra di $G$ su $X$ tale che $(g, x) \mapsto g \cdot x$, si può definire
			naturalmente un'azione destra da $X \times G$ in $X$ in modo tale che $(x, g) \mapsto x \cdot g = g\inv \cdot x$.
			Infatti $x \cdot e = e\inv \cdot x = e \cdot x = x$, e $(x \cdot g) \cdot g' = (g\inv \cdot x) \cdot g' =
			{g'}\inv \cdot (g\inv \cdot x) = ({g'}\inv g\inv) \cdot x = (g g')\inv \cdot x = x \cdot (g g')$.
		\end{enumerate}
	\end{example}
	
	\begin{definition} [$G$-insieme]
		Se esiste un azione di $G$ su $X$, si dice che $X$ è un $G$\textit{-insieme}.
	\end{definition}
	
	\begin{definition} [orbita di $x$]
		Sia $\sim_G$ la relazione d'equivalenza tale che $x \sim_G y \defiff \exists g \in G \mid g \cdot x = y$.
		Allora le classi di equivalenza si dicono \textbf{orbite}, ed in particolare si indica l'orbita a cui
		appartiene un dato $x \in X$ come $\Orb(x) = O_x$, ed è detta \textit{orbita di} $x$.
	\end{definition}
	
	\begin{example} Si possono individuare facilmente alcune orbite per alcune azioni classiche.
		\begin{enumerate}[(i)]
			\item Se $G = \GL(n, \KK)$ è il gruppo delle matrici invertibili su $\KK$ di taglia $n$ rispetto
			all'operazione di moltiplicazione matriciale, $G$ opera naturalmente su $M(n, \KK)$ tramite
			la similitudine, ossia $G$ agisce in modo tale che $P \cdot M = P M P\inv$ $\forall P \in \GL(n, \KK)$,
			$M \in M(n, \KK)$. In particolare, data $M \in M(n, \KK)$, $\Orb(M)$ coincide esattamente
			con la classe di similitudine di $M$.
			
			\item Se $G = \GL(n, \KK)$, $G$ opera naturalmente anche su $\Sym(n, \KK)$
			tramite la congruenza, ossia tramite la mappa $(P, A) \mapsto P^\top A P$. L'orbita $\Orb(A)$ è la classe di congruenza delle matrice simmetria $A \in \Sym(n, \KK)$. Analogamente si può costruire un'azione per le
			matrici hermitiane.
			
			\item Se $G = O_n$, il gruppo delle matrici ortogonali di taglia $n$ su $\KK$, $G$ opera su $\RR^n$ tramite la mappa $O \cdot \vec v \mapsto O \vec v$. L'orbita $\Orb(\vec v)$ è in particolare la sfera $n$-dimensionale di raggio $\norm{x}$.
		\end{enumerate}
	\end{example}

	\begin{definition} [stabilizzatore di $x$]
		Lo \textbf{stabilizzatore} di un punto $x \in X$ è l'insieme degli elementi di $G$ che
		agiscono su $x$ lasciandolo invariato, ossia lo stabilizzatore $\Stab_G(X)$ è il sottogruppo
		di $G$ tale che:
		
		\[ \Stab_G(X) = \{g \in G \mid g \cdot x = x \}. \]
	\end{definition}

	\begin{example}
		Sia $H \subseteq G$ e sia $X = G/H$. $X$ è un $G$-insieme
		tramite l'azione $g'.(gH) = g'gH$. Vale in particolare
		che $\Stab_G(eH) = H$.
	\end{example}

	\begin{proposition}
		Sia $X$ un $G$-insieme. Sia $x \in X$. $H = \Stab_G(x)$ e sia
		$O_x$ l'orbita di $x$. Allora esiste un'applicazione bigettiva
		naturale $G/H \to O_x$.
	\end{proposition}

	\begin{proof}
		Sia $\varphi$ tale che $\varphi(gH) = g.x$. Si mostra che
		$\varphi$ è ben definita: $g' = gh$, $\varphi(g'H) = (gh).x =
		g.(h.x) = g.x$. Chiaramente $\varphi$ è anche surgettiva.
		Inoltre, $g.x = g'.x \implies x = (g\inv g').x \implies g\inv g' = h \in H \implies gH = g'H$, e pertanto $\varphi$ è iniettiva.
		Allora $\varphi$ è bigettiva.
	\end{proof}

	\begin{definition}
		Si dice che $G$ opera \textit{liberamente} su $X$ se
		$\forall x \in X$, l'applicazione $G \to O_x$ tale che
		$g \mapsto g.x$, ossia se $\Stab_G(x) = \{e\}$:
	\end{definition}
	
	\begin{definition}
		$G$ opera \textit{transitivamente} su $X$ se $x \sim_G y$ $\forall x$, $y \in X$, cioè se c'è un'unica orbita, che coincide con $X$. In
		tal caso si dice che $X$ è \textbf{omogeneo} per l'azione di $G$.
	\end{definition}

	\begin{example}
		\begin{enumerate}[(i)]
			\item $O_n$ opera su $S^{n-1} \subseteq \RR^n$ transitivamente.
			%TODO: aggiunge che lo stabilizzatore è isomorfo alle ortogonali
			%TODO: di dimensioni n-1
			
			\item $\Gr_k(\RR^n) = \{ W \subseteq \RR^n \mid \dim W = k \}$ (Grassmanniana). $O_n$ opera transitivamente su $\Gr_K(\RR^n)$.
		\end{enumerate}
	\end{example}

	\begin{definition}
		$G$ opera in maniera \textit{semplicemente transitiva} su $X$
		se $\exists x \in X$ tale che $g \mapsto g.x$ è una bigezione,
		ossia se $G$ opera transitivamente e liberamente.
	\end{definition}

	\begin{definition}
		Un insieme $X$ con un'azione semplicemente transitiva di $G$ è
		detto un $G$-insieme omogeneo \textit{principale}.
	\end{definition}

	\begin{example}
		\begin{enumerate}[(i)]
			\item $X = G$. L'azione naturale di $G$ su $X$ per moltiplicazione
			è semplicemente transitivo (per $g$, $g' \in G$, esiste un
			unico $h \in G$ tale che $g = h.g' = hg'$). Quindi $X$
			è $G$-omogeneo principale.
			
			\item Se $X$ è $G$-omogeneo principale, l'azione è fedele.
			
			\item Se $X$ è omogeneo per un gruppo $G$ commutativo, allora
			$G$ agisce fedelmente su $X$ $\implies$ $X$ è un $G$-insieme
			omogeneo principale.
		\end{enumerate}
	\end{example}

	\begin{definition} [spazio affine]
		Sia $V$ uno spazio vettoriale su un campo $\KK$ qualsiasi.
		Allora uno spazio affine $E$ associato a $V$ è un qualunque
		$V$-insieme omogeneo principale.
	\end{definition}

	Pertanto, $\forall P, Q \in E$, esiste un unico vettore $\v \in V$
	tale che $Q = \v . P $, denotato come $Q = P + \v = \v + P$. Si
	osserva che $\v + (\w + P) = (\v + \w) + P$. Essendo $\v$ unico,
	si scrive $\v = Q - P = \vvec{PQ}$.
	
	%TODO: aggiunge applicazione bigettiva
	
	Fissato $O \in E$, l'applicazione $\v \mapsto \v + O$, $V \to E$
	è una bigezione.
	
	\begin{remark}\nl
		\li $P-P = \vec 0 \in V$, $P-Q = -(Q-P)$, $(P_3 - P_2) + (P_2 - P_1) = P_3 - P_1$. \\
		
		\li $O \in E$ l'applicazione $P \mapsto P-O$ è una bigezione di $E$
		su $V$.
	\end{remark}

	Siano $P_1$, ..., $P_n \in E$. $\forall \lambda_1$, ..., $\lambda_k \in \KK$. $\forall O \in E$ possiamo individuare il punto $P = O + \sum_{i=1}^n \lambda_i (P_i - O)$.
	
	$P = P' = \iff O + \sum_{i=1}^n \lambda_i (P_i - O) = O' + \sum_{i=1}^n \lambda_i (P_i - O') \iff O + \sum_{i=1}^n \lambda_i (O' - O) = O' \iff
	(\sum \lambda_i) (O' - O) = O' - O \iff \sum \lambda_i = 1$.
	
	\begin{definition}
		Un punto $P \in E$ è \textbf{combinazione affine} dei punti
		$P_1$, ..., $P_k$ se $P = O + \sum \lambda_i (P_i - O)$ se
		$\sum \lambda_i = 1$. Si scriverà, in particolare, che
		$P = \sum \lambda_i P_i$.
	\end{definition}

	Si chiama retta affine l'insieme dei punti che sono combinazione affine di
	due punti. Analogamente si fa per un piano e uno spazio.
	
	\begin{definition}
		Un sottoinsieme $D \subseteq E$ si dirà \textbf{sottospazio affine}
		se è chiuso per combinazioni affini (finite).
	\end{definition}

	\begin{definition}
		Il sottospazio affine $D \subseteq E$ generato da un sottoinsieme $S \subseteq E$ è l'insieme delle combinazioni affini (finite) di punti
		di $S$, detto $D = \Aff(S)$. %TODO: mostrare che è chiuso per combinazioni affini.
	\end{definition}
\end{document}
