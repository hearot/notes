\documentclass[11pt]{article}
\usepackage{personal_commands}
\usepackage[italian]{babel}

\title{\textbf{Note del corso di Geometria 1}}
\author{Gabriel Antonio Videtta}
\date{28 aprile 2023}

\begin{document}
	
	\maketitle
	
	\begin{center}
		\Large \textbf{Indipendenza e applicazioni affini}
	\end{center}
	
	Fissato un origine $O$ dello spazio affine, si possono sempre considerare due
	bigezioni:
	
	\begin{itemize}
		\item La bigezione $i_O : E \to V$ tale che $i(P) = P - O \in V$,
		\item La bigezione $j_O : V \to E$ tale che $j(\v) = O + \v \in E$.
	\end{itemize}
	
	Si osserva inoltre che $i_O$ e $j_O$ sono l'una la funzione inversa dell'altra.
	
	Dato uno spazio vettoriale $V$ su $\KK$ di dimensione $n$, si può considerare $V$ stesso
	come uno spazio affine, denotato  con le usuali operazioni:
	
	\begin{enumerate}[(a)]
		\item $\v + \w$, dove $\v \in V$ è inteso come $\mathit{punto}$ di $V$ e $\w \in W$ come
		il vettore che viene applicato su $\w$, coincide con la somma tra $\v$ e $\w$ (e analogamente
		$\w - \v$ è esattamente $\w - \v$).
		
		\item Le bigezioni considerate inizialmente sono in particolare due mappe tali che
		$i_{\vv 0}(\v) = \v - \vv 0$ e che $j_{\vv 0}(\v) = \vv 0 + \v$.
	\end{enumerate}
	
	\begin{definition} [spazio affine standard]
		Si denota con $\AnK$ lo \textbf{spazio affine standard} costruito sullo spazio vettoriale
		$\KK^n$. Analogamente si indica con $A_V$ lo spazio affine costruito su uno spazio
		vettoriale $V$.
	\end{definition}

	\begin{remark}\nl
		\li Una combinazione affine di $A_V$ è in particolare una combinazione lineare di $V$. Infatti,
		se $\v = \sum_{i=1}^n \lambda_i \vv i$ con $\sum_{i=1}^n \lambda_i = 1$, allora, fissato
		$\vv 0 \in V$, $\v = \vv 0 + \sum_{i=1}^n \lambda_i (\vv i - \vv 0) = \vv 0 + \sum_{i=1}^n \lambda_i \vv i - \vv 0 = \sum_{i=1}^n \lambda_i \vv i$.
		
		\li Come vi è una bigezione data dal passaggio alle coordinate da $V$ a $\KK^n$, scelta una base
		$\basis$ di $V$ e un punto $O$ di $E$, vi è anche una bigezione $\varphi_{O, \basis}$ da $E$ a $\AnK$ data
		dalla seguente costruzione:
		
		\[ \varphi_{O, \basis}(P) = [P-O]_\basis. \]
	\end{remark}
	
	\begin{proposition}
		Sia $D \subseteq E$. Allora $D$ è un sottospazio affine di $E$ $\iff$ fissato $P_0 \in D$, l'insieme
		$D_0 = \{ P - P_0 \mid P \in D \} \subseteq V$ è un sottospazio vettoriale di $V$.
	\end{proposition}
	
	\begin{proof}
		Si dimostrano le due implicazioni separatamente. \\
		
		\rightproof Siano $\vv 1$, ..., $\vv k \in D_0$. Allora, per definizione, esistono $P_1$, ...,
		$P_k \in D$ tali che $\vv i = P_i - P_0$ $\forall 1 \leq i \leq k$. Siano $\lambda_1$, ...,
		$\lambda_k \in \KK$. Sia inoltre $P = P_0 + \sum_{i=1}^k \lambda_i \vv i \in E$. Sia infine
		$O \in D$. Allora $P = O + (P_0 - O) + \sum_{i=1}^k \lambda_i \vv i = O + (P_0 - O) + \sum_{i=1}^k \lambda_i (P_i - O + O - P_0) = O + (P_0 - O) + \sum_{i=1}^k \lambda_i (P_i - O) - \sum_{i=1}^k \lambda_i (P_0 - O) =
		O + (1-\sum_{i=1}^k \lambda_i) (P_0 - O) + \sum_{i=1}^k \lambda_i (P_i - O)$. In particolare $P$
		è una combinazione affine di $P_1$, ..., $P_k \in D$, e quindi, per ipotesi, appartiene a $D$. Allora
		$P - P_0 =  \sum_{i=1}^k \lambda_i \vv i \in D_0$. Poiché allora $D_0$ è chiuso per combinazioni lineari,
		$D_0$ è un sottospazio vettoriale di $V$. \\
		
		\leftproof Sia $P = \sum_{i=1}^k \lambda_i P_i$ con $\sum_{i=1}^k \lambda_i = 1$, con $P_1$, ..., $P_k \in D$ e
		$\lambda_1$, ..., $\lambda_k \in \KK$. Allora $P - P_0 = \sum_{i=1}^k \lambda_i (P_i - P_0) \in D_0$ per ipotesi, essendo combinazione lineare di elementi di $D_0$. Pertanto, poiché esiste un solo punto $P'$
		tale che $P' = P_0 + \sum_{i=1}^k \lambda_i (P_i - P_0)$, affinché $\sum_{i=1}^k \lambda_i (P_i - P_0)$
		appartenga a $D_0$, deve valere anche che $P \in D$. Si conclude quindi che $D$ è un sottospazio
		affine, essendo chiuso per combinazioni affini.
	\end{proof}
	
	\begin{remark}Sia $D$ un sottospazio affine di $E$. \\

		\li Vale la seguente identità $D_0 = \{ P - Q \mid P, Q \in D \}$. Sia infatti $A = \{ P - Q \mid P, Q \in D \}$. Chiaramente $D_0 \subseteq A$.
		Inoltre, se $P-Q \in A$, $P-Q = (P-P_0) - (Q-P_0)$. Pertanto, essendo $P-Q$ combinazione lineari di elementi
		di $D_0$, ed essendo $D_0$ spazio vettoriale per la proposizione precedente, $P-Q \in D_0 \implies A \subseteq D_0$, da cui si conclude che $D_0 = A$. \\
		\li Pertanto $D_0$ è unico, a prescindere dalla scelta di $P_0 \in D$. \\
		\li Vale che $D = P_0 + D_0$, ossia $D$ è il traslato di $D$ mediante il punto $P_0$.
	\end{remark}
	
	\begin{definition} [direzione di un sottospazio affine]
		Si definisce $D_0$ come la \textbf{direzione} del sottospazio affine $D$.
	\end{definition}

	\begin{definition} [dimensione un sottospazio affine]
		Dato $D$ sottospazio affine di $E$, si dice dimensione di $D$,
		indicata con $\dim D$, la dimensione della sua direzione $D_0$, ossia
		$\dim D_0$. In particolare $\dim E = \dim V$.
	\end{definition}
	
	\begin{definition} [sottospazi affini paralleli]
		Due sottospazi affini si dicono \textbf{paralleli} se condividono
		la stessa direzione.
	\end{definition}

	\begin{remark}\nl
		\li I sottospazi affini di dimensione zero sono tutti i punti di $E$. \\
		\li I sottospazi affini di dimensione uno sono le \textit{rette affini},
		mentre quelli di dimensione due sono i \textit{piani affini}. \\
		\li Si dice \textit{iperpiano affine} un sottospazio affine di codimensione $1$,
		ossia di dimensione $n-1$.
	\end{remark}

	\begin{definition} [punti affinemente indipendenti]
		I punti $P_1$, ..., $P_n \in E$ si dicono affinemente indipendenti se l'espressione $P = \sum \lambda_i P_i$ con $\sum \lambda_i = 1$
		è unica $\forall P \in \Aff(P_1, \ldots, P_n)$. Analogamente
		un sottoinsieme $S \subseteq E$ si dice affinemente indipendente
		se ogni suo sottoinsieme finito lo è.
	\end{definition}

	\begin{proposition}
		$P_1$, ..., $P_n$ sono affinemente indipendenti $\iff$
		$\forall i = 1, \ldots, k$ i vettori $P_j - P_i$ con $j \neq i$
		sono linearmente indipendenti $\iff$ $\exists i = 1, \ldots, k$ i vettori $P_j - P_i$ con $j \neq i$
		sono linearmente indipendenti $\forall i P_i \notin \Aff\{P_1, \ldots, P_n\}$ con $P_i$ escluso.
	\end{proposition}

	\begin{proof}
		%TODO: considerare il passaggio ai vettori spostamento
	\end{proof}

	\begin{remark}\nl
		\li Il numero massimo di punti affinemente indipendenti in $E$ è $\dim E + 1$. \\
		
		\li Se $E = A_n(\KK)$ e $V = \KK^n$. Allora $\ww 1$, ...,
		$\ww k \in E$ sono aff. indip. $\iff$ i vettori $\ww1$, ...,
		$\ww k$ immersi in $\KK^{n+1}$ aggiungendo una coordinata $1$
		in fondo sono linearmente indipendenti.
	\end{remark}

	\begin{remark} Sia $E$ spazio affine con $V$ di dimensione $n$. Si
		scelgano $n+1$ punti affinemente indipendenti $P_0$, ..., $P_n$.
		Allora $\Aff(P_0, ..., P_n) = E$. Quindi $P \in E$ si scrive
		in modo unico come $P = \sum \lambda_i P_i$ con $\sum \lambda_i = 1$.
		Le $\lambda_i$ si diranno allora le coordinate affini di $P$
		nel riferimento $P_0$, ..., $P_n$.
	\end{remark}

	Se si impone $\lambda_i \geq 0$, si definisce che la
	combinazione è una combinazione convessa. Si definisce
	baricentro il punto con $\lambda_i = \frac{1}{n}$.
	
	\begin{definition} [inviluppo convesso] Si dice $\IC(S)$ di un insieme
		$S \subseteq E$ l'insieme delle combinazioni convesse di $S$ (finite).
		%TODO: dimostrare che è un insieme convesso
	\end{definition}

	% TODO: aggiungere baricentro
	
	\begin{definition} Sia $E$ uno spazio affine su $V$, $E'$ spazio
		affine su $V'$ (sullo stesso $\KK$) un'applicazione $f : E \to E'$
		si dice app. affine se conserva le combinazioni affini
		($f(\sum \lambda_i P_i) = \sum \lambda_i f(P_i)$, $\sum \lambda_i = 1$).
	\end{definition}

	\begin{theorem} Sia $f : E \to E'$ affine. Allora $\exists$ unica
		app. lineare $g : V \to V'$ lineare tale che valga
		$f(O + \v) = f(O) + g(\v)$, per ogni scelta di $O \in E$.
	\end{theorem}

	\begin{proof}
		Sia $O \in E$. L'applicazione $g_O : V \to V'$ data da
		$g_O(\v) = f(O + \v) - f(O)$. Si dimostra che $g_O$ è
		lineare.
	\end{proof}
\end{document}
