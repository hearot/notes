\documentclass[11pt]{article}
\usepackage{personal_commands}
\usepackage[italian]{babel}

\title{\textbf{Note del corso di Geometria 1}}
\author{Gabriel Antonio Videtta}
\date{\today}

\begin{document}
	
	\maketitle
	
	\begin{center}
		\Large \textbf{Esercitazione: forma canonica di Jordan reale}
	\end{center}

	\wip

	\begin{exercise}
		Sia $M \in M(n, \RR)$ tale che $\exists a_1$, ..., $a_k \in \RR$ distinti
		tale che:
		
		\[ (M^2 + a_1^2 I) \cdots (M^2 + a_k^2 I) = 0. \]
		
		Dimostrare allora che esistono $S$, $A \in M(n, \RR)$ tale che
		$M = SA$ con $S$ simmetrica e $A$ antisimmetrica.
	\end{exercise}

	\begin{solution}
		Per ipotesi, $p(x) = (x^2+a_1^2) \cdots (x^2 + a_k^2) \in \Ker \sigma_M$.
		Dal momento che $p(x)$ si scompone in fattori lineari distinti in
		$\CC$, $p(x)$ è anche il polinomio minimo di $M$. Si deduce
		allora che $M$ è diagonalizzabile, e che i suoi autovalori sono
		esattamente $\pm a_1 i$, ..., $\pm a_k i$. Allora la forma
		canonica di Jordan reale di $M$ è:
		
		\[ J = \Matrix{1} \]
		
		... \\
	\end{solution}

	\begin{remark}\nl
		\li $f(\Rad \varphi) = \Rad \psi$.
		\li $[]$ è un'isometria tra $(V, \varphi)$ e $(\KK^n, M_\basis(\varphi))$.
		\\ Si dice cono isotropo $CI(\varphi)$ l'insieme dei vettori
		isotropi di $V$. $CI(\varphi) = V \iff \varphi = 0$ ($\Char \KK\neq 2$).
	\end{remark}

	\begin{exercise}
		Sia $V = \RR_2[x]$ e sia $\varphi : V \times V \to \RR$ tale che
		$\varphi(p, q) = p(1) q(2) + p(2) q(1)$ $\forall p$, $q \in V$.
		Si mostri che $\varphi$ è un prodotto scalare di $V$.
	\end{exercise}

	\begin{solution}
		Si osserva che $\varphi$ è simmetrica. Inoltre, $\varphi(p + p', q) =
		p(1) q(2) + p'(1) q(2) + p(2) q(1) + p'(2) q(1) = \varphi(p, q) +
		\varphi(p', q)$, e $\varphi(\alpha p, q) = \alpha \varphi(p, q)$;
		quindi $\varphi$ è un prodotto scalare. \\
		
		Sia $\basis$ la base con $1$, $x$, $x^2$. Allora la matrice
		associata è:
		
		\[ M = \Matrix{ 2 & 3 &  5 \\ 3 & 4 & 6 \\ 5 & 6 & 8 }. \]
		
		Vale che $\rg(M) = 2$ e che $\Ker M = \Span\Vector{2 \\ -3 \\ 1}$,
		ossia che $\Rad \varphi = \Span(x^2-3x+2)$. Si poteva
		ottenere questo risultato direttamente dalla definizione di $\varphi$.
		Sia infatti $\varphi(p, q) = 0$ $\forall q \in V$. Sia allora
		$q = x-2$: allora $\varphi(p, q) = p(2) q(1) = -p(2) = 0 \implies
		x-2 \mid p$. Con $q = x-1$, invece, $x-1 \mid p$. Quindi $(x-1)(x-2) \mid p \implies p \in \Span((x-1)(x-2)) = \Span(x^2-3x+2)$.
	\end{solution}

	\begin{exercise}
		Sia $\KK = \RR$. Sia $V = S(2, \RR)$ e sia $\varphi : V \times V \to \KK$ tale che che $\varphi(A, B) = (1 2) A B \Vector{1 \\ 2}$ $\forall
		A, B \in V$. Infatti $\varphi(B, A) = (1 2) B A \Vector{1 \\ 2} =
		(1 2) A^\top B^\top \Vector{1 \\ 2} = \varphi(A, B)$. Chiaramente
		è lineare. \\
		
		Sia $\basis$ $A_1$, $A_2$ (0 0 \\ 0 1) e $A_3$ la base standard di $V$.
		Allora $\varphi(A_1, A_1) = 1$, $\varphi(A_1, A_2) = 0$, ..., da cui:
		
		\[ M = \Matrix{ 1 & 0 & 2 \\ 0 & 4 & 2 \\ 2 & 2 & 5}. \]
		
		Si consideri $A_1$: $A_1$ non è isotropo. Si ricerca allora
		$A_1^\perp$. $0 = \varphi(A, B) = (1 2) A_1 B \Vector{1 \\ 2} =
		(1 2) \Matrix{ a & c \\ 0 & 0} \Vector{1 \\ 2} = a + 2c \implies
		a = -2c$, ossia $A_1^\perp = \Span(\Matrix{-2 & 1 \\ 1 & 0},
		\Matrix{0 & 0 \\ 0 & 1})$. \\
		
		Anche $A_2$ non è isotropo, quindi si considera $A_2^\perp$
	\end{exercise}
\end{document}
