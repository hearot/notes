\documentclass[11pt]{article}
\usepackage{personal_commands}
\usepackage[italian]{babel}

\title{\textbf{Note del corso di Geometria 1}}
\author{Gabriel Antonio Videtta}
\date{\today}

\begin{document}
	
	\maketitle
	
	\begin{center}
		\Large \textbf{Esercitazione: forma canonica di Jordan reale}
	\end{center}

	\wip

	\begin{exercise}
		Sia $M \in M(n, \RR)$ tale che $\exists a_1$, ..., $a_k \in \RR$ distinti
		tale che:
		
		\[ (M^2 + a_1^2 I) \cdots (M^2 + a_k^2 I) = 0. \]
		
		Dimostrare allora che esistono $S$, $A \in M(n, \RR)$ tale che
		$M = SA$ con $S$ simmetrica e $A$ antisimmetrica.
	\end{exercise}

	\begin{solution}
		Per ipotesi, $p(x) = (x^2+a_1^2) \cdots (x^2 + a_k^2) \in \Ker \sigma_M$.
		Dal momento che $p(x)$ si scompone in fattori lineari distinti in
		$\CC$, $p(x)$ è anche il polinomio minimo di $M$. Si deduce
		allora che $M$ è diagonalizzabile, e che i suoi autovalori sono
		esattamente $\pm a_1 i$, ..., $\pm a_k i$. Allora la forma
		canonica di Jordan reale di $M$ è:
		
		\[ J = \Matrix{} \]
	\end{solution}
\end{document}
