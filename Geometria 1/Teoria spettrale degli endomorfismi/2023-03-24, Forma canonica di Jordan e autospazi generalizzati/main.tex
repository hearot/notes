\documentclass[11pt]{article}
\usepackage{personal_commands}
\usepackage[italian]{babel}

\title{\textbf{Note del corso di Geometria 1}}
\author{Gabriel Antonio Videtta}
\date{24 marzo 2023}

\begin{document}
	
	\maketitle
	
	\begin{center}
		\Large \textbf{Autospazi generalizzati e decomposizione di Fitting per la forma canonica di Jordan}
	\end{center}

	\begin{note}
		Nel corso del documento, per $f$ si intenderà un generico endomorfismo di $\End(V)$, e per $V$
		verrà inteso uno spazio vettoriale di dimensione finita $n$ su un campo $\KK$ algebricamente
		chiuso, qualora non specificato diversamente.
	\end{note}
	
	Sia $f \in \End(V)$. Si osservino allora le seguenti catene ascendenti:
	\begin{gather}
		\{\vec0\} \subsetneq \Ker f \subsetneq \Ker f^2 \subsetneq \cdots \subsetneq \Ker f^{k-1} \subsetneq \Ker f^k = \Ker f^{k+1} = \cdots, \\
		\{\vec0\} \subsetneq \Im f \subsetneq \Im f^2 \subsetneq \cdots \subsetneq \Im f^{k-1} \subsetneq \Im f^k = \Im f^{k+1} = \cdots,
	\end{gather}

	Sia la $(1)$ che la $(2)$ devono stabilizzarsi allo stesso $k \in \NN$, per la cosiddetta decomposizione di Fitting.
	Sempre per tale decomposizione vale in particolare che:
	
	\[ V = \Ker f^k \oplus \Im f^k. \]
	
	\begin{remark} Si possono fare alcune osservazioni riguardo la decomposizione di Fitting. \\

	\li Sia $\Ker f^k$ che $\Im f^k$ sono $f$-invarianti: $\vec v \in \Ker f^k \implies f^k(f(\vec v)) = f(f^k(\vec v)) = \vec0 \implies f(\vec v) \in \Ker f^k$ e $\vec v \in \Im f^k \implies \vec v = f^k(\vec w)$, $f(\vec v) = f(f^k(\vec w)) = f^k(f(\vec w)) \in \Im f^k$. \\ 
	\li $\restr{f}{\Ker f^k}$ è nilpotente: $(\restr{f}{\Ker f^k})^k = \restr{f^k}{\Ker f^k} = 0$. \\
	\li $\restr{f}{\Im f^k}$ è invertibile: $\Ker \restr{f}{\Im f^k} = \Ker f \cap \Im f^k \subseteq \Ker f^k \cap \Im f^k = \{\vec 0\}$, e quindi $\restr{f}{\Im f^k}$ è iniettiva; quindi $\restr{f}{\Im f^k}$ è anche invertibile, essendo un endomorfismo. \\
	\li Poiché $\restr{f}{\Ker f^k}$ è nilpotente, $p_{\restr{f}{\Ker f^k}}(\lambda) = \lambda^d$, dove
	$d = \dim \Ker f^k$. Inoltre
	$\varphi_{\restr{f}{\Ker f^k}}(\lambda) = \lambda^k$: se infatti $\varphi_{\restr{f}{\Ker f^k}}(\lambda) = \lambda^t$
	con $t < k$, varrebbe sicuramente che ${\restr{f}{\Ker f^k}}^{k-1} = \restr{f^{k-1}}{\Ker f^k} = 0$, ossia che
	$\Ker f^k \subseteq \Ker f^{k-1}$, violando la minimalità di $k$, \Lightning. \\
	\li Dal momento che vale la decomposizione di Fitting e che $\varphi_{\restr{f}{\Ker f^k}}$ e $\varphi_{\restr{f}{\Im f^k}}$ sono coprimi tra loro (il primo è diviso solo da $t$, mentre il secondo non è diviso da $t$), $\varphi_f = \mcm(\varphi_{\restr{f}{\Ker f^k}}, \varphi_{\restr{f}{\Im f^k}}) = \varphi_{\restr{f}{\Ker f^k}} \varphi_{\restr{f}{\Im f^k}}$. Si conclude quindi che $k = \mu'_a(0)$ rispetto a $\varphi_f$, ossia la molteplicità algebrica di $0$ in
	tale polinomio. Analogamente si osserva che $t = \mu_a(0)$ rispetto a $p_f$, ossia la molteplicità algebrica
	dell'autovalore $0$ in $f$, e quindi che $\mu_a(0) \geq k$, \\
	\li Considerando l'endomorfismo $g = f - \lambda \Id$, si osservano facilmente alcune analogie tra le proprietà
	determinanti di $g$ e di $f$: $p_g(t) = \det(f - \lambda \Id - t \Id) = \det(f - (\lambda + t) \Id) = p_f(\lambda + t)
	\implies \mu_{a, g}(0) = \mu_{a, f}(\lambda)$. Si possono dunque riscrive i precedenti risultati in termini delle
	molteplicità di un generico autovalore di $f$ considerando la molteplicità di $0$ in $g$.
	\end{remark}
	
	Reiterando la decomposizione di Fitting (o applicando il teorema di decomposizione primaria), si ottiene
	infine la seguente decomposizione di $V$:
	
	\[ V = \Ker (f - \lambda_1 \Id)^{\mu_a(\lambda_1)} \oplus \cdots \oplus \Ker (f - \lambda_m \Id)^{\mu_a(\lambda_m)}, \]
	
	dove $m$ è il numero di autovalori di $V$. Si può riscrivere questa identità ponendo $n_i := \mu'_a(\lambda_i)$ in
	$\varphi_f$:
	
	\[ V = \Ker (f - \lambda_1 \Id)^{n_1} \oplus \cdots \oplus \Ker (f - \lambda_m \Id)^{n_m}. \]
	
	\begin{definition}
		Si definisce \textbf{autospazio generalizzato} relativo all'autovalore $\lambda_i$ di $f$, lo spazio:
		
		\[ \widetilde{V_{\lambda_i}} = \Ker (f - \lambda_i \Id)^{\mu_{a, f}(\lambda_i)} = \Ker (f - \lambda_m \Id)^{n_m}. \]
	\end{definition}
	
	\begin{remark} Riguardo alla decomposizione primaria di $V$ e agli autospazio generalizzati di $f$ si possono fare alcune osservazioni aggiuntive. \\
	
		\li Si può riscrive la decomposizione primaria di $V$ in termini degli autospazi generalizzati di $f$ come $V = \oplus_{i=1}^m \widetilde{V_{\lambda_i}}$. \\
		\li Vale in particolare che $\widetilde{V_{\lambda_i}} = \{ \v \in V \mid \exists k \in \NN \mid (f-\lambda_i \Id)^k(\v) = \vec{0} \} = \bigcup_{k=0}^{\infty} \Ker (f - \lambda_i \Id)^k$, tenendo in conto la decomposizione
		di Fitting e la minimalità di $n_i$. \\
		\li Considerando la traslazione vista nell'ultima osservazione, si deduce che $\Ker(f - \lambda_i \Id)^{n_i}$ ammette
		come unico autovalore $\lambda_i$ (separazione degli autovalori). \\
		\li Poiché $f$ è diagonalizzabile se e solo se $V = \bigoplus_{i=1}^m \Ker(f - \lambda_i \Id)$, si può dedurre
		un altro criterio per la diagonalizzabilità, ossia $f$ diagonalizzabile $\impliedby n_i = 1$ $\forall i \leq m$. \\
		\li Del precedente criterio vale anche il viceversa: se $f$ è diagonalizzabile e $\lambda_1$, ..., $\lambda_k$ sono
		i suoi autovalori, $V$ ammette una base di autovettori; dati allora gli indici $i_p$ che associano ogni vettore
		$\vv p$ all'indice del suo rispettivo autovalore, allora sia
		$\vv 1 ^{(\lambda_{i_1})}$, ..., $\vv n ^{(\lambda_{i_n})}$ una base di $V$. Poiché $q(t) = \prod_{i=1}^k (t - \lambda_i)$ è tale che $q(f)$ si annulla in ogni vettore della base e ogni suo fattore lineare è composto da un autovalore di $f$ ed è distinto, deve valere che $\varphi_f = q$.
	\end{remark}

	\begin{exercise}
		Si calcoli il polinomio minimo di $A = \Matrix{0 & -2 & 0 & -2 & 1 \\ 1 & 1 & 0 & 2 & 1 \\ 0 & 0 & 1 & 0 & 0 \\ 0 & 0 & 0 & -1 & 0 \\ 1 & 2 & 0 & 2 & 0}$.
	\end{exercise}

	\begin{solution}
		Innanzitutto, si calcola il polinomio caratteristico di $A$, ossia $p_A(t) = (1-t)^3 (1+t)^2$, da cui si ricava
		che gli autovalori di $A$ sono $1$ e $-1$, con $\mu_a(1) = 3$ e $\mu_a(-1) = 2$. Si può dunque decomporre $V$
		come:
		
		\[ V = \Ker (A - I)^3 \oplus \Ker (A + I)^2, \]
		
		\vskip 0.05in
		
		e $\varphi_A$ sarà della forma $\varphi_A(t) = (t-1)^{n_1} (t+1)^{n_2}$ con $n_1 \leq 3$ e $n_2 \leq 2$.
		\begin{enumerate}[(i)]
			\item $\rg(A - I) = 3 \implies \dim \Ker (A - I) = 2 < 3 = \mu_a(-1)$. Si controlli adesso il rango di
			$(A-I)^2$: $\rg(A - I)^2 = 2 \implies \dim \Ker (A - I)^2 = 3 = \mu_a(1)$, da cui $n_1 = 2$. 
			\item $\rg(A + I) = 3 \implies \dim \Ker (A + I) = 2$. Allora, poiché $\dim \Ker (A + I) = 2 = \mu_a(-1)$,
			si conclude che $n_2 = 1$.
		\end{enumerate}
	
		Quindi $\varphi_A(t) = (t-1)^2(t+1)$.
	\end{solution}
	
	\begin{exercise}
		Sia $A \in M(n, \CC)$ invertibile. Dimostrare allora che se $A^3$ è diagonalizzabile, anche $A$ lo è. 
	\end{exercise}

	\begin{solution}
		Se $A^3$ è diagonalizzabile, per la precedente osservazione, $\varphi_{A^3}(t) = \prod_{i=1}^m (t - \lambda_i)$,
		dove $m$ è il numero di autovalori distinti di $A^3$. Allora, detto $p(t) = \prod_{i=1}^m (t^3 - \lambda_i)$, vale che
		$p(A) = 0$, ossia che $\varphi_A \mid p$. Dal momento che $A$ è invertibile, anche $A^3$ lo è, e quindi
		$\lambda_i \neq 0$ $\forall i \leq m$. Poiché $p$ è allora fattorizzato in soli termini lineari distinti,
		anche $\varphi_A$ deve esserlo, e quindi $A$ deve essere diagonalizzabile. \\
	\end{solution}

	\vskip 0.1in

	Nello studio della forma canonica di Jordan è rilevante costruire una base a bandiera tale per cui la matrice
	associata in tale base sia una matrice a blocchi diagonale formata da blocchi di Jordan. Si consideri
	allora $g = f - \lambda \Id$, e sia $k$ la molteplicità algebrica di $\lambda$ nel polinomio minimo
	di $f$ (i.e.~il $k$ minimo già visto precedentemente nella decomposizione di Fitting di $g$). \\
	
	Si possono allora definire dei sottospazi $U_i$ secondo le seguenti decomposizioni:
	
	\begin{align*}
		\Ker g^k &= \Ker g^{k-1} \oplus U_1, \\
		\Ker g^{k-1} &= \Ker g^{k-2} \oplus g(U_1) \oplus U_2, \\
		\vdots & \qquad \vdots \qquad \vdots \qquad \vdots \qquad \vdots \\
		\Ker g &= \underbrace{\Ker g^0}_{= \{ \vec 0\}} \oplus \, g^{k-1}(U_1) \oplus \cdots \oplus U_{k}. 
	\end{align*}

	Si noti che $g$ mantiene la dimensione di $U_i$ ad ogni passo fino a $k-i$ composizioni di $g$ (infatti
	$\Ker g^{k-i} \cap U_i \subseteq \Ker g^{k-1} \cap U_i = \{\vec 0\}$, per costruzione dei sottospazi
	supplementari $U_i$). In particolare, $\dim U_i = m_i$ rappresenta il numero di blocchi di Jordan
	relativi a $\lambda$ di taglia $k-i+1$, e quindi valgono le seguenti identità:
	
	\begin{align*}
		\dim \Ker g^k &= \dim \Ker g^{k-1} + m_1 = \mu_a(\lambda), \\
		\dim \Ker g^{k-1} &= \dim \Ker g^{k-2} + m_1 + m_2, \\
		\vdots & \qquad \vdots \qquad \vdots \qquad \vdots \qquad \vdots \\
		\dim \Ker g &= m_1 + m_2 + \ldots + m_k = \mu_g(\lambda).
	\end{align*}

	%TODO: aggiungere osservazioni su come trovare una base di Ker g^j seguendo i blocchi di Jordan.

\end{document}
