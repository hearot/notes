\documentclass[11pt]{article}
\usepackage{personal_commands}
\usepackage[italian]{babel}

\title{\textbf{Note del corso di Geometria 1}}
\author{Gabriel Antonio Videtta}
\date{\today}

\begin{document}
	
	\maketitle
	
	\begin{center}
		\Large \textbf{Titolo della lezione}
	\end{center}

	\begin{note}
		Nel corso del documento, per $V$ si intenderà uno spazio vettoriale di dimensione
		finita $n$ e per $\varphi$ un suo prodotto, hermitiano o scalare
		dipendentemente dal contesto.
	\end{note}

	\begin{definition} [azione di un gruppo]
		Sia $G$ un gruppo e sia $X$ un insieme. Un'\textbf{azione} di $G$
		su $X$ (a sinistra) è un'applicazione $G \times V \to X$ tale
		che $(g, x) \mapsto g.x$ e che:
		
		\begin{enumerate}[(i)]
			\item $e.x = x$ $\forall x \in X$,
			\item $g.(h.x) = (gh).x$ $\forall x \in X$, $\forall g$, $h \in G$.
		\end{enumerate}
	\end{definition}

	Si può dunque definire un'applicazione $f_g$, che, dato $g \in X$,
	è tale che $f_g(x) = g.x$ $\forall x \in X$. Tale applicazione è
	bigettiva, dacché $f_{g\inv}$ è una sua inversa, sia destra che sinistra.
	La definizione equivale a dare un omomorfismo da $G$ a $S_X$ associando
	a $g$ l'applicazione $f_g$, dove $S_X$ è il gruppo delle bigezioni
	di $X$ con la composizione. \\
	
	L'azione di $G$ si dice \textit{fedele} se $g \mapsto f_g$ è iniettivo
	(ossia se $f_g(x) = x \forall x \in X \implies g=e$).
	
	\begin{enumerate}[(i)]
		\item Per ogni insieme $X$, $G = S_X$ agisce su $X$ in modo tale
		che $g.x = g(x)$ $\forall x \in X$,
		
		\item $\forall$ gruppo $G$, $G$ agisce su $X = G$ tramite
		$g.g' = gg'$,
		
		\item Si può chiaramente definire un'azione destra in modo
		analogo, con la notazione $(g, x) \mapsto x.g$.
	\end{enumerate}

	Se $X$ subisce un'azione di $G$, si dice che $X$ è un $G$-insieme.
	Si introduce la relazione di equivalenza $x \sim_G y \defiff \exists g \in G \mid g.x = y$. Le classi di equivalenza si chiamano \textbf{orbite}
	di $G$ (i.e.~$O_X = \{ g.x \mid g \in G \}$). \\
	
	\begin{example}
		\begin{enumerate}[(i)]
			\item Se $G = \GL(n, \KK)$, $G$ opera su $M(n, \KK)$ tramite
			la similitudine. Le orbite sono le classi di similitudine
			della matrici.
			
			\item Se $G = \GL(n, \KK)$, $G$ opera su $\Sym(n, \KK)$
			tramite la congruenza. Le orbite sono le classi di congruenza
			delle matrici simmetriche. Analogamente si può fare per la
			matrici hermitiane.
			
			\item Se $G = O_n$, esso opera su $\RR^n$ tramite la
			moltiplicazione. Le orbite sono le sfere di raggio $\norm x$.
		\end{enumerate}
	\end{example}

	\begin{definition}
		Lo \textbf{stabilizzatore} di un punto $x \in X$ è
		$\Stab_G(X) = \{g \in G \mid g.x = x \}$, sottogruppo
		di $G$.
	\end{definition}

	\begin{example}
		Sia $H \subseteq G$ e sia $X = G/H$. $X$ è un $G$-insieme
		tramite l'azione $g'.(gH) = g'gH$. Vale in particolare
		che $\Stab_G(eH) = H$.
	\end{example}

	\begin{proposition}
		Sia $X$ un $G$-insieme. Sia $x \in X$. $H = \Stab_G(x)$ e sia
		$O_x$ l'orbita di $x$. Allora esiste un'applicazione bigettiva
		naturale $G/H \to O_x$.
	\end{proposition}

	\begin{proof}
		Sia $\varphi$ tale che $\varphi(gH) = g.x$. Si mostra che
		$\varphi$ è ben definita: $g' = gh$, $\varphi(g'H) = (gh).x =
		g.(h.x) = g.x$. Chiaramente $\varphi$ è anche surgettiva.
		Inoltre, $g.x = g'.x \implies x = (g\inv g').x \implies g\inv g' = h \in H \implies gH = g'H$, e pertanto $\varphi$ è iniettiva.
		Allora $\varphi$ è bigettiva.
	\end{proof}

	\begin{definition}
		Si dice che $G$ opera \textit{liberamente} su $X$ se
		$\forall x \in X$, l'applicazione $G \to O_x$ tale che
		$g \mapsto g.x$, ossia se $\Stab_G(x) = \{e\}$:
	\end{definition}
	
	\begin{definition}
		$G$ opera \textit{transitivamente} su $X$ se $x \sim_G y$ $\forall x$, $y \in X$, cioè se c'è un'unica orbita, che coincide con $X$. In
		tal caso si dice che $X$ è \textbf{omogeneo} per l'azione di $G$.
	\end{definition}

	\begin{example}
		\begin{enumerate}[(i)]
			\item $O_n$ opera su $S^{n-1} \subseteq \RR^n$ transitivamente.
			%TODO: aggiunge che lo stabilizzatore è isomorfo alle ortogonali
			%TODO: di dimensioni n-1
			
			\item $\Gr_k(\RR^n) = \{ W \subseteq \RR^n \mid \dim W = k \}$ (Grassmanniana). $O_n$ opera transitivamente su $\Gr_K(\RR^n)$.
		\end{enumerate}
	\end{example}

	\begin{definition}
		$G$ opera in maniera \textit{semplicemente transitiva} su $X$
		se $\exists x \in X$ tale che $g \mapsto g.x$ è una bigezione,
		ossia se $G$ opera transitivamente e liberamente.
	\end{definition}

	\begin{definition}
		Un insieme $X$ con un'azione semplicemente transitiva di $G$ è
		detto un $G$-insieme omogeneo \textit{principale}.
	\end{definition}

	\begin{example}
		\begin{enumerate}[(i)]
			\item $X = G$. L'azione naturale di $G$ su $X$ per moltiplicazione
			è semplicemente transitivo (per $g$, $g' \in G$, esiste un
			unico $h \in G$ tale che $g = h.g' = hg'$). Quindi $X$
			è $G$-omogeneo principale.
			
			\item Se $X$ è $G$-omogeneo principale, l'azione è fedele.
			
			\item Se $X$ è omogeneo per un gruppo $G$ commutativo, allora
			$G$ agisce fedelmente su $X$ $\implies$ $X$ è un $G$-insieme
			omogeneo principale.
		\end{enumerate}
	\end{example}

	\begin{definition} [spazio affine]
		Sia $V$ uno spazio vettoriale su un campo $\KK$ qualsiasi.
		Allora uno spazio affine $E$ associato a $V$ è un qualunque
		$V$-insieme omogeneo principale.
	\end{definition}

	Pertanto, $\forall P, Q \in E$, esiste un unico vettore $\v \in V$
	tale che $Q = \v . P $, denotato come $Q = P + \v = \v + P$. Si
	osserva che $\v + (\w + P) = (\v + \w) + P$. Essendo $\v$ unico,
	si scrive $\v = Q - P = \vvec{PQ}$.
	
	%TODO: aggiunge applicazione bigettiva
	
	Fissato $O \in E$, l'applicazione $\v \mapsto \v + O$, $V \to E$
	è una bigezione.
	
	\begin{remark}\nl
		\li $P-P = \vec 0 \in V$, $P-Q = -(Q-P)$, $(P_3 - P_2) + (P_2 - P_1) = P_3 - P_1$. \\
		
		\li $O \in E$ l'applicazione $P \mapsto P-O$ è una bigezione di $E$
		su $V$.
	\end{remark}

	Siano $P_1$, ..., $P_n \in E$. $\forall \lambda_1$, ..., $\lambda_k \in \KK$. $\forall O \in E$ possiamo individuare il punto $P = O + \sum_{i=1}^n \lambda_i (P_i - O)$.
	
	$P = P' = \iff O + \sum_{i=1}^n \lambda_i (P_i - O) = O' + \sum_{i=1}^n \lambda_i (P_i - O') \iff O + \sum_{i=1}^n \lambda_i (O' - O) = O' \iff
	(\sum \lambda_i) (O' - O) = O' - O \iff \sum \lambda_i = 1$.
	
	\begin{definition}
		Un punto $P \in E$ è \textbf{combinazione affine} dei punti
		$P_1$, ..., $P_k$ se $P = O + \sum \lambda_i (P_i - O)$ se
		$\sum \lambda_i = 1$. Si scriverà, in particolare, che
		$P = \sum \lambda_i P_i$.
	\end{definition}

	Si chiama retta affine l'insieme dei punti che sono combinazione affine di
	due punti. Analogamente si fa per un piano e uno spazio.
	
	\begin{definition}
		Un sottoinsieme $D \subseteq E$ si dirà \textbf{sottospazio affine}
		se è chiuso per combinazioni affini (finite).
	\end{definition}

	\begin{definition}
		Il sottospazio affine $D \subseteq E$ generato da un sottoinsieme $S \subseteq E$ è l'insieme delle combinazioni affini (finite) di punti
		di $S$, detto $D = \Aff(S)$. %TODO: mostrare che è chiuso per combinazioni affini.
	\end{definition}
\end{document}
