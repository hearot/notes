\chapter{Introduzione al prodotto scalare}

\begin{note}
	Nel corso del documento, per $V$, qualora non specificato, si intenderà uno spazio vettoriale di dimensione
	finita $n$.
\end{note}

\section{Prime definizioni}

\subsection{Prodotto scalare e vettori ortogonali rispetto a \texorpdfstring{$\varphi$}{φ}}

\begin{definition} [prodotto scalare]
	Un \textbf{prodotto scalare} su $V$ è una forma bilineare simmetrica $\varphi$ con argomenti in $V$.
\end{definition}

\begin{example}
	Sia $\varphi : M(n, \KK) \times M(n, \KK) \to \KK$ tale che $\varphi(A, B) = \tr(AB)$. \\
	
	\li $\varphi(A + A', B) = \tr((A + A')B) = \tr(AB + A'B) = \tr(AB) + \tr(A'B) = \varphi(A, B) + \varphi(A', B)$ (linearità
	nel primo argomento), \\
	\li $\varphi(\alpha A, B) = \tr(\alpha A B) = \alpha \tr(AB) = \alpha \varphi(A, B)$ (omogeneità nel primo argomento), \\
	\li $\varphi(A, B) = \tr(AB) = \tr(BA) = \varphi(B, A)$ (simmetria), \\
	\li poiché $\varphi$ è simmetrica, $\varphi$ è lineare e omogenea anche nel secondo argomento, e quindi è una
	forma bilineare simmetrica, ossia un prodotto scalare su $M(n, \KK)$.
\end{example}

\begin{definition} [vettori ortogonali]
	Due vettori $\v$, $\w \in V$ si dicono \textbf{ortogonali} rispetto al prodotto scalare $\varphi$, ossia $\v \perp \w$, se $\varphi(\v, \w) = 0$.
\end{definition}

\begin{definition}
	Si definisce prodotto scalare \textit{canonico} di $\KK^n$ la forma bilineare simmetrica $\varphi$ con
	argomenti in $\KK^n$ tale che:
	
	\[ \varphi((x_1, ..., x_n), (y_1, ..., y_n)) = \sum_{i=1}^n x_i y_i. \]
\end{definition}

\begin{remark}
	Si può facilmente osservare che il prodotto scalare canonico di $\KK^n$ è effettivamente un prodotto
	scalare. \\
	
	\li $\varphi((x_1, ..., x_n) + (x_1', ..., x_n'), (y_1, ..., y_n)) = \sum_{i=1}^n (x_i + x_i') y_i =
	\sum_{i=1}^n \left[x_iy_i + x_i' y_i\right] = \sum_{i=1}^n x_i y_i + \sum_{i=1}^n x_i' y_i =
	\varphi((x_1, ..., x_n), (y_1, ..., y_n)) + \varphi((x_1', ..., x_n'), (y_1, ..., y_n))$ (linearità nel
	primo argomento), \\
	\li $\varphi(\alpha(x_1, ..., x_n), (y_1, ..., y_n))$ $= \sum_{i=1}^n \alpha x_i y_i = \alpha \sum_{i=1}^n x_i y_i$ $=
	\alpha \varphi((x_1, ..., x_n), (y_1, ..., y_n))$ (omogeneità nel primo argomento), \\
	\li $\varphi((x_1, ..., x_n), (y_1, ..., y_n)) = \sum_{i=1}^n x_i y_i = \sum_{i=1}^n y_i x_i = \varphi((y_1, ..., y_n), (x_1, ..., x_n))$ (simmetria), \\
	\li poiché $\varphi$ è simmetrica, $\varphi$ è lineare e omogenea anche nel secondo argomento, e quindi è una
	forma bilineare simmetrica, ossia un prodotto scalare su $\KK^n$.
\end{remark}

\begin{example}
	Altri esempi di prodotto scalare sono i seguenti: \\
	
	\li $\varphi(A, B) = \tr(A^\top B)$ per $M(n, \KK)$, \\
	\li $\varphi(p(x), q(x)) = p(a) q(a)$ per $\KK[x]$, con $a \in \KK$, \\
	\li $\varphi(p(x), q(x)) = \sum_{i=1}^n p(x_i) q(x_i)$ per $\KK[x]$, con $x_1$, ..., $x_n$ distinti, \\
	\li $\varphi(p(x), q(x)) = \int_a^b p(x)q(x) dx$ per lo spazio delle funzioni integrabili su $\RR$, con $a$, $b$ in $\RR$, \\
	\li $\varphi(\vec{x}, \vec{y}) = \vec{x}^\top A \vec{y}$ per $\KK^n$, con $A \in M(n, \KK)$ simmetrica.
\end{example}

\subsection{Prodotto definito o semidefinito}

\begin{definition}
	Sia\footnote{In realtà, la definizione è facilmente estendibile a qualsiasi campo, purché esso
		sia ordinato.} $\KK = \RR$. Allora un prodotto scalare $\varphi$ si dice \textbf{definito positivo} ($\varphi > 0$) se $\v \in V$, $\vec{v} \neq \vec{0} \implies
	\varphi(\vec{v}, \vec{v}) > 0$. Analogamente $\varphi$ è \textbf{definito negativo} ($\varphi < 0$) se $\vec{v} \neq \vec 0 \implies \varphi(\v, \v) < 0$. In generale si dice che $\varphi$ è \textbf{definito} se è definito positivo o
	definito negativo. \\
	
	Infine, $\varphi$ è \textbf{semidefinito positivo} ($\varphi \geq 0$) se $\varphi(\v, \v) \geq 0$ $\forall \v \in V$ (o
	\textbf{semidefinito negativo}, e quindi $\varphi \leq 0$, se invece $\varphi(\v, \v) \leq 0$ $\forall \v \in V$). Analogamente ai
	prodotti definiti, si dice che $\varphi$ è \textbf{semidefinito} se è semidefinito positivo o semidefinito
	negativo.
\end{definition}

\begin{example}
	Il prodotto scalare canonico di $\RR^n$ è definito positivo: infatti $\varphi((x_1, ..., x_n), (x_1, ..., x_n)) =
	\sum_{i=1}^n x_i^2  > 0$, se $(x_1, ..., x_n) \neq \vec 0$. \\
	
	Al contrario, il prodotto scalare $\varphi : \RR^2 \to \RR$ tale che $\varphi((x_1, x_2), (y_1, y_2)) = x_1 y_1 - x_2 y_2$ non è definito positivo: $\varphi((x, y), (x, y)) = 0$, $\forall$ $(x, y) \mid x^2 = y^2$, ossia se
	$y = x$ o $y = -x$.
\end{example}

\section{Il radicale di un prodotto scalare}

\subsection{La forma quadratica $q$ associata a \texorpdfstring{$\varphi$}{φ} e vettori (an)isotropi}

\begin{definition}
	Ad un dato prodotto scalare $\varphi$ di $V$ si associa una mappa
	$q : V \to \KK$, detta \textbf{forma quadratica}, tale che $q(\vec{v}) = \varphi(\vec{v}, \vec{v})$.
\end{definition}

\begin{remark}
	Si osserva che $q$ non è lineare in generale: infatti $q(\vec{v} + \vec{w}) \neq q(\vec{v}) + q(\vec{w})$ in
	$\RR^n$.
\end{remark}

\begin{definition}[vettore (an)isotropo]
	Un vettore $\vec{v} \in V$ si dice \textbf{isotropo} rispetto al prodotto scalare $\varphi$ se $q(\vec{v}) =
	\varphi(\vec{v}, \vec{v}) = 0$. Al contrario, $\v$ si dice \textbf{anisotropo} se non è isotropo, ossia
	se $q(\v) \neq 0$.
\end{definition}

\begin{definition}[cono isotropo]
	Si definisce \textbf{cono isotropo} di $V$ rispetto al prodotto scalare $\varphi$ il seguente insieme:
	
	\[ \CI(\varphi) = \{ \v \in V \mid \varphi(\v, \v) = 0 \}, \]
	
	\vskip 0.05in
	
	ossia l'insieme dei vettori isotropi di $V$.
\end{definition}

\begin{example}
	Rispetto al prodotto scalare $\varphi : \RR^3 \to \RR$ tale che $\varphi((x_1, x_2, x_3), (y_1, y_2, y_3)) =
	x_1 y_1 + x_2 y_2 - x_3 y_3$, i vettori isotropi sono i vettori della forma $(x, y, z)$ tali che $x^2 + y^2 = z^2$, e quindi $\CI(\varphi)$ è l'insieme dei
	vettori stanti sul cono di equazione $x^2 + y^2 = z^2$.
\end{example}

\subsection{Matrice associata a \texorpdfstring{$\varphi$}{φ} e relazione di congruenza}

\begin{remark}
	Come già osservato in generale per le applicazioni multilineari, il prodotto scalare è univocamente determinato
	dai valori che assume nelle coppie $\vv{i}, \vv{j}$ estraibili da una base $\basis$. Infatti, se
	$\basis = (\vv1, ..., \vv{k})$, $\vec{v} = \sum_{i=1}^k \alpha_i \vv{i}$ e $\vec{w} = \sum_{i=1}^k \beta_i \vv{i}$,
	allora:
	
	\[ \varphi(\vec{v}, \vec{w}) = \sum_{i=1}^k \sum_{j=1}^k \alpha_i \beta_j \, \varphi(\vv{i}, \vv{j}). \]
\end{remark}

\begin{definition}
	Sia $\varphi$ un prodotto scalare di $V$ e sia $\basis = (\vv1, ..., \vv{n})$ una base ordinata di $V$. Allora si definisce la \textbf{matrice associata}
	a $\varphi$ come la matrice:
	
	\[ M_\basis(\varphi) = (\varphi(\vv{i}, \vv{j}))_{i,\,j = 1\text{---}n} \in M(n, \KK). \] 
\end{definition}

\begin{remark}\nl
	\li $M_\basis(\varphi)$ è simmetrica, infatti $\varphi(\vv{i}, \vv{j}) = \varphi(\vv{j}, \vv{i})$,
	dal momento che il prodotto scalare è simmetrico, \\
	\li $\varphi(\vec{v}, \vec{w}) = [\vec{v}]_\basis^\top M_\basis(\varphi) [\vec{w}]_\basis$.
\end{remark}

\begin{theorem} (di cambiamento di base per matrici di prodotti scalari) Siano $\basis$, $\basis'$ due
	basi ordinate di $V$. Allora, se $\varphi$ è un prodotto scalare di $V$ e $P = M^{\basis'}_{\basis}(\Id_V)$, vale la seguente identità:
	
	\[ \underbrace{M_{\basis'}(\varphi)}_{A'} = P^\top \underbrace{M_{\basis}}_{A} P. \]
\end{theorem}

\begin{proof} Siano $\basis = (\vv{1}, ..., \vv{n})$ e $\basis' = (\vec{w}_1, ..., \vec{w}_n)$. Allora
	$A'_{ij} = \varphi(\vec{w}_i, \vec{w}_j) = [\vec{w}_i]_{\basis}^\top A [\vec{w}_j]_{\basis} =
	(P^i)^\top A P^j = P_i^\top (AP)^j = (P^\top AP)_{ij}$, da cui la tesi.
\end{proof}

\begin{definition}
	Si definisce \textbf{congruenza} la relazione di equivalenza $\cong$ (denotata anche come $\equiv$) definita nel seguente
	modo su $A, B \in M(n, \KK)$:
	
	\[ A \cong B \defiff \exists P \in GL(n, \KK) \mid A = P^\top A P. \]
\end{definition}

\begin{remark}
	Si può facilmente osservare che la congruenza è in effetti una relazione di equivalenza. \\
	
	\li $A = I^\top A I \implies A \cong A$ (riflessione), \\
	\li $A \cong B \implies A = P^\top B P \implies B = (P^\top)\inv A P\inv = (P\inv)^\top A P\inv \implies B \cong A$ (simmetria), \\
	\li $A \cong B$, $B \cong C$ $\implies A = P^\top B P$, $B = Q^\top C Q$, quindi $A = P^\top Q^\top C Q P =
	(QP)^\top C (QP) \implies A \cong C$ (transitività). 
\end{remark}

\begin{remark}
	Si osservano alcune proprietà della congruenza. \\
	
	\li Per il teorema di cambiamento di base del prodotto scalare, due matrici associate a uno stesso
	prodotto scalare sono sempre congruenti (esattamente come due matrici associate a uno stesso
	endomorfismo sono sempre simili). \\
	\li Se $A$ e $B$ sono congruenti, $A = P^\top B P \implies \rg(A) = \rg(P^\top B P) = \rg(BP) = \rg(B)$,
	dal momento che $P$ e $P^\top$ sono invertibili; quindi il rango è un invariante per congruenza. Allora
	si può ben definire il rango $\rg(\varphi)$ di un prodotto scalare come il rango della matrice
	associata di $\varphi$ in una qualsiasi base di $V$. \\
	\li Se $A$ e $B$ sono congruenti, $A = P^\top B P \implies \det(A) = \det(P^\top B P) = \det(P^\top) \det(B) \det(P)=
	\det(P)^2 \det(B)$. Quindi, per $\KK = \RR$, il segno del determinante è un altro invariante per congruenza.
\end{remark}

\subsection{Studio del radicale \texorpdfstring{$V^\perp$}{V⟂} attraverso \texorpdfstring{$M_\basis(\varphi)$}{M\_B(φ)}}

\begin{definition}
	Si definisce il \textbf{radicale} di un prodotto scalare $\varphi$ come lo spazio:
	
	\[ V^\perp = \Rad(\varphi) = \{ \vec{v} \in V \mid \varphi(\vec{v}, \vec{w}) = 0 \, \forall \vec{w} \in V \} \]
	
	\vskip 0.05in
\end{definition}

\begin{remark}
	Il radicale del prodotto scalare canonico su $\RR^n$ ha dimensione nulla, dal momento che $\forall \vec{v} \in \RR^n \setminus \{\vec{0}\}$, $q(\vec{v}) = \varphi(\vec{v}, \vec{v}) > 0 \implies \v \notin V^\perp$. In
	generale ogni prodotto scalare definito positivo (o negativo) è non degenere, dal momento che ogni vettore
	non nullo non è isotropo, e dunque non può appartenere a $V^\perp$.
\end{remark}

\begin{definition}
	Un prodotto scalare si dice \textbf{degenere} se il radicale dello spazio su tale prodotto scalare ha
	dimensione non nulla.
\end{definition}

\begin{remark}
	Sia $\alpha_\varphi : V \to \dual{V}$ la mappa\footnote{In letteratura questa mappa, se invertibile, è nota come \textit{isomorfismo musicale}, ed è in realtà indicata come $\flat$.} tale che
	$\alpha_\varphi(\vec{v}) = p$, dove $p(\vec{w}) = \varphi(\vec{v}, \vec{w})$ $\forall \v$, $\w \in V$. \\
	
	Si osserva che $\alpha_\varphi$ è un'applicazione lineare. Infatti, $\forall \v$, $\w$, $\U \in V$,
	$\alpha_\varphi(\v + \w)(\U) = \varphi(\v + \w, \U) = \varphi(\v, \U) + \varphi(\w, \U) =
	\alpha_\varphi(\v)(\U) + \alpha_\varphi(\w)(\U) \implies \alpha_\varphi(\v + \w) = \alpha_\varphi(\v) + \alpha_\varphi(\w)$. Inoltre $\forall \v$, $\w \in V$, $\lambda \in \KK$, $\alpha_\varphi(\lambda \v)(\w) =
	\varphi(\lambda \v, \w) = \lambda \varphi(\v, \w) = \lambda \alpha_\varphi(\v)(\w) \implies
	\alpha_\varphi(\lambda \v) = \lambda \alpha_\varphi(\v)$.
	
	Si osserva inoltre che $\Ker \alpha_\varphi$ raccoglie tutti
	i vettori $\v \in V$ tali che $\varphi(\v, \w) = 0$ $\forall \w \in W$, ossia esattamente i vettori di $V^\perp$, per cui si conclude che $V^\perp = \Ker \alpha_\varphi$ (per cui $V^\perp$ è effettivamente uno
	spazio vettoriale). Se $V$ ha dimensione finita, $\dim V = \dim \dual{V}$,
	e si può allora concludere che $\dim V^\perp > 0 \iff \Ker \alpha_\varphi \neq \{\vec{0}\} \iff \alpha_\varphi$ non è
	invertibile (infatti lo spazio di partenza e di arrivo di $\alpha_\varphi$ hanno la stessa dimensione). In
	particolare, $\alpha_\varphi$ non è invertibile se e solo se $\det(\alpha_\varphi) = 0$. \\
	
	Sia $\basis = (\vv{1}, ..., \vv{n})$ una base ordinata di $V$. Si consideri allora la base ordinata del
	duale costruita su $\basis$, ossia $\dual{\basis} = (\vecdual{v_1}, ..., \vecdual{v_n})$. Allora
	$M_{\basisdual}^\basis(\alpha_\varphi)^i = [\alpha_\varphi(\vv{i})]_{\basisdual} = \Matrix{\varphi(\vec{v_i}, \vec{v_1}) \\ \vdots \\ \varphi(\vec{v_i}, \vec{v_n})} \underbrace{=}_{\varphi \text{ è simmetrica}}
	\Matrix{\varphi(\vec{v_1}, \vec{v_i}) \\ \vdots \\ \varphi(\vec{v_n}, \vec{v_i})} = M_\basis(\varphi)^i$. Quindi
	$M_{\basisdual}^\basis(\alpha_\varphi) = M_\basis(\varphi)$. \\
	
	Si conclude allora che $\varphi$ è degenere se e solo se $\det (M_\basis(\varphi)) = 0$ e che
	$V^\perp \cong \Ker M_\basis(\varphi)$ mediante l'isomorfismo del passaggio alle coordinate.
\end{remark}

\subsection{Condizioni per la (semi)definitezza di un prodotto scalare}

\begin{proposition} Sia $\KK = \RR$. Allora
	$\varphi$ è definito $\iff$ $\CI(\varphi) = \zerovecset$. \label{prop:definitezza_varphi}
\end{proposition}

\begin{proof}
	Si dimostrano le due implicazioni separatamente. \\
	
	\rightproof Se $\varphi$ è definito, allora $\varphi(\v, \v)$ è sicuramente diverso da zero
	se $\v \neq \vec 0$. Pertanto $\CI(\varphi) = \zerovecset$. \\
	
	\leftproof Sia $\varphi$ non definito. Se non esistono $\v \neq \vec 0$, $\w \neq \vec 0 \in V$ tali che
	$q(\v) > 0$ e che $q(\w) < 0$, allora $\varphi$ è necessariamente semidefinito. In tal caso,
	poiché $\varphi$ non è definito, deve anche esistere $\U \in V$, $\U \neq \vec 0 \mid q(\U) = 0 \implies \CI(\varphi) \neq \zerovecset$. \\
	
	Se invece tali $\v$, $\w$ esistono, questi sono anche linearmente indipendenti. Se infatti
	non lo fossero, uno sarebbe il multiplo dell'altro, e quindi le loro due forme quadratiche
	sarebbero concordi di segno, \Lightning. Si consideri allora la combinazione lineare
	$\v + \lambda \w$ al variare di $\lambda \in \RR$, imponendo che essa sia isotropa:
	
	\[ q(\v + \lambda \w) = 0 \iff \lambda^2 q(\w)+ 2 \lambda q(\v, \w) + q(\v) = 0. \]
	
	\vskip 0.05in
	
	Dal momento che $\frac{\Delta}{4} = \overbrace{q(\v, \w)^2}^{\geq 0} - \overbrace{q(\w)q(\v)}^{> 0}$ è
	sicuramente maggiore di zero, tale equazione ammette due soluzioni reali $\lambda_1$, $\lambda_2$.
	In particolare $\lambda_1$ è tale che $\v + \lambda_1 \w \neq \vec 0$, dal momento che $\v$ e $\w$
	sono linearmente indipendenti. Allora $\v + \lambda_1 \w$ è un vettore isotropo non nullo
	di $V$ $\implies \CI(\varphi) \neq \zerovecset$. \\
	
	Si conclude allora, tramite la contronominale, che se $\CI(\varphi) = \zerovecset$, $\varphi$
	è necessariamente definito. 
\end{proof}

\begin{proposition} Sia $\KK = \RR$. Allora
	$\varphi$ è semidefinito $\iff$ $\CI(\varphi) = V^\perp$. \label{prop:semidefinitezza_varphi}
\end{proposition}

\begin{proof}
	Si dimostrano le due implicazioni separatamente. \\
	
	\rightproof Sia $\varphi$ semidefinito. Chiaramente $V^\perp \subseteq \CI(\varphi)$. Si assuma
	che $V^\perp \subsetneq \CI(\varphi)$. Sia allora $\v$ tale che $\v \in \CI(\varphi)$ e che $\v \notin V^\perp$.
	Poiché $\v \notin V^\perp$, esiste un vettore $\w \in V$ tale che $\varphi(\v, \w) \neq 0$. Si osserva
	che $\v$ e $\w$ sono linearmente indipendenti tra loro. Se infatti non lo fossero, esisterebbe $\mu \in \RR$
	tale che $\w = \mu \v \implies \varphi(\v, \w) = \mu \, \varphi(\v, \v) = 0$, \Lightning. \\
	
	Si consideri allora la combinazione lineare $\v + \lambda \w$. Si consideri $\varphi$ semidefinito positivo.
	In tal caso si può imporre che la valutazione di $q$ in $\v + \lambda \w$ sia strettamente negativa:
	
	\[ q(\v + \lambda \w) < 0 \iff \overbrace{q(\v)}^{=0} + \lambda^2 q(\w) + 2 \lambda \, \varphi(\v, \w) < 0. \]
	
	\vskip 0.05in
	
	In particolare, dal momento che $\frac{\Delta}{4} = \varphi(\v, \w)^2 > 0$, tale disequazione ammette
	una soluzione $\lambda_1 \neq 0$. Inoltre $\v + \lambda_1 \w \neq \vec 0$, dal momento che $\v$ e
	$\w$ sono linearmente indipendenti. Allora si è trovato un vettore non nullo per cui la valutazione in esso
	di $q$ è negativa, contraddicendo l'ipotesi di semidefinitezza positiva di $\varphi$, \Lightning. Analogamente
	si dimostra la tesi per $\varphi$ semidefinito negativo. \\
	
	\leftproof Sia $\varphi$ non semidefinito. Allora devono esistere $\v$, $\w \in V$ tali che
	$q(\v) > 0$ e che $q(\w) < 0$. In particolare, $\v$ e $\w$ sono linearmente indipendenti
	tra loro, dal momento che se non lo fossero, uno sarebbe multiplo dell'altro, e le
	valutazioni in essi di $q$ sarebbero concordi di segno, \Lightning. Si consideri allora
	la combinazione lineare $\v + \lambda \w$, imponendo che $q$ si annulli in essa:
	
	\[ q(\v + \lambda \w) = 0 \iff \lambda^2 q(\w)+ 2 \lambda q(\v, \w) + q(\v) = 0. \]
	
	\vskip 0.05in
	
	In particolare, dal momento che $\frac{\Delta}{4} = \varphi(\v, \w)^2 > 0$, tale disequazione ammette
	una soluzione $\lambda_1 \neq 0$. Allora, per tale $\lambda_1$, $\v + \lambda_1 \w \in \CI(\varphi)$.
	Tuttavia $\varphi(\v + \lambda_1 \w, \v - \lambda_1 \w) = q(\v) - \underbrace{\lambda_1^2 q(\w)}_{<0} > 0 \implies
	\v + \lambda_1 \w \notin V^\perp \implies \CI(\varphi) \supsetneq V^\perp$. \\
	
	Si conclude allora, tramite la contronominale, che se $\CI(\varphi) = V^\perp$, $\varphi$
	è necessariamente semidefinito. 
\end{proof}

\section{Formula delle dimensioni e di polarizzazione rispetto a $\varphi$}

\begin{definition}[sottospazio ortogonale a $W$]
	Sia $W \subseteq V$ un sottospazio di $V$. Si identifica allora come \textbf{sottospazio ortogonale a $W$}
	il sottospazio $W^\perp = \{ \v \in V \mid \varphi(\v, \w) \, \forall \w \in W \}$.
\end{definition}

\begin{proposition}[formula delle dimensioni del prodotto scalare]
	Sia $W \subseteq V$ un sottospazio di $V$. Allora vale la seguente identità:
	
	\[ \dim W + \dim W^\perp = \dim V + \dim (W \cap V^\perp). \]
\end{proposition}

\begin{proof}
	Si consideri l'applicazione lineare $a_\varphi$ introdotta precedentemente. Si osserva che $W^\perp = \Ker (i^\top \circ a_\varphi)$, dove $i : W \to V$ è tale che $i(\vec w) = \vec w$. Allora,
	per la formula delle dimensioni, vale la seguente identità: 
	
	\begin{equation}
		\label{eq:dim_formula_dimensioni_1}
		\dim V = \dim W^\perp + \rg (i^\top \circ a_\varphi). 
	\end{equation}
	
	\vskip 0.05in
	
	Sia allora $f = i^\top \circ a_\varphi$.
	Si consideri ora l'applicazione $g = a_\varphi \circ i : W \to \dual V$. Sia ora $\basis_W$ una base di $W$ e
	$\basis_V$ una base di $V$. Allora le matrici associate di $f$ e di $g$ sono le seguenti:
	
	\begin{enumerate}[(i)]
		\item $M_{\dual \basis_W}^{\basis_V}(f) = M_{\dual \basis_W}^{\basis_V}(i^\top \circ a_\varphi) =
		\underbrace{M_{\dual \basis_W}^{\dual \basis_V}(i^\top)}_A \underbrace{M_{\dual \basis_V}^{\basis_V}(a_\varphi)}_B = AB$,
		\item $M_{\dual \basis_V}^{\basis_W}(g) = M_{\dual \basis_V}^{\basis_W}(a_\varphi \circ i) =
		\underbrace{M_{\dual \basis_V}^{\basis_V}(a_\varphi)}_B \underbrace{M_{\basis_V}^{\basis_W}(i)}_{A^\top} = BA^\top \overbrace{=}^{B^\top = B} (AB)^\top$.
	\end{enumerate}
	
	Poiché $\rg(A) = \rg(A^\top)$, si deduce che $\rg(f) = \rg(g) \implies \rg(i^\top \circ a_\varphi) = \rg(a_\varphi \circ i) = \rg(\restr{a_\varphi}{W}) = \dim W - \dim \Ker \restr{a_\varphi}{W}$, ossia che:
	
	\begin{equation}
		\label{eq:dim_formula_dimensioni_2}
		\rg(i^\top \circ a_\varphi) = \dim W - \dim (W \cap \underbrace{\Ker a_\varphi}_{V^\perp}) = \dim W - \dim (W \cap V^\perp).
	\end{equation}
	
	Si conclude allora, sostituendo l'equazione \eqref{eq:dim_formula_dimensioni_2} nell'equazione \eqref{eq:dim_formula_dimensioni_1}, che $\dim V = \dim W^\top + \dim W - \dim (W \cap V^\perp)$, ossia la tesi.
\end{proof}

\begin{remark} Si identifica $\w^\perp$ come il sottospazio di tutti i vettori di $V$ ortogonali a $\w$.
	In particolare, se $W = \Span(\vec w)$ è il sottospazio generato da $\vec w \neq \vec 0$, $\vec w \in V$, allora $W^\perp = \w^\perp$. Inoltre valgono le seguenti equivalenze: $\vec w \notin W^\perp \iff$ $\Rad (\restr{\varphi}{W}) = W \cap W^\perp = \zerovecset$ $\iff \vec w \text{ non è isotropo } \iff$ $V = W \oplus W^\perp$.
\end{remark}

\begin{proposition}[formula di polarizzazione]
	Se $\Char \KK \neq 2$, un prodotto scalare è univocamente determinato dalla sua forma quadratica $q$.
	In particolare vale la seguente identità:
	
	\[ \varphi(\v, \w) = \frac{q(\v + \w) - q(\v) - q(\w)}{2}. \]
	
	\vskip 0.05in
\end{proposition}

\section{Il teorema di Lagrange e basi ortogonali}

\begin{definition}
	Si definisce \textbf{base ortogonale} di $V$ una base $\vv 1$, ..., $\vv n$ tale per cui $\varphi(\vv i, \vv j) = 0
	\impliedby i \neq j$, ossia una base per cui la matrice associata del prodotto scalare è diagonale. 
\end{definition}

\begin{theorem}[di Lagrange]
	Ogni spazio vettoriale $V$ su $\KK$ tale per cui $\Char \KK \neq 2$ ammette una base ortogonale.
\end{theorem}

\begin{proof}
	Si dimostra il teorema per induzione su $n := \dim V$. Per $n \leq 1$, la tesi è triviale (se esiste una base, tale base è
	già ortogonale). Sia
	allora il teorema vero per $i \leq n$. Se $V$ ammette un vettore non isotropo $\vec w$, sia $W = \Span(\vec w)$ e si consideri la decomposizione $V = W \oplus W^\perp$. Poiché $W^\perp$ ha dimensione $n-1$, per ipotesi induttiva
	ammette una base ortogonale. Inoltre, tale base è anche ortogonale a $W$, e quindi l'aggiunta di $\vec w$ a
	questa base ne fa una base ortogonale di $V$. Se invece $V$ non ammette vettori non isotropi, ogni forma quadratica
	è nulla, e quindi il prodotto scalare è nullo per la proposizione precedente. Allora in questo caso
	ogni base è una base ortogonale, completando il passo induttivo, e dunque la dimostrazione.
\end{proof}

\subsection{L'algoritmo di ortogonalizzazione di Gram-Schmidt}

\begin{definition}[coefficiente di Fourier]
	Siano $\v \in V$ e $\w \in V \setminus \CI(\varphi)$. Allora si definisce il \textbf{coefficiente di Fourier}
	di $\v$ rispetto a $\w$ come il rapporto $C(\w, \v) = \frac{\varphi(\v, \w)}{\varphi(\w, \w)}$.
\end{definition}

Se $\CI(\varphi) = \zerovecset$ (e quindi nel caso di $\KK = \RR$, dalla
\textit{Proposizione \ref{prop:definitezza_varphi}}, se $\varphi$ è definito) ed è
data una base $\basis = \{ \vv 1, \ldots, \vv n \}$ per $V$, è possibile
applicare l'\textbf{algoritmo di ortogonalizzazione di Gram-Schmidt} per ottenere
da $\basis$ una nuova base $\basis' = \{ \vv 1', \ldots, \vv n' \}$ con le seguenti proprietà:

\begin{enumerate}[(i)]
	\item $\basis'$ è una base ortogonale,
	\item $\basis'$ mantiene la stessa bandiera di $\basis$ (ossia $\Span(\vv 1, \ldots, \vv i) = \Span(\vv 1', \ldots, \vv i')$ per ogni $1 \leq i \leq n$).
\end{enumerate}

L'algoritmo si applica nel seguente modo: si prenda in considerazione $\vv 1$ e si sottragga ad ogni altro vettore
della base il vettore $C(\vv 1, \vv i) \vv 1 = \frac{\varphi(\vv 1, \vv i)}{\varphi(\vv 1, \vv 1)} \vv 1$,
rendendo ortogonale ogni altro vettore della base con $\vv 1$. Si sta quindi applicando la mappa
$\vv i \mapsto \vv i - \frac{\varphi(\vv 1, \vv i)}{\varphi(\vv 1, \vv 1)} \vv i = \vv i ^{(1)}$.
Si verifica infatti che $\vv 1$ e $\vv i ^{(1)}$ sono ortogonali per $2 \leq i \leq n$:

\[ \varphi(\vv 1, \vv i^{(1)}) = \varphi(\vv 1, \vv i) - \varphi\left(\vv 1, \frac{\varphi(\vv 1, \vv i)}{\varphi(\vv 1, \vv 1)} \vv i\right) = \varphi(\vv 1, \vv i) - \varphi(\vv 1, \vv i) = 0. \]

Poiché $\vv 1$ non è isotropo, si deduce che vale la decomposizione $V = \Span(\vv 1) \oplus \Span(\vv 1)^\perp$.
In particolare $\dim \Span(\vv 1)^\perp = n-1$: essendo allora i vettori $\vv 2 ^{(1)}, \ldots, \vv n ^{(1)}$
linearmente indipendenti e appartenenti a $\Span(\vv 1)^\perp$, ne sono una base. Si conclude quindi
che vale la seguente decomposizione:

\[ V = \Span(\vv 1) \oplus^\perp \Span(\vv 2 ^{(1)}, \ldots, \vv n ^{(1)}). \]

\vskip 0.05in

Si riapplica dunque l'algoritmo di Gram-Schmidt prendendo come spazio vettoriale lo spazio generato dai
vettori a cui si è applicato precedentemente l'algoritmo, ossia $V' = \Span(\vv 2 ^{(1)}, \ldots, \vv n ^{(1)})$,
fino a che non si ottiene $V' = \zerovecset$.
