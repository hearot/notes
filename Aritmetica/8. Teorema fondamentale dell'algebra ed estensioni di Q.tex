\section{Teorema fondamentale dell'Algebra e radici reali in \texorpdfstring{$\QQx$}{Q[x]}}

Si enuncia adesso il \nameref{th:algebra}, senza tuttavia
fornirne una dimostrazione\footnote{Per la dimostrazione si rimanda
    a \cite[pp.~142-143]{di2013algebra}, avvisando della sua
    estrema tecnicità. Una dimostrazione a tema strettamente
    algebrico è dovuta invece al matematico francese Laplace (1749 -- 1827), per la quale
    si rimanda a \cite[pp.~120-122]{Remmert1991}.}.

\begin{theorem}[\textit{Teorema fondamentale dell'Algebra}]
    \label{th:algebra}
    Un polinomio non costante $f(x) \in \CCx$ ammette sempre almeno una radice in
    $\CC$.
\end{theorem}

\begin{corollary}
    Sia $f(x) \in \CCx$ di grado $n\geq1$. Allora $f(x)$ ammette
    esattamente $n$ radici, contate con la giusta molteplicità.
\end{corollary}

\begin{proof}
    Sia $\zeta_1$ una radice complessa di $f(x)$, la cui esistenza
    è garantita dal \nameref{th:algebra}. Si divida $f(x)$ per
    $(x-\zeta_1)$ e se ne prende il quoziente $q_1(x)$, mentre si
    ignori il resto, che
    per la \textit{Proposizione \ref{prop:radice_x_meno_alpha}},
    è nullo. \\

    Si reiteri il procedimento utilizzando $q_1(x)$ al
    posto di $f(x)$ fino a quando il grado del quoziente non è nullo,
    e si chiami infine questo quoziente di grado nullo $\alpha$.
    Infatti, poiché i gradi dei quozienti diminuiscono di $1$ ad
    ogni iterazione, è garantito che l'algoritmo termini esattamente
    dopo $n$ iterazioni. Pertanto, $f(x)$ a priori ha almeno $n$ radici. \\

    In questo modo, numerando le radici, si può scrivere $f(x)$ come:

    \begin{equation}
        \label{eq:fattorizzazione_fx__reali}
        f(x)=\alpha(x-\zeta_1)(x-\zeta_2)\cdots(x-\zeta_n).
    \end{equation}

    \vskip 0.1in

    Dal momento che $x-\zeta_i$ è irriducibile $\forall 1 \leq i \leq n$
    e dacché $\KKx$, in quanto anello euclideo, è un UFD, si dimostra
    che \eqref{eq:fattorizzazione_fx__reali} è l'unica fattorizzazione di
    $f(x)$, a meno di associati. Pertanto $f(x)$ ammette esattamente
    $n$ radici.
\end{proof}