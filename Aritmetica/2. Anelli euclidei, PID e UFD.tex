\section{Anelli euclidei, PID e UFD}

\subsection{Prime proprietà}

Nel corso della storia della matematica, numerosi studiosi hanno tentato
di generalizzare -- o meglio, accomunare a più strutture algebriche -- il
concetto di divisione euclidea che era stato formulato per l'anello
dei numeri interi $\ZZ$ e, successivamente, per l'anello dei polinomi
$\KK[x]$. Lo sforzo di questi studiosi ad oggi è converso in un'unica
definizione, quella di anello euclideo, di seguito presentata.

\begin{definition}
    Un \textbf{anello euclideo} è un dominio d'integrità $D$\footnote{Difatti, nella
        letteratura inglese, si parla di \textit{Euclidean domain} piuttosto che di
        anello.} sul quale è
    definita una funzione $g$ detta \textbf{funzione grado} o \textit{norma}
    soddisfacente le seguenti proprietà:

    \begin{itemize}
        \item $g : D \setminus \{0\} \to \NN$,
        \item $\forall a$, $b \in D \setminus \{0\}$, $g(a) \leq g(ab)$,
        \item $\forall a \in D$, $b \in D \setminus \{0\}$, $\exists q$, $r \in D \mid
                  a=bq+r$ e $r=0 \,\lor\, g(r)<g(q)$.
    \end{itemize}
\end{definition}

Di seguito vengono presentate alcune definizioni, correlate
alle proprietà immediate di un anello euclideo.

\begin{definition}
    Dato un anello euclideo $E$, siano $a \in E$ e $b \in E \setminus \{0\}$. Si dice che
    $b \mid a$, ossia che $b$ \textit{divide} $a$, se $\exists c \in E \mid
        a=bc$.
\end{definition}

\begin{remark*}
    Si osserva che, per ogni anello euclideo $E$, qualsiasi $a \in E$ divide
    $0$. Infatti, $0 = a0$.
\end{remark*}

\begin{proposition}
    Dato un anello euclideo $E$, $a \mid b \,\land\, b \nmid a \implies g(a) < g(b)$.
\end{proposition}

\begin{proof}
    Poiché $b \nmid a$, esistono $q$, $r$ tali che $a = bq + r$, con
    $g(r) < g(b)$. Dal momento però che $a \mid b$, $\exists c \mid
        b = ac$. Pertanto $a = ac + r \implies r = a(1-c)$. Dacché $1-c \neq 0$ --
    altrimenti $r=0$, \Lightning{} --, così come $a \neq 0$, si deduce
    dalle proprietà della funzione grado che $g(a) \leq g(r)$.
    Combinando le due disuguaglianze, si ottiene la
    tesi: $g(a) < g(b)$.
\end{proof}

\begin{proposition}
    \label{prop:g1_minimo}
    $g(1)$ è il minimo di $\Imm g$, ossia il minimo grado assumibile
    da un elemento di un anello euclideo $E$.
\end{proposition}

\begin{proof}
    Sia $a \in E \setminus \{0\}$, allora, per le proprietà della funzione
    grado, $g(1) \leq g(1a) = g(a)$.
\end{proof}

\begin{theorem}
    Sia $a \in E \setminus \{0\}$, allora $a \in E^* \iff g(a) = g(1)$.
\end{theorem}

\begin{proof}
    Dividiamo la dimostrazione in due parti, ognuna corrispondente a una implicazione. \\

    ($\implies$) \;Sia $a \in E^*$, allora $\exists b \in E^*$ tale che $ab=1$. Poiché
    sia $a$ che $b$ sono diversi da $0$, dalle proprietà della funzione grado si
    desume che $g(a) \leq g(ab) = g(1)$. Poiché, dalla \textit{Proposizione \ref{prop:g1_minimo}},
    $g(1)$ è minimo, si conclude che $g(a) = g(1)$. \\

    ($\;\Longleftarrow\;$) \;Sia $a \in E \setminus \{0\}$ con $g(a) = g(1)$. Allora
    esistono $q$, $r$ tali che $1 = aq + r$. Vi sono due possibilità: che $r$ sia $0$, o
    che $g(r) < g(a)$. Tuttavia, poiché $g(a)=g(1)$, dalla \textit{Proposizione \ref{prop:g1_minimo}} si desume che $g(a)$ è minimo, e quindi che
    $r$ è nullo. Si conclude quindi che $aq = 1$, e dunque che $a \in E^*$.
\end{proof}

\subsection{Irriducibili e prime definizioni}

Come accade nell'aritmetica dei numeri interi, anche in un dominio è possibile definire
una nozione di \textit{primo}. In un dominio possono essere tuttavia definiti due tipi di "primi",
gli elementi \textit{irriducibili} e gli elementi \textit{primi}.

\begin{definition}
    In un dominio $A$, si dice che $a \in A \setminus A^*$ è \textbf{irriducibile} se
    $\exists b$, $c \mid a=bc \implies b \in A^*$ o $c \in A^*$.
\end{definition}

\begin{remark*}
    Dalla definizione si escludono gli invertibili di $A$ per permettere
    di definire meglio il concetto di fattorizzazione in seguito. Infatti,
    se li avessimo inclusi, avremmo che ogni dominio sarebbe a fattorizzazione
    non unica, dal momento che $a=bc$ potrebbe essere scritto anche come
    $a=1bc$.
\end{remark*}

\begin{definition}
    Si dice che due elementi non nulli $a$, $b$ appartenenti a un anello euclideo
    $E$ sono \textbf{associati} se $a \mid b$ e $b \mid a$.
\end{definition}

\begin{proposition}
    \label{prop:associati}
    $a$ e $b$ sono associati $\iff \exists c \in E^* \mid a=bc$ e $a$, $b$ entrambi non nulli.
\end{proposition}

\begin{proof} Si dimostrano le due implicazioni separatamente. \\

    ($\implies$) Se $a$ e $b$ sono associati, allora $\exists d$, $e$ tali che $a=bd$ e che $b=ae$. Combinando le due relazioni si ottiene:

    \[ a=aed \implies a(1-ed)=0.\]

    Poiché $a$ è diverso da zero, si ricava che $ed=1$, ossia
    che $d$, $e \in E^*$, e quindi la tesi. \\

    ($\;\Longleftarrow\;$) Se $a$ e $b$ sono entrambi non
    nulli e $\exists c \in E^* \mid a=bc$, $b$ chiaramente
    divide $a$. Inoltre, $a=bc \implies b=ac^{-1}$, e quindi
    anche $a$ divide $b$. Pertanto $a$ e $b$ sono associati.
\end{proof}

\begin{proposition}
    \label{prop:divisione_associati}
    Siano $a$ e $b$ due associati in $E$. Allora $a \mid c \implies b \mid c$.
\end{proposition}

\begin{proof}
    Poiché $a$ e $b$ sono associati, per la \textit{Proposizione \ref{prop:associati}}, $\exists d \in E^*$ tale che
    $a = db$. Dal momento che $a \mid c$, $\exists \alpha \in E$ tale che
    $c = \alpha a$, quindi:

    \[ c = \alpha a = \alpha d b,\]

    da cui la tesi.
\end{proof}

\begin{proposition}
    \label{prop:associati_generatori}
    Siano $a$ e $b$ due associati in $E$. Allora
    $(a)=(b)$.
\end{proposition}

\begin{proof}
    Poiché $a$ e $b$ sono associati, $\exists d \in E^*$ tale che $a = db$. Si dimostra l'uguaglianza dei due insiemi. \\

    Sia $\alpha = ak \in (a)$, allora $\alpha = dbk$ appartiene anche a $(b)$, quindi $(a) \subseteq (b)$. Sia
    invece $\beta = bk \in (b)$, allora $\beta = d^{-1}ak$
    appartiene anche a $(a)$, da cui $(b) \subseteq (a)$.
    Dalla doppia inclusione si verifica la tesi, $(a)=(b)$.
\end{proof}

\begin{definition}
    In un dominio $A$, si dice che $a \in A \setminus A^*$ è \textbf{primo} se
    $a \mid bc \implies a \mid b \,\lor\, a \mid c$.
\end{definition}

\begin{proposition}
    Se $a \in A$ è primo, allora $a$ è anche irriducibile.
\end{proposition}

\begin{proof}
    Si dimostra la tesi contronominalmente. Sia $a$ non irriducibile. Se
    $a \in A^*$, allora $a$ non può essere primo. Altrimenti $a=bc$ con
    $b$, $c \in A \setminus A^*$. \\

    Chiaramente $a \mid bc$, ossia sé stesso. Senza perdità di generalità, se $a \mid b$, dal momento che anche $b \mid a$,
    si dedurrebbe che $a$ e $b$ sono associati secondo la
    \textit{Proposizione \ref{prop:associati}}. Tuttavia questo
    implicherebbe che $c \in A^*$, \Lightning{}.
\end{proof}

\subsection{PID e MCD}

Come accade per $\ZZ$, in ogni anello euclideo è possibile definire il
concetto di \textit{massimo comun divisore}, sebbene con qualche accortezza
in più. Pertanto, ancor prima di definirlo, si enuncia la definizione di
PID e si dimostra un teorema fondamentale degli anelli euclidei, che
si ripresenterà in seguito come ingrediente fondamentale per la fondazione
del concetto di MCD.

\begin{definition}
    Si dice che un dominio è un \textit{principal ideal domain} (\textbf{PID})\footnote{Ossia un \textit{dominio
            a soli ideali principali}, quindi monogenerati, proprio come da definizione.} se ogni suo ideale è monogenerato.
\end{definition}

\begin{theorem}
    Sia $E$ un anello euclideo. Allora $E$ è un PID.
\end{theorem}

\begin{proof}
    Sia $I$ un ideale di $E$. Se $I = (0)$, allora $I$ è già monogenerato.
    Altrimenti si consideri l'insieme $g(I \setminus \{0\})$. Poiché
    $g(I \setminus \{0\}) \subseteq \NN$,
    esso ammette un minimo per il principio del buon ordinamento. \\

    Sia $m \in I$ un valore che assume tale minimo e sia $a \in I$.
    Poiché $E$ è euclideo, $\exists q$, $r \mid a = mq + r$ con $r=0$ o
    $g(r)<g(m)$. Tuttavia, poiché $r = a-mg \in I$ e $g(m)$ è minimo, necessariamente $r=0$ -- altrimenti $r$ sarebbe
    ancor più minimo di $m$, \Lightning{} --,
    quindi $m \mid a$, $\forall a \in I$. Quindi $I \subseteq (m)$. \\

    Dal momento che per le proprietà degli ideali $\forall a \in E$, $ma \in I$,
    si conclude che $(m) \subseteq I$. Quindi $I = (m)$.
\end{proof}

Adesso è possibile definire il concetto di massimo comun divisore, basandoci
sul fatto che ogni anello euclideo è un PID.

\begin{definition}
    Sia $D$ un dominio e siano $a$, $b \in D$. Si definisce
    \textit{massimo comun divisore} (\textbf{MCD}) di $a$ e $b$ un
    generatore dell'ideale $(a,b)$.
\end{definition}

\begin{remark*}
    Questa definizione di MCD è una buona definizione dal momento che sicuramente
    esiste un generatore dell'ideale $(a,b)$, dacché $D$ è un PID.
\end{remark*}

\begin{remark*}
    Non si parla di un unico massimo comun divisore, dal momento che
    potrebbero esservi più generatori dell'ideale $(a,b)$. Segue tuttavia che tutti questi generatori sono in realtà
    associati\footnote{Infatti ogni generatore divide ogni
        altro elemento di un ideale, e così i vari generatori si
        dividono tra di loro. Pertanto sono associati.}.
    Quando si scriverà
    $\MCD(a,b)$ s'intenderà quindi uno qualsiasi di questi associati.
\end{remark*}

\begin{theorem}[\textit{Identità di Bézout}]
    \label{th:bezout}
    Sia $d$ un MCD di $a$ e $b$. Allora
    $\exists \alpha$, $\beta$ tali che $d = \alpha a + \beta b$.
\end{theorem}

\begin{proof}
    Il teorema segue dalla definizione di MCD come generatore
    dell'ideale $(a,b)$. Infatti, poiché $d \in (a,b)$, esistono
    sicuramente, per definizione, $\alpha$ e $\beta$ tali che
    $d = \alpha a + \beta b$.
\end{proof}

\begin{proposition}
    \label{prop:mcd}
    Siano $a$, $b \in D$. Allora vale la seguente equivalenza:

    \[ d = \MCD(a, b) \iff \begin{cases} d \mid a \,\land\, d \mid b \\ \forall c \text{ t.c.\ } c \mid a \,\land\, c \mid b,\;c \mid d\end{cases}\]
\end{proposition}

\begin{proof} Si dimostrano le due implicazioni separatamente. \\

    ($\implies$) Poiché $d$ è generatore dell'ideale $(a, b)$, la prima proprietà segue banalmente. \\

    Inoltre, per l'\nameref{th:bezout}, $\exists \alpha$, $\beta$ tali che
    $d = \alpha a + \beta b$. Allora, se $c \mid a$ e $c \mid b$, sicuramente
    esistono $\gamma$ e $\delta$ tali che $a=\gamma c$ e $b=\delta c$. Pertanto
    si verifica la seconda proprietà, e quindi la tesi:

    \[ d = \alpha a + \beta b = \alpha \gamma c + \beta \delta c = c(\alpha\gamma+\beta\delta). \]

    \vskip 0.1in

    ($\;\Longleftarrow\;$) Sia $m = \MCD(a,b)$. Dal momento che $d$ divide
    sia $a$ che $b$, $d$ deve dividere, per l'implicazione scorsa, anche $m$.
    Per la seconda proprietà, $m$ divide $d$ a sua volta. Allora $d$ è un
    associato di $m$, e quindi, dalla \textit{Proposizione \ref{prop:associati_generatori}}, $(m)=(d)=(a,b)$, da cui $d = \MCD(a,b)$.
\end{proof}

\begin{proposition}
    \label{prop:divisione_gcd}
    Se $a \mid bc$ e $d = \MCD(a, b) \in D^*$, allora $a \mid c$.
\end{proposition}

\begin{proof}
    Per l'\nameref{th:bezout} $\exists \alpha$, $\beta$ tali che
    $\alpha a + \beta b = d$. Allora, poiché $a \mid bc$, $\exists
        \gamma$ tale che $bc=a\gamma$. Si verifica quindi la tesi:

    \[ \alpha a + \beta b = d \implies \alpha ac + \beta bc = dc \implies
        a d^{-1} (\alpha c + \beta \gamma) = c.\]
\end{proof}

\begin{lemma}
    \label{lem:primalità_mcd}
    Se $a$ è un irriducibile di un PID $D$, allora $\forall b \in D$,
    $(a,b)=D \,\lor\, (a,b)=(a)$, o equivalentemente $\MCD(a,b) \in D^*$ o
    $\MCD(a,b) = a$.
\end{lemma}

\begin{proof}
    Dacché $\MCD(a,b) \mid a$, le uniche opzioni, dal momento che $a$ è irriducibile,
    sono che $\MCD(a,b)$ sia un invertibile o che sia un associato
    di $a$ stesso.
\end{proof}

\begin{theorem}
    \label{th:irriducibili_primi}
    Se $a$ è un irriducibile di un PID $D$, allora $a$ è anche un primo.
\end{theorem}

\begin{proof}
    Siano $b$ e $c$ tali che $a \mid bc$. Per il \textit{Lemma \ref{lem:primalità_mcd}},
    $\MCD(a,b)$ può essere solo un associato di $a$ o essere un invertibile. Se è
    un associato di $a$, allora, per la \textit{Proposizione \ref{prop:divisione_associati}}, poiché $\MCD(a,b)$ divide $b$, anche $a$ divide $b$.
    Altrimenti $\MCD(a,b) \in D^*$, e quindi, per la \textit{Proposizione \ref{prop:divisione_gcd}}, $a \mid c$.
\end{proof}

\subsection{L'algoritmo di Euclide}

Per algoritmo di Euclide si intende un algoritmo che è in grado di
produrre in un numero finito di passi un MCD tra due elementi
$a$ e $b$ non entrambi nulli di un anello euclideo\footnote{Si richiede che l'anello sia
    euclideo e non soltanto che sia un PID, dal momento che l'algoritmo
    usufruisce delle proprietà della funzione grado.}. L'algoritmo
classico è di seguito presentato:

\newpage

\begin{algorithm}
    $e \gets \max(a,b)$\;
    $d \gets \min(a,b)$\;
    \BlankLine\BlankLine
    \While{$d>0$}
    {
        $m \gets d$\;
        $d \gets e \bmod d$\;
        $e \gets m$\;
    }
\end{algorithm}

dove $e$ è l'MCD ricercato e l'operazione $\mathrm{mod}$ restituisce un resto della
divisione euclidea\footnote{Ossia $a \bmod b$ restituisce un $r$ tale che $\exists q
        \mid a = bq+r$ con $r=0$ o $g(r)<g(q)$.}.

\begin{lemma}
    \label{lem:euclide_finito}
    L'algoritmo di Euclide termina sempre in un numero finito di passi.
\end{lemma}

\begin{proof}
    Se $d$ è pari a $0$, l'algoritmo termina immediatamente. \\

    Altrimenti si può costruire una sequenza $(g(d_i))_{i\geq1}$ dove $d_i$ è il valore di $d$ all'inizio
    di ogni $i$-esimo ciclo $\textbf{while}$. Ad ogni ciclo vi sono due casi: se $d_i$ si annulla dopo
    l'operazione di $\mathrm{mod}$, il ciclo si conclude al passo successivo, altrimenti,
    poiché $d_i$ è un resto di una divisione euclidea, segue che $g(d_i)<g(d_{i-1})$, dove
    si pone $d_{0}=\min(a, b)$. \\

    Per il principio della discesa infinita, $(g(d_i))_{i\geq1}$ non può essere
    una sequenza infinita, essendo strettamente decrescente. Quindi la sequenza è
    finita, e pertanto il ciclo $\textbf{while}$ s'interrompe dopo un numero finito
    di passi.
\end{proof}

\begin{lemma}
    \label{lem:generatori_euclide}
    Sia $r = a \bmod b$. Allora vale che $(a,b)=(b,r)$.
\end{lemma}

\begin{proof}
    Poiché $r = a \bmod b$, $\exists q$ tale che $a = qb + r$.
    Siano $k_1$ e $k_2$ tali che $(k_1)=(a,b)$ e $(k_2)=(b,r)$. Dal
    momento che $k_1$ divide sia $a$ che $b$, si ha che divide anche
    $r$. Siano $\alpha$, $\beta$ tali che $a = \alpha k_1$ e
    $b = \beta k_1$. Si verifica infatti che:

    \[ r = a - qb = \alpha k_1 - q \beta k_1 = k_1 (\alpha - q \beta). \]

    Poiché $k_1$ divide sia $b$ che $r$, per le proprietà del $\MCD$,
    $k_1$ divide anche $k_2$. Analogamente, $k_2$ divide $k_1$. Pertanto
    $k_1$ e $k_2$ sono associati, e dalla \textit{Proposizione \ref{prop:associati_generatori}} generano quindi lo stesso ideale, da
    cui la tesi.
\end{proof}

\begin{theorem}
    L'algoritmo di Euclide restituisce sempre correttamente un MCD tra due elementi $a$ e $b$ non entrambi nulli in un numero finito di passi.
\end{theorem}

\begin{proof}
    Per il \textit{Lemma \ref{lem:euclide_finito}}, l'algoritmo sicuramente termina.
    Se $d$ è pari a $0$, allora l'algoritmo termina restituendo $e$. Il valore è
    corretto, dal momento che, senza perdità di generalità, se $b$ è nullo, allora
    $\MCD(a, b)=a$: infatti $a$ divide sia sé stesso che $0$, e ogni divisore di $a$ è
    sempre un divisore di $0$. \\

    Se invece $d$ non è pari a $0$, si scelga il $d_n$ tale che $g(d_n)$ sia l'ultimo
    elemento della sequenza $(g(d_i))_{i\geq1}$ definita nel \textit{Lemma \ref{lem:euclide_finito}}. Per il \textit{Lemma \ref{lem:generatori_euclide}},
    si ha la seguente uguaglianza:

    \[ (e_0, d_0) = (d_0, d_1) = \cdots = (d_n, 0) = (d_n). \]

    \vskip 0.1in

    Poiché quindi $d_n$ è generatore di $(e_0, d_0)=(a,b)$, $d_n = \MCD(a,b)$.
\end{proof}

\subsection{UFD e fattorizzazione}

Si enuncia ora la definizione fondamentale di UFD, sulla
quale costruiremo un teorema fondamentale per gli anelli
euclidei.

\begin{definition}
    Si dice che un dominio $D$ è uno \textit{unique factorization domain} (\textbf{UFD})\footnote{Ossia
        un \textit{dominio a fattorizzazione unica}.} se ogni $a \in D$ non nullo e non invertibile può essere scritto
    in forma unica come prodotto di irriducibili, a meno di associati.
\end{definition}

\begin{lemma}
    \label{lem:fattorizzazione}
    Sia $E$ un anello euclideo. Allora ogni elemento $a \in E$ non nullo e
    non invertibile può essere scritto come prodotto di irriducibili.
\end{lemma}

\begin{proof}
    Si definisca $A$ nel seguente modo:

    \[A = \{g(a) \mid a \in E \setminus (E^* \cup \{0\}) \text{ non sia prodotto di irriducibili}\}.\]

    \vskip 0.1in

    Se $A \neq \emptyset$, allora, poiché $A \subseteq \NN$, per il principio
    del buon ordinamento, esiste un $m \in E$ tale che $g(m)$ sia minimo.
    Sicuramente $m$ non è irriducibile -- altrimenti $g(m) \notin A$, \Lightning{} --,
    quindi $m=ab$ con $a$, $b \in E \setminus E^*$. \\

    Poiché $a \mid m$, ma $m \nmid a$ -- altrimenti $a$ e $m$ sarebbero
    associati, e quindi $b$ sarebbe invertibile --, si deduce che $g(a) < g(m)$, e
    quindi che $g(a) \notin A$. Allora $a$ può scriversi come prodotto di irriducibili.
    Analogamente anche $b$ può scriversi come prodotto di irriducibili, e quindi
    $m$, che è il prodotto di $a$ e $b$, è prodotto di irriducibili, \Lightning{}. \\

    Quindi $A = \emptyset$, e ogni $a \in E$ non nullo e non invertibile è prodotto
    di irriducibili.
\end{proof}

\begin{theorem}
    \label{th:euclidei_ufd}
    Sia $E$ un anello euclideo. Allora $E$ è un UFD\footnote{In realtà questo teorema
        è un caso particolare di un teorema più generale: ogni PID è un UFD. Poiché
        la dimostrazione esula dalle intenzioni di queste dispense, si è preferito
        dimostrare il caso più familiare. Per la dimostrazione del teorema più generale si
        rimanda a \cite[pp.~124-126]{di2013algebra}.}.
\end{theorem}

\begin{proof}
    Innanzitutto, per il \textit{Lemma \ref{lem:fattorizzazione}}, ogni
    $a \in E$ non invertibile e non nullo ammette una fattorizzazione. \\

    Sia allora $a \in E$ non invertibile e non nullo. Affinché $E$ sia un UFD,
    deve verificarsi la seguente condizione: se
    $a=p_1p_2 \cdots p_r=q_1q_2 \cdots q_s \in E$, allora
    $r=s$ ed esiste una permutazione $\sigma \in S_r$ tale per cui
    $\sigma$ associ a ogni indice $i$ di un $p_i$ un indice $j$ di
    un $q_j$ in modo tale che $p_i$ e $q_j$ siano associati. \\

    Si procede per induzione. \\

    \,(\textit{passo base}) \,Se $r=1$, allora $a$ è irriducibile. Allora necessariamente
    $s=1$, altrimenti $a$ sarebbe prodotto di irriducibili, e quindi contemporaneamente
    anche non irriducibile. Inoltre esiste la permutazione banale $e \in S_1$ che
    associa $p_1$ a $q_1$. \\

    \,(\textit{passo induttivo}) \,Si assume che valga la tesi se $a$ è
    prodotto di $r-1$ irriducibili.
    Si consideri $p_1$: poiché $p_1$ divide $a$, $p_1$ divide anche
    $q_1q_2 \cdots q_s$. Dal momento che $E$, in quanto
    anello euclideo, è anche un dominio, dal \textit{Teorema \ref{th:irriducibili_primi}}, $p_1$ è anche primo,
    e quindi $p_1 \mid q_1$ o $p_1 \mid q_2 \cdots q_s$. \\

    Se $p_1 \nmid q_1$ si reitera il procedimento su $q_2 \cdots q_s$, trovando in
    un numero finito di passi un $q_j$ tale per cui $p_1 \mid q_j$. Allora si procede
    la dimostrazione scambiando $q_1$ e $q_j$. \\

    Poiché $q_1$ è irriducibile, $p_1$ e $q_1$ sono associati, ossia $q_1 = kp_1$ con
    $k \in E^*$. Allora $p_1 \cdots p_r = q_1 \cdots q_s = kp_1 \cdots q_s$, quindi,
    dal momento che $p_1 \neq 0$ ed $E$ è un dominio:

    \[p_1(p_2 \cdots p_r - kq_2 \cdots q_s)=0 \implies p_2 \cdots p_r = kq_2 \cdots q_s .\]

    Tuttavia il primo membro è un prodotto $r-1$ irriducibili, pertanto $r=s$ ed
    esiste un $\sigma \in S_{r-1}$ che associa ad ogni irriducibile $p_i$ un suo
    associato $q_i$. Allora si estende $\sigma$ a $S_r$ mappando $p_1$ a $q_1$,
    verificando la tesi.
\end{proof}

\subsection{Il teorema cinese del resto}

    Il noto \nameref{th:cinese} è un risultato più generale di quanto
    si sia visto nel contesto dell'aritmetica modulare. Difatti, esso è
    applicabile in forma estesa a tutti gli anelli euclidei, non solo ai
    numeri interi (che comunque rimangono un esempio classico di anello euclideo). \\
    
    \begin{lemma}
        \label{lem:pre_cinese}

        Sia $a$ un elemento riducibile di un anello euclideo $E$ e
        sia $a=bc$, dove $\MCD(b, c) \in E^*$. Allora vale
        il seguente isomorfismo:
        
        \[ A/(a) \cong A/(b) \times A/(c). \]
    \end{lemma}
    
    \begin{proof}
        Si consideri la funzione $\pi$ definita nel seguente
        modo:
        
        \[ \pi : A/(a) \to A/(b) \times A/(c),\,e + (a) \mapsto (e + (b), e + (c)). \]
        
        \vskip 0.1in
        
        Si verifica che $\pi$ è un omomorfismo. Infatti
        $\pi(1 + (a)) = (1 + (b), 1 + (c))$. \\
        
        Siano $e$, $k \in A$. Allora
        $\pi$ soddisfa la linearità:
        
        \begin{multline*}
            \pi\Bigl(\bigl(e + (a)\bigr) + \bigl(k + (a)\bigr)\Bigr) = \pi\bigl(e + k + (a)\bigr) =
            \bigl(e + k + (b), e + k + (c)\bigr) =
            \bigl(e + (b), e + (c)\bigr) + \\ \bigl(k + (b), k + (c)\bigr) = \pi\bigl(e + (a)\bigr) +
            \pi\bigl(k + (a)\bigr).
        \end{multline*}
        
        \vskip 0.1in
        
        e la moltiplicatività:
        
        \begin{multline*}
            \pi\Bigl(\bigl(e + (a)\bigr) \cdot \bigl(k + (a)\bigr)\Bigr) = \pi\bigl(ek + (a)\bigr) =
            \bigl(ek + (b), ek + (c)\bigr) =
            \bigl(e + (b), e + (c)\bigr) \cdot \\ \bigl(k + (b), k + (c)\bigr) = \pi\bigl(e + (a)\bigr) \cdot
            \pi\bigl(k + (a)\bigr).
        \end{multline*}
        
        \vskip 0.1in
        
        Si studia $\Ker \pi$ per dimostrare l'iniettività di $\pi$.
        Si pone dunque $\pi\bigl(e + (a)\bigr) = \bigl(0 + (b), 0 + (c)\bigr)$.
        Questa condizione è equivalente ad asserire che sia $b$ che $c$ dividano
        $e$. \\
        
        Sia allora $k \in E$ tale che $e=bk$. Dal momento che $c$ divide $e$, si
        $e$ divide $bk$. Allora, dacché per ipotesi $\MCD(a, b) \in E^*$, per la
        \propref{prop:divisione_gcd} $c$ divide $k$. Quindi esiste $j \in E$ tale che
        $k=cj$. In particolare, unendo le due condizioni si ottiene
        $e=bk=bcj=aj$. Pertanto $a$ divide $e$, da cui si deduce che $e + (a)$
        è equivalente a $0 + (a)$. Allora, poiché $\Ker \pi = (0)$, $\pi$ è un
        monomorfismo. \\
        
        Si studia invece adesso la surgettività di $\pi$. Siano $\alpha$,
        $\beta \in E$. Si pone dunque $\pi\bigl(e + (a)\bigr) =
        \bigl(\alpha + (b), \beta + (c)\bigr)$. Questa condizione è equivalente
        al seguente sistema:
        
        \[ \begin{cases} e = \alpha + bk, \\ e = \beta + cj, \end{cases} \quad \text{con } k, j \in E. \]
        
        \vskip 0.1in
        
        Unendo le due condizioni si ottiene la seguente equazione:
        
        \[ \alpha + bk = \beta + cj \iff cj - bk = \alpha - \beta.  \]
        
        \vskip 0.1in
        
        Si consideri ora $d = \MCD(b, c)$. Per l'\nameref{th:bezout} esistono
        $x$, $y$ tali che:
        
        \[ cx+by=d, \]
        
        \vskip 0.1in
        
        da cui si ricava che:
        
        \[ (\alpha-\beta)(cx + by) = (\alpha-\beta)d \implies cxd\inv(\alpha-\beta)+byd\inv(\alpha-\beta)=\alpha-\beta, \]
        
        \vskip 0.1in
        
        ponendo allora $j=xd\inv(\alpha-\beta)$ e $k=-yd\inv(\alpha-\beta)$
        si ricava una possibile soluzione per $e$. Quindi
        $\pi$ è un epimorfismo. \\
        
        Poiché $\pi$ è sia un monomorfismo che un epimorfismo, si conclude
        che $\pi$ è un isomorfismo, da cui la tesi.
        
    \end{proof}

    \begin{theorem}[\textit{Teorema cinese del resto}]
        \label{th:cinese}
    
        Sia $a$ un elemento di un anello euclideo $A$ e sia
        $p_1^{m_1}p_2^{m_2}\cdots p_n^{m_n}$ una sua fattorizzazione
        in irriducibili non associati.
        Allora vale il seguente isomorfismo:
        
        \[ A/(a) \cong A/(p_1^{m_1}) \times \cdots \times A/(p_n^{m_n}). \]
    \end{theorem}
    
    \begin{proof}
        Si dimostra il teorema applicando il principio di induzione su $n$,
        il numero di fattori irriducibili distinti che appaiono
        nella fattorizzazione di $a$. \\
        
        \,(\textit{passo base}) \, Se $a$ consta di un solo fattore irriducibile,
        allora banalmente $A/(a) \cong A/(p_1^{m_1})$. \\
        
        \,(\textit{passo induttivo}) \, Possiamo riscrivere $a$ come
        il prodotto di $(p_1^{m_1}\cdots p_{n-1}^{m_{n-1}})$ e di
        $p_n^{m_n}$. \\
        
        Si nota innanzitutto che $d = \MCD(p_1^{m_1}\cdots p_{n-1}^{m_{n-1}}, p_n^{m_n})$
        è un invertibile. Se così non fosse, infatti, si potrebbe
        considerare un irriducibile $q$ della fattorizzazione di $d$:
        tale $q$, in quanto primo per il \thref{th:irriducibili_primi},
        deve dividere un $p_j$ con $1 \leq j \leq n-1$, così
        come deve dividere $p_n$. Allora $p_j$ e $q$ sono associati,
        così come $q$ e $p_n$. Dunque anche $p_j$ e $p_n$ sono associati.
        Tuttavia questo è un assurdo, dal momento che per ipotesi
        la fattorizzazione di $a$ include irriducibili distinti e
        non associati, \Lightning{}. \\
        
        Allora dal \lemref{lem:pre_cinese} si ricava che:
        
        \[ A/(a) \cong A/(p_1^{m_1}\cdots p_{n-1}^{m_{n-1}}) \times A/(p_n^{m_n}), \]
        
        \vskip 0.1in
        
        mentre dal passo induttivo si sa già che:
        
        \[ A/(p_1^{m_1}\cdots p_{n-1}^{m_{n-1}}) \cong A/(p_1^{m_1}) \times \cdots \times A/(p_{n-1}^{m_{n-1}}). \]
        
        \vskip 0.1in
        
        Pertanto, unendo le due informazioni, si verifica la tesi:
        
            \[ A/(a) \cong
            A/(p_1^{m_1}) \times \cdots \times A/(p_{n-1}^{m_{n-1}}) \times A/(p_n^{m_n}). \]
        
    \end{proof}

\subsection{La seminorma di \texorpdfstring{$\ZZ[\sqrt{n}]$}{Z[√n]}}

Si definisce innanzitutto $\ZZ[\sqrt{n}]$ nel seguente modo:

\[ \ZZ[\sqrt{n}] = \{ a + b\sqrt{n} \mid a, b \in \ZZ \}. \]

\begin{definition}
    Si definisce \textbf{seminorma} di $\ZZ[\sqrt{n}]$ la seguente
    funzione:
    
    \[ \ell : \ZZ[\sqrt{n}] \to \ZZ, \, a + b\sqrt{n} \mapsto a^2 - n b^2. \]
\end{definition}

\begin{proposition}
    La seminorma di $\ZZ[\sqrt{n}]$ è una funzione moltiplicativa.
\end{proposition}

\begin{proof}
    Dimostrare la tesi è equivalente al verificare la seguente identità:
    
    \[ (a^2-nb^2)(c^2-nd^2)=(ac+nbd)^2-n(ad+bc)^2, \]
    
    \vskip 0.1in
    
    come si verifica nelle seguenti righe:
    
    \begin{multline*}
        (ac+nbd)^2-n(ad+bc)^2 = a^2c^2+n^2b^2d^2+\cancel{2acnbd}-na^2d^2-nb^2c^2-\cancel{2acnbd} = \\
        a^2(c^2-nd^2) -nb^2(c^2-nd^2) = (a^2-nb^2)(c^2-nd^2).
    \end{multline*}
\end{proof}

\begin{theorem}
    \label{th:invertibili_z_sqrtn}
    Un elemento $a \in \ZZ[\sqrt{n}]$ è invertibile se e solo se
    $\ell(a) \in \{1, -1\}$.
\end{theorem}

\begin{proof}
    Si dimostrano le due implicazioni separatamente. \\
    
    ($\implies$) \; Sia $a \in a \in \ZZ[\sqrt{n}]^*$. Allora esiste un
    $b \in \ZZ[\sqrt{n}]^*$ tale che $ab = 1$. Applicando la seminorma
    a entrambi i membri si ricava che:
    
    \[ \ell(ab) = 1 \implies \ell(a)\ell(b) = 1. \]
    
    \vskip 0.1in
    
    Gli unici invertibili di $\ZZ$ sono tuttavia $1$ e $-1$,
    da cui la tesi. \\
    
    ($\;\Longleftarrow\;$) \; Si consideri $a+b\sqrt{n} \in \ZZ[\sqrt{n}]$.
    Sia $d = \ell(a) \in \{1, -1\}$ si ricava che:
    
    \[ a^2-nb^2 = d \implies (a+b\sqrt{n})(a-b\sqrt{n})=d \implies (a+b\sqrt{n})d\inv(a-b\sqrt{n})=1, \]
    
    \vskip 0.1in
    
    da cui la tesi.
\end{proof}

\begin{example}[$\ZZsqrt{10}$ non è un UFD]
    Il numero $6$ ammette due fattorizzazioni in irriducibili
    completamente distinte in $\ZZsqrt{10}$. Dunque
    $\ZZsqrt{10}$ non è un UFD. Conseguentemente non è né un anello
    euclideo\footnote{Violerebbe altrimenti il \thref{th:euclidei_ufd}.}, né un
    PID\footnote{Si usa ancora la proposizione, non dimostrata
    in queste dispense, secondo cui un PID è sempre un UFD. Per
    tale dimostrazione si rimanda ancora a \cite[pp.~124-126]{di2013algebra}.}.
\end{example}    

\begin{proof}
    Dal momento che $6=16-10$,
    possiamo fattorizzare $6$ come il prodotto
    di $4+\sqrt{10}$ e $4-\sqrt{10}$. Tuttavia, dalla
    fattorizzazione in $\ZZ$, sappiamo anche
    che $6=2 \cdot 3$. \\
    
    Dimostriamo che sia $2$ che $3$ sono irriducibili
    in $\ZZ[\sqrt{10}]$. Se $2$ fosse riducibile,
    si potrebbe scrivere come prodotto
    di due fattori non invertibili: \\
    
    \begin{equation}
        \label{eq:es_z_sqrt10_fattorizzazione_2}
        2 = (a + b\sqrt{10})(c + d\sqrt{10}) \implies 4 = (a^2 - 10b^2)(c^2 - 10d^2).
    \end{equation}
    
    \vskip 0.1in
    
    Poiché nessun fattore di $2$ è invertibile per ipotesi, per il
    \thref{th:invertibili_z_sqrtn} nessuno dei due
    fattori in \eqref{eq:es_z_sqrt10_fattorizzazione_2} può essere uguale a $1$ o $-1$.
    Allora l'unica possibilità è che $a^2 - 10b^2$ sia uguale a $2$ o
    $-2$. Se però così fosse, $a^2 \equiv \pm 2 \pod{10}$, che
    non ammette soluzione. \\
    
    Reiterando lo stesso ragionamento per $3$,
    si ottiene $a^2 \equiv \pm 3 \pod{10}$, che anche
    stavolta non ammette soluzione. Quindi sia $2$ che
    $3$ sono irriducibili in $\ZZ[\sqrt{10}]$. \\
    
    Analogamente dimostriamo che sia $4+\sqrt{10}$ che
    $4-\sqrt{10}$ sono irriducibili. Si assuma che
    $4+\sqrt{10}$ sia riducibile, allora si ricava che:
    
    \[ 4+\sqrt{10} = (a + b\sqrt{10})(c + d\sqrt{10}), \]
    
    \vskip 0.1in
    
    da cui, passando alle seminorme si ottiene che:
    
    \[ 6 = (a^2 - 10b^2)(c^2 - 10d^2). \]
    
    \vskip 0.1in
    
    Poiché entrambi i fattori sono non invertibili per ipotesi,
    per il \thref{th:invertibili_z_sqrtn} ognuno di essi è
    diverso da $1$ e $-1$, come visto prima. Quindi l'unica
    possibilità è che $a^2 - 10b^2$ sia uguale a $\pm 2$ o
    $\pm 3$. Tuttavia, da prima sappiamo che nessuna di queste
    equazioni ammette soluzione. Quindi $4+\sqrt{10}$ è irriducibile,
    e allo stesso modo si dimostra che anche $4-\sqrt{10}$ lo
    è. \\
    
    Ora si dimostra che $2$ non è associato né a $4 + \sqrt{10}$
    né a $4 - \sqrt{10}$. Se fossero associati, esisterebbe
    un invertibile $a$ tale che $2 = (4 \pm \sqrt{10})a$. \\
    
    Passando alle norme, si ricava che:
    
    \[ 4 = 6 \, \ell(a),  \]
    
    \vskip 0.1in
    
    dove, ricordando che $\ell(a)=\pm 1$ per il \thref{th:invertibili_z_sqrtn},
    si ottiene:
    
    \[ 4 = \pm 6, \]
    
    \vskip 0.1in
    
    ossia un assurdo, \Lightning{}. \\
    
    Poiché $2$ non è associato né a né a $4 + \sqrt{10}$
    né a $4 - \sqrt{10}$, le due fattorizzazioni sono due
    fattorizzazioni in irriducibili completamente
    distinte. Quindi $\ZZsqrt{10}$ non può essere un UFD.
\end{proof}
