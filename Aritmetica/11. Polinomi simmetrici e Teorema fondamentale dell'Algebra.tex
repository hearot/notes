\section{Polinomi simmetrici}

\subsection{Definizione e prime proprietà}

Sia $\KK$ un campo. Dati $\sigma \in S_n$ e un polinomio $f \in \KK[x_1, \ldots, x_n]$,
si definisce il seguente polinomio:

\[ (\sigma \cdot f) (x_1, \ldots, x_n) = f(x_{\sigma(1)}, \ldots, x_{\sigma(n)}), \]

\vskip 0.1in

ossia il polinomio ottenuto permutando le variabili $x_i$ secondo $\sigma$.

\begin{definition}
    Si definisce $\Sym[X_n]$ su $K$ come il sottoanello di $\KK[x_1, \ldots, x_n]$ dei
    \textbf{polinomi simmetrici}, ossia di quei polinomi tali che
    $\sigma \cdot f = f$, $\forall \sigma \in S_n$.
\end{definition}

\begin{definition}
    Sia $d \in \NN$ tale che $0 \leq d \leq n$. Si definisce \textbf{polinomio simmetrico elementare} su $\Sym[X_n]$ ogni polinomio
    della seguente forma:
    
    \[ e_d(x_1, \ldots, x_n) = \sum_{1 \leq i_1 < \cdots < i_d \leq n} \underbrace{x_{i_1} \cdots x_{i_n}}_{d\text{ volte}}, \]
    
    \vskip 0.1in
    
    dove si pone $e_0(x_1, \ldots, x_n) := 1$
\end{definition}

\begin{remark*}
    Qualora siano noti al contesto le variabili su cui è definito $\Sym[X_n]$ si
    può omettere la parentesi di $e_d$, scrivendo pertanto semplicemente
    $e_d$.
\end{remark*}

\begin{remark*}
    Sia $p(x) = a_n x^n + \ldots + a_0$ un polinomio in $\KK[x]$. Siano $\lambda_1$,
    ..., $\lambda_n$ le sue radici nel suo campo di spezzamento. Allora vale
    che:
    
    \[ a_{n-i} = (-1)^i \, a_n \, e_i(\lambda_1, \ldots, \lambda_n). \]
\end{remark*}

\begin{definition}
    Sia $\alpha = (\alpha_1, \ldots, \alpha_n) \in \NN^n$, si definisce:
    
    \[ x^\alpha = x_1^{\alpha_1} x_2^{\alpha_2} \cdots x_n^{\alpha_n}, \quad
        \card{\alpha} = \sum_{i=1}^n \alpha_i. \]
\end{definition}

\begin{remark*}
    Ogni monomio nelle variabili $x_1$, ..., $x_n$ può essere rappresentato
    nella forma $x^\alpha$, ponendo $\alpha_i$ uguale al numero di volte
    in cui la variabile $x_i$ compare nel monomio.
\end{remark*}

\begin{definition}
    Si definisce \textit{degree lexicographic order} (\textbf{deglex}) la seguente
    relazione di ordine sui monomi monici di un polinomio:

    \[ x^\alpha > x^\beta \defiff \card{\alpha} > \card{\beta} \text{ oppure }  \\
        \card{\alpha} = \card{\beta} \text{ e } \alpha > \beta \text{ secondo il LO,} \]
        
    \vskip 0.1in
    
    dove con LO si indica il \textit{lexicographic order}.
\end{definition}

\begin{proposition}
    Il \textit{deglex} è una relazione di ordine totale.
\end{proposition}

\begin{proof}
    [TODO]
\end{proof}

\begin{proposition}
    \label{prop:moltiplicazione_disuguaglianza_deglex}
    Vale la seguente equivalenza:
    
    \[ x^\alpha x^\gamma > x^\beta x^\gamma \iff x^\alpha > x^\beta. \]
\end{proposition}

\begin{proof}
    Si dimostrano le due implicazioni separamente. \\
    
    \ ($\implies$)\; Se $\card{\alpha} + \card{\gamma} > \card{\beta} + \card{\gamma}$, allora
    anche $\card{\alpha} > \card{\beta}$, e dunque
    $x^\alpha > x^\beta$. Altrimenti, esiste un $i \in \NN$ tale
    per cui $\alpha_i + \gamma_i > \beta_i + \gamma_i$ e $\alpha_j + \gamma_j
    = \beta_j + \gamma_j$ $\forall j < i$. Allora
    anche $\alpha_j = \beta_j$ $\forall j < i$ e
    $\alpha_i > \beta_i$. Dunque, per il LO, $\alpha
    > \beta$, e quindi $x^\alpha > x^\beta$. \\
    
    \ ($\,\,\Longleftarrow\,\;$)\; Se $\card{\alpha} > \card{\beta}$, allora
    anche $\card{\alpha} + \card{\gamma} > \card{\beta} + \card{\gamma}$, e dunque
    $x^\alpha x^\gamma > x^\beta x^\gamma$. Altrimenti, esiste un $i \in \NN$ tale
    per cui $\alpha_i > \beta_i$ e $\alpha_j = \beta_j$ $\forall j < i$. Allora
    anche $\alpha_j + \gamma_j = \beta_j + \gamma_j$ $\forall j < i$ e
    $\alpha_i + \gamma_i > \beta_i + \gamma_i$. Dunque, per il LO, $\alpha + \gamma
    > \beta + \gamma$, e quindi $x^\alpha x^\gamma > x^\beta x^\gamma$.
\end{proof}

\begin{proposition}
    \label{prop:numero_finito_soluzioni_deglex}

    Sia $\alpha \in \NN^n$. Allora esiste un numero finito di $\beta \in \NN^n$
    tale che $x^\alpha > x^\beta$.
\end{proposition}

\begin{proof}
    Siano fissati gli $\alpha_i$. Se $x^\alpha > x^\beta$,
    allora vale sicuramente l'equazione:
    
    \[ \alpha_1 + \ldots + \alpha_n > \beta_1 + \ldots + \beta_n, \]
    
    \vskip 0.1in
    
    che ammette un numero finito di soluzioni.
\end{proof}

\begin{definition}
    Si definisce \textbf{leading term} di un polinomio in
    $x_1$, ..., $x_n$ il termine $cx^\alpha$ tale che
    $x^\alpha > x^\beta$, per ogni altro monomio $x^\beta$
    del polinomio.
\end{definition}

\begin{proposition}
    \label{prop:leading_term_prodotto}

    Siano $f$ e $g \in \KK[x_1, \ldots, x_n]$.
    Il \textit{leading term} di $fg$ è il
    prodotto dei \textit{leading term} di $f$ e di $g$.
\end{proposition}

\begin{proof}
    Siano $x^\alpha$ e $x^\beta$ i rispettivi \textit{leading term}
    di $f$ e di $g$. Sia inoltre $x^\gamma$ il \textit{leading term}
    di $fg$. Si assuma che $x^\gamma \neq x^\alpha x^\beta$. \\
    
    Poiché ogni monomio del prodotto di $fg$ è un prodotto di due
    monomi di $f$ e di $g$, $x^\gamma$ potrà scriversi come
    prodotto di $x^\delta x^\zeta$, dove $x^\delta$ è un monomio
    di $f$ e $x^\zeta$ è un monomio di $g$. \\
    
    Poiché $x^\alpha$ è il \textit{leading term} di $f$, vale
    la seguente disuguaglianza:
    
    \[ x^\alpha > x^\delta, \]
    
    \vskip 0.1in
    
    da cui, dalla \propref{prop:moltiplicazione_disuguaglianza_deglex}, si
    ricava che:
    
    \[ x^\alpha x^\zeta > x^\delta x^\zeta. \]
    
    \vskip 0.1in
    
    Analogamente vale la seguente altra disuguaglianza:
    
    \[ x^\beta > x^\zeta, \]
    
    \vskip 0.1in
    
    da cui si ottiene che:
    
    \[ x^\alpha x^\beta > x^\alpha x^\zeta. \]
    
    \vskip 0.1in
    
    Combinando le due disuguaglianze si ottiene infine che:
    
    \[ x^\alpha x^\beta > x^\delta x^\zeta, \]
    
    \vskip 0.1in
    
    che è assurdo, dal momento che $x^\delta x^\zeta = x^\gamma$ è il \textit{leading
    term} di $fg$, \Lightning{}. Quindi $x^\gamma = x^\alpha x^\beta$.
\end{proof}

\begin{lemma}
    \label{lem:leading_term_simmetrico_disuguaglianza}
    Sia $c x^\alpha$ il \textit{leading term}
    di $f \in \Sym[X_n]$, con $\alpha = (\alpha_1, \ldots, \alpha_n) \in \NN^n$.
    Allora $\alpha_1 \geq \alpha_2 \geq \cdots \geq \alpha_n$.
\end{lemma}

\begin{proof}
    Si dimostra la tesi contronominalmente. \\
    
    Sia $c x^\beta$ un monomio di $f$ con $\beta = (\beta_1, \ldots, \beta_n)$ tale che esistano $i < j
    \mid \beta_i < \beta_j$. Si consideri $\gamma \in \NN^n$ come
    la tupla riordinata in modo decrescente di $\beta$ e sia
    $\sigma \in S_n$ tale che $\gamma = (\beta_{\sigma(1)},
    \ldots, \beta_{\sigma(n)})$. \\
    
    Poiché $f$ è un polinomio simmetrico, $\sigma \cdot f = f$. Quindi
    $f$ ammette un monomio della forma $c x^\gamma$. Dal momento
    che $\gamma > \beta$ per il LO, $x^\gamma > x^\beta$. Quindi
    $c x^\beta$ non è il \textit{leading term} di $f$.
\end{proof}

\begin{theorem}[\textit{Teorema fondamentale dei polinomi simmetrici}]
    Sia $\KK$ un campo. Vale il seguente isomorfismo:
    
    \[ \Sym[X_n] \cong \KK\left[e_1, \ldots, e_n\right]. \]
\end{theorem}

\begin{proof}
    Sia $c x^\alpha$ il \textit{leading term} di $f$, con
    $\alpha = (\alpha_1, \ldots, \alpha_n) \in \NN^n$.
    Per il \lemref{lem:leading_term_simmetrico_disuguaglianza},
    $\alpha_i - \alpha_{i+1} \geq 0$ $\forall 1 \leq i < n$. \\
    
    Si definisca dunque $\beta \in \NN^n$ in modo tale
    che $\beta_i = \alpha_i - \alpha_{i+1} \geq 0$ $\forall 1 \leq i < n$
    e $\beta_n = \alpha_n$. \\
    
    Si consideri il monomio $e_1^{\beta_1} e_2^{\beta_2} \ldots e_n^{\beta_n}$:
    il suo \textit{leading term}, per la \propref{prop:leading_term_prodotto},
    è il prodotto dei \textit{leading term} dei suoi fattori,
    ossia $x_1^{\alpha_1} \cdots x_n^{\alpha_n} = x^\alpha$. \\
    
    Si consideri adesso come polinomio $f - c e_1^{\beta_1} e_2^{\beta_2} \ldots e_n^{\beta_n}$,
    e si reiteri l'algoritmo fino a quando il risultato non è zero. Che l'algoritmo
    termini è garantito dalla \propref{prop:numero_finito_soluzioni_deglex}, da cui
    si desume che vi è numero finito di \textit{leading term} possibili una
    volta tolto ad ogni iterazione il termine $c e_1^{\beta_1} e_2^{\beta_2} \ldots e_n^{\beta_n}$. \\
    
    Infine si sarà ottenuto una rappresentazione di $f$ come combinazione di
    $e_1$, ..., $e_n$. Questa rappresentazione è unica perché
    i termini $e_1^{\beta_1} e_2^{\beta_2} \ldots e_n^{\beta_n}$ sono
    linearmente indipendenti, dal momento che i loro
    \textit{leading term} sono distinti. \\
    
    Si costruisca dunque l'omomorfismo $\Pi : \Sym[X_n] \to \KK\left[e_1, \ldots, e_n\right]$
    che associa ad ogni polinomio simmetrico la sua rappresentazione in
    $\KK\left[e_1, \ldots, e_n\right]$. \\
    
    Si verifica che $\Pi$ è un omomorfismo. Poiché tale omomorfismo è iniettivo
    e surgettivo, è un isomorfismo, da cui la tesi.
\end{proof}

\subsection{Teorema fondamentale dell'Algebra}


