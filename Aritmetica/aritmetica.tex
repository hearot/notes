% Style formatting by Evan Chen (evanchen.cc)
% Inspired by https://github.com/diego-unipi/Appunti-Aritmetica

\PassOptionsToPackage{main=italian}{babel}
\documentclass[11pt]{scrartcl}
\usepackage[sexy]{evan}
\usepackage[italian]{babel}
\usepackage[utf8]{inputenc}
\usepackage{marvosym}

\setlength{\headsep}{0.3in}

\newcommand{\gfrac}[2]{\displaystyle \frac{#1}{#2}}
\newcommand{\nnorm}[1]{\lVert #1 \rVert}

\let\oldforall\forall
\let\forall\undefined
\DeclareMathOperator{\forall}{\oldforall}

\newcommand{\KK}{\mathbb{K}}

\let\oldexists\exists
\let\exists\undefined
\DeclareMathOperator{\exists}{\oldexists}

\let\oldnexists\nexists
\let\nexists\undefined
\DeclareMathOperator{\nexists}{\oldnexists}

\let\oldlnot\lnot
\let\lnot\undefined
\DeclareMathOperator{\lnot}{\oldlnot}

\let\oldcirc\circ
\let\circ\undefined
\DeclareMathOperator{\circ}{\oldcirc}

\DeclareMathOperator{\existsone}{\exists !}

\DeclareMathOperator{\cl}{cl}
\DeclareMathOperator{\Dom}{Dom}
\DeclareMathOperator{\Codom}{Cod}
\DeclareMathOperator{\Gr}{Gr}
\DeclareMathOperator{\Id}{Id}

\let\oldemptyset\emptyset
\let\emptyset\varnothing

\begin{document}

\author{Gabriel Antonio Videtta \\ \textnormal{\href{mailto:g.videtta1@studenti.unipi.it}{g.videtta1@studenti.unipi.it}}}
\title{Appunti del corso di Aritmetica}
\subtitle{tenutosi sotto la supervisione dei proff. Gaiffi e \textit{D'Adderio}}
\date{A.A. 2022/2023}
\maketitle
\thispagestyle{empty}

\begin{center}
	\includegraphics[scale=0.3]{logo.png}
\end{center}

\author{Gabriel Antonio Videtta}
\newpage
\thispagestyle{empty}
~\newpage

\section*{Premessa}

Affinché possano chiamarsi queste dispense, voglio mettere alcuni punti
in chiaro. Non sono un professore, né ho mai insegnato nella mia vita, per
quanto punti a farlo, pertanto queste dispense non forniscono né coprono
l'esperienza che un professore potrebbe condividere durante un vero e proprio
corso universitario.

Piuttosto queste dispense hanno lo scopo di immagazzinare e incapsulare
le nozioni che un normale corso di Aritmetica -- o Algebra 1 che sia --
potrebbe fornire, e non hanno quindi la pretesa di sostituirsi a uno
studio più approfondito e personale.

Naturalmente sono accettati a braccia aperte suggerimenti e correzioni
(che potete inviare alla mia mail,
\texttt{\href{mailto:g.videtta1@studenti.unipi.it}{g.videtta1@studenti.unipi.it}}).

\section*{Ringraziamenti}

Chiaramente ci sono alcuni ringraziamenti che ho piacere a fare. Innanzitutto
vorrei ringraziare il mio caro amico \textbf{Diego Monaco}
(\texttt{\href{mailto:d.monaco2@studenti.unipi.it}{d.monaco2@studenti.unipi.it}}),
da cui ho preso pesante ispirazione per lo stile e il contenuto di queste dispense
(trovate difatti i suoi appunti su \underline{\href{https://github.com/diego-unipi/Appunti-Aritmetica}{GitHub}}).

In secondo luogo, voglio ringraziare \textbf{Evan Chen}, dal quale ho reperito
già pronti i fogli di stile per queste dispense (e che anche voi potete trovare
sul suo \underline{\href{https://web.evanchen.cc/faq-latex.html}{sito personale}}).

\newpage
\tableofcontents

\section{Gruppi}

\subsection{Definizione e motivazione}

Innanzitutto, prima di dare una definizione formale, un
\vocab{gruppo} è una struttura algebrica, ossia un insieme
di oggetti di varia natura che rispettano alcune determinate
regole.

Il motivo (con ogni probabilità l'unico) per cui la teoria dei
gruppi risulta interessante è la facilità con cui un'astrazione
come la struttura di gruppo permette di desumere teoremi universali
per oggetti matematici apparentemente scollegati.

Infatti, dimostrato un teorema in modo astratto per un gruppo
generico, esso è valido per ogni gruppo. Per quanto questo
fatto risulti di una banalità assoluta, esso è di fondamentale
aiuto nello studio della matematica. Si pensi ad esempio all'
aritmetica modulare, o alle funzioni bigettive, o ancora
alle trasformazioni del piano: tutte queste nozioni condividono
teoremi e metodi che si fondano su una stessa logica. Come vedremo,
esse condividono la natura di gruppo.

\begin{definition}
    Dato un insieme non vuoto $G$, esso si dice \textbf{gruppo} se data
    un'operazione ben definita $\cdot : G \times G \to G$ è t.c:

    \begin{itemize}
        \item (\vocab{associatività}) $\forall a, b, c \in G, \, (a \cdot b) \cdot c = a \cdot (b \cdot c)$
        \item (\vocab{esistenza dell'elem. neutro}) $\exists e \in G \mid a \cdot e = a = e \, \cdot a \forall a \in G$
        \item (\vocab{esistenza dell'elem. inverso}) $\forall a \in G, \, \exists a^{-1} \in G \mid a \cdot a^{-1} = a^{-1} \cdot a = e$
    \end{itemize}
\end{definition}

\end{document}