\section{Campi di spezzamento}

\begin{theorem}
    \label{th:esistenza_spezzamento}
    Sia $A$ un campo, e sia $f(x) \in A[x]$.
    Allora esiste sempre un estensione di $A$ in cui siano
    contenute tutte le radici di $f(x)$.
\end{theorem}

\begin{proof}
    Si dimostra il teorema applicando il principio di induzione sul
    grado di $f(X)$. \\

    \ (\textit{passo base}) \,Sia $\deg f(x) = 0$. Allora $A$ stesso è un
    campo in cui sono contenute tutte le radici, dacché esse non esistono. \\

    \ (\textit{passo induttivo}) \,Sia $\deg f(x) = n$. Sia $f_1(x)$ un
    irriducibile di $f(x)$ e sia $\gamma(x) \in A[x]$ tale che
    $f(x)=f_1(x)\gamma(x)$. Allora, per il \thref{th:campo_quoziente_irriducibile}
    $A[x]/(f_1(x))$ è un campo, in cui, per la \propref{prop:radice_quoziente},
    $f_1(x)$ ammette radice. \\

    Poiché $\deg \gamma(x) < n$, per il passo induttivo
    esiste un campo $C$ che estende $A[x]/(f_1(x))$ in cui risiedono tutte le sue radici. Dacché $C$ contiene $A[x]/(f_1(x))$, sia le radici
    di $f_1(x)$ che di $\gamma(x)$ risiedono in $C$. Tuttavia queste sono
    tutte le radici di $f(x)$, si conclude che $C$, che è un'estensione di $A[x]/(f_1(x))$, e quindi anche di $A$, è il campo ricercato.
\end{proof}

Pertanto ora è possibile enunciare la definizione di \textit{campo di spezzamento}.

\begin{definition}
    Si definisce \textbf{campo di spezzamento} di un polinomio $f(x) \in A[x]$ un
    campo $C$ con le seguenti caratteristiche:

    \begin{itemize}
        \item $f(x)$ si fattorizza in $C[x]$ come prodotto di irriducibili di
              primo grado (i.e. in $C[x]$ risiedono tutte le radici di $f(x)$),
        \item Se $B$ è un campo tale che $A \subseteq B \subsetneq C$, allora
              $f(x)$ non si fattorizza in $B[x]$ come prodotto di irriducibili di
              primo grado.
    \end{itemize}
\end{definition}

\begin{remark*}
    Per il \thref{th:esistenza_spezzamento} esiste sempre un campo di spezzamento
    di un polinomio, dunque la definizione data è una buona definizione.
\end{remark*}

\begin{remark*}
    In generale i campi di spezzamento non sono uguali, sebbene siano tutti
    isomorfi tra loro\footnote{Per la dimostrazione di questo risultato
        si rimanda a TODO}.
\end{remark*}