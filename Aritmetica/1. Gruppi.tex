\section{Gruppi}

\subsection{Definizione e motivazione}

Innanzitutto, prima di dare una definizione formale, un
\vocab{gruppo} è una struttura algebrica, ossia un insieme
di oggetti di varia natura che rispettano alcune determinate
regole.

Il motivo (con ogni probabilità l'unico) per cui la teoria dei
gruppi risulta interessante è la facilità con cui un'astrazione
come la struttura di gruppo permette di desumere teoremi universali
per oggetti matematici apparentemente scollegati.

Infatti, dimostrato un teorema in modo astratto per un gruppo
generico, esso è valido per ogni gruppo. Per quanto questo
fatto risulti di una banalità assoluta, esso è di fondamentale
aiuto nello studio della matematica. Si pensi ad esempio all'
aritmetica modulare, o alle funzioni bigettive, o ancora
alle trasformazioni del piano: tutte queste nozioni condividono
teoremi e metodi che si fondano su una stessa logica. Come vedremo,
esse condividono la natura di gruppo.

\begin{definition}
    Dato un insieme non vuoto $G$, esso si dice \textbf{gruppo} se data
    un'operazione ben definita $\cdot : G \times G \to G$ è t.c:

    \begin{itemize}
        \item (\vocab{associatività}) $\forall a, b, c \in G, \, (a \cdot b) \cdot c = a \cdot (b \cdot c)$
        \item (\vocab{esistenza dell'elem. neutro}) $\exists e \in G \mid a \cdot e = a = e \, \cdot a \forall a \in G$
        \item (\vocab{esistenza dell'elem. inverso}) $\forall a \in G, \, \exists a^{-1} \in G \mid a \cdot a^{-1} = a^{-1} \cdot a = e$
    \end{itemize}
\end{definition}