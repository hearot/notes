\section{Gruppi}

\subsection{Definizione e motivazione}

Innanzitutto, prima di dare una definizione formale, un
\vocab{gruppo} è una struttura algebrica, ossia un insieme
di oggetti di varia natura che rispettano alcune determinate
regole.

Il motivo (con ogni probabilità l'unico) per cui la teoria dei
gruppi risulta interessante è la facilità con cui un'astrazione
come la struttura di gruppo permette di desumere teoremi universali
per oggetti matematici apparentemente scollegati.

Infatti, dimostrato un teorema in modo astratto per un gruppo
generico, esso è valido per ogni gruppo. Per quanto questo
fatto risulti di una banalità assoluta, esso è di fondamentale
aiuto nello studio della matematica. Si pensi ad esempio all'
aritmetica modulare, o alle funzioni bigettive, o ancora
alle trasformazioni del piano: tutte queste nozioni condividono
teoremi e metodi che si fondano su una stessa logica. Come vedremo,
esse condividono la natura di gruppo.

\begin{definition}
    Dato un insieme non vuoto $G$, $(G, \cdot)$ si dice \textbf{gruppo} se data
    un'operazione ben definita $\cdot : G \times G \to G$ essa è t.c:

    \begin{itemize}
        \item (\vocab{associatività}) $\forall a, b, c \in G, \, (a \cdot b) \cdot c = a \cdot (b \cdot c)$
        \item (\vocab{esistenza dell'elem. neutro}) $\exists e \in G \mid a \cdot e = a = e \, \cdot a \,\, \forall a \in G$
        \item (\vocab{esistenza dell'elem. inverso}) $\forall a \in G, \, \exists a^{-1} \in G \mid a \cdot a^{-1} = e$
    \end{itemize}
\end{definition}

\begin{remark}
    Nella definizione di gruppo si è chiaramente specificato che l'operazione dev'essere
    ben definita e, soprattutto, che l'insieme $G$ dev'essere chiuso rispetto ad esso.
    
    Pertanto, non è sufficiente aver verificato le tre proprietà sopraelencate senza
    aver prima verificato che l'operazione sia effettivamente un'operazione di
    gruppo.
\end{remark}

\begin{example}[Gruppo ciclico elementare]
    L'insieme $\ZZ/n\ZZ$ (che talvolta indicheremo semplicemente come $\ZZ_n$) degli interi modulo $n$ è un gruppo con l'operazione di
    somma $+$. Infatti:

    \begin{itemize}
        \item $\forall \left[a\right]_n, \left[b\right]_n \in \ZZ/n\ZZ, \, \left[a\right]_n + \left[b\right]_n = \left[a+b\right]_n \in \ZZ/n\ZZ$ (\textit{chiusura rispetto all'operazione})
        \item $\forall \left[a\right]_n, \left[b\right]_n, \left[c\right]_n \in \ZZ/n\ZZ, \, \left(\left[a\right]_n + \left[b\right]_n\right) + \left[c\right]_n = \left[a+b\right]_n +
        \left[c\right]_n = \left[a+b+c\right]_n = \left[a\right]_n + \left[b+c\right]_n = \left[a\right]_n + \left(\left[b\right]_n + \left[c\right]_n\right)$ (\textit{associatività})
        \item $\forall \left[a\right]_n \in \ZZ/n\ZZ, \, \left[a\right]_n + 0 = \left[a\right]_n$ (\textit{esistenza dell'elem. neutro})
        \item $\forall \left[a\right]_n \in \ZZ/n\ZZ, \, \exists \left[-a\right]_n \in \ZZ/n\ZZ \mid \left[a\right]_n + \left[-a\right]_n = 0$ (\textit{esistenza dell'elem. inverso})
    \end{itemize}
\end{example}

\begin{example}[Gruppo simmetrico]
    L'insieme $S_n$ delle funzioni bigettive da $X_n = \{1, 2, \ldots, n\}$ in sé stesso è un
    gruppo rispetto all'operazione di composizione, detto \vocab{gruppo simmetrico}. Infatti:

    \begin{itemize}
        \item $\forall f, g \in S_n, \, f \circ g \in S_n$ (\textit{chiusura rispetto all'operazione})
        \item $\forall f, g, h \in S_n, \, (f \circ g) \circ h = f \circ (g \circ h)$ (\textit{associatività})
        \item $\exists e = \Id \in S_n \mid f \circ e = f = e \circ f \forall f \in S_n$ (\textit{esistenza dell'elem. neutro})
        \item $\forall f \in S_n, \, \exists f^{-1} \in S_n \mid f \circ f^{-1} = e$ (\textit{esistenza dell'elem. inverso})
    \end{itemize}
\end{example}

Le proprietà date dalla definizione di un gruppo ci permettono immediatamente di desumere
altre proprietà fondamentali, e che sulle quali faremo affidamento d'ora in poi.

\begin{theorem}
    \label{th:gruppo:inverso_unico}
    L'inverso $a^{-1}$ di un elemento $a$ di un gruppo $G$ è unico.
\end{theorem}

\begin{proof}
    Supponiamo che $b$ e $c$ siano due elementi inversi distinti di $a$. Allora
    $b=b\cdot e=b\cdot \underbrace{(a \cdot c)}_{=e}=\underbrace{(b \cdot a)}_{=e} \cdot c=c$, \Lightning. Pertanto l'inverso è unico.
\end{proof}

\begin{theorem}
    L'inverso dell'inverso $\left(a^{-1}\right)^{-1}$ è pari a $a$.
\end{theorem}

\begin{proof}
    Dal momento che l'inverso è unico (per il \vocab{Teorema~\ref{th:gruppo:inverso_unico}}),
    $\left(a^{-1}\right)^{-1} a^{-1} = e \implies \left(a^{-1}\right)^{-1} = a$.
\end{proof}

\begin{theorem}
    L'inverso di $ab$ è $b^{-1}a^{-1}$.
\end{theorem}

\begin{proof}
    Si verifica facilmente che $ab b^{-1}a^{-1}= a e a^{-1} = a a^{-1} = e$. Poiché
    l'inverso è unico (per il \vocab{Teorema~\ref{th:gruppo:inverso_unico}}), allora
    $\left(ab\right)^{-1} = b^{-1}a^{-1}$.
\end{proof}

\begin{remark}
    In realtà, sebbene a prima vista potrebbe sembrare inusuale l'inversione dei due
    fattori nell'ultima identità, essa è una conseguenza del modo in cui operiamo
    naturalmente. Si prenda per esempio la composizione $f \circ g$, per ottenere
    l'identità è necessario prima decomporre $f$, l'ultima funzione aggiunta, ed infine
    $g$, ossia seguendo l'ordine da sinistra a destra.

    Nel corso di Geometria vi sarà spiegato come anche la matrici si comportano in
    questo modo (non è un caso, dal momento che anch'esse, sotto talune condizioni,
    formano un gruppo, il cosiddetto \vocab{gruppo lineare} $\GL_n(\KK)$).
\end{remark}

\begin{theorem}
    Un'equazione della forma $ax=bx$ è vera se e solo se $a=b$.
\end{theorem}

\begin{proof}
    Infatti, moltiplicando per l'inverso di $x$, $ax=bx \iff axx^{-1}=bxx^{-1} \iff a=b$.
\end{proof}


