\section{Introduzione alla teoria dei campi}

\subsection{La caratteristica di un campo}

Si consideri il seguente omomorfismo:

\[ \psi : \ZZ \to \KK, \]

\vskip 0.1in

completamente determinato dalla condizione $\psi(1) = 1$, dacché
$\ZZ$ è generato da $1$. Si studia innanzitutto il caso in cui
$\Ker \psi = (0)$. In questo caso, $\psi$ è un monomorfismo, e per
il \corref{cor:primo_isomorfismo_iniettivo}, $\ZZ \cong \Imm \psi$. \\

Pertanto, $\KK$ ammetterebbe come sottoanello una copia isomorfa di $\ZZ$.
Inoltre, poiché $\KK$ è un campo, deve anche ammetterne gli inversi, e quindi
ammetterebbe come sottocampo una copia isomorfa di $\QQ$. La seguente
definizione classificherà questi tipi di campo. \\

\begin{definition}
    Si dice che un campo $\KK$ è di \textbf{caratteristica zero} ($\Char \KK = 0$),
    quando $\Ker \psi = (0)$.
\end{definition}

Altrimenti, se $\Ker \psi \neq (0)$, dacché $\ZZ$ è un anello euclideo,
$\Ker \psi$ deve essere monogenerato da un intero $n$, ossia $\Ker \psi = (n)$. \\

Tuttavia non tutti gli interi sono ammissibili. Sia infatti $n$ non primo, allora
$n = ab$ con $a$, $b \neq \pm 1$. Si nota innanzitutto che $\psi(a) \neq 0$,
se infatti fosse nullo, $n$ dovrebbe dividere $a$, impossibile dal momento
che $\card{a} < \card{n}$, \Lightning{}. Analogamente anche $\psi(b) \neq 0$. \\

Se $n$ fosse generatore di $\Ker \psi$ si ricaverebbe allora che:

\[ \underbrace{\psi(a)}_{\neq\,0} \underbrace{\psi(b)}_{\neq\,0} = \psi(n) = 0, \]

\vskip 0.1in

che è assurdo, dal momento che $\KK$, in quanto campo, è anche un dominio.
Quindi $n$ deve essere un numero primo. In particolare, allora, per
il \nameref{th:primo_isomorfismo}, $\ZZp = \ZZ/(p) \cong \Imm \psi$,
ossia $\KK$ contiene una copia isomorfa di $\ZZp$, a cui ci riferiremo
semplicemente con $\FFpp$. \\

Allora, poiché sia $\KK$ che $\FFpp$ sono campi, $\KK$ è uno spazio
vettoriale su $\FFpp$. Si può dunque classificare quest'ultimo tipo di
campi con la seguente definizione:

\begin{definition}
    Si dice che un campo $\KK$ è di \textbf{caratteristica $p$} ($\Char \KK = p$)
    quando $\Ker \psi = (p)$, con $p$ primo.
\end{definition}

\begin{remark*}
    La caratteristica di un campo \textbf{non} distingue i campi finiti
    dai campi infiniti. Esistono infatti campi infiniti di caratteristica
    $p$, come il campo delle funzioni razionali su $\ZZp$:

    \[ \ZZ_p(x) =  \left\{ \frac{f(x)}{g(x)} \mid f(x),\, g(x) \in \ZZpx,\, g(x) \neq 0 \right\}. \]

    \vskip 0.1in

    Infatti $\psi(p) = p \, \psi(1) = 0$.
\end{remark*}

\subsection{Prime proprietà dei campi di caratteristica \texorpdfstring{$p$}{p}}

Come si è appena visto, un campo $\KK$ di caratteristica $p$ contiene
al suo interno un sottocampo $\FFpp$ isomorfo a $\ZZp$, ed è per questo
uno spazio vettoriale su di esso. A partire da questa informazione si
può dimostrare la seguente proposizione.

\begin{proposition}
    \label{prop:campo_char_p_prodotto_per_p}
    Sia $\KK$ un campo di caratteristica $p$. Allora, per ogni
    elemento $v$ di $\KK$, $pv=0$.
\end{proposition}

\begin{proof}
    Considerando ogni elemento di $\KK$ come vettore e $p$ come
    scalare, si ricava che:

    \[ pv=(\underbrace{1+\ldots+1}_{p\text{ volte}})v=
        (\underbrace{\psi(1)+\ldots+\psi(1)}_{p\text{ volte}})v=
        \psi(p)v=0v=0. \]

\end{proof}

Mentre, partendo da questa proposizione, si può dimostrare il
seguente teorema.

\begin{theorem}[\textit{Teorema del binomio ingenuo}]
    \label{th:binomio_ingenuo}
    Siano $a$ e $b$ elementi di un campo di caratteristica $p$. Allora
    $(a+b)^p = a^p + b^p$.
\end{theorem}

\begin{proof}
    Per dimostrare la tesi si applica la formula del binomio di Newton
    nel seguente modo:

    \[ (a+b)^p = \sum_{i=0}^p \binom{p}{i} a^{p-i}b^p. \]

    \vskip 0.1in

    Tuttavia, dal momento che $p$ è un fattore di tutti i binomiali per
    $1 \leq i \leq p-1$, tutti i termini computati con queste $i$
    sono nulli per la \propref{prop:campo_char_p_prodotto_per_p}.
    Si desume così l'identità della tesi.
\end{proof}

\subsection{L'omomorfismo di Frobenius}

\begin{definition}
    Dato un campo $\KK$ di caratteristica $p$, si definisce
    \textbf{omomorfismo di Frobenius} per il campo $\KK$
    la funzione:

    \[ \Frob : \KK \to \KK,\, a \mapsto a^p. \]
\end{definition}

\begin{remark*}
    In effetti, l'omomorfismo di Frobenius è un omomorfismo. \\

    Infatti, $\Frob(1) = 1^p = 1$. Inoltre tale funzione
    rispetta la linearità per il \nameref{th:binomio_ingenuo}:

    \[ \Frob(a + b) = (a+b)^p = a^p + b^p = \Frob(a) + \Frob(b), \]

    \vskip 0.1in

    e chiaramente anche la moltiplicatività:

    \[ \Frob(ab) = (ab)^p = a^p b^p = \Frob(a) \Frob(b). \]
\end{remark*}

\begin{proposition}
    \label{prop:frobenius_monomorfismo}
    L'omomorfismo di Frobenius di un campo $\KK$ di caratteristica
    $p$ è un monomorfismo.
\end{proposition}

\begin{proof}
    Si prenda in considerazione $\Ker \Frob$. Esso è sicuramente
    un ideale diverso da $\KK$, dacché $1 \notin \Ker \Frob$.
    Tuttavia, se $\Ker \Frob \neq (0)$, $\Ker \Frob$, dal
    momento che $\KK$, in quanto campo, è un anello euclideo,
    e quindi un PID, è monogenerato da un invertibile. \\

    Se però così fosse, $\Ker \Frob$ coinciderebbe con il
    campo $\KK$ stesso, \Lightning{}. Quindi $\Ker \Frob = (0)$,
    da cui la tesi.
\end{proof}

\begin{proposition}
    Sia $\KK$ un campo finito di caratteristica $p$. Allora
    l'omomorfismo di Frobenius è un automorfismo.
\end{proposition}

\begin{proof}
    Dalla \propref{prop:frobenius_monomorfismo} è noto che
    $\Frob$ sia già un monomorfismo. Dal momento che
    il dominio e il codominio sono lo stesso e constano
    entrambi dunque di un numero finito di elementi,
    se $\Frob$ non fosse surgettivo, vi sarebbe un elemento
    di $\KK$ a cui non è associato nessun elemento di $\KK$
    mediante $\Frob$. \\

    Per il principio dei cassetti, allora, spartendo
    $\card{\KK}$ elementi in $\card{\KK}-1$ elementi,
    vi sarebbe almeno un elemento dell'immagine a cui
    sarebbero associati due elementi del dominio. Tuttavia
    questo è assurdo dal momento che $\Frob$ è un
    monomorfismo. Quindi $\Frob$ è un epimorfismo. \\

    Dacché $\Frob$ è contemporaneamente un endomorfismo,
    un monomorfismo e un epimorfismo, è allora anche
    un automorfismo.
\end{proof}

\begin{proposition}
    \label{prop:punti_fissi_frobenius_campo}
    Sia $\KK$ un campo di caratteristica $p$ e si
    definisca l'insieme dei punti fissi del suo
    omomorfismo di Frobenius:

    \[ \Fix(\Frobexp^n) = \{ a \in \KK \mid \Frobexp^n(a) = a \} .\]

    \vskip 0.1in

    Allora $\Fix(\Frobexp^n)$ è un sottocampo di $\KK$.
\end{proposition}

\begin{proof}
    Affinché $\Fix(\Frobexp^n)$ sia un sottocampo di $\KK$,
    la sua somma e la sua moltiplicazione devono essere ben
    definite, e ogni suo elemento deve ammettere un inverso
    sia additivo che moltiplicativo. \\

    Siano allora $a$, $b \in \Fix(\Frobexp^n)$.
    $\Frobexp^n$ è un omomorfismo, in quanto è composizione
    di omomorfismi (in particolare, dello stesso omomorfismo
    $\Frobexp$). Sfruttando le proprietà
    degli omomorfismi si dimostra dunque
    che $a+b \in \Fix(\Frobexp^n)$:

    \[ \Frobexp^n(a+b) = \Frobexp^n(a) + \Frobexp^n(b) = a+b, \]

    \vskip 0.1in

    e che $ab \in \Fix(\Frobexp^n)$:

    \[ \Frobexp^n(ab) = \Frobexp^n(a)\Frobexp^n(b) = ab. \]

    \vskip 0.1in

    Analogamente si dimostra che $-a \in \Fix(\Frobexp^n)$:

    \[ \Frobexp^n(-a) = -\Frobexp^n(a) = -a, \]

    \vskip 0.1in

    e che $a\inv \in \Fix(\Frobexp^n)$:

    \[ \Frobexp^n(a\inv) = \Frobexp^n(a)\inv = a\inv. \]

\end{proof}


\subsection{Classificazione dei campi finiti}

\begin{theorem}
    Ogni campo finito $\KK$ di caratteristica $p$ consta
    di $p^n$ elementi, con $n \in \NN^+$.
\end{theorem}

\begin{proof}
    Come già detto precedentemente, $\KK$ è uno
    spazio vettoriale su una copia isomorfa di $\ZZp$,
    $\FFpp$. \\

    Si consideri allora il grado $[\KK : \FFpp]$. Sicuramente
    questo grado non è infinito, dal momento che $\KK$ non
    ha infiniti elementi. Quindi $[\KK : \FFpp] = n \in \NN$. \\

    Sia dunque $(k_1, k_2, \ldots, k_n)$ una base di $\KK$
    su $\FFpp$. Ogni elemento $a$ di $\KK$ si potrà dunque scrivere
    come:

    \[ a = \alpha_1 k_1 + \ldots + \alpha_n k_n, \quad \alpha_1, \ldots, \alpha_n \in \FFpp,\]

    \vskip 0.1in

    e dunque vi saranno in totale $p^n$
    elementi, dove ogni $p$ è contato dal numero di elementi che è
    possibile associare ad ogni coefficiente, ossia $\card{\FFpp} = p$,
    per il numero di elementi appartenenti alla base, ossia $[\KK : \FFpp] =
        n$, da cui la tesi.
\end{proof}

\begin{theorem}
    Per ogni $n \in \NN^+$ e per ogni numero primo $p$ esiste un
    campo finito con $p^n$ elementi.
\end{theorem}

\begin{proof}
    Si consideri il polinomio $x^{p^n}-x$ su $\ZZp$ e un suo
    campo di spezzamento $A$. $\Fix(\Frobexp^n)$, per
    la \propref{prop:punti_fissi_frobenius_campo}, è
    un sottocampo, e
    contiene esattamente le radici di $x^{p^n}-x$, che
    in $A$ si spezza in fattori lineari, per definizione. \\

    La derivata di $x^{p^n}-x$ è $p^n x^{p^n - 1}-1 \equiv -1$,
    dacché $A$ è uno spazio vettoriale su $\ZZp$, e pertanto
    vale ancora la \propref{prop:campo_char_p_prodotto_per_p}.
    Dal momento che $-1$ e $x^{p^n}-x$ non hanno fattori lineari
    in comune, per il \textit{Criterio della derivata},
    $x^{p^n}-x$ non ammette radici multiple. \\

    Allora $\Fix(\Frobexp^n)$ è un campo con $p^n$ elementi,
    ossia tutte le radici di $x^{p^n}-x$ (e coincide quindi
    con il campo di spezzamento $A$), da cui la tesi.
\end{proof}
