\chapter{I polinomi di un campo: \texorpdfstring{$\KKx$}{K[x]}}

\section{Elementi preliminari}

Prima di procedere ad enunciare le proprietà più
rilevanti dell'anello dei polinomi $\KKx$, si ricorda
che esso è un \textbf{anello euclideo} in cui la funzione
grado coincide con il grado del polinomio, ossia
$g = \deg$. Si enuncia
ora invece la definizione di radice.

\begin{definition}
    Si dice che $\alpha \in \KK$ è una \textbf{radice} del polinomio
    $f(x) \in \KKx$ se $f(\alpha)=0$.
\end{definition}

\begin{proposition}
    \label{prop:radice_x_meno_alpha}
    Se $\alpha \in \KK$ è una radice di $f(x) \in \KKx$, allora
    $(x-\alpha)$ divide $f(x)$.
\end{proposition}

\begin{proof}
    Dal momento che $\KKx$ è un anello euclideo, si può eseguire
    la divisione euclidea tra $f(x)$ e $(x-\alpha)$, ossia
    esistono $q(x)$, $r(x) \in \KKx$ tali che $f(x)=q(x)(x-\alpha)+r(x)$
    con $\deg r(x) < \deg (x-\alpha)$ o con $r(x)=0$. \\

    Se $r(x) \neq 0$, poiché $\deg r(x) < \deg (x-\alpha)$, si deduce
    che $\deg r(x) = 0$, ossia che $r(x)$ è un invertibile. In entrambi
    i casi, $r(x)$ è comunque una costante. Pertanto, valutando il
    polinomio in $\alpha$, si ricava:

    \[ 0=f(\alpha)=\underbrace{q(\alpha)(\alpha-\alpha)}_{=\,0} + r(\alpha), \]

    da cui $r(\alpha)=0$. Quindi $f(x)=q(x)(x-\alpha)$, e si verifica
    la tesi.
\end{proof}

\begin{theorem}
    \label{th:al_più_n_radici}
    Sia $f(x) \in \KKx$ di grado $n$. Allora $f(x)$ ha al più
    $n$ radici.
\end{theorem}

\begin{proof} Se $n$ è nullo, allora $f(x)$ è una costante
    non nulla, e quindi non ammette radici, in accordo alla tesi. \\

    Sia allora $n \geq 1$. Se $f(x)$ non ha radici in $\KK$, allora
    la tesi è ancora soddisfatta. Altrimenti sia $\zeta_1$ una radice  di $f(x)$. Si divida $f(x)$ per
    $(x-\zeta_1)$ e se ne prende il quoziente $q_1(x)$, mentre si
    ignori il resto, che,
    per la \textit{Proposizione \ref{prop:radice_x_meno_alpha}},
    è nullo. \\

    Si reiteri il procedimento utilizzando $q_1(x)$ al
    posto di $f(x)$ fino a quando il grado del quoziente non è nullo o
    il quoziente non ammette radici in $\KK$, e si chiami quest'ultimo
    quoziente $\lambda(x)$.
    Infatti, poiché i gradi dei quozienti diminuiscono di $1$ ad
    ogni iterazione, è garantito che l'algoritmo termini al più
    dopo $n$ iterazioni. \\

    In questo modo, numerando le radici, si può scrivere $f(x)$ come:

    \begin{equation}
        \label{eq:fattorizzazione_fx}
        f(x)=\alpha(x-\zeta_1)(x-\zeta_2)\cdots(x-\zeta_k)\lambda(x).
    \end{equation}

    \vskip 0.1in

    Si osserva che $x-\zeta_i$ è irriducibile $\forall 1 \leq i \leq k$.
    Se $f(x)$ ammettesse un'altra fattorizzazione in cui compaia
    un fattore $x-\alpha$ con $\alpha \neq \zeta_i$ $\forall 1 \leq i \leq k$, allora $f(x)$ ammetterebbe due fattorizzazioni in
    irriducibili, dacché $x-\alpha$ non sarebbe un associato
    di nessuno dei $x-\zeta_i$, né tantomeno di un
    irriducibile $\lambda(x)$. \\

    Se infatti $x-\alpha$ fosse un associato di un
    irriducibile $\lambda(x)$, $x-\alpha$ dividerebbe
    $\lambda(x)$, e quindi $\lambda(x)$ ammetterebbe $\alpha$ come radice. Se $\lambda(x)$
    è una costante, questo è a priori assurdo, \Lightning{}. Se invece
    $\lambda(x)$ non è una costante, il fatto che ammetta una radice contraddirebbe il funzionamento
    dell'algoritmo di fattorizzazione espresso in precedenza, \Lightning{}. Quindi $x-\alpha$ non è associato di nessun irriducibile di $\lambda(x)$. \\

    Allora il fatto che $f(x)$ ammetta due fattorizzazioni in
    irriducibili è assurdo, dacché $\KKx$ è un anello euclideo, e
    quindi un UFD, \Lightning{}. Quindi le radici sono esattamente $k \leq n$, da cui la tesi.
\end{proof}

\section{Sottogruppi moltiplicativi finiti di \texorpdfstring{$\KK$}{K}}

Si illustra adesso un teorema che riguarda i sottogruppi
moltiplicativi finiti di $\KK$, da cui conseguirà,
per esempio, che $\ZZ_p^*$ è sempre ciclico, per
qualsiasi $p$ primo. \\

\begin{lemma}
    \label{lem:somma_phi_n}
    Per ogni $n \in \NN$ vale la seguente identità:

    \[ n = \sum_{d \mid n} \varphi(d). \]
\end{lemma}

\begin{proof}
    Si consideri il gruppo ciclico $\ZZ_n$ per $n \in \NN$.
    Si osserva che $\card{\ZZ_n} = n$. \\

    Si definisca $X_d$ come l'insieme degli elementi di $G$
    di ordine $d$. Dal momento che ogni elemento appartiene
    a uno e uno solo di questi $X_d$, per ogni divisore
    $d$ di $n$, allora si può partizionare $G$ nel
    seguente modo:

    \begin{equation*}
        G = \bigcup_{d \mid n} X_d.
    \end{equation*}

    Dal momento che $\ZZ_n$ è ciclico, ogni $X_d$ ha esattamente
    $\varphi(d)$ elementi, e dunque si deduce che:

    \begin{equation*}
        n = \card{G} = \sum_{d \mid n} \card{X_d} =  \sum_{d \mid n} \varphi(d),
    \end{equation*}

    ossia la tesi.
\end{proof}

\begin{theorem}
    Un sottogruppo moltiplicativo finito di un campo
    $\KK$ è sempre ciclico.
\end{theorem}

\begin{proof}
    Sia $G$ un sottogruppo finito di un campo $\KK$ definito
    sulla sua operazione di moltiplicazione, e sia
    $\card{G} = n$. \\

    Si definisca $X_d$ come l'insieme degli elementi di $G$
    di ordine $d$. Dal momento che ogni elemento appartiene
    a uno e uno solo di questi $X_d$, per ogni divisore
    $d$ di $n$, allora si può partizionare $G$ nel
    seguente modo:

    \begin{equation*}
        G = \bigcup_{d \mid n} X_d,
    \end{equation*}

    da cui:

    \begin{equation}
        \label{eq:partizione_g_xd}
        n = \card{G} = \sum_{d \mid n} \card{X_d}.
    \end{equation}

    \vskip 0.1in

    Dal \lemref{lem:somma_phi_n} e da \eqref{eq:partizione_g_xd},
    si ricava infine la seguente equazione:

    \begin{equation}
        \label{eq:uguaglianza_xd}
        \sum_{d \mid n} \card{X_d} = n = \sum_{d \mid n} \varphi(d).
    \end{equation}

    Adesso vi sono due casi: o $\card{X_n} > 0$ o $\card{X_n} = 0$. \\

    Nel primo caso si concluderebbe che esiste almeno un elemento in
    $G$ di ordine $n$, e quindi che esiste un generatore con cui
    $G$ è ciclico, ossia la tesi. \\

    Nel secondo caso si dimostra un assurdo. Dal momento che
    $\card{X_n} = 0$, esiste sicuramente un divisore proprio
    $d$ di $n$ tale che $\card{X_d} > \varphi(d)$. Altrimenti,
    se $\card{X_d} \leq \varphi(d)$ per ogni divisore $d$,
    si ricaverebbe la seguente disuguaglianza:

    \[ \sum_{\substack{d \mid n \\ d \neq n}} \card{X_d} \leq \sum_{
            \substack{d \mid n \\ d \neq n}} \varphi(d) \implies \sum_{d \mid n} \card{X_d}
        \overbrace{=}^{\card{X_n}=0} \sum_{\substack{d \mid n \\ d \neq n}} \card{X_d}
        \leq \sum_{\substack{d \mid n \\ d \neq n}} \varphi(d)
        \overbrace{<}^{\varphi(n) \geq 1} \sum_{d \mid n} \varphi(d).\]

    \vskip 0.1in

    Tuttavia questo è un assurdo, dal momento che per \eqref{eq:uguaglianza_xd}
    deve valere l'uguaglianza, \Lightning{}. \\

    Sia $g \in X_d$ e si consideri $(g)$, il sottogruppo generato da $g$.
    Vale in particolare che $\card{(g)} = d$. \\

    Si consideri adesso il polinomio $f(x)= x^d-1 \in \KK[x]$. Tutti e $d$ gli
    elementi di $(g)$ sono già soluzione di $f(x)$. Tuttavia, poiché
    $\card{X_d} > \varphi(d)$, esiste sicuramente un elemento $h$ in $X_d$ che
    non appartiene a $(g)$. Infatti se tutti gli elementi di $X_d$ appartenessero
    a $(g)$ vi sarebbero più di $\varphi(d)$ generatori, \Lightning{}. \\

    Infine, poiché $h \in X_d$, anch'esso è soluzione di $f(x)$. Questo è
    però un assurdo, poiché, per il \thref{th:al_più_n_radici}, $f(x)$
    ammette al più $d$ radici, mentre così ne avrebbe almeno $d+1$, \Lightning{}. \\

    Quindi $\card{X_d}>0$, e $G$ è ciclico.
\end{proof}

\section{Il quoziente \texorpdfstring{$\KKx/(f(x))$}{K[x]/(f(x))}}

Nell'ambito dello studio delle radici di un polinomio,
il quoziente $\KKx/(f(x))$ gioca un ruolo fondamentale.
Infatti, come vedremo in seguito, se $f(x)$ è irriducibile,
questo diventa un campo, e, soprattutto, ammette sempre una
radice per $f(x)$. \\

In realtà, il quoziente $\KKx/(f(x))$ si comporta pressocché
allo stesso modo dei più familiari $\ZZ/n\ZZ$. Infatti
le principali regole dell'aritmetica modulare potrebbero
essere estese anche a tale quoziente, senza particolari
sacrifici. \\

Si enuncia adesso un teorema importante, che è equivalente --
anche nella dimostrazione -- all'analogo per i campi
$\ZZ/p\ZZ$.

\begin{theorem}
    \label{th:campo_quoziente_irriducibile}
    $\KKx/(f(x))$ è un campo se e solo se $f(x)$ è irriducibile.
\end{theorem}

\begin{proof}
    Si dimostrano le due implicazioni separatamente. \\

    ($\implies$)\; Sia $f(x) \in \KKx$ irriducibile. Affinché l'anello
    commutativo $\KKx/(f(x))$ sia un campo è sufficiente dimostrare
    che ogni suo elemento non nullo ammette un inverso moltiplicativo. \\

    Sia $\alpha(x) + (f(x)) \in \KKx/(f(x))$ non nullo. Allora
    $\alpha(x)$ non è divisibile da $f(x)$, e pertanto
    $\MCD(\alpha(x), f(x))=1$\footnote{Si ricorda che in un PID la
        nozione di \textit{massimo comun divisore} (MCD) è più ambigua
        di quella di $\ZZ$. Infatti $\MCD(a,b)$ comprende tutti i
        generatori dell'ideale $(a,b)$, e quindi tutti i suoi associati.
        Pertanto si dirà $\MCD(a,b)$ uno qualsiasi di questi associati,
        e nel nostro caso $1$ è un buon valore, dacché l'MCD deve essere
        un associato di un'unità.}. \\

    Allora, per l'\textit{Identità di Bézout}, esistono $\beta(x)$,
    $\lambda(x) \in \KKx$ tali che:

    \[ \alpha(x)\beta(x) + \lambda(x)f(x) = 1. \]

    Dacché $\alpha(x)\beta(x)-1 \in (f(x))$, si deduce che
    $\alpha(x)\beta(x)+(f(x))=1+(f(x))$, e quindi
    $\beta(x) + (f(x))$ è l'inverso moltiplicativo di $\alpha(x) +
        (f(x))$, da cui la dimostrazione dell'implicazione. \\

    ($\,\Longleftarrow\,\,$)\; Si dimostra l'implicazione
    contronominalmente. Sia $f(x) \in \KKx$ riducibile. Allora
    esistono $\alpha(x)$ e $\beta(x)$ non
    invertibili tali che $f(x)=\alpha(x)\beta(x)$, da cui si ricava che:

    \[[\alpha(x)+(f(x))][\beta(x)+(f(x))]=f(x)+(f(x))=0+(f(x)),\]

    \vskip 0.1in

    ossia l'identità di $\KKx/(f(x))$. \\

    Tuttavia, se $\KKx/(f(x))$ fosse un campo, e quindi un dominio,
    ciò non sarebbe ammissibile, dacché non potrebbero esservi
    divisori di zero. Quindi $\KKx/(f(x))$ non è un campo.

\end{proof}

\begin{remark*}
    Una notazione per indicare un elemento di $\KKx/(f(x))$ alternativa
    e più sintetica di $a+(f(x))$ è $\overline{a}$, qualora
    sia noto nel contesto a quale $f(x)$ si fa riferimento.
\end{remark*}

\begin{proposition}
    \label{prop:radice_quoziente}
    Nell'anello $\KKx/(f(x))$ esiste sempre una radice di $f(x)$,
    convertendo opportunamente i coefficienti da $\KK$ a $\KKx/(f(x))$.
\end{proposition}

\begin{proof}
    Sia $\overline{x} = x + (f(x)) \in \KKx/(f(x))$ e si descriva $f(x)$ come:

    \[ f(x)=a_nx^n+\ldots+a_0. \]

    Allora, computando $f(x)$ in $\overline{x}$ e convertendone
    i coefficienti, si ricava che:

    \[f(\overline{x})=\overline{a_n} \, \overline{x}^n + \ldots + \overline{a_0} =
        \overline{a_n x^n} + \ldots + \overline{a_0} = \overline{f(x)} =
        \overline{0}.\]

    Quindi $\overline{x}$ è una radice di $f(x)$, da cui la tesi.

\end{proof}