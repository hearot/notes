\documentclass{book}

\usepackage{amsmath}
\usepackage{amsthm}
\usepackage[italian]{babel}

\begin{document}

\author{Gabriel Antonio Videtta}
\title{Appunti di Fisica}
\maketitle
\thispagestyle{empty}

\newcommand{\gfrac}[2]{\displaystyle \frac{#1}{#2}}

\newpage

\chapter{I moti principali della fisica}

\section{Moto rettilineo uniforme (m.u.a.)}

Conoscendo le definizioni di accelerazione ($\vec{a} = \frac{d\vec{v}}{dt}$)
e di velocità ($\vec{v} = \frac{d\vec{x}}{dt}$) è possibile, ponendo l'accelerazione
costante (i.e. il \textit{jerk} è nullo, $\frac{d\vec{a}}{dt} = 0$), ricavare numerose formule.

\subsection{Le equazioni del moto in un sistema di riferimento unidimensionale}

Le equazioni del moto sono le seguenti:

\begin{equation}
    \begin{cases}
        x(t)=x_0+v_0t+\frac{1}{2}a_0t^2 \\
        v(t)=v_0+at
    \end{cases}
\end{equation}

\begin{proof}
    Da $a=\frac{dv}{dt}$, si ricava $dv=a\cdot dt$, da cui:

    \begin{equation*}
        \int dv=\int a\, dt = a \int dt \Rightarrow v=v_0+at
    \end{equation*}

    Dimostrata questa prima equazione, è possibile dimostrare in modo analogo l'altra:

    \begin{equation*}
        \int dx=\int v\cdot dt = \int v_0\, dt + \int at\, dt = x_0+v_0t+\frac12at^2
    \end{equation*}

    La dimostrazione può essere inoltre resa immediata se si sviluppano $x(t)$ e
    $v(t)$ come serie di Taylor-Maclaurin.

\end{proof}

\subsection{Lo spostamento in funzione della velocità e dell'accelerazione}

Senza ricorrere alla variabile di tempo $t$, è possibile
esprimere lo spostamento in funzione della velocità e dell'accelerazione
mediante le seguente formula:

\begin{equation}
    x-x_0=\frac{v^2-v_0^2}{2a}
\end{equation}

\begin{proof}

    Considerando $a=\frac{dv}{dt}$, è possibile riscrivere mediante l'impiego
    delle formule di derivazione delle funzioni composte quest'ultima formula in

    \begin{equation*}
        a=\frac{dv}{dt}=\frac{dx}{dt}\frac{dv}{dx}=v\,\frac{dv}{dx}
    \end{equation*}

    Da ciò si può ricavare infine l'ultima formula:

    \begin{equation*}
        a\,dx=v\,dv \Rightarrow a \int dx = \int v \, dv
    \end{equation*}

    E quindi:

    \begin{equation*}
        a(x-x_0)=\frac{v^2-v_0^2}{2} \Rightarrow x-x_0=\frac{v^2-v_0^2}{2a}
    \end{equation*}

\end{proof}


\end{document}