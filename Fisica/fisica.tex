\documentclass[oneside]{book}

\usepackage{amsmath}
\usepackage{amsthm}
\usepackage{hyperref}
\usepackage{mathtools}
\usepackage[italian]{babel}
\usepackage[utf8]{inputenc}
\usepackage[parfill]{parskip}
\usepackage{wrapfig}

\usepackage{pgfplots}
\pgfplotsset{compat=1.15}
\usepackage{mathrsfs}
\usetikzlibrary{arrows,angles,quotes}


\newcommand{\gfrac}[2]{\displaystyle \frac{#1}{#2}}
\newcommand{\abs}[1]{\lvert#1\rvert}
\newcommand{\norm}[1]{\lVert \vec{#1} \rVert}
\newcommand{\nnorm}[1]{\lVert #1 \rVert}


\begin{document}

\author{Gabriel Antonio Videtta}
\title{Appunti di Fisica}

\maketitle

\tableofcontents

\chapter{I moti principali della fisica}

\section{Il moto uniformemente accelerato (m.u.a.)}

Conoscendo le definizioni di accelerazione ($\vec{a} = \frac{d\vec{v}}{dt}$)
e di velocità ($\vec{v} = \frac{d\vec{x}}{dt}$) è possibile, ponendo l'accelerazione
costante (i.e. il \textit{jerk} è nullo, $\frac{d\vec{a}}{dt} = 0$), ricavare numerose formule.

\subsection{Le equazioni del moto in un sistema di riferimento unidimensionale}

Le equazioni del moto sono le seguenti:

\begin{equation}
    \begin{dcases}
        x(t)=x_0+v_0t+\frac{1}{2}at^2 \\
        v(t)=v_0+at
    \end{dcases}
    \label{eq:mua}
\end{equation}

\begin{proof}
    Da $a=\frac{dv}{dt}$, si ricava $dv=a\cdot dt$, da cui:

    \begin{equation*}
        \int dv=\int a\, dt = a \int dt \Rightarrow v=v_0+at
    \end{equation*}

    Dimostrata questa prima equazione, è possibile dimostrare in modo analogo l'altra:

    \begin{equation*}
        \int dx=\int v\cdot dt = \int v_0\, dt + \int at\, dt = x_0+v_0t+\frac12at^2
    \end{equation*}

    La dimostrazione può essere inoltre resa immediata se si sviluppano $x(t)$ e
    $v(t)$ come serie di Taylor-Maclaurin.

\end{proof}

\subsection{Lo spostamento in funzione della velocità e dell'accelerazione}

Senza ricorrere alla variabile di tempo $t$, è possibile
esprimere lo spostamento in funzione della velocità e dell'accelerazione
mediante le seguente formula:

\begin{equation}
    x-x_0=\frac{v^2-v_0^2}{2a}
\end{equation}

\begin{proof}

    Considerando $a=\frac{dv}{dt}$, è possibile riscrivere, mediante l'impiego
    delle formule di derivazione delle funzioni composte, quest'ultima formula:

    \begin{equation*}
        a=\frac{dv}{dt}=\frac{dx}{dt}\frac{dv}{dx}=v\,\frac{dv}{dx}
    \end{equation*}

    Da ciò si può ricavare infine l'ultima formula:

    \begin{equation*}
        a\,dx=v\,dv \Rightarrow a \int dx = \int v \, dv
    \end{equation*}

    E quindi:

    \begin{equation*}
        a(x-x_0)=\frac{v^2-v_0^2}{2} \Rightarrow x-x_0=\frac{v^2-v_0^2}{2a}
    \end{equation*}

\end{proof}

\section{Il moto dei proiettili}

Il \textit{moto dei proiettili}, o moto parabolico, non
è altro che la forma vettoriale del m.u.a. sfruttando due accelerazioni per
entrambe le dimensioni: una nulla (quella dello spostamento parallelo al
terreno) ed una pari a $-g$ (quella data dalla gravità nello spostamento
normale al terreno).

\subsection{Le equazioni del moto dei proiettili}

Riprendendo le precedenti considerazioni, si può dunque scrivere
l'equazione del moto in forma vettoriale:

\begin{equation}
    \begin{pmatrix}
        x \\
        y
    \end{pmatrix} = \begin{pmatrix}
        x_0 \\
        y_0
    \end{pmatrix} + \vec{v_0} t + \frac12 \vec{a} t^2
\end{equation}

O nel casso del moto parabolico sulla Terra:

\begin{equation}
    \begin{pmatrix}
        x \\
        y
    \end{pmatrix} = \begin{pmatrix}
        x_0 \\
        y_0
    \end{pmatrix} + \vec{v_0} t + \frac12 \begin{pmatrix}
        0 \\
        -g
    \end{pmatrix} t^2
\end{equation}

O si può separare quest'ultima in due equazioni:

\begin{equation}
    \begin{dcases}
        x(t)=x_0+v_0\cos(\theta)t \\
        y(t)=y_0+v_0\sin(\theta)t-\frac12gt^2
    \end{dcases}
\end{equation}

\subsection{Il calcolo della gittata e della traiettoria}

Definita la \textit{gittata} come la distanza tra il punto di lancio ed
il punto in cui il corpo assume la stessa ordinata del punto di lancio e
la \textit{traiettoria} come la distanza tra il punto di lancio ed il
punto in cui il corpo assume la massima ordinata, si possono facilmente
dimostrare le seguenti equazioni:

\begin{equation}
    \displaystyle
    \begin{dcases}
        x_{\text{gittata}} = \frac{v_0^2 \sin(2\theta)}{g} \\
        x_{\text{traiettoria}} = \frac12 x_{\text{gittata}} = \frac{v_0^2 \sin(2\theta)}{2g}
    \end{dcases}
\end{equation}

\section{Il moto circolare}

Definendo alcune grandezze fisiche in modo analogo a come vengono
proposte nel m.u.a., è possibile riproporre le equazioni \ref{eq:mua}
mediante l'impiego di grandezze esclusivamente angolari.

\subsection{Le equazioni del moto circolare}

Si definiscano dunque le seguenti grandezze:

\begin{itemize}
    \item L'angolo $\theta$ in funzione del tempo
    \item La velocità angolare $\displaystyle \omega=\dot{\theta}=\frac{d\theta}{dt}$
    \item L'accelerazione angolare $\displaystyle \alpha=\ddot{\theta}=\frac{d\omega}{dt}=\frac{d^2\theta}{dt^2}$
\end{itemize}

Innanzitutto, è possibile coniugare il mondo angolare con quello
cartesiano, tenendo conto del fatto che $x=\theta r$. In questo modo
si ricavano le seguenti relazioni:

\begin{itemize}
    \item $\displaystyle v=\omega r$, la velocità angolare
    \item $\displaystyle a_t=\alpha r$, l'accelerazione tangenziale
    (da distinguersi da quella centripeta!)
\end{itemize}

Per sostituzione, dalle equazioni \ref{eq:mua} si ottengono dunque
le analoghe seguenti:

\begin{equation}
    \begin{dcases}
        \theta = \theta_0 + \omega t + \frac12 \alpha t^2 \\
        \omega = \omega_0 + \alpha t
    \end{dcases}
\end{equation}

Nel caso del moto circolare uniforme ($\alpha=0$) è utile definire
altre due quantità:

\begin{itemize}
    \item Il periodo $T=\dfrac{2\pi}{\omega}$
    \item La frequenza $f=\dfrac{1}{T}=\dfrac{\omega}{2\pi}$
\end{itemize}

\subsection{L'accelerazione centripeta}

Oltre all'accelerazione tangenziale, direttamente proporzionale a
quella lineare, è possibile definire anche un altro tipo di
accelerazione: l'\textbf{accelerazione centripeta} ($a_c$), diretta dal
corpo verso il centro della circonferenza sulla quale questo muove.

Questo tipo di accelerazione è costante nel moto circolare uniforme
($\alpha=0$) ed è calcolata mediante le seguente equazione:

\begin{equation}
    a_c=\frac{v^2}{r}
    \label{eq:acc_c}
\end{equation}

Qualora non ci si riferisse ad un moto circolare uniforme,
l'accelerazione centripeta non sarà costante, ma variabile in
funzione della velocità con la quale si muove il corpo.

Inoltre, vale la seguente relazione:

\begin{equation}
    a=\sqrt{a_t^2 + a_c^2}
    \label{eq:acc_moto_cir}
\end{equation}

Nel caso del moto circolare uniforme, l'unica
accelerazione agente sul corpo è quindi quella centripeta.

\newpage

\subsection{Il moto circolare visualizzato vettorialmente}

\vskip 0.1in

\begin{wrapfigure}[12]{l}{0.5\textwidth}
    \begin{tikzpicture}
        \coordinate (a) at (2.236, 0);
        \coordinate (b) at (0.89, 0.92);
        \coordinate (o) at (0, 0);

        \draw [rotate around={0:(0,0)}] (0,0) ellipse (2.236 and 1);
        \draw [->,thick] (0,0) -- (0,2.5) node[midway, right] {$\vec{\omega}$};
        \draw [->,thick] (0,0) -- (a) node[right] {$\vec{r}(t)$};
        \draw [->,thick] (0,0) -- (b) node[above right] {$\vec{r}(t+dt)$};

        \draw [->] (a) -- (2.6, 0.7) node[right] {$\vec{v}(t)$};

        \draw pic["$d\theta$", draw=black, angle eccentricity=1.5, angle radius=0.5cm] {angle=a--o--b};

    \end{tikzpicture}

    \caption{Il moto circolare nel piano $O_{xy}$} \label{fig:moto_circolare}
\end{wrapfigure}

Per visualizzare in modo più intuitivo, ma anche più formale, il
moto circolare, è possibile costruire un sistema di riferimento
basandosi su alcune assunzioni.

Basandosi sulla figura \ref{fig:moto_circolare}, assumiamo
$\vec{d\theta} = d\theta \cdot \hat{z}$, attraverso cui
possiamo concludere che $\vec{\omega}=\frac{\vec{d\theta}}{dt}$ è anch'esso
parallelo a $\hat{z}$.

Inoltre, $d\vec{r}$ deve essere perpendicolare sia a $\vec{r}$ che
a $\vec{\omega}$, poiché appartiene al piano $O_{xy}$.

Perciò è possibile riscrivere $d\vec{r}$ nella seguente forma, tenendo
conto che il suo modulo è pari a $d\theta \cdot \norm{r}$ (ovvero
l'arco di circonferenza percorso per $d\theta$):

\begin{equation*}
    d\vec{r}=\frac{d\theta \cdot \norm{r}}{\lVert \vec{w} \times
    \vec{r} \rVert} \vec{w} \times \vec{r} =
    \frac{d\theta}{\norm{\omega}} \vec{w} \times \vec{r}
\end{equation*}

Poiché la velocità $\vec{v}$ è pari a $\frac{d\vec{r}}{dt}$, si ottiene,
conoscendo $d\vec{r}$, la seguente relazione:

\begin{equation}
    \vec{v}=\vec{w}\times\vec{r}
\end{equation}

Dalla quale si ricava che $\vec{v}$ è perpendicolare sia a $\vec{r}$ che
a $\vec{\omega}$.

Analogamente, è possibile ricavare l'accelerazione:

\begin{equation}
    \vec{a}=\frac{d\vec{v}}{dt}=\frac{d(\vec{\omega}
    \times \vec{r})}{dt}=\vec{\alpha} \times \vec{r}
    + \vec{\omega} \times \vec{v}
\end{equation}

È interessante notare che $\vec{\alpha} \times \vec{r}$
è perpendicolare a $\vec{\omega} \times \vec{v}$, permettendoci
di calcolare facilmente il modulo dell'accelerazione:

\begin{equation}
    \norm{a} = \sqrt{\nnorm{\vec{\alpha} \times \vec{r}}^2 + \nnorm{\vec{\omega} \times \vec{v}}^2}
\end{equation}

Non solo: $\vec{\alpha} \times \vec{r}$ è perpendicolare a $\vec{r}$ e
$\vec{\omega} \times \vec{v}$ gli è parallelo, ma possiede un verso opposto.
Per questa serie di motivi, $\vec{\alpha} \times \vec{r}$ viene chiamata
\textbf{accelerazione tangenziale} ($\vec{a_t}$), mentre
$\vec{\omega} \times \vec{v}$ viene chiamata \textbf{accelerazione centripeta}
($\vec{a_c}$).

Nel moto circolare uniforme, ove $\vec{\alpha}=0$, infatti l'accelerazione
centripeta è costante (vd. eq. \ref{eq:acc_c}) e l'accelerazione
tangenziale è nulla (quindi $\vec{a}=\vec{a_c}$).

Attraverso questa visualizzazione del moto, è possibile ricavare tutte
le formule proposte all'inizio della sezione (ed è soprattutto
possibile giustificare l'equazione \ref{eq:acc_moto_cir}).

\end{document}