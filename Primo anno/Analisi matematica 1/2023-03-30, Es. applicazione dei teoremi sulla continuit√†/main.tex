\documentclass[11pt]{article}
\usepackage{personal_commands}
\usepackage[italian]{babel}

\title{\textbf{Note del corso di Analisi Matematica 1}}
\author{Gabriel Antonio Videtta}
\date{30 marzo 2023}

\begin{document}
	
	\maketitle
	
	\wip
	
	\begin{center}
		\Large \textbf{Esercitazioni: applicazione dei teoremi sulla continuità}
	\end{center}

	\begin{exercise}
		Siano $I = [a, b] \in \RRbar$, $f : I \to \RRbar$ continua strettamente
		crescente. Allora l'eq.~$f(x)=y$ ha una sola soluzione se 
		$f(a) \leq y \leq f(b)$ e nessuna soluzione se $y < f(a)$ o
		se $y > f(b)$.
	\end{exercise}
	
	\begin{solution}
		Poiché $f$ è strettamente crescente, $f$ è iniettiva. Allora,
		se $y$ è tale che $f(a) \leq y \leq f(b)$, per il teorema
		dei valori intermedi, $\exists x \mid f(x) = y$; e tale $x$
		è unica dal momento che $f$ è iniettiva. In particolare, poiché $f$ è crescente,
		$f(a)$ e $f(b)$ sono rispettivamente $\inf f(I)$ e $\sup f(I)$, e
		quindi sono anche
		$\min f(I)$ e $\max f(I)$, da cui, se $y < f(a)$ o $y > f(b)$,
		$y=f(x)$ non ammette soluzione.
	\end{solution}

	\begin{exercise}
		Si consideri l'eq.~$xe^x = 4$ (*).
		
		\begin{enumerate}[(a)]
			\item Mostrare che (*) ammette un'unica soluzione $\xbar \in \RRbar$, e trovare $x_0$, $x_1$ tali che $x_0 < \xbar < x_1$.
			\item Calcolare $\xbar$ con errore minore a $10^{-2}$.
		\end{enumerate}		
	\end{exercise}

	\begin{solution}
		Si studia la funzione $f(x) = xe^x - 4$. $f$ è continua, e vale che
		$f(0) = -4$ e che $f(2) = 2e^2 - 4 \geq 4$. Quindi, per il teorema
		degli zeri su $[0, 2]$, $f$ ammette uno zero $\xbar$ in $(0, 2)$. \\
		
		Si studia adesso la derivata $f'(x) = e^x + xe^x = (1+x)e^x$.
		$f'(x) > 0 \iff x > -1$, ossia $f$ è crescente per $x > -1$.
		Al contrario, $f$ decresce per $x < -1$; poiché allora
		$\lim_{x \to -\infty} f(x) = -4$, $\sup f((-\infty), -1)) = -4$,
		e quindi $f$ non ha zeri per $x < -1$, tantomeno per $x = -1$
		(infatti $f(-1) = -1 e^{-1} - 4 \neq 0$). \\
		
		Poiché per $x > -1$ $f$ è allora strettamente crescente,
		$f$ può ammette un solo zero, ossia quello trovato all'inizio
		della soluzione.
		
		Per ricavare $\xbar$ con errore minore a $10^{-2}$, si applica
		il metodo di bisezione per $7$ volte (infatti $\eps(n) = \frac{1}{2^n}$
		per ogni passaggio $n$-esimo dell'algoritmo, e $\eps(7) \approx 0.0079 < 0.01$), ricavando $\xbar = 1.2031$.
	\end{solution}

	\begin{exercise}
		Si consideri l'eq~$x^5+x=10$ (*).
		
		\begin{enumerate}[(a)]
			\item Mostrare che $\exists\, \xbar$ soluzione di (*) e che
			tale $\xbar$ è unica.
			\item Mostrare che $\xbar \in (0, 2)$.
			\item Trovare $\xbar$ con errore minore a $10^{-2}$.
		\end{enumerate}
	\end{exercise}

	\begin{solution}
		Si consideri la funzione $f(x) = x^5 + x - 10$. Si osserva che
		tale funzione è sempre continua. Si osserva che $f(0) = -10$ e che
		$f(2) = 24$. Quindi $f$ ammette una soluzione $\xbar$ in
		$(0, 2)$. \\
		
		Si studia la derivata di $f$, ossia $f'(x) = 5x^4+1$. Poiché
		$f'(x) > 0$ $\forall x \in \RR$, $f$ è strettamente crescente,
		e quindi $f$ ammette un'unica soluzione, $\xbar$. \\
		
		Per trovare la soluzione $\xbar$ con errore minore a $10^{-2}$,
		come nell'esercizio precedente, è necessario applicare
		il metodo di bisezione per $7$ volte, ricavando
		$\xbar = 1.5469$.
	\end{solution}

	\begin{remark}
		La scelta del punto medio nell'algoritmo di bisezione è (quasi)
		forza. Nella costruzione degli intervalli è infatti
		necessario che l'intervallo, all'infinito, tenda ad un solo
		punto; qualora non venga scelto il punto medio degli intervalli,
		questo non è assolutamente garantito.
	\end{remark}

	\begin{exercise}
		Sia $I = [a, b]$. Siano $f_1$, $f_2 : I \to \RR$ continue tali che
		$f_1(a) < f_2(a)$ e che $f_1(b) > f_2(b)$. Dimostrare che $\exists \xbar \in I$ tale che $f_1(\xbar) = f_2(\xbar)$.
	\end{exercise}

	\begin{solution}
		Si consideri $g(x) = f_1(x) - f_2(x)$. $g$ è continua in $I$, e
		$g(a) < 0$ e $g(b) > 0$ per ipotesi. Allora, per il teorema degli
		zeri, $\exists x \in (a, b)$ tale che $g(x) = 0$, ossia
		che $f_1(x) = f_2(x)$, da cui la tesi.
	\end{solution}
	
	\begin{exercise}
		Sia $I = [a, b]$ e sia $f: I \to \RR$ continua. Sia $P$ un punto che si muove in modo continuo nella striscia $I \times \RR$. Sia in
		particolare $P : [0, 1] \to I \times \RR$ tale che
		$t \mapsto (x(t), y(t))$ con $a \leq x(t) \leq b$ $\forall t \in [0, 1]$, con $y(0) > f(a)$, $y(1) < f(b)$, $x(0) = a$ e $x(1) = b$. Dimostrare che $\exists t \in
		[0, 1]$ tale che $(x(t), y(t)) = (x(t), f(x(t)))$, ossia che tale
		curva si interseca con la funzione $f$.
	\end{exercise}

	\begin{solution}
		Si consideri la funzione $g(t) = f(x(t)) - y(t)$. Poiché $x$ ed
		$f$ sono continue, lo è anche la loro composizione, e così,
		poiché anche $y$ è continua, lo è in particolare $g$. Dal momento
		che $g(0) = f(x(0)) - y(0) = f(a) - y(0) < 0$ e $g(1) =
		f(x(1)) - y(1) = f(b) - y(1) > 0$, per il teorema dei valori
		intermedi, esiste $\tbar \in (0, 1)$ tale che $g(0) = 0$,
		ossia tale che $f(x(\tbar)) = y(\tbar)$, da cui la tesi.
	\end{solution}
	
	\begin{exercise}
		Sia $I = (a, b)$ e sia $f : (a, b) \to \RRbar$ continua tale che
		$\exists \ell_a = \lim_{x \to a} f(x)$, $\ell_b = \lim_{x \to b} f(x)$.
		Si consideri allora l'estensione continua $\tilde f$:
		
		\[ \tilde f = \system{ f(x) & \text{se } x \neq a, b, \\ \ell_a & \text{se } x = a, \\ \ell_b & \text{se } x = b.} \]
		
		Allora\footnote{Come
			già riscontrato, vale un risultato ancora più forte:
			data un'estensione $\tilde f$ di $f$ in $\overline I$, $\tilde f$
			è continua se e solo se i valori estesi sono esattamente i limiti
			della funzione nei punti di $I \setminus \overline I$; e quindi
			l'estensione continua è ben definita, e unica del suo genere.} vale che $\tilde f$ è continua in $\overline I$.
	\end{exercise}

	\begin{solution}
		Sicuramente $\tilde f$ è continua in $I$, dacché vale quanto
		$f$ in questa porzione di intervallo. Poiché $\ell_a = \lim_{x \to a} f(x)$, per ogni intorno $I$ di $\ell_a$ esiste un intorno $J$ di $a$
		tale che $f(J \cap I \setminus \{a\}) = f(J \cap I) = \tilde f(J \cap I) \subseteq I$, ossia, per definizione, $\tilde f$ è continua
		anche in $a$, e, analogamente, anche in $b$.
	\end{solution}

	\begin{remark} Come mostrato nella traccia dell'esercizio precedente,
		si possono estendere continuamente alcune funzioni elementari.
		Per esempio, detta $f(x) = \frac{1}{x^2}$, si può estendere $f$
		a $\tilde f : \RRbar \to \RRbar$ in modo tale che:
		
		\[ \tilde f(x) = \system {0 & \text{se } x = \pm \infty, \\ +\infty & \text{se } x = 0, \\ f(x) & \text{altrimenti.}} \]
	\end{remark}

	%TODO: dimostrare che se limite sinistro e destro coincidono, allora esiste il limite ed è lo stesso del limite sinistro e destro.
	
	\begin{exercise}
		Si trovi un esempio di funzione $f : X \to \RRbar$, dove, dato $\xbar$ punto
		di accumulazione di $X$, $f(x) \tendsto{\xbar} \ell$, ma
		$\exists \, (x_n) \subseteq X$ tale che $x_n \tendston \xbar$, ma
		$f(x_n)$ non tende a $\ell$ per $n \to \infty$.
	\end{exercise}

	\begin{solution}
		Sia $f : \RR \to \RR$ tale che:
		
		\[ f(x) = \system{0 & \text{se } x = 0, \\ 1 & \text{altrimenti}.}\]
		
		Si consideri allora la successione $(x_n) \subseteq X$ tale che:
		
		\[ x_n = \system{ 0 & \text{se } n \text{ è pari}, \\ \frac{1}{n} & \text{altrimenti}. } \]
		
		Si mostra che $x_n \tendston 0$. Infatti, sia $I = [-\eps, \eps]$, con $\eps > 0$, un intorno di $0$.
		Allora per $n > \frac{1}{\eps}$ vale che $x_n \in I$ (infatti $0$ vi appartiene sempre, e $0 < \frac{1}{n} < \eps$);
		da cui si ricava proprio che $x_n \tendston 0$. \\
		
		Chiaramente $f(x) \tendsto{0} 1$. È sufficiente mostrare allora che $f(x_n)$ non tende a $1$ per
		$n \to \infty$. Si consideri la sottosuccessione $f(x_{2n})$: poiché $f(x_{2n}) = f(0) = 0$, la
		sottosuccessione presa in considerazione è costante, e quindi $f(x_{2n}) \tendston 0$.
		Anche la sottosuccessione $f(x_{2n + 1})$ è costante, e vale che $f(x_{2n + 1}) = f(\frac{1}{n}) = 1$,
		e quindi $f(x_{2n+1}) \tendston 1$. Poiché allora il limite di $f(x_n)$, se esistesse, dovrebbe essere
		uguale a quello di ambo le sottosuccessioni considerate, ed il limite è unico, $f(x_n)$ non ammette
		limite, proprio come volevasi dimostrare.
	\end{solution}

	\begin{exercise}
		Sia $X \subseteq \RRbar$ tale che ogni punto di $X$ sia isolato.
		Dimostrare allora che $X$ è al più numerabile.
	\end{exercise}

	\begin{solution}
		Sia $\xbar \in X$. Poiché $\xbar$ è per ipotesi isolato, esiste
		un intorno $I(\xbar)$ di $\xbar$ tale che $I \cap X = \{\xbar\}$. Si può
		sempre trovare un intorno $J(\xbar)$ più piccolo di $I(\xbar)$ tale
		che $J(\xbar) \cap I(x) = \emptyset$ $\forall x \in X \setminus \{\xbar\}$.
		Se infatti non si potesse, esisterebbe un $x \in X \setminus \{\xbar\}$ tale che $J \cap I(x) \neq \emptyset$ per ogni
		intorno $J \subseteq I(\xbar)$ di $\xbar$: sicuramente tale $x \notin J$,
		altrimenti $I(\xbar)$ conterrebbe un elemento di $X$ diverso
		da $\xbar$, assurdo dal momento che $I(\xbar)$ non ne contiene uno
		per costruzione; ma $x$ non può neanche appartenere a $X \setminus J$,
		dacché in tal modo si può sempre costruire con errore a piacimento
		un intorno più piccolo di $J$ tale che sia disgiunto con $I(x)$,
		\Lightning. Dal momento che $\QQ$ è denso in $\RRbar$, si può allora
		sempre associare a $J(\xbar)$ un numero razionale $q$ al suo interno.
		In questo modo si può costruire una funzione $f : X \to \QQ$,
		tale che $f(\xbar) = q$. Poiché i $J(x)$ sono digiunti per costruzione,
		$f$ è iniettiva, e quindi $\abs X \leq \abs \QQ = \abs \NN$, e quindi
		$X$ è al più numerabile. %TODO: approfondire
	\end{solution}
\end{document}
