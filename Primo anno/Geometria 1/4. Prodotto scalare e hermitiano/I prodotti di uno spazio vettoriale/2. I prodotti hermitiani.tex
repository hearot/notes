\chapter{I prodotti hermitiani e complessificazione (non indicizzato)}

\begin{definition} (prodotto hermitiano) Sia $\KK = \CC$. Una mappa $\varphi : V \times V \to \CC$ si dice \textbf{prodotto hermitiano} se:
	
	\begin{enumerate}[(i)]
		\item $\varphi$ è $\CC$-lineare nel secondo argomento, ossia se $\varphi(\v, \U + \w) = \varphi(\v, \U) + \varphi(\v, \w)$ e
		$\varphi(\v, a \w) = a \, \varphi(\v, \w)$,
		\item $\varphi(\U, \w) = \conj{\varphi(\w, \U)}$.
	\end{enumerate}
\end{definition}

\begin{definition} (prodotto hermitiano canonico in $\CC^n$) Si definisce
	\textbf{prodotto hermitiano canonico} di $\CC^n$ il prodotto $\varphi : \CC^n \times \CC^n \to \CC$ tale per cui, detti $\v = (z_1 \cdots z_n)^\top$ e $\w = (w_1 \cdots w_n)^\top$, $\varphi(\v, \w) = \sum_{i=1}^n \conj{z_i} w_i$.
\end{definition}

\begin{remark}\nl
	\li $\varphi(\U + \w, \v) = \conj{\varphi(\v, \U + \w)} =
	\conj{\varphi(\v, \U) + \varphi(\v, \w)} = \conj{\varphi(\v, \U)} + \conj{\varphi(\v, \U)} = \varphi(\w, \v) + \varphi(\U, \v)$, ossia
	$\varphi$ è additiva anche nel primo argomento. \\
	\li $\varphi(a \v, \w) = \conj{\varphi(\w, a \v)} = \conj{a} \conj{\varphi(\w, \v)} = \conj{a} \, \varphi(\v, \w)$. \\
	\li $\varphi(\v, \v) = \conj{\varphi(\v, \v)}$, e quindi $\varphi(\v, \v) \in \RR$. \\
	\li Sia $\v = \sum_{i=1}^n x_i \vv i$ e sia $\w = \sum_{i=1}^n y_i \vv i$, allora $\varphi(\v, \w) = \sum_{i =1}^n \sum_{j=1}^n \conj{x_i} y_i \varphi(\vv i, \vv j)$. \\
	\li $\varphi(\v, \w) = 0 \iff \varphi(\w, \v) = 0$.
\end{remark}

\begin{proposition}
	Data la forma quadratica $q : V \to \RR$  del prodotto hermitiano $\varphi$ tale che $q(\v) = \varphi(\v, \v) \in \RR$, tale
	forma quadratica individua univocamente il prodotto hermitiano $\varphi$.
\end{proposition}

\begin{proof}
	Innanzitutto si osserva che:
	
	\[ \varphi(\v, \w) = \frac{\varphi(\v, \w) + \conj{\varphi(\v, \w)}}{2} +  \frac{\varphi(\v, \w) . \conj{\varphi(\v, \w)}}{2}. \]
	
	\vskip 0.05in
	
	Si considerano allora le due identità:
	
	\[ q(\v + \w) - q(\v) - q(\w) =
	\varphi(\v, \w) + \conj{\varphi(\w, \v)} = 2 \, \Re(\varphi(\v, \w)), \]
	
	\[ q(i\v + \w) - q(\v) - q(\w) = -i(\varphi(\v, \w) - \conj{\varphi(\v, \w)}) = 2 \, \imm(\varphi(\v, \w)), \]
	
	\vskip 0.05in
	
	da cui si conclude che il prodotto $\varphi$ è univocamente
	determinato dalla sua forma quadratica.
\end{proof}

\begin{definition}
	Si definisce \textbf{matrice aggiunta} di $A \in M(n, \KK)$ la matrice coniugata della trasposta di $A$, ossia:
	
	\[ A^* = \conj{A^\top} = \conj{A}^\top. \]
\end{definition}

\begin{remark}
	Per quanto riguarda la matrice aggiunta valgono le principali proprietà della matrice trasposta:
	
	\begin{itemize}
		\item $(A + B)^* = A^* + B^*$,
		\item $(AB)^* = B^* A^*$,
		\item $(A\inv)^* = (A^*)\inv$, se $A$ è invertibile.
	\end{itemize}
\end{remark}

%TODO: aggiungere tr(conj(A^t) B)

\begin{definition} (matrice associata del prodotto hermitiano) Analogamente
	al caso del prodotto scalare, data una base $\basis = \{\vv 1, \ldots, \vv n\}$ si definisce
	come \textbf{matrice associata del prodotto hermitiano} $\varphi$
	la matrice $M_\basis(\varphi) = (\varphi(\vv i, \vv j))_{i,j = 1 \textrm{---} n}$.
\end{definition}

\begin{remark}
	Si osserva che, analogamente al caso del prodotto scalare, vale
	la seguente identità:
	
	\[ \varphi(\v, \w) = [\v]_\basis^* M_\basis(\varphi) [\w]_\basis. \]
\end{remark}

\begin{proposition}
	(formula del cambiamento di base per i prodotto hermitiani) Siano
	$\basis$, $\basis'$ due basi di $V$. Allora vale la seguente
	identità:
	
	\[ M_{\basis'} = M_{\basis}^{\basis'}(\Idv)^* M_\basis(\varphi) M_{\basis}^{\basis'}(\Idv). \]
\end{proposition}

\begin{proof}
	Siano $\basis = \{ \vv 1, \ldots, \vv n \}$ e $\basis' = \{ \ww 1, \ldots, \ww n \}$. Allora $\varphi(\ww i, \ww j) = [\ww i]_\basis^* M_\basis(\varphi) [\ww j]_\basis = \left( M_\basis^{\basis'}(\Idv)^i \right)^* M_\basis(\varphi) M_\basis^{\basis'}(\Idv)^j =
	\left(M_\basis^{\basis'}(\Idv)\right)^*_i M_\basis(\varphi) M_\basis^{\basis'}(\Idv)^j$, da cui si ricava l'identità
	desiderata.
\end{proof}

\begin{definition} (radicale di un prodotto hermitiano)
	Analogamente al caso del prodotto scalare, si definisce il \textbf{radicale} del prodotto $\varphi$ come il seguente sottospazio: 
	
	\[ V^\perp = \{ \v \in V \mid \varphi(\v, \w) = 0 \, \forall \w \in V \}. \]
\end{definition}

\begin{proposition}
	Sia $\basis$ una base di $V$ e $\varphi$ un prodotto hermitiano. Allora $V^\perp = [\cdot]_\basis \inv (\Ker M_\basis(\varphi))$\footnote{Stavolta non è sufficiente considerare la mappa $f : V \to V^*$ tale che $f(\v) = \left[ \w \mapsto \varphi(\v, \w) \right]$, dal momento che $f$ non è lineare, bensì antilineare, ossia $f(a \v) = \conj a f(\v)$.}.
\end{proposition}

\begin{proof}
	Sia $\basis = \{ \vv 1, \ldots, \vv n \}$ e sia $\v \in V^\perp$.
	Siano $a_1$, ..., $a_n \in \KK$ tali che $\v = a_1 \vv 1 + \ldots + a_n \vv n$. Allora, poiché $\v \in V$, $0 = \varphi(\vv i, \v)
	= a_1 \varphi(\vv i, \vv 1) + \ldots + a_n \varphi(\vv i, \vv n) = M_i [\v]_\basis$, da cui si ricava che $[\v]_\basis \in \Ker M_\basis(\varphi)$, e quindi che $V^\perp \subseteq [\cdot]_\basis \inv (\Ker M_\basis(\varphi))$. \\
	
	Sia ora $\v \in V$ tale che $[\v]_\basis \in \Ker M_\basis(\varphi)$.
	Allora, per ogni $\w \in V$, $\varphi(\w, \v) = [\w]_\basis^* M_\basis(\varphi) [\v]_\basis = [\w]_\basis^* 0 = 0$, da cui si
	conclude che $\v \in V^\perp$, e quindi che  $V^\perp \supseteq [\cdot]_\basis \inv (\Ker M_\basis(\varphi))$, da cui
	$V^\perp = [\cdot]_\basis \inv (\Ker M_\basis(\varphi))$, ossia
	la tesi.
\end{proof}

\begin{remark}
	Come conseguenza della proposizione appena dimostrata, valgono
	le principali proprietà già viste per il prodotto scalare. \\
	
	\li $\det(M_\basis(\varphi)) = 0 \iff V^\perp \neq \zerovecset \iff \varphi$ è degenere, \\
	\li Vale il teorema di Lagrange, e quindi quello di Sylvester, benché con alcune accortezze: si
	introduce, come nel caso di $\RR$, il concetto di segnatura, che diventa l'invariante completo
	della nuova congruenza hermitiana, che ancora una volta si dimostra essere una relazione
	di equivalenza. \\
	\li Come mostrato nei momenti finali del documento (vd.~\textit{Esercizio 3}), vale
	la formula delle dimensioni anche nel caso del prodotto hermitiano.
\end{remark}

\hr

\begin{definition} (restrizione ai reali di uno spazio) Sia $V$
	uno spazio vettoriale su $\CC$ con base $\basis$. Si definisce allora lo spazio $V_\RR$, detto
	\textbf{spazio di restrizione su $\RR$} di $V$, come uno spazio su $\RR$ generato da
	$\basis_\RR = \basis \cup i \basis$. 
\end{definition}

\begin{example}
	Si consideri $V = \CC^3$. Una base di $\CC^3$ è chiaramente $\{ \e1, \e2, \e3 \}$. Allora
	$V_\RR$ sarà uno spazio vettoriale su $\RR$ generato dai vettori $\{ \e1, \e2, \e3, i\e1, i\e2, i\e3 \}$.
\end{example}

\begin{remark}
	Si osserva che lo spazio di restrizione su $\RR$ e lo spazio di partenza condividono lo stesso insieme
	di vettori. Infatti, $\Span_\CC(\basis) = \Span_\RR(\basis \cup i\basis)$. Ciononostante, $\dim V_\RR = 2 \dim V$\footnote{Si sarebbe potuto ottenere lo stesso risultato utilizzando il teorema delle torri algebriche: $[V_\RR : \RR] = [V: \CC] [\CC: \RR] = 2 [V : \CC]$.}, se $\dim V \in \NN$.
\end{remark}

\begin{definition} (complessificazione di uno spazio) Sia $V$ uno spazio vettoriale su $\RR$.
	Si definisce allora lo \textbf{spazio complessificato} $V_\CC = V \times V$ su $\CC$ con le seguenti operazioni:
	
	\begin{itemize}
		\item $(\v, \w) + (\v', \w') = (\v + \v', \w + \w')$,
		\item $(a+bi)(\v, \w) = (a\v - b\w, a\w + b\v)$.
	\end{itemize}
\end{definition}

\begin{remark}
	La costruzione dello spazio complessificato emula in realtà la costruzione di $\CC$ come spazio
	$\RR \times \RR$. Infatti se $z = (c, d)$, vale che $(a + bi)(c, d) = (ac - bd, ad + bc)$, mentre
	si mantiene l'usuale operazione di addizione. In particolare si può identificare l'insieme
	$V \times \zerovecset$ come $V$, mentre $\zerovecset \times V$ viene identificato come l'insieme
	degli immaginari $iV$ di $V_\CC$. Infine, moltiplicare per uno scalare reale un elemento di
	$V \times \zerovecset$ equivale a moltiplicare la sola prima componente con l'usuale operazione
	di moltiplicazione di $V$. Allora, come accade per $\CC$, si può sostituire la notazione
	$(\v, \w)$ con la più comoda notazione $\v + i \w$.
\end{remark}

\begin{remark}
	Sia $\basis = \{ \vv 1, \ldots, \vv n \}$ una base di $V$. Innanzitutto si osserva che
	$(a+bi)(\v, \vec 0) = (a\v, b\v)$. Pertanto si può concludere che $\basis \times \zerovecset$ è
	una base dello spazio complessificato $V_\CC$ su $\CC$. \\
	
	Infatti, se $(a_1 + b_1 i)(\vv 1, \vec 0) + \ldots + (a_n + b_n i)(\vv n, \vec 0) = (\vec 0, \vec 0)$,
	allora $(a_1 \vv 1 + \ldots + a_n \vv n, b_1 \vv 1 + \ldots + b_n \vv n) = (\vec 0, \vec 0)$.
	Poiché però $\basis$ è linearmente indipendente per ipotesi, l'ultima identità implica che
	$a_1 = \cdots = a_n = b_1 = \cdots = b_n = 0$, e quindi che $\basis \times \zerovecset$ è linearmente
	indipendente. \\
	
	Inoltre $\basis \times \zerovecset$ genera $V_\CC$. Se infatti $\v = (\U, \w)$, e vale che:
	
	\[ \U = a_1 \vv 1 + \ldots + a_n \vv n, \quad \w = b_1 \vv 1 + \ldots + b_n \vv n, \]
	
	\vskip 0.1in
	
	allora $\v = (a_1 + b_1 i) (\vv 1, \vec 0) + \ldots + (a_n + b_n i) (\vv n, \vec 0)$. Quindi
	$\dim V_\CC = \dim V$.
\end{remark}

\begin{definition}
	Sia $f$ un'applicazione $\CC$-lineare di $V$ spazio vettoriale su $\CC$. Allora
	si definisce la \textbf{restrizione su} $\RR$ di $f$, detta $f_\RR : V_\RR \to V_\RR$,
	in modo tale che $f_\RR(\v) = f(\v)$.
\end{definition}

\begin{remark}
	Sia $\basis = \{\vv 1, \ldots, \vv n\}$ una base di $V$ su $\CC$. Sia $A = M_\basis(f)$. Si
	osserva allora che, se $\basis' = \basis \cup i \basis$ e $A = A' + i A''$ con $A'$, $A'' \in M(n, \RR)$,
	vale la seguente identità:
	
	\[ M_{\basis'}(f_\RR) = \Matrix{ A' & \rvline & -A'' \\ \hline A'' & \rvline & A' }. \]
	
	Infatti, se $f(\vv i) = (a_1 + b_1 i) \vv 1 + \ldots + (a_n + b_n i) \vv n$, vale che
	$f_\RR(\vv i) = a_1 \vv 1 + \ldots + a_n \vv n + b_1 (i \vv 1) + \ldots + b_n (i \vv n)$,
	mentre $f_\RR(i \vv i) = i f(\vv i) = - b_1 \vv 1 + \ldots - b_n \vv n + a_1 (i \vv 1) + \ldots + a_n (i \vv n)$.
\end{remark}

\begin{definition}
	Sia $f$ un'applicazione $\RR$-lineare di $V$ spazio vettoriale su $\RR$. Allora
	si definisce la \textbf{complessificazione} di $f$, detta $f_\CC : V_\CC \to V_\CC$,
	in modo tale che $f_\CC(\v + i \w) = f(\v) + i f(\w)$.
\end{definition}

\begin{remark}
	Si verifica infatti che $f_\CC$ è $\CC$-lineare.
	\begin{itemize}
		\item $f_\CC((\vv1 + i \ww1) + (\vv2 + i \ww2)) = f_\CC((\vv1 + \vv2) + i (\ww1 + \ww2)) =
		f(\vv1 + \vv2) + i f(\ww1 + \ww2) = (f(\vv1) + i f(\ww1)) + (f(\vv2) + i f(\ww2)) =
		f_\CC(\vv1 + i\ww1) + f_\CC(\vv2 + i\ww2)$.
		
		\item $f_\CC((a+bi)(\v + i\w)) = f_\CC(a\v-b\w + i(a\w+b\v)) = f(a\v - b\w) + i f(a\w + b\v) =
		af(\v) - bf(\w) + i(af(\w) + bf(\v)) = (a+bi)(f(\v) + if(\w)) = (a+bi) f_\CC(\v + i\w)$.
	\end{itemize}
\end{remark}

\begin{proposition}
	Sia $f_\CC$ la complessificazione di $f \in \End(V)$, dove $V$ è uno spazio vettoriale su $\RR$.
	Sia inoltre $\basis = \{ \vv 1, \ldots, \vv n \}$ una base di $V$. Valgono allora i seguenti risultati:
	
	\begin{enumerate}[(i)]
		\item $\restr{(f_\CC)_\RR}{V}$ assume gli stessi valori di $f$,
		\item $M_\basis(f_\CC) = M_\basis(f) \in M(n, \RR)$,
		\item $M_{\basis \cup i \basis}((f_\CC)_\RR) = \Matrix{M_\basis(f) & \rvline & 0 \\ \hline 0 & \rvline & M_\basis(f)}$.
	\end{enumerate}
\end{proposition}

\begin{proof}Si dimostrano i risultati separatamente.
	\begin{enumerate}[(i)]
		\item Si osserva che $(f_\CC)_\RR(\vv i) = f_\CC(\vv i) = f(\vv i)$. Dal momento che
		$(f_\CC)_\RR$ è $\RR$-lineare, si conclude che $(f_\CC)_\RR$ assume gli stessi valori
		di $f$.
		
		\item Dal momento che $\basis$, nell'identificazione di $(\v, \vec 0)$ come $\v$, è
		sempre una base di $V_\CC$, e $f_\CC(\vv i) = f(\vv i)$, chiaramente
		$[f_\CC(\vv i)]_\basis = [f(\vv i)]_\basis$, e quindi $M_\basis(f_\CC) = M_\basis(f)$,
		dove si osserva anche che $M_\basis(f) \in M(n, \RR)$, essendo $V$ uno spazio vettoriale
		su $\RR$.
		
		\item Sia $f(\vv i) = a_1 \vv 1 + \ldots + a_n \vv n$ con $a_1$, ..., $a_n \in \RR$. Come
		osservato in (i), $\restr{(f_\CC)_\RR}{\basis} = \restr{(f_\CC)_\RR}{\basis}$, e quindi
		la prima metà di $M_{\basis \cup i \basis}((f_\CC)_\RR)$ è formata da due blocchi: uno
		verticale coincidente con $M_\basis(f)$ e un altro completamente nullo, dal momento che
		non compare alcun termine di $i \basis$ nella scrittura di $(f_\CC)_\RR(\vv i)$. Al
		contrario, per $i \basis$, $(f_\CC)_\RR(i \vv i) = f_\CC(i \vv i) = i f(\vv i) = a_1 (i \vv 1) +
		\ldots + a_n (i \vv n)$; pertanto la seconda metà della matrice avrà i due blocchi della prima metà,
		benché scambiati.
	\end{enumerate}
\end{proof}

\begin{remark}
	Dal momento che $M_\basis(f_\CC) = M_\basis(f)$, $f_\CC$ e $f$ condividono lo stesso polinomio caratteristico
	e vale che $\Sp(f) \subseteq \Sp(f_\CC)$, dove vale l'uguaglianza se e solo se tale polinomio caratteristico
	è completamente riducibile in $\RR$. Inoltre, se $V_\lambda$ è l'autospazio su $V$ dell'autovalore $\lambda$, l'autospazio
	su $V_\CC$, rispetto a $f_\CC$, è invece ${V_\CC}_\lambda = V_\lambda + i V_\lambda$, la cui
	dimensione rimane invariata rispetto a $V_\lambda$, ossia $\dim V_\lambda = \dim {V_\CC}_\lambda$
	(infatti, analogamente a prima, una base di $V_\lambda$ può essere identificata come base
	anche per ${V_\CC}_\lambda$).
\end{remark}

\begin{proposition}
	Sia $f_\CC$ la complessificazione di $f \in \End(V)$, dove $V$ è uno spazio vettoriale su $\RR$.
	Sia inoltre $\basis = \{ \vv 1, \ldots, \vv n \}$ una base di $V$. Allora un endomorfismo
	$\tilde g : V_\CC \to V_\CC$ complessifica un endomorfismo $g \in \End(V)$ $\iff$ $M_\basis(\tilde g) \in M(n, \RR)$.
\end{proposition}

\begin{proof}
	Se $\tilde g$ complessifica $g \in \End(V)$, allora, per la proposizione precedente,
	$M_\basis(\tilde g) = M_\basis(g) \in M(n, \RR)$. Se invece $A = M_\basis(\tilde g) \in M(n, \RR)$,
	si considera $g = M_\basis\inv(A) \in \End(V)$. Si verifica facilemente che $\tilde g$ non è altro che
	il complessificato di tale $g$:
	
	\begin{itemize}
		\item $\tilde g (\vv i) = g(\vv i)$, dove l'uguaglianza è data dal confronto delle matrici associate,
		e quindi $\restr{\tilde g}{V} = g$;
		\item $\tilde g(\v + i\w) = \tilde g(\v) + i \tilde g(\w) = g(\v) + i g(\w)$, da cui la tesi.
	\end{itemize}
\end{proof}

\begin{proposition}
	Sia $\varphi$ un prodotto scalare di $V$ spazio vettoriale su $\RR$. Allora esiste un
	unico prodotto hermitiano $\varphi_\CC : V_\CC \times V_\CC \to \CC$ che estende $\varphi$ (ossia tale che
	$\restr{\varphi_\CC}{V \times V} = \varphi$), il quale assume la stessa segnatura
	di $\varphi$.
\end{proposition}

\begin{proof}
	Sia $\basis$ una base di Sylvester per $\varphi$. Si consideri allora il prodotto
	$\varphi_\CC$ tale che:
	
	\[ \varphi_\CC(\vv1 + i\ww1, \vv2 + i\ww2) = \varphi(\vv1, \vv2) + \varphi(\ww1, \ww2) + i(\varphi(\vv1, \ww1) - \varphi(\ww1, \vv2)). \]
	
	Chiaramente $\restr{\varphi_\CC}{V \times V} = \varphi$. Si verifica allora che $\varphi_\CC$ è hermitiano:
	
	\begin{itemize}
		\item $\varphi_\CC(\v + i\w, (\vv1 + i\ww1) + (\vv2 + i\ww2))$ $= \varphi(\v, \vv1 + \vv2) + \varphi(\w, \ww1 + \ww2)$ $+ i(\varphi(\v, \ww1 + \ww2)$ $- \varphi(\w, \vv1 + \vv2))$ $= [\varphi(\v, \vv1) + \varphi(\w, \ww1) + i(\varphi(\v, \ww1) - \varphi(\w, \vv1))]$ $+ [\varphi(\v, \vv2) + \varphi(\w, \ww2) + i(\varphi(\v, \ww2) - \varphi(\w, \vv2))] = \varphi_\CC(\v + i\w, \vv1 + i\ww1) +
		\varphi_\CC(\v + i\w, \vv2 + i\ww2)$ (additività nel secondo argomento),
		
		\item $\varphi_\CC(\v + i\w, (a+bi)(\vv1 + i\ww1)) = \varphi_\CC(\v + i\w, a\vv1-b\ww1 + i(b\vv1+a\ww1)) =
		\varphi(\v, a\vv1-b\ww1) + \varphi(\w, b\vv1+a\ww1) + i(\varphi(\v, b\vv1+a\ww1) - \varphi(\w, a\vv1-b\ww1))=
		a\varphi(\v, \vv1) - b\varphi(\v, \ww1) + b\varphi(\w, \vv1) + a\varphi(\w, \ww1) + i(b\varphi(\v, \vv1) + a\varphi(\v, \ww1) - a\varphi(\w, \vv1) + b\varphi(\w, \ww1)) = a(\varphi(\v, \vv1) + \varphi(\w, \ww1)) -
		b(\varphi(\v, \ww1) - \varphi(\w, \vv1)) + i(a(\varphi(\v, \ww1) - \varphi(\w, \vv1)) + b(\varphi(\v, \vv1) + \varphi(\w, \ww1))) = (a+bi)(\varphi(\v, \vv1) + \varphi(\w, \ww1) + i(\varphi(\v, \ww1) - \varphi(\w, \vv1))) = (a+bi) \varphi_\CC(\v + \w, \vv1 + i\ww1)$ (omogeneità nel secondo argomento),
		
		\item $\varphi_\CC(\vv1 + i\ww1, \vv2 + i\ww2) = \varphi(\vv1, \vv2) + \varphi(\ww1, \ww2) + i(\varphi(\vv1, \ww2) - \varphi(\ww1, \vv2)) = \conj{\varphi(\vv1, \vv2) + \varphi(\ww1, \ww2) + i(\varphi(\ww1, \vv2) - \varphi(\vv1, \ww2))} = \conj{\varphi(\vv2, \vv1) + \varphi(\ww2, \ww1) + i(\varphi(\vv2, \ww1) - \varphi(\ww2, \vv1))} = \conj{\varphi_\CC(\vv2 + \ww2, \vv1 + \ww1)}$ (coniugio nello scambio degli argomenti).
	\end{itemize}
	
	Ogni prodotto hermitiano $\tau$ che estende il prodotto scalare $\varphi$ ha la stessa matrice associata nella
	base $\basis$, essendo $\tau(\vv i, \vv i) = \varphi(\vv i, \vv i)$ vero per ipotesi. Pertanto $\tau$ è
	unico, e vale che $\tau = \varphi_\CC$. Dal momento che $M_\basis(\varphi_\CC) = M_\basis(\varphi)$ è
	una matrice di Sylvester, $\varphi_\CC$ mantiene anche la stessa segnatura di $\varphi$.
\end{proof}
