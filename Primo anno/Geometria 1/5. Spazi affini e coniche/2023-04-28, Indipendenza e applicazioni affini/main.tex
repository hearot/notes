\documentclass[11pt]{article}
\usepackage{personal_commands}
\usepackage[italian]{babel}

\title{\textbf{Note del corso di Geometria 1}}
\author{Gabriel Antonio Videtta}
\date{28 aprile 2023}

\begin{document}
	
	\maketitle
	
	\begin{center}
		\Large \textbf{Indipendenza e applicazioni affini}
	\end{center}
	
	\begin{note} Qualora non specificato diversamente, si intenderà per
		$E$ uno spazio affine sullo spazio vettoriale $V$ e
		per $E'$ uno spazio affine sullo spazio vettoriale $V'$, dove sia $V$ che $V'$ sono costruiti
		sul campo $\KK$.
	\end{note}
	
	Fissata un'origine $O$ dello spazio affine, si possono sempre considerare due
	bigezioni:
	
	\begin{itemize}
		\item La bigezione $i_O : E \to V$ tale che $i(P) = P - O \in V$,
		\item La bigezione $j_O : V \to E$ tale che $j(\v) = O + \v \in E$.
	\end{itemize}
	
	Si osserva inoltre che $i_O$ e $j_O$ sono l'una la funzione inversa dell'altra.
	
	Dato uno spazio vettoriale $V$ su $\KK$ di dimensione $n$, si può considerare $V$ stesso
	come uno spazio affine, denotato  con le usuali operazioni:
	
	\begin{enumerate}[(a)]
		\item $\v + \w$, dove $\v \in V$ è inteso come $\mathit{punto}$ di $V$ e $\w \in W$ come
		il vettore che viene applicato su $\w$, coincide con la somma tra $\v$ e $\w$ (e analogamente
		$\w - \v$ è esattamente $\w - \v$).
		
		\item Le bigezioni considerate inizialmente sono in particolare due mappe tali che
		$i_{\vv 0}(\v) = \v - \vv 0$ e che $j_{\vv 0}(\v) = \vv 0 + \v$.
	\end{enumerate}
	
	\begin{definition} [spazio affine standard]
		Si denota con $\AnK$ lo \textbf{spazio affine standard} costruito sullo spazio vettoriale
		$\KK^n$. Analogamente si indica con $A_V$ lo spazio affine costruito su uno spazio
		vettoriale $V$.
	\end{definition}

	\begin{remark}\nl
		\li Una combinazione affine di $A_V$ è in particolare una combinazione lineare di $V$. Infatti,
		se $\v = \sum_{i=1}^n \lambda_i \vv i$ con $\sum_{i=1}^n \lambda_i = 1$, allora, fissato
		$\vv 0 \in V$, $\v = \vv 0 + \sum_{i=1}^n \lambda_i (\vv i - \vv 0) = \vv 0 + \sum_{i=1}^n \lambda_i \vv i - \vv 0 = \sum_{i=1}^n \lambda_i \vv i$.
		
		\li Come vi è una bigezione data dal passaggio alle coordinate da $V$ a $\KK^n$, scelta una base
		$\basis$ di $V$ e un punto $O$ di $E$, vi è anche una bigezione $\varphi_{O, \basis}$ da $E$ a $\AnK$ data
		dalla seguente costruzione:
		
		\[ \varphi_{O, \basis}(P) = [P-O]_\basis. \]
		
		\vskip 0.05in
	\end{remark}
	
	\begin{proposition}
		Sia $D \subseteq E$. Allora $D$ è un sottospazio affine di $E$ $\iff$ fissato $P_0 \in D$, l'insieme
		$D_0 = \{ P - P_0 \mid P \in D \} \subseteq V$ è un sottospazio vettoriale di $V$.
	\end{proposition}
	
	\begin{proof}
		Si dimostrano le due implicazioni separatamente. \\
		
		\rightproof Siano $\vv 1$, ..., $\vv k \in D_0$. Allora, per definizione, esistono $P_1$, ...,
		$P_k \in D$ tali che $\vv i = P_i - P_0$ $\forall 1 \leq i \leq k$. Siano $\lambda_1$, ...,
		$\lambda_k \in \KK$. Sia inoltre $P = P_0 + \sum_{i=1}^k \lambda_i \vv i \in E$. Sia infine
		$O \in D$. Allora $P = O + (P_0 - O) + \sum_{i=1}^k \lambda_i \vv i = O + (P_0 - O) + \sum_{i=1}^k \lambda_i (P_i - O + O - P_0) = O + (P_0 - O) + \sum_{i=1}^k \lambda_i (P_i - O) - \sum_{i=1}^k \lambda_i (P_0 - O) =
		O + (1-\sum_{i=1}^k \lambda_i) (P_0 - O) + \sum_{i=1}^k \lambda_i (P_i - O)$. In particolare $P$
		è una combinazione affine di $P_1$, ..., $P_k \in D$, e quindi, per ipotesi, appartiene a $D$. Allora
		$P - P_0 =  \sum_{i=1}^k \lambda_i \vv i \in D_0$. Poiché allora $D_0$ è chiuso per combinazioni lineari,
		$D_0$ è un sottospazio vettoriale di $V$. \\
		
		\leftproof Sia $P = \sum_{i=1}^k \lambda_i P_i$ con $\sum_{i=1}^k \lambda_i = 1$, con $P_1$, ..., $P_k \in D$ e
		$\lambda_1$, ..., $\lambda_k \in \KK$. Allora $P - P_0 = \sum_{i=1}^k \lambda_i (P_i - P_0) \in D_0$ per ipotesi, essendo combinazione lineare di elementi di $D_0$. Pertanto, poiché esiste un solo punto $P'$
		tale che $P' = P_0 + \sum_{i=1}^k \lambda_i (P_i - P_0)$, affinché $\sum_{i=1}^k \lambda_i (P_i - P_0)$
		appartenga a $D_0$, deve valere anche che $P \in D$. Si conclude quindi che $D$ è un sottospazio
		affine, essendo chiuso per combinazioni affini.
	\end{proof}
	
	\begin{remark}Sia $D$ un sottospazio affine di $E$. \\

		\li Vale la seguente identità $D_0 = \{ P - Q \mid P, Q \in D \}$. Sia infatti $A = \{ P - Q \mid P, Q \in D \}$. Chiaramente $D_0 \subseteq A$.
		Inoltre, se $P-Q \in A$, $P-Q = (P-P_0) - (Q-P_0)$. Pertanto, essendo $P-Q$ combinazione lineari di elementi
		di $D_0$, ed essendo $D_0$ spazio vettoriale per la proposizione precedente, $P-Q \in D_0 \implies A \subseteq D_0$, da cui si conclude che $D_0 = A$. \\
		\li Pertanto $D_0$ è unico, a prescindere dalla scelta di $P_0 \in D$. \\
		\li Vale che $D = P_0 + D_0$, ossia $D$ è il traslato di $D$ mediante il punto $P_0$.
	\end{remark}
	
	\begin{definition} [direzione di un sottospazio affine]
		Si definisce $D_0 = \Giac(D) = \{ P - Q \mid P, Q \in D \} \subseteq V$ come la \textbf{direzione} (o \textit{giacitura}) del sottospazio affine $D$.
	\end{definition}

	\begin{definition} [dimensione un sottospazio affine]
		Dato $D$ sottospazio affine di $E$, si dice dimensione di $D$,
		indicata con $\dim D$, la dimensione della sua direzione $D_0$, ossia
		$\dim D_0$. In particolare $\dim E = \dim V$.
	\end{definition}
	
	\begin{definition} [sottospazi affini paralleli]
		Due sottospazi affini si dicono \textbf{paralleli} se condividono
		la stessa direzione.
	\end{definition}

	\begin{remark}\nl
		\li I sottospazi affini di dimensione zero sono tutti i punti di $E$. \\
		\li I sottospazi affini di dimensione uno sono le \textit{rette affini},
		mentre quelli di dimensione due sono i \textit{piani affini}. \\
		\li Si dice \textit{iperpiano affine} un sottospazio affine di codimensione $1$,
		ossia di dimensione $n-1$. \\
		\li Due sottospazi affini sono paralleli se e solo se uno può
		essere ottenuto mediante una traslazione dell'altro sottospazio. \\
		\li Se $D = \Aff(P_1, \ldots, P_k)$ con $P_1$, ..., $P_k \in E$,
		i vettori $P_2 - P_1$, ..., $P_k - P_1$ generano $D_0$. Infatti,
		se $P - P_1 \in D_0$, con $P \in D$, esistono $\lambda_1$, ...,
		$\lambda_k \in \KK$ con $\sum_{i=1}^k \lambda_i = 1$ tali che
		$P = \sum_{i=1}^k \lambda_i P_i$. Allora $P-P_1 = \sum_{i=1}^k \lambda_i (P_i - P_1)$, da cui si deduce che tali vettori
		generano $D_0$.
	\end{remark}
	
	\begin{definition} [punti affinemente indipendenti]
		Un insieme di punti $P_1$, ..., $P_k$ di $E$ si dice \textbf{affinemente indipendente} se ogni
		combinazione affine di tali punti è unica. Analogamente un sottoinsieme $S \subseteq E$ si dice
		affinemente indipendente se ogni suo sottoinsieme finito lo è.
	\end{definition}
	
	\begin{proposition}
		Dati i punti $P_1$, ..., $P_k \in E$, sono equivalenti le seguenti affermazioni.
		
		\begin{enumerate}[(i)]
			\item $P_1$, ..., $P_k$ sono affinemente indipendenti,
				
			\item $\forall i \in \NN^+ \mid 1 \leq i \leq k$, $P_i \notin \Aff(P_1, \ldots, P_k)$,
				con $P_i$ escluso,

			\item $\forall i \in \NN^+ \mid 1 \leq i \leq k$ l'insieme di vettori
				$\{ P_j - P_i \mid 1 \leq j \leq k, j \neq i \}$ è linearmente indipendente,
				
			\item $\exists i \in \NN^+ \mid 1 \leq i \leq k$ per il quale l'insieme di vettori
				$\{ P_j - P_i \mid 1 \leq j \leq k, j \neq i \}$ è linearmente indipendente.
		\end{enumerate}
	\end{proposition}
	
	\begin{proof}
		Siano $P_1$, ..., $P_k$ affinemente indipendenti. Sia $i \in \NN^+ \mid 1 \leq i \leq k$.
		Allora chiaramente (i) $\iff$ (ii), dacché se $P_i$ appartenesse a $\Aff(P_1, \ldots, P_k)$, con
		$P_i$ escluso, si violerebbe l'unicità della combinazione affine di $P_i$, e analogamente se
		esistessero due combinazioni affini in diversi scalari dello stesso punto si potrebbe
		un punto $P_j$ con $1 \leq j \leq k$ come combinazione affine degli altri punti. \\
		
		Siano allora $\lambda_1$, ..., $\lambda_k \in \KK$, con $\lambda_i$ escluso, tali che:
		
		\[ \sum_{\substack{j = 1 \\ j \neq i}}^n \lambda_j (P_j - P_i) = \vec 0. \]
		
		Allora si può riscrivere $P_i$ nel seguente modo:
		
		\[ P_i = \left(1 - \sum_{\substack{j = 1 \\ j \neq i}}^n \lambda_j\right) P_i + \sum_{\substack{j = 1 \\ j \neq i}}^n \lambda_j P_j. \]
		
		\vskip 0.05in
		
		Dal momento che la scrittura di $P_i$ è unica per ipotesi, $\lambda_j = 0$ $\forall 1 \leq j \leq k$ con $j \neq i$, e dunque l'insieme di vettori $\{ P_j - P_i \mid 1 \leq j \leq k, j \neq i \}$ è linearmente
		indipendente, per cui (ii) \mbox{$\implies$} (iii). Analogamente si deduce anche che (iii) \mbox{$\implies$} (i) e che (iii) \mbox{$\implies$} (iv). Pertanto (i) \mbox{$\iff$} (ii) \mbox{$\iff$} (iii). \\
		
		Si assuma ora l'ipotesi (iv) e sia $t \in \NN^+ \mid 1 \leq t \leq k$ tale che $t \neq i$. Siano
		dunque $\lambda_1$, ..., $\lambda_k$, con $\lambda_t$ escluso, tale che:
		
		\[ \sum_{\substack{j = 1 \\ j \neq t}}^k \lambda_j (P_j - P_t) = \vec 0. \]
		
		Allora si può riscrivere la somma come:
		
		\[ \sum_{\substack{j = 1 \\ j \neq t}}^k \lambda_j (P_j - P_i) - \sum_{\substack{j = 1 \\ j \neq t}}^k \lambda_j (P_t - P_i) = \vec 0, \]
		
		\vskip 0.05in
		ossia come combinazione lineare dei vettori della forma $P_j - P_i$. Allora, poiché per ipotesi tali
		vettori sono linearmente indipendenti, vale che:
		
		\[ \system{\lambda_j = 0 & \se j \neq t \E j \neq i, \\  \sum_{\substack{j = 1 \\ j \neq t}}^k \lambda_j = 0 & \implies \lambda_i = 0.} \]
		
		Pertanto l'insieme di vettori $\{ P_j - P_t \mid 1 \leq j \leq k, j \neq t \}$ è linearmente indipendente,
		da cui vale che (iv) \mbox{$\implies$} (iii). Si conclude dunque che
		(i) \mbox{$\iff$} (ii) \mbox{$\iff$} (iii) \mbox{$\iff$} (iv), ossia la tesi.
	\end{proof}

	\begin{remark}\nl
		\li Si osserva che il numero massimo di punti affinemente indipendenti di un sottospazio affine $D$
		di dimensione $k$ è $k+1$, dacché, fissato un punto, vi possono essere al più $k$ vettori linearmente indipendenti. \\
		\li Un punto di $E$ è sempre affinemente indipendente, dacché la sua unica combinazione affine è
		sé stesso. \\
		\li Due punti di $E$ sono affinemente indipendenti se e solo
		se il vettore che li congiunge è non nullo. \\
		\li Se $P_1$, ..., $P_k$ sono punti affinemente indipendenti,
		allora $\dim \Aff(P_1, \ldots, P_k) = k-1$. Infatti esistono
		almeno $k-1$ vettori linearmente indipendenti nella direzione
		di questo sottospazio affine, ed esattamente $k-1$ vettori
		generano tale direzione.
	\end{remark}
	
	\begin{definition} [riferimento affine] Sia $D \subseteq E$ un sottospazio affine di $E$ di dimensione $k-1$.
		Siano i punti $P_1$, ..., $P_k$ dei punti affinemente indipendenti. Allora
		si dice che tali punti formano un \textbf{riferimento affine} di $D$.
	\end{definition}
	
	\begin{definition} [coordinate affini] Sia $D \subseteq E$ un sottospazio affine di $E$ di dimensione $k-1$
		e siano i punti $P_1$, ..., $P_k$ un riferimento affine $R$ di $D$. Allora, se $P = \lambda_1 P_1 + \ldots + \lambda_k P_k \in D$ con $\lambda_1 + \ldots + \lambda_k = 1$, si dice che le \textbf{coordinate affine}
		di $P$ sono rappresentate dal punto $[P]_\basis$, dove:
		
		\[ [P]_R = \Vector{\lambda_1 \\ \vdots \\ \lambda_k} \in \AnK. \]
	\end{definition}
	
	\begin{remark}\nl
		\li Esiste sempre un riferimento affine di un sottospazio affine $D$ di $E$. Infatti, dato un punto $P_1$
		di $E$, e una base $\basis = \{\vv 1, \ldots, \vv k\}$ della direzione $D_0$, i punti $P_1$, $P_1 + \vv 1$, ...,
		$P_1 + \vv k$ formano un riferimento affine. \\
		\li Dalla definizione sopra si deduce che, scelto un riferimento affine $R$, esiste una mappa iniettiva $[\cdot]_R : D \to \AnK$, dove l'immagine di $P$ mediante $[\cdot]_R$ è esattamente il vettore
		contenente le coordinate affini di $P$.  
	\end{remark}
	
	\begin{proposition}
		Sia $E = \AnK$. Allora i punti $P_1$, ..., $P_k$ sono affinemente indipendenti se e solo
		se i vettori $\hat P_1 = \Vector{P_1 \\ \hline 1}$, ..., $\hat P_k = \Vector{P_k \\ \hline 1}$ sono
		linearmente indipendenti.
	\end{proposition}
	
	\begin{proof}
		Si dimostrano le due implicazioni separatamente. \\
		
		\rightproof Siano $\lambda_1$, ..., $\lambda_k \in \KK$ tali che
		$\lambda_1 \hat P_1 + \ldots + \lambda_k \hat P_k = \vec 0$. Allora
		$\sum_{i=1}^k \lambda_i = 0$ e $\lambda_1 P_1 + \ldots + \lambda_k P_k = 0$. \\
		
		Pertanto, sapendo che $\lambda_1 = - \lambda_2 + \ldots - \lambda_k$, vale
		la seguente identità:
		
		\[ \lambda_2 (P_2 - P_1) + \ldots + \lambda_k (P_k - P_1) = 0. \]
		
		\vskip 0.05in
		
		Poiché i punti $P_1$, ..., $P_k$ sono affinemente indipendenti, per la proposizione precedente,
		allora i vettori $P_2 - P_1$, ..., $P_k - P_1$ sono linearmente indipendenti, per cui $\lambda_2 = \cdots = \lambda_k = 0$. Pertanto anche $\lambda_1 = 0$, e quindi i vettori $\hat P_1$, ..., $\hat P_k$ sono
		linearmente indipendenti. \\
		
		\leftproof Siano $\lambda_2$, ..., $\lambda_k \in \KK$ tali che
		$\lambda_2 (P_2 - P_1) + \ldots + \lambda_k (P_k - P_1) = 0$. Sia allora
		$\lambda_1 = - \lambda_2 + \ldots - \lambda_k$. Si osserva dunque
		che $\lambda_1 + \ldots + \lambda_k = 0$ e che $\lambda_1 P_1 + \ldots + \lambda_k P_k = 0$,
		da cui si deduce che $\lambda_1 \hat P_1 + \ldots + \lambda_k \hat P_k = 0$. Dal momento
		però che $\hat P_1$, ..., $\hat P_k$ sono linearmente indipendenti, $\lambda_2 = \cdots = \lambda_k = 0$,
		da cui la tesi, per la proposizione precedente.
	\end{proof}
	
	\begin{definition} [combinazione convessa]
		Si dice che una combinazione affine $\sum_{i=1}^k \lambda_i P_i$ nei punti $P_1$, ..., $P_k$ con
		$\sum_{i=1}^k \lambda_i = 1$ è
		una \textbf{combinazione convessa} se $\lambda_i \geq 0$ $\forall 1 \leq i \leq k$.
	\end{definition}
	
	\begin{definition} [baricentro]
		Si definisce \textbf{baricentro} $G_S$ dei punti $P_1$, ..., $P_k$,
		che compongono l'insieme $S \subseteq E$, la combinazione convessa
		$\sum_{i=1}^k \frac{1}{k} P_i$.
	\end{definition}

	\begin{definition} [inviluppo convesso] Si definisce l'\textbf{inviluppo complesso} $\IC(S)$ di un insieme
		$S \subseteq E$ l'insieme delle combinazioni convesse finite di $S$.
	\end{definition}

	\begin{remark}\nl
		\li L'insieme $\IC(S)$ è, effettivamente, un insieme convesso, se $S \subseteq E$. Se infatti $P$, $Q \in \IC(S)$,
		allora $\lambda_1 P + \lambda_2 Q \in \IC(S)$, con $\lambda_1$, $\lambda_2 \geq 0$, e quindi
		$[P, Q] \subseteq \IC(S)$. \\
		\li Se $E = \Aa_2(\RR)$, e $P_1$, $P_2$, $P_3$ sono tre punti di $E$, l'inviluppo convesso dei
		tre punti è esattamente il triangolo costruito sui tre punti. Analogamente, presi quattro
		punti di $\Aa_3(\RR)$, l'inviluppo convesso dei quattro punti è un tetraedro. \\
		\li Se $A = B \sqcup C \subseteq E$ (ossia se $A = B \cup C$ con
		$B \cap C = \emptyset$), si osserva che $G_A = \frac{\abs{B}}{\abs{A}} G_B + \frac{\abs{C}}{\abs{A}} G_C$. Infatti, se $B_1$, ..., $B_{\abs B}$ sono i punti di $A$ appartenenti a $B$ e $C_1$, ..., $C_{\abs C}$ sono
		quelli appartenenti a $C$, $G_A = \sum_{i=1}^{\abs B} \frac{1}{\abs{A}} B_i + \sum_{i=1}^{\abs C} \frac{1}{\abs{A}} C_i = \frac{\abs{B}}{\abs A} \sum_{i=1}^{\abs B} \frac{1}{\abs{B}} B_i + \frac{\abs{C}}{\abs A} \sum_{i=1}^{\abs C} \frac{1}{\abs{C}} C_i = \frac{\abs{B}}{\abs{A}} G_B + \frac{\abs{C}}{\abs{A}} G_C$. \\
		\li In $\Aa_2(\RR)$, il baricentro tra tre punti affinemente
		indipendenti è esattamente il baricentro del loro inviluppo
		convesso, ossia del triangolo formato da questi punti. Infatti,
		se $S = \{ P_1, P_2, P_3 \}$, $G_S = \frac{1}{3} P_1 + \frac{1}{3} P_2 +
		\frac{1}{3} P_3$. Inoltre, per l'osservazione precedente, si può
		scrivere il baricentro di questo triangolo come una combinazione
		convessa del punto medio di due punti e del terzo punto non
		considerato, ossia $G_S = \frac{2}{3} \left( \frac{1}{2} P_i + \frac{1}{2} P_j \right) + \frac{1}{3} P_k$. Pertanto il baricentro
		di un triangolo è l'intersezione di tutte e tre le mediane di
		tale triangolo. Se si dota il piano della misura euclidea si deduce
		anche che il segmento che congiunge il baricentro al
		punto medio è la metà del segmento che congiunge il baricentro
		al terzo punto.
	\end{remark}
	
	\begin{definition} [applicazione affine] Si definisce \textbf{applicazione affine} da $E$ a $E'$ un'applicazione $\varphi : E \to E'$
		che conservi le combinazioni affini, ossia tale che:
		
		\[ \varphi\left( \sum_{i=1}^k \lambda_i P_i \right) = \sum_{i=1}^k \lambda_i \, \varphi(P_i), \quad \se \sum_{i=1}^k \lambda_i = 0. \]
	\end{definition}
	
	\begin{remark}\nl
		\li Come per le applicazioni lineari, la somma e la composizione di più applicazioni affini è
		ancora una applicazione affine. \\
		\li Se si sceglie un riferimento affine di $E$, $\varphi$ è univocamente determinata
		da come agisce su tale riferimento.
	\end{remark}
	
	\begin{theorem}
		Sia $\varphi : E \to E'$ un'applicazione affine. Allora esiste un'unica applicazione lineare $g : V \to V'$
		tale per cui $\varphi(P) = \varphi(O) + g(P-O)$ $\forall P \in E$, invariante per la scelta di $O \in E$.
	\end{theorem}
	
	\begin{proof}
		Sia $O \in E$. Si consideri l'applicazione $g : V \to V'$ tale per cui $g(\v) = \varphi(O + \v) - \varphi(O)$.
		Si verifica che $g$ è lineare:
		
		\begin{itemize}
			\item $g(\v + \w) = \varphi(O + \v + \w) - \varphi(O) = \varphi((O + \v) + (O + \w) - O) - \varphi(O) =
			\varphi(O + \v) - \varphi(O) + \varphi(O + \w) - \varphi(O) = g(\v) + g(\w)$ (additività),
			\item $g(a\v) = \varphi(O + a\v) - \varphi(O) = \varphi(a(O + \v) + (1-a)O) - \varphi(O) =
			a\varphi(O + \v) + (1-a)\varphi(O) - \varphi(O) = ag(\v)$ (omogeneità).
		\end{itemize}
		
		Inoltre, $\varphi(P) = \varphi(O + P - O) = \varphi(O) + \varphi(P) - \varphi(O) = \varphi(O) + g(P-O)$. Si
		osserva infine che $g$ è unica per costruzione. Si
		verifica allora che scegliendo $O' \in E$ al posto di $O$, la costruzione di $g$ è invariante, ossia
		che $\varphi(O' + \v) - \varphi(O') = \varphi(O + \v) - \varphi(O)$ $\forall \v \in V$. Infatti
		$\varphi(O' + \v) - \varphi(O') = \varphi(O' - O + (O + \v)) - \varphi(O') =
		\varphi(O') - \varphi(O) + \varphi(O + \v) - \varphi(O') = \varphi(O + \v) - \varphi(O)$, da cui
		la tesi.
	\end{proof}
	
	\begin{remark}
		Data un'applicazione lineare $g$ da $V$ in $V'$ e dati $O \in E$, $O' \in E$, si può sempre costruire un'applicazione affine $\varphi$ tale che $\varphi(P) = O' + g(P - O)$. Infatti, se $\sum_{i=1}^n \lambda_i = 1$,
		$\varphi\left( \sum_{i=1}^n \lambda_i P_i \right) = O' + g\left( \sum_{i=1}^n \lambda_i (P_i - O) ) \right) =
		O' + \sum_{i=1}^n \lambda_i \, g(P_i - O) = O' + \sum_{i=1}^n \lambda_i \, (\varphi(P_i) - O') = \sum_{i=1}^n \lambda_i \, \varphi(P_i)$.
	\end{remark}
	
	\begin{definition} [applicazione lineare associata ad un'applicazione affine]
		Data un'applicazione affine $\varphi : E \to E'$ e dato $O \in E$, si definisce $g : V \to V'$ tale che
		$g(\v) = \varphi(O + \v) - \varphi(O)$ come l'\textbf{applicazione lineare associata a $\varphi$} (detta anche \textit{differenziale} di $f$, ed indicata con $df$).
	\end{definition}
	
	\begin{remark}\nl
		\li Siano $E = \AnK$ ed $E' = \Aa_m(\KK)$. Allora, se $\varphi$ è un'applicazione affine da $E$ a $E'$,
		$\varphi(\vec x) = \varphi(\vec 0) + g(\vec x - \vec 0) = A \vec x + \vec b$ $\forall \vec x \in E$, dove $A$ è la matrice associata
		di $g$ nelle basi canoniche di $\KK^n$ e $\KK^m$ e $\vec b = \varphi(\vec 0)$. \\

		\li Sia $E''$ un altro spazio affine costruito su un altro spazio
		vettoriale $V''$, sempre fondato sul campo $\KK$. Se dunque $g$ e $g'$ sono le applicazioni lineari associate alle applicazioni affini $\varphi : E \to E'$ e $\varphi' : E' \to E''$,
		allora $g \circ g'$ è l'applicazione lineare associata a $\varphi \circ \varphi'$ e
		$\varphi + \varphi'$. Infatti, se $O \in E$, $\varphi(\varphi'(P)) = \varphi(\varphi'(O) + g'(P-O)) =
		\varphi(\varphi'(O)) + g(g'(P-O))$.
	\end{remark}
	
	\begin{definition} [isomorfismo affine] Un'applicazione affine da $E$ in $E'$ si dice \textbf{isomorfismo affine} se è bigettiva.
	\end{definition}
	
	\begin{definition} [affinità] Un'applicazione affine da $E$ in $E$ si dice \textbf{affinità} se è un isomorfismo affine.
	\end{definition}
	
	\begin{remark}
		Affinché un'applicazione affine sia un'affinità è necessario e sufficiente che la sua applicazione
		lineare sia invertibile. Infatti, se $\varphi : E \to E$ è un'applicazione affine e l'applicazione
		lineare associata $g : V \to V'$ è invertibile, allora $\varphi(P) = \varphi(Q) \implies \varphi(O) + g(P - O) = \varphi(O) + g(Q - O) \implies g(P-O) = g(Q-O) \implies P-O = Q-O \implies P=Q$ (iniettività), e $\forall P \in E$,
		$\varphi(O + g\inv(P-\varphi(O))) = \varphi(O) + g(g\inv(P-\varphi(O))) = P$ (surgettività). Analogamente
		si dimostra il viceversa.
	\end{remark}

	\begin{remark}
		Se $\varphi : E \to E$ è un'affinità, anche il suo inverso $\varphi\inv$
		lo è. Dacché $\varphi\inv$ è già bigettiva, è sufficiente mostra
		che è anche un'applicazione affine. Siano allora $\lambda_1$, ...,
		$\lambda_k \in \KK$ tali che $\sum_{i=1}^k \lambda_i = 1$. Siano
		inoltre $P_1$, ..., $P_k$ punti di $E$. Allora, poiché $\varphi$
		è un'affinità, esistono $Q_1 = \varphi\inv(P_1)$, ..., $Q_k = \varphi\inv(P_k) \in E$ tali che
		$\varphi\left( \sum_{i=1}^k \lambda_i Q_i \right) = \sum_{i=1}^k \lambda_i P_i$. Allora $\varphi\inv\left( \sum_{i=1}^k \lambda_i P_i \right) = \varphi\inv\left(\varphi\left( \sum_{i=1}^k \lambda_i Q_i \right)\right) = \sum_{i=1}^k \lambda_i \, \varphi\inv(P_i)$. \\
		
		In particolare, se $g \in \End(V)$ è l'applicazione lineare associata
		a $\varphi$, $g\inv$ è l'applicazione lineare associata a $\varphi\inv$.
		Sia infatti $f \in \End(V)$ è l'applicazione lineare associata
		a $\varphi\inv$. Dal momento che $\varphi\inv(\varphi(O + \v)) = O + \v$
		e che $\varphi\inv(\varphi(O + \v)) = \varphi\inv(\varphi(O) + g(\v)) =
		\varphi\inv(\varphi(O)) + f(g(\v)) = O + f(g(\v))$, deve valere infatti
		che $f(g(\v)) = \v$ $\forall \v \in V$, ossia $f \circ g = \Idv \implies f = g\inv$.
	\end{remark}
	
	\begin{definition} [gruppo delle affinità di uno spazio affine] Si indica con $A(E)$ il gruppo,
		mediante l'operazione di composizione, delle affinità di $E$.
	\end{definition}
	
	\begin{remark}\nl
		\li Un esempio notevole di affinità è la \textbf{traslazione} $\tau_{\v} : E \to E$ tale che
		$\tau_{\v}(Q) = Q + \v$, dove $\v \in V$. In particolare l'applicazione associata a tale
		affinità è l'identità. Infatti, se $O \in E$, $g(\v) = \tau_{\v}(O + \v) - \tau_{\v}(O) = (O + 2\v) - (O + \v) = \v$. \\
		\li L'applicazione $\zeta : A(E) \to \GL(V)$ che associa ad un'affinità l'applicazione ad essa
		associata è un epimorfismo di gruppi. Infatti, dato un endomorfismo invertibile di $V$, vi
		si può costruire sopra, come visto prima, un'affinità. Inoltre vale che $\zeta(f \circ f') = \zeta(f) \circ \zeta(f')$, per $f$, $f' \in A(E)$. \\
		\li Vale che $\Ker \zeta$ è esattamente il sottogruppo normale di $A(E)$ delle traslazioni, dal momento
		che sono le uniche affinità la cui applicazione lineare associata è l'identità.
	\end{remark}
\end{document}
