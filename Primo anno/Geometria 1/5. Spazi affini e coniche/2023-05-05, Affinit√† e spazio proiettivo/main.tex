\documentclass[11pt]{article}
\usepackage{personal_commands}
\usepackage[italian]{babel}

\title{\textbf{Note del corso di Geometria 1}}
\author{Gabriel Antonio Videtta}
\date{5 maggio 2023}

\begin{document}
	
	\maketitle
	
	\begin{center}
		\Large \textbf{Affinità e spazio proiettivo}
	\end{center}
	
	\begin{note}
		Qualora non specificato diversamente, con $E$ si indicherà un
		generico spazio affine di dimensione $n$ su cui agisce lo
		spazio vettoriale $V$.
	\end{note}
	
	Sia $f$ un'applicazione affine di $E$. Allora, per ogni $O \in E$, $\v \in V$,
	$f(O + \v) = f(O) + g(\v)$, dove $g \in \End(V)$ è l'applicazione lineare
	associata ad $f$. Pertanto $f(O + \v) = O + (f(O) - O) + g(\v)$, ossia
	$f$ è una traslazione di vettore $f(O) - O$ composta ad un'applicazione
	lineare. \\
	
	In particolare, passando alle coordinate rispetto al punto $O$ e una
	base $\basis$ di $V$, si può riscrivere $[f(P)]_{O, \basis}$ secondo
	la seguente identità:
	
	\[ [f(P)]_{O, \basis} = \underbrace{[f(O) - O]_{\basis}}_{\vec b} + \underbrace{[g(P - O)]_{\basis}}_{A [\v]_\basis} = A [P - O]_\basis + \vec b, \]
	
	dove $A = M_\basis(g)$. In particolare, in $\AnK$, scegliendo $O = \vec 0$ come origine e la base canonica
	come base $\basis$, si ottiene che:
	
	\[ f(\v) = A \v + \vec b, \]
	
	per ogni $\v \in \AnK$. Se $f \in A(E)$, allora vale anche che:
	\[ f\inv(O + \w) = f\inv(f(O) + (O - f(O)) + \w) = O - g\inv(f(O) - O) + g\inv(\w), \]
	
	dove si è usato che $g$ è invariante per cambiamento del punto d'origine $O$. Pertanto,
	in questo caso, passando alle coordinate, vale che:
	
	\[ [f\inv(P)]_{O, \basis} = A\inv [P - O]_\basis - A\inv \vec b. \]
	
	Considerando questa identità in $\AnK$, risulta che:
	
	\[ f\inv(\vec v) = A\inv \vec v - A\inv \vec b, \]
	
	per ogni $\v \in \AnK$.
	
	\hr \vskip 0.1in

	Sia $\iota : \AnK \to H_{n+1}$ l'applicazione che associa $\vec x$ a $\Vector{\vec x \\ 1} \in H_{n+1}$,
	dove vale che:
	
	\[ H_{n+1} = \left\{ \Vector{x_1 \\ \vdots \\ x_{n+1}} \;\middle\vert\; x_{n+1} = 1 \right\}, \]
	
	\vskip 0.15in
	
	ossia l'iperpiano affine di $\Aa_{n+1}(\KK)$ dei vettori con l'ultima coordinata pari a $1$. Per comodità
	si indica $\iota(\x)$ con $\hat \x$.
	
	\begin{proposition}
		$\iota$ è un'isomorfismo affine.
	\end{proposition}
	
	\begin{proof}
		Si verifica innanzitutto che $\iota$ è un'applicazione affine. Siano $\lambda_1$, ..., $\lambda_k \in \KK$ tali
		che $\sum_{i=1}^k \lambda_i = 1$, e siano $\xx 1$, ..., $\xx k \in E$. Allora vale che:
		
		\[ \iota\left( \sum_{i=1}^k \lambda_i \xx i \right) = \Vector{\sum_{i=1}^k \lambda_i \xx i \\ 1} = \Vector{\sum_{i=1}^k \lambda_i \xx i \\ \sum_{i=1}^k \lambda_i} = \sum_{i=1}^k \lambda_i \, \iota(\xx i). \]
		
		\vskip 0.05in
		
		Si consideri\footnote{Per concludere in modo più diretto la dimostrazione è sufficiente anche esibire l'inverso di $g$, ottenuto ignorando l'ultima coordinata di un vettore di $H_{n+1}$.} ora l'applicazione lineare $g$ associata a $\iota$. Allora, posto $O = \vec 0$, $g(\v) = f(O + \v) - f(O) =
		f(\v) - f(\vec 0) = f(\v) - \Vector{0 & \cdots & 0 & 1}^\top$. Dal momento che la direzione di $H_{n+1}$ è
		$n$-dimensionale (scegliendo $O$ come origine, tutti i vettori ottenibili scartano l'ultima coordinata, sempre
		pari a $0$), $g$ mappa due spazi vettoriali di stessa dimensione. \\
		
		Pertanto, è sufficiente dimostrare che $g$ è surgettiva affinché sia invertibile (e dunque $\iota$ sia un isomorfismo affine). Chiaramente $g$ è surgettiva, dal momento che ad ogni vettore $\hat \v = \Matrix{\v & 0} \in \Giac(H_{n+1})$ è tale che $g(\v) = \hat \v$. Si conclude dunque che $g$ è invertibile, e che $\iota$ è
		un isomorfismo affine.
	\end{proof}

	\begin{proposition}
		Sia $f \in A(\AnK)$ e sia $f' = \iota \circ f \circ \iota\inv \in A(H_{n+1})$
		l'identificazione di $f$ in $H_{n+1}$. Allora si può estendere $f'$ ad un'applicazione lineare invertibile $\hat f$ di $\KK^{n+1}$ (ossia ad un'applicazione $\hat f$ tale per cui $\restr{\hat f}{H_{n+1}} = f'$). Viceversa, data un'applicazione lineare invertibile $g \in \End(\KK^{n+1})$ tale che $\restr{g}{H_{n+1}} = H_{n+1}$, allora la restrizione $\restr{g}{H_{n+1}}$ è un'affinità di $H_{n+1}$ ed
		induce un'affinità $f$ di $\AnK$ in modo tale che $f = \iota\inv \circ \restr{g}{H_{n+1}} \circ \iota$. \\
		
		In particolare, una tale $\hat f$ è tale che $\hat f(\x') = A' \x'$ $\forall \x' \in \KK^{n+1}$, dove
		vale che:
		
		\[ A' = \Matrix{ A & \rvline & \vec b \, \\ \hline 0 & \rvline & 1 \, }, \qquad f(\v) = A \v + \vec b \quad \forall \v \in \AnK. \]
	\end{proposition}
	
	\begin{proof}
		Si consideri $\hat f \in \End(\KK^{n+1})$ tale che $\hat f(\x') = A' \x'$. $\hat f$ è invertibile dal
		momento che $A'$ lo è. Infatti vale che:
		
		\[ (A')\inv = \Matrix{ A\inv & \rvline & -A\inv \, \vec b \, \\ \hline 0 & \rvline & 1 \, }. \]
		
		\vskip 0.05in
		
		Sia $\hat x = \Vector{\x & 1}^\top \in H_{n+1}$. Sia ora $\hat x \in H_{n+1}$. Allora $\hat f(\hat x) = \Vector{A \x + \vec b & 1}^\top = \Vector{f(\x) & 1}^\top = \iota(f(\vec x)) = \iota(f(\iota\inv(\hat x))) = f'(\hat x) \in H_{n+1}$ $\forall \hat x \in H_{n+1}$. Pertanto $\restr{\hat f}{H_{n+1}} = f'$. \\
		
		Si consideri adesso $g \in \GL(\KK^{n+1})$ tale che $\restr{g}{H_{n+1}} = H_{n+1}$. Sia $A'$ tale che
		$g(\x') = A' \x'$ $\forall \x' \in \KK^{n+1}$. Poiché $\restr{g}{H_{n+1}} = H_{n+1}$, allora
		$(A')_{n+1,n+1} = g(\e{n+1})_{n+1} = 1$. Poiché $g(\e n + \e{n+1})_{n+1} = 1$, allora $(A')_{n+1,n} = 0$.
		In particolare, partendo da $j=n$ fino a $j=1$, si deduce, per induzione, che $g(\e j + \ldots + \e{n+1})_{n+1} = 1 \implies (A')_{n+1,j} = 0$. \\
		
		Allora $A'$ è della seguente forma:
		
		\[ A' = \Matrix{ A & \rvline & \vec b \, \\ \hline 0 & \rvline & 1 \, }, \quad A \in M(n, \KK), \, \vec b \in \KK^n. \]
		
		\vskip 0.05in
		
		Considerando allora l'applicazione affine $f \in \AnK$ tale che $f(\v) = A \v + \vec b$,
		$g$ è l'applicazione lineare invertibile che estende $f' = \iota \circ f \circ \iota\inv$, come
		visto prima, da cui la tesi.
	\end{proof}
	
	\begin{remark}
		Le matrici della forma:
		
		\[ \Matrix{ A & \rvline & \vec b \, \\ \hline 0 & \rvline & 1 \, }, \quad A \in M(n, \KK), \, \vec b \in \KK^n, \]
		
		\vskip 0.05in
		
		formano un sottogruppo di $(M(n+1, \KK), \cdot)$ canonicamente isomorfo a $A(\AnK)$. In particolare si osserva che un'affinità dipende
		da esattamente $n^2 + n$, dove $n^2$ sono i parametri su cui
		si basa $A$, e $n$ sono i parametri su cui si basa $\vec b$. \\
		
		Se $D \subseteq E$ è un sottospazio affine di $E$, l'insieme
		$T = \{ f \in A(E) \mid f(D) = D \}$ forma un sottogruppo di $(A(E), \circ)$. In particolare, se $\dim D = k$, un'affinità di $T$
		dipende da esattamente $(k+1)k + (n-k)n$ parametri. \\
		
		Infatti in tal caso, scegliendo una base opportuna di $D_0$, estesa
		poi a base di $E_0$, e riferendosi ad un'origine di $D$,
		$A$ conterrà un blocco $k^2$ relativo alle immagini della base
		di $D$ ed un blocco $(n-k)n$ relativo alle immagini degli altri
		vettori, non appartenenti a $D$. Inoltre dovranno essere scelti
		i parametri riguardanti il vettore $\vec b$, che, essendo stato
		scelto come riferimento un punto d'origine appartenente a $D$,
		richiederà la scelta di $k$ parametri.
	\end{remark}
	
	\hr
	
	\begin{definition} [spazio proiettivo]
		Si definisce lo \textbf{spazio proiettivo} $\PP(\KK^{n+1}) = \PP^n(\KK)$ come l'insieme
		dei sottospazi di dimensione unitaria di $\KK^{n+1}$.
	\end{definition}
	
	\begin{remark}
		Se si definisce la relazione di equivalenza $\sim$ su $V$ in modo tale che $\x \sim \y \defiff \exists \alpha \in \KK^* \mid \x = \alpha \y$, $V \quot \sim$ è in bigezione con lo spazio proiettivo. In particolare,
		ogni elemento di $V \quot \sim$ è un unico elemento dello spazio proiettivo a cui è stato tolto il vettore $\vec 0$.
	\end{remark}
	
	\begin{remark}
		Ogni elemento $\hat x = \Vector{\x & 1}^\top$ di $H_{n+1}$ identifica un unico elemento dello spazio proiettivo, ossia $\Span(\hat x)$, dal momento che due vettori di $H_{n+1}$ appartengono alla stessa retta se e solo se
		sono linearmente dipendenti, ossia se sono uguali. \\
		
		Gli elementi di $\PP^n(\KK)$ che non contengono elementi di $H_{n+1}$ sono esattamente i sottospazi
		contenenti vettori la cui ultima coordinata è nulla. Pertanto questi elementi, detti \textbf{punti all'infinito}
		di $\PP^n(\KK)$, si possono identificare in particolare come elementi di $\PP^{n+1}(\KK)$. \\
	\end{remark}
	
	\begin{remark}
		Si può ricoprire $\PP^n(\KK)$ con iperpiani analoghi ad $H_{n+1}$, ossia con gli iperpiani della
		seguente forma:
		
		\[ T_i = \left\{ \Vector{x_1 \\ \vdots \\ x_{n+1}} \;\middle\vert\; x_i = 1 \right\}. \]
		
		\vskip 0.1in
		
		Ogni elemento di $\PP^n(\KK)$ interseca infatti almeno uno di questi iperpiani, dacché in esso deve
		esistervi obbligatoriamente un vettore non nullo. In particolare, se esiste un'intersezione tra $T_i$
		e un elemento di $\PP^n(\KK)$, questa è unica.  
	\end{remark}

	\hr
	
	\begin{theorem}
		Sia $E$ uno spazio affine sullo spazio $V$ di dimensione $n$. Allora
		valgono i seguenti due risultati.
		
		\begin{enumerate}[(i)]
			\item Se $f \in A(E)$ e i punti $P_1$, ..., $P_k$ sono affinemente
			indipendenti, allora anche i punti $f(P_1)$, ..., $f(P_k)$ sono
			affinemente indipendenti.
			
			\item Se i punti $P_1$, ..., $P_{n+1}$ sono affinemente indipendenti, e lo sono anche i punti $Q_1$, ..., $Q_{n+1}$,
			allora esiste un'unica affinità
			$f \in A(E)$ tale che $f(P_i) = Q_i$ $\forall 1 \leq i \leq n+1$. 
		\end{enumerate}
	\end{theorem}

	\begin{proof}
		Si dimostrano i due risultati separatamente.
		
		\begin{enumerate}[(i)]
			\item Poiché $f \in A(E)$, allora $g \in \GL(V)$, ed è
			dunque invertibile. Si considerino i vettori $f(P_i) - f(P_1) = g(P_i - P_1)$
			con $2 \leq i \leq k$. Dal momento che è invertibile,
			$g$ mappa vettori linearmente indipendenti a vettori
			ancora linearmente indipendenti.
			
			Allora, poiché i punti
			$P_1$, ..., $P_k$ sono affinemente indipendenti, i
			vettori $P_i - P_1$ sono linearmente indipendenti per
			$2 \leq i \leq k$. Pertanto anche i vettori $g(P_i - P_1) = f(P_i) - f(P_1)$ con $2 \leq i \leq k$ sono linearmente indipendenti, da cui si conclude che i punti $f(P_1)$, ...,
			$f(P_k)$ sono affinemente indipendenti.
			
			\item Dal momento che i punti $P_1$, ..., $P_{n+1}$ sono
			affinemente indipendenti, allora i vettori $P_i - P_1$ con
			$2 \leq i \leq n+1$ sono linearmente indipendenti, e formano
			dunque una base di $V$, essendo tanti quanti la dimensione
			di $V$. Analogamente anche i vettori $Q_i - Q_1$ con $2 \leq i \leq n+1$ formano una base di $V$.

			In particolare esiste una sola applicazione lineare $g$ che
			associa a $P_i - P_1$ il vettore $Q_i - Q_1$, con $2 \leq i \leq n+1$. Dacché le immagini formano una base di $V$, $g$ è suriettiva,
			e dunque, poiché $g \in \End(V)$, $g$ è anche invertibile.
			Un'affinità $f \in A(E)$ tale che $f(P_i) = Q_i$ con $1 \leq i \leq n+1$ è per esempio $f(P) = Q_1 + g(P - P_1)$. \\
			
			Si mostra che tale $f$ è anche unica. Se esistesse $f' \in A(E)$
			con le stesse proprietà di $f$, varrebbe che $Q_i - Q_1 = f'(P_i) - f'(P_1) =  g'(P_i - P_1)$ $\forall 2 \leq i \leq n+1$. Tuttavia
			una $g'$ tale che mappi $P_i - P_1$ a $Q_i - P_1$ $\forall 2 \leq i \leq n+1$ è unica, e quindi $g' = g$. Allora $f'(P) = Q_1 + g(P - P_1) = f(P)$ $\forall P \in E$ $\implies f' = f$. \qedhere
		\end{enumerate}
	\end{proof}
	
	\begin{proposition}
		Sia $f \in A(E)$ e sia $D$ un sottospazio affine di $E$. Allora anche $f(D)$ è un sottospazio affine di $E$ della
		stessa dimensione di $D$.
	\end{proposition}
	
	\begin{proof}
		Sia $P_0 \in D$. Allora $(f(D))_0 = \{ f(P) - f(P_0) \forall P \in D \} = \{ g(\v) \forall \v \in D_0 \} =
		g(D_0)$. Dal momento che $f$ è un'affinità, $g$ è invertibile, e quindi preserva la dimensione di $D_0$.
		Pertanto $\dim (f(D))_0 = \dim D_0 \implies \dim f(D) = \dim D$.
	\end{proof}
	
	\begin{remark}
		Siano $D$ e $D'$ due sottospazi affini di $E$. Allora $D \cap D'$ è sempre o vuoto o un sottospazio
		affine. Se infatti $D \cap D'$ non è vuoto, presa una sua combinazione affine, essa è in particolare
		una combinazione affine sia di punti di $D$ che di punti di $D'$, per cui appartiene a $D \cap D'$.
	\end{remark}
	
	\begin{proposition}
		Siano $D$ e $D'$ due sottospazi affini di $E$ con $D \cap D' \neq \emptyset$. Allora
		valgono i seguenti due risultati:
		
		\begin{enumerate}[(i)]
			\item $\Aff(D \cup D')_0 = D_0 + D'_0$,
			\item $(D \cap D')_0 = D_0 \cap D_0'$.
		\end{enumerate}
	\end{proposition}
	
	\begin{proof}
		Si dimostrano i due risultati separatamente.

		\begin{enumerate}[(i)]
			\item Si dimostra l'identità mostrando che vale la doppia inclusione dei due spazi vettoriali.
			Sia innanzitutto $\vec u \in D_0 + D_0'$. Allora esistono $\v \in D_0$, $\w \in D_0'$ tali che
			$\vec u = \v + \w$. Dal momento che $D \cap D' \neq \emptyset$, esiste un punto $P \in D \cap D'$.
			
			Dacché allora $\v \in D_0$, esiste $P_1 \in D$ tale che $\v = P_1 - P$. Analogamente $\exists P_2 \in D'$
			tale che $\w = P_2 - P$. Allora $\vec u = \v + \w = (P_1 - P) + (P_2 - P) = (P_1 + P_2 - P) - P$,
			dove $P_1 + P_2 - P$ è una combinazione affine di $\Aff(D \cup D')$. Allora, poiché $P \in \Aff(D \cup D')$,
			$\vec u \in \Aff(D \cup D')_0$, da cui si deduce che $D_0 + D_0' \subseteq \Aff(D \cup D')$.
			
			Sia ora $\vec u \in \Aff(D \cup D')_0$. Allora esistono $P_1$, ..., $P_k$ punti di $D$, $Q_1$, ..., $Q_{k'}$
			punti di $D'$ e $\lambda_1$, ..., $\lambda_k$, $\mu_1$, ..., $\mu_{k'} \in \KK$ tali che:
			
			\[ \vec u = \left( \sum_{i=1}^k \lambda_i P_i + \sum_{j=1}^{k'} \mu_j Q_j \right) - P, \qquad \sum_{i=1}^k \lambda_i + \sum_{j=1}^{k'} \mu_j = 1. \]
			
			\vskip 0.05in
			
			Allora si può riscrivere $\vec u$ come:
			\[ \vec u =  \underbrace{\left(\sum_{i=1}^k \lambda_i P_i + \sum_{j=1}^{k'} \mu_j P\right)}_{\in D}  - P +  \underbrace{\left(\sum_{i=1}^k \lambda_i P + \sum_{j=1}^{k'} \mu_j Q_j\right)}_{\in D'} - P, \]
			
			dove, ricordando che $P \in D \cap D'$, vale che:
			\[ \left(\sum_{i=1}^k \lambda_i P_i + \sum_{j=1}^{k'} \mu_j P\right)  - P \in D_0, \quad 
			\left(\sum_{i=1}^k \lambda_i P + \sum_{j=1}^{k'} \mu_j Q_j\right) - P \in D'_0, \]
			
			da cui si conclude che $\vec u \in D_0 + D_0' \implies \Aff(D \cup D')_0 \subseteq D_0 + D'_0$,
			e quindi che $\Aff(D \cup D')_0 = D_0 + D'_0$.
			
			\item Come prima, si dimostra l'identità mostrando che vale la doppia inclusione dei due
			spazi vettoriali. Sia $\vec u \in D_0 \cap D_0'$. Sia $P \in D \cap D'$. Allora esiste $P_1 \in D$
			tale che $\vec u = P - P_1$. Analogamente esiste $P_2 \in D'$ tale che $\vec u = P - P_2$. Poiché
			$V$ agisce liberamente su $E$, esiste un solo punto $P'$ tale che $P = P' + \vec u$. Si conclude dunque
			che $P_1 = P_2$, e dunque che $P_1$ appartiene anche a $D'$. Pertanto $\vec u \in (D \cap D')_0 \implies
			D_0 \cap D_0' \subseteq (D \cap D')_0$.
			
			Sia ora invece $\vec u \in (D \cap D')_0$. Allora esiste $P_1 \in D \cap D'$ tale che
			$\vec u = P - P_1$. In particolare, dal momento che $P$ e $P_1$ appartengono a $D$,
			$\vec u \in D_0$. Analogamente $\vec u \in D_0'$. Pertanto $\vec u \in D_0 \cap D_0' \implies
			(D \cap D')_0 \subseteq D_0 \cap D_0'$, da cui si conclude che $(D \cap D')_0 = D_0 \cap D_0'$. \qedhere
		\end{enumerate}
	\end{proof}

	\begin{definition} [somma affine]
		Siano $D$ e $D'$ due sottospazi affini di $E$. Si definisce
		allora la \textbf{somma affine} $D + D'$ come $\Aff(D \cup D')$.
	\end{definition}
	
	\begin{proposition} [formula di Grassmann per i sottospazi affini]
		Siano $D$ e $D'$ due sottospazi affini di $E$ con $D \cap D' \neq \emptyset$. Allora
		$\dim (D + D') = \dim D + \dim D' - \dim (D \cap D')$.
	\end{proposition}
	
	\begin{proof}
		Per la proposizione precedente, $\dim \Aff(D \cup D') = \dim (D_0 + D_0')$. Allora, applicando
		la formula di Grassmann per i sottospazi vettoriali, $\dim (D_0 + D_0') = \dim D_0 + \dim D_0' - \dim (D_0 \cap D_0') = \dim D + \dim D' - \dim (D_0 \cap D_0')$. Sempre per la proposizione precedente,
		$D_0 \cap D_0' = (D \cap D')_0$, da cui si deduce che $\dim (D_0 \cap D_0') = \dim (D \cap D')_0 = \dim D \cap D'$. Pertanto $\dim (D + D') = \dim \Aff(D \cup D') = \dim D + \dim D' - \dim (D \cap D')$.
	\end{proof}

	\begin{remark}
		Si definisce $\ell_{P,Q} = \{ \lambda P + (1-\lambda) Q \mid \lambda \in \KK \}$ con $P$, $Q \in E$ come la retta passante per due
		punti. Allora, in generale, se $D$ e $D'$ sono due sottospazi
		affini di $E$, $D + D' = \bigcup_{\substack{P \in D \\ Q \in D'}} \ell_{P,Q}$. \\
		
		Infatti ogni elemento di $\ell_{P, Q}$ è una combinazione affine
		di due elementi di $D + D'$, e quindi appartiene a $D + D' \implies
		D + D' \supseteq \bigcup_{\substack{P \in D \\ Q \in D'}} \ell_{P,Q}$. \\
		
		Infine, se $T \in D + D'$, esistono $\lambda_1$, ..., $\lambda_k$,
		$\mu_1$, ..., $\mu_{k'} \in \KK$, $P_1$, ..., $P_k \in D$ e
		$Q_1$, ..., $Q_{k'} \in D'$ tali che $T = \sum_{i=1}^k \lambda_i P_i +
		\sum_{j=1}^{k'} \mu_j Q_j$\footnote{Al più $T$ è un elemento di solo $D$ o $D'$. In tal caso $T$ appartiene già a una qualsiasi retta passante per $T$. Pertanto si può anche assumere successivamente
		che $\alpha$, $\beta \neq 0$ -- se infatti uno dei due parametri
		fosse nullo, $T$ apparterrebbe a $D$ o $D'$.}, con $\sum_{i=1}^k \lambda_i + \sum_{j=1}^{k'} \mu_j = 1$. Se $\alpha = \sum_{i=1}^k \lambda_i$ e
		$\beta = \sum_{j=1}^{k'} \mu_j$, si può riscrivere $T$ nel
		seguente modo:
		
		\[ T = \alpha \underbrace{\sum_{i=1}^k \frac{\lambda_i}{\alpha} P_i}_{=\, P'} +
		\beta \underbrace{\sum_{j=1}^{k'} \frac{\mu_j}{\beta} Q_j}_{=\, Q'}, \]
		
		\vskip 0.05in
		
		dove si osserva che $P' \in D$, essendo combinazione affine di
		elementi di $D$, e che analogamente $Q' \in D'$. Allora
		$T$ giace sulla retta passante per $P'$ e per $Q'$, ossia
		$T \in \ell_{P',Q'} \implies D + D' \subseteq \bigcup_{\substack{P \in D \\ Q \in D'}} \ell_{P,Q}$.
	\end{remark}

	\begin{remark}
		Siano fissati $P_0 \in D$ e $P_0' \in D'$, e siano $P \in D$ e
		$Q \in D'$. Allora vale la seguente identità:
		
		\[ P - Q = \underbrace{(P - P_0)}_{\in \, D_0} + \underbrace{(P_0 - P_0')}_{\in \, \Span(P_0 - P_0')} + \underbrace{(P_0' - Q)}_{\in \, D_0'}. \]
		
		\vskip 0.05in
		
		Si osserva che in generale vale che $(D + D')_0 = D_0 + D_0' + \Span(P_0 - P_0')$. Chiaramente vale che $(D + D')_0 \supseteq D_0 + D_0' + \Span(P_0 - P_0')$, dal momento che $D_0$, $D_0'$ e $\Span(P_0 - P_0')$ sono tutti sottospazi vettoriali di $(D + D')_0$. \\
		
		Sia ora $P' \in D + D'$. Allora esistono $P'' \in D$, $Q'' \in Q$ tali
		per cui $P' \in \ell_{P'', Q''}$, e quindi esiste $\lambda \in \KK$
		per cui $P' = P'' + \lambda (Q'' - P'')$. Poiché $P'' \in D$,
		esiste $\v \in D_0$ tale per cui $P'' = P_0 + \v$. Allora
		$P' - P_0 = \v - \lambda (P'' - Q'') \in D_0 + D_0' + \Span(P_0 - P_0')$, pertanto $(D + D')_0 \subseteq D_0 + D_0' + \Span(P_0 - P_0')$,
		da cui si conclude che $(D + D')_0 = D_0 + D_0' + \Span(P_0 - P_0')$.
	\end{remark}

	\begin{proposition} [formula di Grassmann modificata]
		Se $D \cap D' = \emptyset$, allora $\dim (D + D') = \dim D + \dim D' -
		\dim (D_0 \cap D_0') + 1$.
	\end{proposition}

	\begin{proof}
		Dalla precedente osservazione, vale che $(D + D')_0 = D_0 + D_0' + \Span(P_0 - P_0')$. Si dimostra che $P_0 - P_0' \notin D_0 + D_0'$.
		Se infatti $P_0 - P_0'$ appartenesse a $D_0 + D_0'$, esisterebbero
		$P \in D$, $Q \in D'$ tali per cui $P_0 - P_0' = (P_0 - P) + (Q - P_0')$. \\
		
		Allora, facendo agire questo vettore su $P_0'$,
		$P_0 = Q + (P_0 - P)$. Tuttavia, poiché l'azione di $V$ su $E$
		è un'azione di gruppo, esiste un solo punto $P'$ tale per
		cui $P_0 = P' + (P_0 - P)$, e in particolare $P' = P$. Pertanto
		$P = Q \implies P \in D'$. Tuttavia $D \cap D' = \emptyset$, \Lightning. Pertanto $P_0 - P_0' \notin D_0 + D_0'$.
		In particolare questo equivale a constatare che
		$(D_0 + D_0') \cap \Span(P_0 - P_0') = \zerovecset$,
		ossia ad osservare che:
		
		\[ (D + D')_0 = D_0 + D_0' \oplus \Span(P_0 - P_0'). \]
		
		\vskip 0.1in
		
		Si conclude dunque che $\dim (D + D') = \dim (D_0 + D_0') + \dim \Span(P_0 - P_0') =
		\dim D + \dim D' - \dim (D_0 \cap D_0') + 1$, da cui la tesi.
	\end{proof}

	\begin{remark}
		In generale vale che $\Span(P_0 - P_0') \subseteq D_0 + D_0' \iff
		D \cap D' \neq \emptyset$. Infatti $\Span(P_0 - P_0') \subseteq D_0 + D_0' \iff D \cap D' \neq \emptyset$, come appena dimostrato.
		Inoltre, se $D \cap D' \neq \emptyset$, esiste un punto
		$P \in D \cap D'$. Allora $P_0 - P_0' = \underbrace{(P_0 - P)}_{\in \, D_0} + \underbrace{(P - P_0')}_{\in \, D_0'} \implies \Span(P_0 - P_0') \subseteq D_0 + D_0'$. \\
		
		Si poteva dunque dimostrare la \textit{formula di Grassmann} (non
		modificata, per $D \cap D' \neq \emptyset$) utilizzando questa osservazione, così come si sarebbe
		potuto dimostrare che $(D + D')_0 = D_0 + D_0'$.
	\end{remark}
	
	\hr
	
	\begin{remark} [punti fissi di un'applicazione affine]
		Si consideri un'applicazione affine $f$ di $\AnK$. Allora esistono $A \in M(n, \KK) \setminus \{0\}$ e $\vec b \in \AnK$ tali
		per cui $f(\x) = A \x + \vec b$ $\forall \x \in \AnK$. In particolare, $f$ ammette punti fissi se
		esiste $\x \in \AnK \mid f(\x) = \x \iff A \x + \vec b = \x \iff (A-I) \x = -\vec b \iff \vec b \in \Im(A-I)$. \\
		
		Ciò è sicuramente vero se $A$ non ammette $1$ come autovalore (infatti in tal caso $A-I$ è invertibile, e quindi
		in particolare è surgettiva). 
	\end{remark}
	
	\begin{example}
		Si consideri $f \in A(\Aa_1(\KK))$. Allora esistono $a$, $b \in \Aa_1(\KK)$ tali per cui $f(x) = ax+b$ $\forall x \in \Aa_1(\KK)$. \\
		
		Se $a \neq 1$ (ossia se non ammette $1$ come autovalore), $f$ ammette un punto fisso, ossia
		$x = -\frac{b}{a-1}$. Altrimenti, se $a = 1$ e $b \neq 0$, $f$ è una traslazione (e quindi non ammette punti fissi).
		Allora, indicando con $\Fix(f) = \{ x \in \Aa_1(\KK) \mid f(x) = x \}$ l'insieme dei punti fissi
		di $f$, vale sicuramente che $\abs{\Fix(f)} \leq 1$, escludendo il caso in cui $a = 1$ e $b = 0$. \\
		
		Inoltre, vale che $A(\Aa_1(\KK))$ agisce transitivamente su $\Aa_1(\KK)$, ossia esiste sempre un'applicazione
		affine tale per cui $x \mapsto y$, dati $x$, $y \in \Aa_1(\KK)$. Ciononostante $A(\Aa_1(\KK))$ non agisce
		liberamente su $\Aa_1(\KK)$, ed in particolare vale che $\Stab(x) \cong \GL(1, \KK)$\footnote{È sufficiente
		mappare ogni affinità alla propria matrice $A$, dal momento che $\vec b$ è già univocamente determinante.}.
		Un risultato analogo vale anche per gli altri spazi affini: per esempio vale ancora che
		$\Stab(\x) \cong \GL(2, \KK)$ $\forall \x \in \Aa_2(\KK)$, sull'azione generata da $A(\Aa_2(\KK))$ su $\Aa_2(\KK)$. \\
		
		Infine, date due coppie di punti $(P, P')$ e $(Q, Q')$ con $P$, $P'$, $Q$, $Q' \in \Aa_1(\KK)$,
		$P \neq P'$ e $Q \neq Q'$, $A(\Aa_1(\KK))$ agisce in maniera semplicemente transitiva mappando una coppia
		di punti all'altra\footnote{Infatti l'unica applicazione affine che manda una coppia di punti nella stessa coppia di punti è obbligatoriamente l'identità, come visto sopra nello studio di $\Fix(f)$. Infine si osserva che esiste sempre un'applicazione che manda una coppia di punti nell'altra.}.
	\end{example}
	
	\begin{remark} [rapporto semplice] Siano $P_1$, $P_2$, $P_3$, $Q_1$, $Q_2$ e $Q_3$ punti di $\Aa_1(\KK)$ con
		$P_1$, $P_2$, $P_3$ distinti e $Q_1$, $Q_2$, $Q_3$ distinti. Allora esiste un'unica applicazione affine $f$
		tale per cui $f(P_1) = Q_1$ e $f(P_2) = Q_2$, dacché $P_1$ e $P_2$ formano un riferimento affine di $\Aa_1(\KK)$. \\
		
		In particolare $f$ è tale che $f(P_3) = Q_3$ se e solo se, detto $\lambda \in \KK$ il parametro tale per
		cui $P_3 = (1-\lambda) P_1 + \lambda P_2$, $Q_3 = f(P_3) = (1-\lambda) f(P_1) + \lambda f(P_2) =
		(1-\lambda) Q_1 + \lambda Q_2$, ossia se e solo se $\lambda$ è lo stesso parametro con cui si scrive $Q_2$ rispetto
		al riferimento affine dato da $Q_1$ e $Q_2$. In particolare ciò è vero se e solo se vale che $\frac{P_3 - P_1}{P_2 - P_1} = \lambda = \frac{Q_3 - Q_1}{Q_2 - Q_1}$. \\
		
		Il rapporto $\frac{P_3 - P_1}{P_2 - P_1}$ si dice \textbf{rapporto semplice} della terna di punti $P_1$, $P_2$ e $P_3$.
		Si conclude dunque che tale $f$ esiste (ed è unica) se e solo se i rapporti semplici delle due terne di punti
		coincidono.
	\end{remark}
\end{document}
