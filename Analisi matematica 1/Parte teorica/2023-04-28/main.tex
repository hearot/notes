\documentclass[11pt]{article}
\usepackage{personal_commands}
\usepackage[italian]{babel}

\title{\textbf{Note del corso di Analisi matematica 1}}
\author{Gabriel Antonio Videtta}
\date{28 aprile 2023}

\begin{document}
	
	\maketitle
	
	\begin{center}
		\Large \textbf{Criterio di confronto per gli integrali}
	\end{center}

	\wip
	
	Siano $\int_a^b f(x) \, dx$ e $\int_a^b g(x) \, dx$ due integrali
	impropri semplici in $b$.
	
	\begin{proposition}
		Se $o \leq f \leq g$ in un intorno di $b$, allora:
		
		\begin{enumerate}[(i)]
			\item Se $\int_a^b f(x) \, dx = +\infty$, allora
			$\int_a^b g(x) \, dx = +\infty$.
			
			\item Se $\int_a^b g(x) \, dx < +\infty$, allora
			$\int_a^b f(x) \, dx < +\infty$.
		\end{enumerate}
	\end{proposition}

	\begin{proposition} [confronto asintotico debole]
		Se $f$, $g \geq 0$ in un intorno di $b$ e $f(x) = O(g(x))$
		per $x \to b^-$, allora:
		
		\begin{enumerate}[(i)]
			\item Se $\int_a^b f(x) \, dx = +\infty$, allora
			$\int_a^b g(x) \, dx = +\infty$.
			
			\item Se $\int_a^b g(x) \, dx < +\infty$, allora
			$\int_a^b f(x) \, dx < +\infty$.
		\end{enumerate}
	\end{proposition}

	\begin{proposition} [confronto asintotico forte]
		Se $f$, $g \geq 0$ in un intorno di $b$ e esiste $0 < m < +\infty$ tale
		che $f(x) \sim m g(x)$ per $x \to b^-$, allora i due integrali
		impropri hanno lo stesso comportamento.
	\end{proposition}
\end{document}
