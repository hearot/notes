\documentclass[11pt]{article}
\usepackage{personal_commands}
\usepackage[italian]{babel}

\title{\textbf{Note del corso di Analisi Matematica 1}}
\author{Gabriel Antonio Videtta}
\date{21 aprile 2023}

\begin{document}
	
	\maketitle
	
	\begin{center}
		\Large \textbf{Integrale secondo Riemann}
	\end{center}

	\begin{definition} (partizione di un intervallo)
		Preso $[a, b] \subset \RR$. Sia $\sigma = \{x_0, x_1, \ldots, x_n \}$
		con $n \in \NN$. Diciamo che $\sigma$ è una \textbf{partizione}
		di $[a, b]$ se $a = x_0 < x_1 < \cdots < x_n = b$.
	\end{definition}

	\begin{definition} (taglia di una partizione)
		Si definisce $\delta(\sigma)$, con $\sigma$ partizione,
		come la massima distanza tra due punti consecutivi della partizione $\sigma$, ed è detta \textbf{parametro di finezza} della partizione $\sigma$.
	\end{definition}

	\begin{definition} (ordinamento sulle partizioni)
		Siano $\sigma_1$, $\sigma_2$ due partizioni di $[a, b]$. Allora
		$\sigma_2$ è più fine di $\sigma_1$ se $\sigma_1 \subset \sigma_2$.
	\end{definition}

	\begin{remark}
		Siano $\sigma_1$ e $\sigma_2$ sono due partizioni di $[a, b]$. \\
		
		\li Chiaramente $\sigma_1 \cup \sigma_2$ è più fine sia di
		$\sigma_1$ che di $\sigma_2$.
		
		\li Inoltre, se $\sigma_1$ è più fine di $\sigma_2$,
		$\delta(\sigma_2) \geq \delta(\sigma_1)$. 
	\end{remark}

	\begin{definition} [somma di Riemann inferiore e superiore]
		Sia $f : [a, b] \to \RR$ limitata e sia $\sigma = \{ x_0, \ldots, x_n \}$ una partizione
		di $[a, b]$. Si definisce allora la \textbf{somma di Riemann inferiore}
		$S'$ come:
		
		\[ S'(\sigma) = \sum_{i=1}^n \left( \inf_{x_{i-1} \leq x \leq x_i} f \right) (x_i - x_{i-1}), \]
		
		e si definisce la \textbf{somma di Riemann superiore} $S''$ come:
		
		\[ S''(\sigma) = \sum_{i=1}^n \left( \sup_{x_{i-1} \leq x \leq x_i} f \right) (x_i - x_{i-1}). \]
	\end{definition}

	\begin{proposition}
		Sia $f : [a, b] \to \RR$ limitata. Allora:
		
		\begin{enumerate}[(i)]
			\item $\forall \sigma$ partizione di $[a, b]$, $S'(\sigma) \leq S''(\sigma)$,

			\item $\forall \sigma_1$, $\sigma_2$ partizioni di $[a, b]$
			con $\sigma_2$ più fine di $\sigma_1$, vale che
			$S'(\sigma_1) \leq S'(\sigma_2) \leq S''(\sigma_1) \geq S'(\sigma_2)$.
			
			\item $\forall \sigma_1$, $\sigma_2$ partizioni di $[a, b]$,
			$S'(\sigma_1) \leq S''(\sigma_2)$.
		\end{enumerate}
	\end{proposition}

	\begin{proof}
		\begin{enumerate}[(i)]
			\item ovvio.
			
			\item Sia $\sigma_1 = \{ x_0, \ldots, x_n \}$ e sia
			$\sigma_2 = \sigma_1 \cup \{ \xi \}$. Aggiungi un elemento
			e la disuguaglianza regge. Fallo aggiungendo ogni elemento.
	
			\item Usa l'unione che è più fine.
		\end{enumerate}
	\end{proof}

	\begin{definition} [integrale di Riemann inferiore e superiore]
		Si definisce l'\textbf{integrale di Riemann inferiore} di $f$ come:
		
		\[ I_- = \sup \{ S'(\sigma) \mid \sigma \text{ partizione di } [a, b] \}, \]
		
		e l'\textbf{integrale di Riemann superiore} di $f$ come:
		
		\[ I_+ = \inf \{ S''(\sigma) \mid \sigma \text{ partizione di } [a, b] \}. \]
	\end{definition}

	\begin{remark}
		Si osserva che $I_+ \geq I_-$.
	\end{remark}

	\begin{definition} [integrale di Riemann]
		Sia $f : [a, b] \to \RR$ limitata. Si dice che
		$f$ è \textbf{integrabile secondo Riemann} in $[a, b]$
		se $I_+ = I_-$.
	\end{definition}

	\begin{definition} [uniformemente continua]
		Sia $X \subseteq \RR$ e sia $f : X \to \RR$. Si dice
		che $f$ è \textbf{uniformemente continua} se $\forall \eps > 0$,
		$\exists \delta(\eps) > 0$ tale che $\forall x, \xbar \in X,
		\abs{x-\xbar} < \delta \implies \abs{f(x)-f(\xbar)} < \eps$.
	\end{definition}

	\begin{remark}
		Se $f$ è uniformemente continua, chiaramente $f$ è
		continua, benché non sia vero il viceversa.
	\end{remark}

	\begin{example}
		Sia $f : [a, +\infty) \to \RR$ tale che $f(x) = \sqrt{x}$. Sia
		$x > \xbar$, allora $\sqrt{x} > \sqrt{\xbar}$. Sia
		$x = \xbar + h$. Si considera $\sqrt{\xbar + h} - \sqrt{\xbar} < \eps$,
		allora $\sqrt{\xbar + h} < \eps + \sqrt{\xbar}$, da cui si
		deduce che $\xbar + h < \eps^2 + \xbar + 2 \eps \sqrt{\xbar}$,
		ossia $h < \eps^2 + \eps \sqrt{\xbar}$. Preso allora $h < \eps^2$,
		si ha che $f$ è uniformemente continua.
	\end{example}

	\begin{example}
		Come prima, ma per $\sin(x)$. Per Lagrange $\exists \tilde x \in (x, \xbar) \mid \frac{\sin(x) - \sin(\xbar)}{x-\xbar}=\cos(\tilde x)$,
		da cui $\sin(x) - \sin(\xbar) = \cos(\xbar) (x - \xbar)$, ossia
		$\sin(x) - \sin(\xbar) \leq x - \xbar \leq \delta = \eps$.
		(In realtà vale per ogni $f$ con $\abs{f'} \leq l$.)
	\end{example}

	\begin{example}
		Dimostra che non sono unif. continue: $e^x$ (con $\log(n+1)$ e $\log(n)$), $\log(x)$ (con $e^{-n+1}$ e $e^{-n}$),
		$\sin(x^2)$ (con $\sqrt(2\pi n + \pi/2)$ e $\sqrt{2\pi n}$).
	\end{example}

	\begin{theorem}
		$f : [a, b] \to \RR$ continua. Allora $f$ è uniformemente continua.
	\end{theorem}

	\begin{proof}
		Per assurdo suppongo che $f$ non sia uniformemente continua.
		Allora considero $x_n$ e $\xbar_n$ tale che
		$\abs{x_n - \xbar_n} \leq \frac{1}{n}$ ma $\abs{f(x_n) - f(\xbar_n)} > \eps$ $\forall n$. Per Bolzano-Weierstrass, $\exists n_k$ sottosuccessione di tale che $x_{n_k} \to x_0 \in [a, b]$.
		Anche $\xbar_{n_k} \to x_0 \in [a, b]$. Poiché $f$ è continua,
		$f(x_{n_k}) \to f(x_0)$, $f(\xbar_{n_k}) \to 0$, e quindi che
		$\abs{f(x_{n_k}) - f(\xbar_{n_k})} \to 0$, contraddizione perché
		$> \eps$.
	\end{proof}

	\begin{theorem}
		$f : [a, b] \to \RR$ continua allora $f$ è integrabile
		secondo Riemann.
	\end{theorem}
	
\end{document}
