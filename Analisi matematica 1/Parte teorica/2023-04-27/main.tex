\documentclass[11pt]{article}
\usepackage{personal_commands}
\usepackage[italian]{babel}

\title{\textbf{Note del corso di Analisi matematica 1}}
\author{Gabriel Antonio Videtta}
\date{\today}

\begin{document}
	
	\maketitle
	
	\begin{center}
		\Large \textbf{Integrali impropri}
	\end{center}

	\wip
	
	%TODO: definizione area di una figura "ragionevole" (differenza)
	%TODO: funzione di Dirichlet non integrabile
	%TODO: se D(f) -- discontinuità -- è finito, f limitata è integrabile
	%TODO: se per ogni e > 0, esistono intervalli I_1, ..., I_{n(e)} tali che U I_i contiene D(f) e somma |I_i| < e, allora f è integrabile.
	%TODO: se per ogni e > 0, esiste una famiglia f di intervalli numerabile tale che U_{f} I_i = D(f) e somma |I_i| < e. 
	
	\begin{definition} [integrale improprio semplice]
		Si dice che l'integrale $\int_a^b f(x) \, dx$ con $a \in \RR$ è un
		\textbf{integrale improprio semplice} in $b$ se
		$f$ è definita e continua su $[a, b)$ e $b = \pm\infty$,
		$f$ non è definita in $b$ o non è continua in $b$. Si
		definisce in modo analogo un integrale improprio semplice se $b \in \RR$. \\ 
		
		In modo più
		generale, si dice che tale integrale è improprio semplice
		se $f$ è integrabile in $[a, b']$ $\forall b' < b$, ma non su
		$[a, b]$
	\end{definition}

	\begin{example}\nl
		\li L'integrale $\int_0^1 \frac{1}{\sin(x)} \, dx$ è un
		integrale improprio semplice dacché $\frac{1}{\sin(x)}$ è
		definito in $1$, ma non in $0$, ed è continuo e definito su $(0, 1)$. \\
		
		\li L'integrale $\int_0^\pi \frac{1}{\sin(x)} \, dx$, invece,
		non è improprio semplice, dal momento che $\frac{1}{\sin(x)}$ non
		è definito né in $0$ né in $\pi$. \\
		
		\li L'integrale $\int_{-1}^1 \frac{1}{x} \, dx$ non è improprio semplice
		poiché $\frac{1}{x}$ non è definito in $0$.
	\end{example}

	\begin{definition}
		Il valore di $\int_a^b f(x) \, dx$ è definito come $\lim_{b' \to b^-} \int_a^{b'} f(x) \, dx$, se esiste.
	\end{definition}

	Vi sono dunque quattro comportamenti possibili dell'integrale improprio
	semplice $\int_a^b f(x) \, dx$:
	
	\begin{enumerate}[(a)]
		\item esiste ed è finito (ossia, \textbf{converge}),
		
		\item esiste ed è $+\infty$ (ossia, \textbf{diverge a} $+\infty$),
		
		\item esiste ed è $-\infty$ (ossia, \textbf{diverge a} $-\infty$),
		
		\item non esiste.
	\end{enumerate}

	\begin{remark} Sia $f : [a, b) \to \RR$ continua con primitiva $F : [a, b) \to \RR$. Allora $\int_a^b f(x) \, dx = \lim_{b' \to b^-} [F(b')] - F(a)$.
	\end{remark}
	
	\begin{example}\nl
		\li $\int_1^{+\infty} \frac{1}{x^\alpha} = \system{+\infty & \se \alpha \leq 1, \\ \frac{1}{a-1} & \altrimenti.}$ \\
		
		\li  $\int_0^{1} \frac{1}{x^\alpha} = \system{+\infty & \se \alpha \geq 1, \\ \frac{1}{1-a} & \altrimenti.}$ \\
		
		\li $\int_a^{+\infty} e^{-x} \, dx = e^{-a}$.
		
		\li $\int_0^{+\infty} \sin(x) \, dx$ non esiste.
	\end{example}

	\begin{note}
		Si impiega la notazione $\int_a^b f(x) \, dx \approx \int_c^d g(x) \, dx$ per indicare che i due integrali hanno lo stesso comportamento.
	\end{note}

	\begin{remark}\nl
		\li Il comportamento di $\int_a^b f(x) \, dx$, se $a \in \RR$, non dipende
		dalla scelta di $a$. \\
		
		\li Sia $f: [a, +\infty) \to \RR$ con limite $L \neq 0 \in \RRbar$ a $+\infty$.
		Allora:
		
		\[ \int_a^{+\infty} f(x) \, dx = \system{+\infty & \se L > 0, \\ -\infty & \altrimenti.} \] \\ %TODO: dimostralo
		
		\li Se $f \geq 0$ in un intorno di $b$, allora $\int_a^b f(x) \, dx$
		esiste sempre e vale o $+\infty$ o un numero finito. %TODO: dimostralo
	\end{remark}
\end{document}
