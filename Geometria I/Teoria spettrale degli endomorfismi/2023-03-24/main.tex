\documentclass[11pt]{article}
\usepackage{personal_commands}
\usepackage[italian]{babel}

\title{\textbf{Note del corso di Geometria 1}}
\author{Gabriel Antonio Videtta}
\date{24 marzo 2023}

\begin{document}
	
	\maketitle
	
	\begin{center}
		\Large \textbf{Esercitazione: la forma canonica di Jordan e gli autospazi generalizzati}
	\end{center}

	\begin{note}
		Nel corso del documento, per $f$ si intenderà un generico endomorfismo di $\End(V)$, e per $V$
		verrà inteso uno spazio vettoriale di dimensione finita $n$ su un campo $\KK$ algebricamente
		chiuso, qualora non specificato diversamente.
	\end{note}
	
	Sia $f \in \End(V)$. Si osservino allora le seguenti catene ascendenti:
	\begin{gather}
		\{\vec0\} \subsetneq \Ker f \subsetneq \Ker f^2 \subsetneq \cdots \subsetneq \Ker f^{k-1} \subsetneq \Ker f^k = \Ker f^{k+1} = \cdots, \\
		\{\vec0\} \subsetneq \Im f \subsetneq \Im f^2 \subsetneq \cdots \subsetneq \Im f^{k-1} \subsetneq \Im f^k = \Im f^{k+1} = \cdots,
	\end{gather}

	Sia la $(1)$ che la $(2)$ devono stabilizzarsi allo stesso $k \in \NN$, per la cosiddetta decomposizione di Fitting.
	Sempre per tale decomposizione vale in particolare che:
	
	\[ V = \Ker f^k \oplus \Im f^k. \]
	
	\begin{remark} Si possono fare alcune osservazioni riguardo la decomposizione di Fitting. \\

	\li Sia $\Ker f^k$ che $\Im f^k$ sono $f$-invarianti: $\vec v \in \Ker f^k \implies f^k(f(\vec v)) = f(f^k(\vec v)) = \vec0 \implies f(\vec v) \in \Ker f^k$ e $\vec v \in \Im f^k \implies \vec v = f^k(\vec w)$, $f(\vec v) = f(f^k(\vec w)) = f^k(f(\vec w)) \in \Im f^k$. \\ 
	\li $\restr{f}{\Ker f^k}$ è nilpotente: $(\restr{f}{\Ker f^k})^k = \restr{f^k}{\Ker f^k} = 0$. \\
	\li $\restr{f}{\Im f^k}$ è invertibile: $\Ker \restr{f}{\Im f^k} = \Ker f \cap \Im f^k \subseteq \Ker f^k \cap \Im f^k = \{\vec 0\}$, e quindi $\restr{f}{\Im f^k}$ è iniettiva; quindi $\restr{f}{\Im f^k}$ è anche invertibile, essendo un endomorfismo. \\
	\li Poiché $\restr{f}{\Ker f^k}$ è nilpotente, $p_{\restr{f}{\Ker f^k}}(\lambda) = \lambda^d$, dove
	$d = \dim \Ker f^k$. Inoltre
	$\varphi_{\restr{f}{\Ker f^k}}(\lambda) = \lambda^k$: se infatti $\varphi_{\restr{f}{\Ker f^k}}(\lambda) = \lambda^t$
	con $t < k$, varrebbe sicuramente che ${\restr{f}{\Ker f^k}}^{k-1} = \restr{f^{k-1}}{\Ker f^k} = 0$, ossia che
	$\Ker f^k \subseteq \Ker f^{k-1}$, violando la minimalità di $k$, \Lightning. \\
	\li Dal momento che vale la decomposizione di Fitting e che $\varphi_{\restr{f}{\Ker f^k}}$ e $\varphi_{\restr{f}{\Im f^k}}$ sono coprimi tra loro (il primo è diviso solo da $t$, mentre il secondo non è diviso da $t$), $\varphi_f = \mcm(\varphi_{\restr{f}{\Ker f^k}}, \varphi_{\restr{f}{\Im f^k}}) = \varphi_{\restr{f}{\Ker f^k}} \varphi_{\restr{f}{\Im f^k}}$. Si conclude quindi che $k = \mu'_a(0)$ rispetto a $\varphi_f$, ossia la molteplicità algebrica di $0$ in
	tale polinomio. Analogamente si osserva che $t = \mu_a(0)$ rispetto a $p_f$, ossia la molteplicità algebrica
	dell'autovalore $0$ in $f$, e quindi che $\mu_a(0) \geq k$.
	\end{remark}
	
	Reiterando la decomposizione di Fitting (o applicando il teorema di decomposizione primaria), si ottiene
	infine la seguente decomposizione di $V$:
	
	\[ V = \Ker (f - \lambda_1 \Id)^{\mu_a(\lambda_1)} \oplus \cdots \oplus \Ker (f - \lambda_m \Id)^{\mu_a(\lambda_m)}, \]
	
	dove $m$ è il numero di autovalori di $V$. Si può riscrivere questa identità ponendo $n_i := \mu'_a(\lambda_i)$ in
	$\varphi_f$:
	
	\[ V = \Ker (f - \lambda_1 \Id)^{n_1} \oplus \cdots \oplus \Ker (f - \lambda_m \Id)^{n_m}. \]
	
	Si deduce da questa identità che $f$ è diagonalizzabile se e solo se $n_i = 1$ $\forall i \leq m$.
	%TODO: aggiungere come proposizione e approfondire
	
	\begin{exercise}
		Sia $A \in M(n, \CC)$ invertibile. Dimostrare allora che se $A^3$ è diagonalizzabile, anche $A$ lo è. 
	\end{exercise}

	\begin{solution}
		Se $A^3$ è diagonalizzabile, per la precedente osservazione, $\varphi_{A^3}(t) = \prod_{i=1}^m (t - \lambda_i)$,
		dove $m$ è il numero di autovalori distinti di $A^3$. Allora, detto $p(t) = \prod_{i=1}^m (t^3 - \lambda_i)$, vale che
		$p(A) = 0$, ossia che $\varphi_A \mid p$. Dal momento che $A$ è invertibile, anche $A^3$ lo è, e quindi
		$\lambda_i \neq 0$ $\forall i \leq m$. Poiché $p$ è allora fattorizzato in soli termini lineari distinti,
		anche $\varphi_A$ deve esserlo, e quindi $A$ deve essere diagonalizzabile.
	\end{solution}

\end{document}
