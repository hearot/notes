\documentclass[11pt]{article}
\usepackage{personal_commands}
\usepackage[italian]{babel}

\title{\textbf{Note del corso di Geometria 1}}
\author{Gabriel Antonio Videtta}
\date{27 marzo 2023}

\begin{document}
	
	\maketitle
	
	\begin{center}
		\Large \textbf{Proprietà e teoremi principali sul prodotto scalare}
	\end{center}
	
	\begin{note}
		Nel corso del documento, per $V$ si intenderà uno spazio vettoriale di dimensione
		finita $n$ e per $\varphi$ un suo prodotto scalare.
	\end{note}
	
	\begin{proposition} (formula delle dimensioni del prodotto scalare)
		Sia $W \subseteq V$ un sottospazio di $V$. Allora vale la seguente identità:
		
		\[ \dim W + \dim W^\perp = \dim V + \dim (W \cap V^\perp). \]
	\end{proposition}

	\begin{proof}
		Si consideri l'applicazione lineare $f : V \to \dual W$ tale che $f(\vec v)$ è un funzionale di $\dual W$ tale che
		$f(\vec v)(\vec w) = \varphi(\vec v, \vec w)$ $\forall \vec w \in W$. Si osserva che $W^\perp = \Ker f$, da cui,
		per la formula delle dimensioni, $\dim V = \dim W^\perp + \rg f$. Inoltre, si osserva anche che
		$f = i^\top \circ a_\varphi$, dove $i : W \to V$ è tale che $i(\vec w) = \vec w$, infatti $f(\vec v) = a_\varphi(\vec v) \circ i$ è un funzionale di $\dual W$ tale che $f(\vec v)(\vec w) = \varphi(\vec v, \vec w)$. Pertanto
		$\rg f = \rg (i^\top \circ a_\varphi)$. \\
		
		Si consideri ora l'applicazione $g = a_\varphi \circ i : W \to \dual W$. Sia ora $\basis_W$ una base di $W$ e
		$\basis_V$ una base di $V$. Allora le matrice associate di $f$ e di $g$ sono le seguenti:
		
		\begin{enumerate}[(i)]
			\item $M_{\dual \basis_W}^{\basis_V}(f) = M_{\dual \basis_W}^{\basis_V}(i^\top \circ a_\varphi) =
			\underbrace{M_{\dual \basis_W}^{\dual \basis_V}(i^\top)}_A \underbrace{M_{\dual \basis_V}^{\basis_V}(a_\varphi)}_B = AB$,
			\item $M_{\dual \basis_V}^{\basis_W}(g) = M_{\dual \basis_V}^{\basis_W}(a_\varphi \circ i) =
			\underbrace{M_{\dual \basis_V}^{\basis_V}(a_\varphi)}_B \underbrace{M_{\basis_V}^{\basis_W}(i)}_{A^\top} = BA^\top \overbrace{=}^{B^\top = B} (AB)^\top$.
		\end{enumerate}
	
		Poiché $\rg(A) = \rg(A^\top)$, si deduce che $\rg(f) = \rg(g) \implies \rg(i^\top \circ a_\varphi) = \rg(a_\varphi \circ i) = \rg(\restr{a_\varphi}{W}) = \dim W - \dim \Ker \restr{a_\varphi}{W} = \dim W - \dim (W \cap \underbrace{\Ker a_\varphi}_{V^\perp}) = \dim W - \dim (W \cap V^\perp)$. Si conclude allora, sostituendo quest'ultima
		identità nell'identità ricavata a inizio dimostrazione che $\dim V = \dim W^\top + \dim W - \dim (W \cap V^\perp)$,
		ossia la tesi.
	\end{proof}

	\begin{remark}
		Si possono fare alcune osservazioni sul radicale di un solo elemento $\vec w$ e su quello del suo sottospazio
		generato $W = \Span(\vec w)$: \\
		
		\li $\vec w ^\perp = W^\perp$, \\
		\li $\vec w \notin W^\perp \iff \Rad (\restr{\varphi}{W}) = W \cap W^\perp \iff \vec w \text{ non è isotropo } = \zerovecset \iff
		V = W \oplus W^\perp$.
	\end{remark}

	\begin{definition}
		Si definisce \textbf{base ortogonale} di $V$ una base $\vv 1$, ..., $\vv n$ tale per cui $\varphi(\vv i, \vv j) = 0
		\impliedby i \neq j$, ossia per cui la matrice associata del prodotto scalare è diagonale. 
	\end{definition}

	\begin{proposition}
		Se $\Char \KK \neq 2$, un prodotto scalare è univocamente determinato dalla sua forma quadratica $q$.
	\end{proposition}

	\begin{proof}
		Si nota infatti che $q(\vec v + \vec w) - q(\vec v) - q(\vec w) = 2 \varphi(\vec v, \vec w)$, e quindi,
		poiché $2$ è invertibile per ipotesi, che $\varphi(\vec v, \vec w) = 2\inv (q(\vec v + \vec w) - q(\vec v) - q(\vec w))$.
	\end{proof}

	\begin{theorem}(di Lagrange)
		Ogni spazio vettoriale $V$ su $\KK$ tale per cui $\Char \KK \neq 2$ ammette una base ortogonale.
	\end{theorem}

	\begin{proof}
		Sia dimostra il teorema per induzione su $n := \dim V$. Per $n \leq 1$, la dimostrazione è triviale. Sia
		allora il teorema vero per $i \leq n$. Se $V$ ammette un vettore non isotropo $\vec w$, sia $W = \Span(\vec w)$ e si consideri la decomposizione $V = W \oplus W^\perp$. Poiché $W^\perp$ ha dimensione $n-1$, per ipotesi induttiva
		ammette una base ortogonale. Inoltre, tale base è anche ortogonale a $W$, e quindi l'aggiunta di $\vec w$ a
		questa base ne fa una base ortogonale di $V$. Se invece $V$ non ammette vettori non isotropi, ogni forma quadratica
		è nulla, e quindi il prodotto scalare è nullo per la proposizione precedente.
	\end{proof}

	%TODO: aggiungere teorema di Sylvester complesso e reale.
\end{document}
