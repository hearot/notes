\documentclass[11pt]{article}
\usepackage{personal_commands}
\usepackage[italian]{babel}

\title{\textbf{Note del corso di Geometria 1}}
\author{Gabriel Antonio Videtta}
\date{27 marzo 2023}

\begin{document}
	
	\maketitle
	
	\begin{center}
		\Large \textbf{Titolo della lezione}
	\end{center}

	SIa $V$ uno spazio vettoriale su $\KK$ e sia $\phi : V \times V \to \KK$
	un suo prodotto scalare.
	
	\begin{definition}
		Due vettori $\vec v$, $\vec w$ si dicono \textbf{ortogonali} se e
		solo se $\varphi(\vec v, \vec w) = 0$.
	\end{definition}

	\begin{definition}
		Preso un sottospazio $W \subseteq V$, si definisce lo spazio:
		
		\[ W^\perp = \{ \vec v \in V \mid \varphi(\vec v, \vec w) = 0, \forall \vec w \in W \}, \]
		
		detto sottospazio perpendicolare a $W$.
	\end{definition}

	\begin{note}
		Tale notazione è valida anche per sottinsiemi generici di $V$,
		perdendo tuttavia la proprietà di sottospazio di $V$.
	\end{note}

	\begin{remark}
		%TODO: da dimostrare.
		Valgono le seguenti osservazioni. \\
		
		\li $S \subseteq T \implies S^\perp \supseteq T^\perp$. \\
		\li $S^\perp = (\Span(S))^\perp$ (infatti, da sopra,
		vale l'inclusione $S^\perp \supseteq (\Span(S))^\perp$;
		l'inclusione vale anche al contrario, dacché ogni vettore
		ortogonale a $S$ è ortogonale ad ogni combinazione lineare
		degli elementi di $S$, per la bilinearità di $\varphi$).
	\end{remark}

	\begin{theorem} (formula della dimensione dello spazio ortogonale)
		Sia $W \subseteq V$ un sottospazio di $V$. Allora vale la seguente
		identità:
		
		\[ \dim W^\perp = \dim V - \dim W + \dim (W \cap V^\perp), \]
		
		da cui, se $\varphi$ è non degenere,
		
		\[ \dim W^\perp = \dim V - \dim W. \]
	\end{theorem}

	\begin{proof}
		%TODO: dimostra che Im f^\top = Ann(Ker f).
		
		Sia $\varphi$ non degenere.
		Si osserva che $\vec w \in W^\perp$ è tale che
		$\alpha_\varphi(\vec v)(\vec w) = 0$ $\forall \vec v \in V$,
		e quindi che $\alpha_\varphi(\vec v) \in \Ann(W)$, che
		ha dimensione $\dim V - \dim W$. \\
		
		Nel caso generale, si consideri l'applicazione
		$g = i^\top \circ \alpha_\varphi \circ i$, dove
		$i : W \to V$ è tale che $i(\vec w) = \vec w$.
		Si osserva allora che $W^\top = \Ker (g)$.
		
		%TODO: recupera dimostrazione.
	\end{proof}

	\begin{proposition}
		$V = W \oplus W^\perp \iff W \cap W^\perp = \zerovecset \iff \restr{\varphi}{W}$ è non degenere.
	\end{proposition}

	\begin{proof}
		%TODO: aggiungere dimostrazione.
	\end{proof}

	%TODO: riguardare appunti.
	
	
\end{document}
