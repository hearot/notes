\documentclass{article}

\usepackage{amsmath}
\usepackage{amssymb}
\usepackage{amsthm}
\usepackage{enumitem}
\usepackage[a4paper, total={6in, 8in}]{geometry}
\usepackage{hyperref}
\usepackage{mathtools}
\usepackage[italian]{babel}
\usepackage[utf8]{inputenc}
\usepackage[parfill]{parskip}
\usepackage{wrapfig}

\usepackage{pgfplots}
\pgfplotsset{compat=1.15}
\usepackage{mathrsfs}
\usetikzlibrary{arrows,angles,quotes}

\renewcommand\qedsymbol{$\blacksquare$}

\newcommand{\gfrac}[2]{\displaystyle \frac{#1}{#2}}
\newcommand{\abs}[1]{\lvert#1\rvert}
\newcommand{\norm}[1]{\lVert \vec{#1} \rVert}
\newcommand{\nnorm}[1]{\lVert #1 \rVert}

\newtheorem{axiom}{Assioma}[section]
\newtheorem{theorem}{Teorema}[section]
\newtheorem{corollary}{Corollario}[theorem]
\newtheorem{lemma}[theorem]{Lemma}

\theoremstyle{definition}
\newtheorem{definition}{Definizione}[section]

\begin{document}

\author{Gabriel Antonio Videtta}
\title{Appunti di Geometria}

\maketitle
\newpage

\tableofcontents
\newpage

\section{Assiomi della geometria}

\subsection{I concetti primitivi}

La geometria euclidea dispone di tre principali concetti primitivi,
ossia concetti inesprimibili per definizione, ma assunti come
definiti e chiari. Essi sono:

\begin{itemize}[noitemsep]
    \item il punto;
    \item la retta;
    \item il piano.
\end{itemize}

Per indicare questi tre concetti sono in atto alcune convenzioni
stilistiche:

\begin{itemize}[noitemsep]
    \item i punti vengono indicati con le lettere
          maiuscole dell'alfabeto latino (\emph{A}, \emph{B}, \emph{C}, ...);
    \item le rette vengono indicate con le lettere
          minuscole dell'alfabeto latino (\emph{a}, \emph{b}, \emph{c}, ...);
    \item i piani vengono indicati con le lettere
          minuscole dell'alfabeto greco ($\alpha$, $\beta$, $\gamma$, ...).
\end{itemize}

A partire da questi concetti è possibile stabilire gli assiomi
della geometria euclidea.

\subsection{Gli assiomi di appartenenza}

Gli assiomi di appartenenza stabiliscono le relazioni tra i
tre concetti primitivi prima elencati.

\begin{axiom}[Primo assioma di relazione di insieme]
    Ogni piano è un insieme infinito di punti
    $( \forall \, \alpha, \, |\alpha| = \infty )$.
\end{axiom}

\begin{axiom}[Secondo assioma di relazione di insieme]
    Ogni retta è un sottoinsieme di un piano
    $(\forall \, r \; \exists! \, \alpha \mid r \in \alpha)$.
\end{axiom}

\begin{axiom}[Primo assioma di appartenenza della retta]
    A ogni retta appartengono almeno due punti distinti
    $(\forall \, r \; \exists \, A, B \mid A \neq B \land A, B \in r)$.
\end{axiom}

\begin{axiom}[Secondo assioma di appartenenza della retta]
    \label{retta:secondo_assioma_appartenenza}
    Dati due punti distinti, esiste una e una sola retta a cui
    essi appartengano contemporaneamente
    $(A \neq B \implies \exists! \, r \mid A, B \in r)$.
\end{axiom}

\begin{theorem}
    Date due rette distinte, esse possono incontrarsi
    in al più un punto
    $(r \neq s \implies |r \cap s| \leq 1)$.
\end{theorem}

\begin{proof}
    Qualora le due rette dovessero incontrarsi in più di un punto, esisterebbero
    allora due punti appartenenti ad
    ambo le rette. Tuttavia, per
    l'\textbf{Assioma \ref{retta:secondo_assioma_appartenenza}},
    attraverso la congiunzione di tali due punti
    si può determinare una e una sola retta,
    generando una contraddizione.
\end{proof}

A partire da questo teorema si possono definire tre combinazioni di rette.

\begin{definition}[Rette coincidenti]
    Due rette si dicono coincidenti se e solo se
    condividono il medesimo sottoinsieme del piano
    $(r \equiv s \iff \nexists P \in r \mid P \notin s \; \land \;
        \nexists P \in s \mid P \notin r)$.
\end{definition}

\begin{definition}[Rette incidenti]
    Due rette si dicono incidenti se e solo se
    condividono un solo punto del piano.
\end{definition}

\begin{definition}[Rette parallele]
    Due rette si dicono parallele se e solo se
    non condividono alcun punto del piano.
    ($r \parallel s \iff |r \cap s| = 0$).
\end{definition}

\begin{definition}[Punti non allineati]
    Tre o più punti si dicono non allineati se
    non esiste alcuna retta che li contenga tutti
    contemporaneamente.
\end{definition}

\begin{axiom}
    \label{piano:tre_punti}
    Tre punti non allineati definiscono sempre e
    univocamente un piano
    $(A, B, C \mid \nexists \, r \mid A, B, C \in r \implies
        \exists \, \alpha \mid A, B, C \in \alpha)$.
\end{axiom}

\subsection{Gli assiomi di ordine}

Un verso di percorrenza in una retta $r$ viene istituito come
un sistema mediante il quale è sempre possibile stabilire una
relazione di ordine tra due punti distinti $A$ e $B$ appartenenti
alla medesima retta in modo tale che $A>B$ o $A<B$.

Stabilito un verso di percorrenza di una retta, vengono
postulati due assiomi detti di ordine che fanno riferimento
a tale verso di percorrenza.

\begin{axiom}[Primo assioma di ordine della retta]
    Presi due punti distinti $A$ e $B$ appartenenti alla retta $r$
    tali che $A<B$, allora esiste un punto $C$, sempre
    appartenente alla retta $r$, tale che $A<C<B$
    $(A,B \in r \mid A<B \implies \exists \, C \in r \mid A<C<B)$.
\end{axiom}

\begin{axiom}[Secondo assioma di ordine della retta]
    \label{retta:secondo_assioma_ordine}
    Dato un punto $C$ appartenente alla retta $r$, esistono
    sempre due punti $A$ e $B$, sempre appartenenti a $r$,
    tali che $A<C<B$.
    $(C \in r \implies \exists \, A,B \in r \mid A<C<B)$.
\end{axiom}

\begin{theorem}
    \label{retta:infiniti_punti}
    Ad ogni retta appartengono infiniti punti.
\end{theorem}

\begin{proof}
    Qualora ad una retta appartenesse un numero finito di punti,
    stabilito un verso di percorrenza, sarebbe possibile enumerare
    tali punti in ordine. Presi i primi due punti minori $A$ e $B$,
    ossia tali che non esista alcun punto $C$ tale che $A<C<B$, per
    l'\textbf{Assioma \ref{retta:secondo_assioma_ordine}} tra di essi deve
    esistere un punto $C$ tale che $A<C<B$, entrando
    in piena contraddizione con l'assunto.
\end{proof}

\begin{theorem}
    Ogni punto $P$ del piano appartiene ad un numero infinito di rette.
\end{theorem}

\begin{proof}
    Per l'\textbf{Assioma \ref{piano:tre_punti}}, per ogni
    punto $P$ del piano devono esistere altri due punti $A$ e $B$
    tali che la retta che li congiunge non contenga $P$.

    Si considerino le rette $a$, che congiunge $P$ e $A$, e $d$,
    che congiunge $A$ e $B$. Per conseguenza del
    \textbf{Teorema \ref{retta:infiniti_punti}},
    per $d$ passano infiniti punti, i quali, presi singolarmente
    e congiunti a $P$, definiscono allo stesso modo infinite
    rette.
\end{proof}

\end{document}