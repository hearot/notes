\documentclass[11pt]{article}
\usepackage{personal_commands}
\usepackage[italian]{babel}

\title{\textbf{Note del corso di Analisi Matematica 1}}
\author{Gabriel Antonio Videtta}
\date{23, 24 e 28 marzo 2023}

\begin{document}
	
	\maketitle
	
	\begin{center}
		\Large \textbf{Proprietà principali della continuità e dei limiti di funzione}
	\end{center}
	
	\begin{note} Nel corso del documento, per un insieme $X$, qualora non
		specificato, si intenderà sempre un sottoinsieme generico dell'insieme
		dei numeri reali esteso $\RRbar$. Analogamente per $f$ si intenderà
		sempre una funzione $f : X \to \RRbar$.
	\end{note}
	
	\begin{proposition}
		Dati $f : X \to \RRbar$, $\xbar$ punto di accumulazione di $X$
		tale che $\forall \, (x_n) \subseteq X \setminus \{\xbar\} \mid x_n \tendston \xbar$ vale che
		$f(x_n)$ converge. Allora il limite di $f(x_n)$ è sempre lo stesso, indipendentemente
		dalla scelta di $(x_n)$.
	\end{proposition}

	\begin{proof}
		Siano per assurdo $(x_n), (y_n) \subseteq X \setminus \{\xbar\}$ due successioni tali che
		$x_n, y_n \tendston \xbar$ e che $f(x_n) \tendston L$ e $f(y_n) \tendston G$ con $L \neq G$. Si
		costruisce allora la successione $(z_n) \subseteq X \setminus \{\xbar\}$ nel seguente modo:
		
		\[ z_n = \system{x_{\frac{n}{2}} & \text{se } n \text{ è pari}, \\ y_{\frac{n-1}{2}} & \text{altrimenti},} \]
		
		\vskip 0.05in
		
		ossia unendo le due successioni $(x_n)$ e $(y_n)$ in modo tale che agli indici pari corrispondano gli
		elementi di $x_n$ e a quelli dispari quelli di $y_n$. \\
		
		Si mostra che $z_n \tendston \xbar$. Sia $I$ un intorno di $\xbar$. Allora, dal momento che
		$(x_n), (y_n) \tendston \xbar$, esistono sicuramente due
		$n_x, n_y \in \NN$ tali che $n \geq n_x \implies x_n \in I$ e $n \geq n_y \implies y_n \in I$. Pertanto,
		detto $n_k = \max\{n_x, n_y\}$, $n \geq n_k \implies x_n, y_n \in I$, ossia che per $n \geq 2 n_k$,
		$z_n \in I$. Si conclude allora che $(z_n) \tendston \xbar$. \\
		
		Tuttavia $f(z_n)$ non può convergere a nessun limite, dal momento che le due sottosuccessioni
		$f(x_n)$ e $f(y_n)$ convergono a valori distinti ed il limite deve essere unico. L'esistenza di
		tale successione contraddice allora l'ipotesi, \Lightning.
	\end{proof}

	\begin{proposition}
		Data $(x_n) \subseteq \RR$, definisco $f : \NN \to \RRbar$ tale
		che $f(n) := x_n$, $\forall n \in \NN$. Allora $f(n) \tendston L \iff x_n \tendston L$.
	\end{proposition}

	\begin{proof} Si dimostrano le due implicazioni separatamente. \\
		
		\rightproof Sia $I$ un intorno di $L$. Allora, poiché $f(n) \tendston L$,
		esiste un intorno $J = [a, \infty]$ tale che $f(J \cap \NN \setminus \{\infty\}) \subseteq I$.
		Poiché $\infty$ è un punto di accumulazione di $\NN$, $A = J \cap \NN \setminus \{\infty\}$ non è mai
		vuoto. Inoltre, poiché $A \subseteq \NN$, $A$ ammette un minimo\footnote{Non è in realtà necessario che
		si consideri il minimo di tale insieme, occorre semplicemente che $A$ sia non vuoto.}, detto $m$.
		Vale in particolare che
		$f(n) \in I$, $\forall n \geq m$, e quindi che $x_n \in I$, $\forall n \geq m$, ossia che $x_n \tendston L$. \\
		
		\leftproof Sia $I$ un intorno di $L$. Dal momento che $x_n \tendston L$, $\exists n_k \in \NN \mid n \geq n_k \implies
		x_n \in I$. Allora, detto $J = [n_k, \infty]$, vale che $f(J \cap \NN \setminus \{\infty\}) \subseteq I$, ossia
		che $f(n) \tendston L$.
	\end{proof}

	\begin{proposition}
		Siano $f : X \to \RRbar$, $\xbar \in X$ punto di accumulazione
		di $X$. Allora sono fatti equivalenti i seguenti:
		
		\begin{enumerate}[(i)]
			\item $f(x) \tendsto{\xbar} f(\xbar)$,
			\item $f$ è continua in $\xbar$.
		\end{enumerate}
	\end{proposition}

	\begin{proof}
		Sia $I$ un intorno di $f(\xbar)$. Dal momento che $\xbar$ è un punto di accumulazione, si ricava allora da
		entrambe le ipotesi che esiste un intorno $J$ di $f(\xbar)$ tale che
		$f(J \cap X \setminus \{\xbar\}) \subseteq I$, e quindi, per definizione, la tesi.
	\end{proof}

	\begin{remark}
		Se $\xbar$ è un punto isolato di $X$, allora $f$ è continua
		in $\xbar$. Pertanto per rendere la proposizione precedente
		vera, è necessario ipotizzare che $\xbar$ sia un punto
		di accumulazione (infatti il limite in un punto isolato
		non esiste per definizione, mentre in tale punto $f$ è
		continua).
	\end{remark}

	\begin{proposition}
		Siano $f : X \to \RR$  e $\xbar$ punto di accumulazione di $X$.
		Siano $L \in \RRbar$ e $\tilde{f} : X \cup \{\xbar\} \to \RRbar$ tale
		che:
		
		\[ \tilde{f}(x) = \begin{cases}
			L & \text{se } x = \xbar, \\
			f(x) & \text{altrimenti}.
		\end{cases} \]
	
		\vskip 0.05in
	
		Allora $f(x) \tendsto{\xbar} L \iff \tilde{f}$ è continua in $\xbar$.
	\end{proposition}

	\begin{proof}
		Si dimostrano le due implicazioni separatamente. \\
		
		\rightproof Sia $I$ un intorno di $L$. Si ricava allora dalle ipotesi che esiste sempre un intorno
		$J$ di $\xbar$ tale che $f(\underbrace{J \cap X \setminus \{\xbar\}}_{A}) \subseteq I$. Dal momento che $\xbar
		\notin A$, si deduce che $f(J \cap X \setminus \{\xbar\}) = \tilde{f}(J \cap X \setminus \{\xbar\}) \subseteq I$,
		ossia che $\tilde{f}$ è continua in $\xbar$. \\
		
		\leftproof Sia $I$ un intorno di $L$. Poiché $\tilde{f}$ è continua in $\xbar$, esiste un intorno $J$ di $\xbar$
		tale che $\tilde{f}(\underbrace{J \cap (X \cup \{\xbar\}) \setminus \{\xbar\}}_{A}) \subseteq I$. Poiché $\xbar \notin A$ e $\xbar$ è punto di accumulazione, si deduce che $I \supseteq \tilde{f}(J \cap (X \cup \{\xbar\}) \setminus \{\xbar\})
		= f(J \cap (X \cup \{\xbar\}) \setminus \{\xbar\}) \supseteq f(J \cap X \setminus \{\xbar\})$, e quindi che
		$f(x) \tendsto{\xbar} L$.
	\end{proof}

	\begin{remark}
		Tutte le funzioni elementari (e.g.~$\sin(x)$, $\cos(x)$, $\exp(x)$, $\ln(x)$, $\abs{x}$, $x^a$) sono funzioni continue nel loro insieme
		di definizione.
	\end{remark}

	\begin{proposition}
		Siano $f : X \to Y \subseteq \RRbar$ e $g : Y \to \RRbar$ e sia $\xbar \in X$. Sia
		$f$ continua in $\xbar$ e sia $g$ continua in $f(\xbar)$. Allora
		$g \circ f$ è continua in $\xbar$.
	\end{proposition}

	\begin{proof}
		Sia $I$ un intorno di $z = g(f(\xbar))$. Allora, poiché $g$ è continua
		in $f(\xbar)$, $\exists J$ intorno di $f(\xbar)$ $\mid g(J \cap Y \setminus \{f(\xbar)\}) \subseteq
		I$. Tuttavia, poiché $f$ è continua in $\xbar$, $\exists K$ intorno
		di $\xbar$ $\mid f(K \cap X \setminus \{\xbar\}) \subseteq J$, da cui si conclude che
		$g(f(K \cap X \setminus \{\xbar\})) \subseteq I$, dacché $\forall x \in K \cap X \setminus \{\xbar\}$,
		o $f(x) = f(\xbar)$, e quindi $g(f(x)) = z$ chiaramente appartiene a $I$, o altrimenti
		$f(x) \in J \cap Y \setminus \{f(\xbar)\} \implies g(f(x)) \in g(J \cap Y \setminus \{f(\xbar)\}) \subseteq I$.
	\end{proof}

	\begin{theorem}
		Sia $f : X \to Y \subseteq \RRbar$, sia $\xbar$ punto di
		accumulazione di $X$ tale che $f(x) \tendsto{\xbar} \ybar$.
		Se $\ybar$ è un punto di accumulazione di $Y$ e $g : Y \to \RRbar$
		è tale che $\ybar \in Y \implies
		g$ continua in $\ybar$ e $g(y) \tendstoy{\ybar} \zbar$, allora
		$g(f(x)) \tendsto{\xbar} \zbar$.
	\end{theorem}

	\begin{proof}
		Siano $\tilde{f} : X \cup \{\xbar\}$, $\tilde{g} : Y \cup \{\ybar\}$ due funzioni costruite nel seguente
		modo:
		
		\[ \tilde{f}(x) = \begin{cases}
			\ybar & \text{se } x = \xbar, \\
			f(x) & \text{altrimenti},
		\end{cases} \qquad
			\tilde{g}(y) = \begin{cases}
				\zbar & \text{se } y = \ybar, \\
				g(y) & \text{altrimenti}.
			\end{cases} \]
		
		Poiché $f(x) \tendsto{\xbar} \ybar$ e $\xbar$ è un punto di accumulazione di $X$, per una proposizione precedente, $\tilde{f}$ è continua in $\xbar$. Analogamente $\tilde{g}$ è continua in $\ybar$. Dal momento che
		vale che $\tilde{f}(\xbar) = \ybar$, per la proposizione precedente $\tilde{g} \circ \tilde{f}$ è continua in
		$\xbar$, e dunque $\lim_{x \to \xbar} \tilde{g}(\tilde{f}(x)) = \tilde{g}(\tilde{f}(\xbar)) = \zbar$. \\
		
		Si consideri adesso la funzione $\widetilde{g \circ f} : X \to \RRbar$ definita nel seguente modo:
		
		\[ \widetilde{g \circ f}(x) = \begin{cases}
			\zbar & \text{se } x = \xbar, \\
			g(f(x)) & \text{altrimenti}.
		\end{cases} \]
	
		Si mostra che $\widetilde{g \circ f} = \tilde{g} \circ \tilde{f}$. Se $x = \xbar$, chiaramente
		$\widetilde{g \circ f}(x) = \zbar = \tilde{g}(\tilde{f}(\xbar))$. Se $x \neq \xbar$, si
		considera il caso in cui $\tilde{f}(x) = f(x)$ è uguale a $\ybar$ ed il caso in cui non vi è
		uguale. \\
		
		Se $\tilde{f}(x) \neq \ybar$, $\tilde{g}(\tilde{f}(x)) = \tilde{g}(f(x)) \overbrace{=}^{f(x) \neq \ybar} g(f(x)) = \widetilde{g \circ f}(x)$. Se invece
		$\tilde{f}(x) = \ybar$, $\ybar \in Y$, e quindi $g$ è continua in $\ybar$, da cui necessariamente
		deriva che $g(\ybar) = \zbar$. Allora $\widetilde{g \circ f}(x) = g(f(x)) = g(\ybar) = \zbar = \tilde{g}(\tilde{f}(\xbar))$. \\ 
		
		Si conclude allora che $\widetilde{g \circ f} = \tilde{g} \circ \tilde{f}$, e
		quindi che $\widetilde{g \circ f}$ è continua in $\xbar$. Pertanto,
		dalla proposizione precedente, $g(f(x)) \tendsto{\xbar} \zbar$.
	\end{proof}

	\begin{exercise}
		Mostrare che tutte le ipotesi della proposizione precedente sono necessarie, fornendo alcuni controesempi.
	\end{exercise}

	\begin{proposition}
		Date $f_1, f_2 : X \to \RR$ continue in $\xbar$. Allora:
		
		\begin{enumerate}[(i)]
			\item $f_1 + f_2$ è continua in $\xbar$,
			\item $f_1 f_2$ è continua in $\xbar$.
		\end{enumerate}
	\end{proposition}

	\begin{proof}
		Si dimostrano i due punti separatamente.
		
		\begin{enumerate}[(i)]
			\item Sia $f := f_1 + f_2$. Poiché $f_1, f_2$ sono continue in $\xbar$,
			$\forall \eps > 0$, $\exists \delta > 0 \mid \abs{x - \xbar} < \delta
			\implies \abs{f_1(x) - f_1(\xbar)}, \abs{f_2(x) - f_2(\xbar)} \leq \eps$ (per ogni $\eps > 0$, si prende $\delta = \min\{\delta_1, \delta_2\}$, ossia il minimo delle semilunghezze degli intorni
			di $\xbar$). Allora $\abs{f(x) - f(\xbar)} \leq
			\abs{f_1(x) - f_1(\xbar)} + \abs{f_2(x) - f_2(\xbar)} \leq 2\eps$.
			Si conclude dunque che $\forall \eps > 0$, $\exists \delta > 0
			\mid \abs{f(x) - f(\xbar)} \leq 2\eps$, e quindi, poiché
			$2\eps \tends{\eps \to 0} 0$, che $f$ è continua in $\xbar$.
			
			\item Dal momento che $f_1, f_2$ sono continue in $\xbar$,
			$\forall \eps > 0$, $\exists \delta > 0$ tale che $\abs{x - \xbar} < \delta \implies \abs{f_1(x) - f_1(\xbar)} < \eps, \abs{f_2(x)
			- f_2(\xbar)} < \eps$ (vale lo stesso ragionamento del punto
			(i)). Allora $f_1(x) = f_1(\xbar) + e_1$ e
			$f_2(x) = f_2(\xbar) + e_2$, con $\abs{e_1}, \abs{e_2} < \eps$.
			Dunque $f_1(x)f_2(x) = f_1(\xbar)f_2(\xbar) + \underbrace{e_1 f_2(\xbar) +
			e_2 f_1(\xbar) + e_1 e_2}_e$. In particolare, per la
			disuguaglianza triangolare, $\abs{e} \leq \abs{e_1 f_2(\xbar)} +
			\abs{e_2 f_1(\xbar)} + \abs{e_1 e_2} \leq \underbrace{\eps \abs{f_2(\xbar)} +
			\eps \abs{f_1(\xbar)} + \eps^2}_{\eps'}$. Poiché $\eps' \tends{\eps \to 0^+} 0$, si conclude che $\abs{f_1(x) f_2(x) - f_1(\xbar) f_2(\xbar)} = \abs{e} \leq \eps' \implies f_1(x)f_2(x)$ continua
			in $\xbar$.
		\end{enumerate}
	\end{proof}

	\begin{proposition}
		Date $f_1, f_2 : X \to \RRbar$, $\xbar$ punto di accumulazione
		di $X$. Se $\lim_{x \to \xbar} f_1(x) = L_1 \in \RR$ e
		$\lim_{x \to \xbar} f_2(x) = L_2 \in \RR$, allora valgono
		i seguenti risultati:
		
		\begin{enumerate}[(i)]
			\item $f_1(x) + f_2(x) \tendsto{\xbar} L_1 + L_2$,
			\item $f_1(x) f_2(x) \tendsto{\xbar} L_1 L_2$.
		\end{enumerate}
	\end{proposition}

	\begin{proof}
		Si definiscono preliminarmente le funzioni $\tilde{f_1}$, $\tilde{f_2} : X \cup \{\xbar\} \to \RRbar$ in modo tale che:
		
		\[ \tilde{f_1}(x) = \begin{cases}
			L_1 & \text{se } x = \xbar, \\
			f_1(x) & \text{altrimenti},
		\end{cases} \qquad
		\tilde{f_2}(x) = \begin{cases}
			L_2 & \text{se } x = \xbar, \\
			f_2(x) & \text{altrimenti}.
		\end{cases} \]
		
		\vskip 0.05in
	
		Si dimostrano allora i due risultati separatamente. \\
		
		\begin{enumerate}[(i)]
			\item Si definisce $\widetilde{f_1 + f_2} : X \cup \{\xbar\} \to \RRbar$ nel seguente modo:
			
			\[ \widetilde{f_1 + f_2}(x) = \system{L_1 + L_2 & \text{se } x = \xbar, \\ f_1(x) + f_2(x) & \text{altrimenti}.} \]
			
			La somma $L_1 + L_2$ è ben definita dacché sia $L_1$ che $L_2$ sono elementi di $\RR$.
			Poiché da una proposizione precedente $\tilde{f_1}$ e $\tilde{f_2}$ sono continue in $\xbar$, $\tilde{f_1} + \tilde{f_2}$ è continua anch'essa in $\xbar$. È sufficiente allora dimostrare che $\widetilde{f_1 + f_2} =
			\tilde{f_1} + \tilde{f_2}$. Se $x \neq \xbar$, $\widetilde{f_1 + f_2}(x) = f_1(x) + f_2(x) = \tilde{f_1}(x) + \tilde{f_2}(x) = (\tilde{f_1} + \tilde{f_2})(x)$. Se invece $x = \xbar$, $\widetilde{f_1 + f_2}(x) = L_1 + L_2 =
			\tilde{f_1}(x) + \tilde{f_2}(x) = (\tilde{f_1} + \tilde{f_2})(x)$. Quindi $\widetilde{f_1 + f_2} =
			\tilde{f_1} + \tilde{f_2}$, e si conclude che $\widetilde{f_1 + f_2}$ è dunque continua in $\xbar$, ossia
			che $(f_1 + f_2)(x) = f_1(x) + f_2(x) \tendsto{\xbar} L_1 + L_2$.
			
			\item Si definisce, analogamente a prima, $\widetilde{f_1 f_2} : X \cup \{\xbar\} \to \RRbar$ nel seguente modo:
			
			\[ \widetilde{f_1 f_2}(x) = \system{L_1 L_2 & \text{se } x = \xbar, \\ f_1(x) f_2(x) & \text{altrimenti}.} \]
			
			Il prodotto $L_1 L_2$ è ben definito dacché sia $L_1$ che $L_2$ sono elementi di $\RR$.
			Poiché da una proposizione precedente $\tilde{f_1}$ e $\tilde{f_2}$ sono continue in $\xbar$, $\tilde{f_1} \tilde{f_2}$ è continua anch'essa in $\xbar$. È sufficiente allora dimostrare che $\widetilde{f_1 f_2} =
			\tilde{f_1}\tilde{f_2}$. Se $x \neq \xbar$, $\widetilde{f_1 f_2}(x) = f_1(x) f_2(x) = \tilde{f_1}(x) 	\tilde{f_2}(x) = (\tilde{f_1}\tilde{f_2})(x)$. Se invece $x = \xbar$, $\widetilde{f_1 f_2}(x) = L_1 L_2 =
			\tilde{f_1}(x) \tilde{f_2}(x) = (\tilde{f_1} \tilde{f_2})(x)$. Quindi $\widetilde{f_1 f_2} =
			\tilde{f_1} \tilde{f_2}$, e si conclude che $\widetilde{f_1 f_2}$ è dunque continua in $\xbar$, ossia
			che $(f_1 f_2)(x) = f_1(x) f_2(x) \tendsto{\xbar} L_1 L_2$.
		\end{enumerate}
	\end{proof}

	\begin{definition}
		(intorno destro e sinistro) Se $\xbar \in \RR$, si dicono
		\textbf{intorni destri} gli intervalli della forma $[\xbar, \xbar + \eps]$ con
		$\eps > 0$. Analogamente, gli \textbf{intorni sinistri} sono gli
		intervalli della forma $[\xbar - \eps, \xbar]$.
	\end{definition}

	\begin{definition}
		(punto di accumulazione destro e sinistro) Sia $\xbar \in X$.
		Si dice che $\xbar$ è un \textbf{punto di accumulazione destro}
		di $X$ se $\forall I$ intorno destro di $\xbar$, $I \cap X \setminus \{\xbar\} \neq \emptyset$. Analogamente si dice \textbf{punto di
		accumulazione sinistro} di $X$ se è tale per gli intorni sinistri.
	\end{definition}

	\begin{definition}
		(limite destro e sinistro) Sia $\xbar$ un punto di accumulazione
		destro di $X$. Allora $\lim_{x \to \xbar^+} f(x) = L \defiff \forall I$
		intorno di $L$, $\exists J$ intorno destro di $\xbar$ tale che
		$f(J \cap X \setminus \{\xbar\}) \subseteq I$. Analogamente si definisce
		il limite sinistro.
	\end{definition}

	\begin{definition}
		(continuità destra e sinistra) Sia $\xbar \in X$. Allora $f$ è continua
		a destra in $\xbar$ se e solo se $\forall I$ intorno di $f(\xbar)$,
		$\exists J$ intorno destro di $\xbar$ tale che $f(J \cap X \setminus \{\xbar\}) \subseteq I$. Analogamente si definisce la continuità
		a sinistra di $f$.
	\end{definition}

	\begin{remark}
		Vi sono chiaramente alcuni collegamenti tra la continuità destra e sinistra e la continuità classica,
		così come ve ne sono tra il limite destro e sinistro ed il limite classico. \\
		
		\li $\xbar$ punto di accumulazione destro o sinistro di $X$ $\iff$ $\xbar$ punto di accumulazione di $X$, \\
		\li $\xbar$ punto di accumulazione destro e sinistro di $X$ $\implies$ $\xbar$ punto di accumulazione di $X$ (non
		è però per forza vero il contrario, è sufficiente considerare $(0, \infty)$, dove $0$ è solo un punto di
		accumulazione destro), \\
		\li $f$ è continua in $\xbar$ $\iff$ $f$ è continua sinistra e destra in $\xbar$, \\
		\li se $\xbar$ è un punto di accumulazione destro e sinistro, $\lim_{x \to \xbar} f(x) = L \iff \lim_{x \to \xbar^+} f(x) = L$ e $\lim_{x \to \xbar^-} f(x) = L$, \\
		\li se $\xbar$ è un punto di accumulazione solo destro, $\lim_{x \to \xbar} f(x) = L \iff \lim_{x \to \xbar^+} f(x) = L$, \\
		\li se $\xbar$ è un punto di accumulazione solo sinistro, $\lim_{x \to \xbar} f(x) = L \iff \lim_{x \to \xbar^-} f(x) = L$.
	\end{remark}
	
	\begin{proposition}
		Sia $f : X \to \RRbar$ monotona e sia $\xbar$ un punto di
		accumulazione destro di $X$. Allora esiste $\lim_{x \to \xbar^+} f(x)$.
		Analogamente esiste da sinistra se $\xbar$ è un punto di
		accumulazione sinistro di $X$.
	\end{proposition}

	\begin{proof}
		Senza perdità di generalità, si assuma $f$ crescente (per il caso decrescente è sufficiente considerare
		$g(x) = -f(x)$) Si consideri l'insieme:
		
		\[E = \{ f(x) \mid x > \xbar \text{ e } x \in X \}.\]
		
		Si consideri adesso $L = \inf E$ e un suo intorno $I$. Se non
		esistesse un intorno destro $J$ di $\xbar$ tale che $f(J \cap X \setminus \{\xbar\}) \subseteq I$, allora
		$\sup I$ sarebbe un minorante di $E$ maggiore di $L$, \Lightning. Quindi tale $J$ esiste, da cui la tesi.
		Analogamente per il caso sinistro.
	\end{proof}

		%TODO: migliorare dimostrazione

	\begin{example} (funzione discontinua in ogni punto di $\RR$) Si consideri la funzione $f : \RR \to \RR$ definita
		nel seguente modo:
	
		\[ f(x) = \system{ 1 & \text{se }x \in \QQ, \\ 0 & \text{altrimenti}, } \]
		
		\vskip 0.05in
		
		ossia la funzione indicatrice dell'insieme $\QQ$ in $\RR$. Si dimostra che $f$ non è continua
		in nessun punto di $\RR$. Sia infatti $\xbar \in \RR \setminus \QQ$. Dal momento che $\QQ$ è denso
		in $\RR$, $\xbar$ è un punto di accumulazione di $\QQ$, e quindi esiste una successione $(x_n) \subseteq \QQ$
		tale che $x_n \tendston \xbar$. Se $f$ fosse continua in $\xbar$, $\lim_{n \to \infty} f(x_n) = 0$,
		ma per l'intorno $I = [-\frac{1}{2}, \frac{1}{2}]$ non esiste alcun $n_k$ tale per cui $f(x_n) \in I$ $\forall n
		\geq n_k$, dal momento che, per definizione di $f$, $f(x_n) = 1$ $\forall n \in \NN$. Quindi $f$ non è continua
		in nessun $\xbar \in \RR \setminus \QQ$. \\
		
		Sia ora $\xbar \in \QQ$. $\xbar$ è un punto di accumulazione di $\RR \setminus \QQ$ (si può infatti
		considerare la successione $(x_n) \subseteq \RR \setminus \QQ$ definita da $x_n = \xbar + \frac{\sqrt{2}}{n}$,
		che è tale che $x_n \tendston \xbar$). Analogamente a come visto prima, allora, per l'intorno $I = [\frac{1}{2}, \frac{3}{2}]$, $f(x_n) \notin I$ $\forall n \in \NN$, e quindi $f$ non è continua neanche su $\xbar \in \QQ$.
	\end{example}

	\begin{exercise}
		Mostrare che l'insieme dei punti di discontinuità di una funzione $f : I \to \RR$ monotona è al più
		numerabile, dove $I$ è un intervallo.
	\end{exercise}

	\begin{solution}
		Si assuma $f$ crescente, senza perdita di generalità (altrimenti è sufficiente considerare $g(x) = -f(x)$).
		Sia $E$ l'insieme dei punti di discontinuità di $f$. $\forall \xbar \in E$, $\xbar$ è un punto di accumulazione
		destro e sinistro di $I$ (infatti $I$ è un intervallo), ed in particolare esistono sempre il limite destro $L^-(\xbar)$
		ed il limite sinistro $L^+(\xbar)$ in $\xbar$ (dal momento che $f$ è monotona), e sono tali che $L^+(\xbar) > L^-(\xbar)$ (sicuramente
		sono diversi, altrimenti
		$f$ sarebbe continua in $\xbar$; inoltre $f$ è crescente). Allora sia $f : E \to \QQ$ tale che $\xbar \mapsto c$, dove $c \in \QQ$ è
		un punto razionale in $(L^-(\xbar), L^+(\xbar))$ (tale $c$ esiste sempre, per la densità di $\QQ$ in $\RR$).
		Inoltre, $x < y \implies L^+(x) \leq L^-(y)$, e quindi ogni intervallo da cui è preso $c$ è distinto al variare
		di $\xbar \in E$. Quindi $f$ è iniettiva, e vale $\abs{E} \leq \abs{\QQ} = \abs{\NN}$. Si conclude allora
		che $E$ è al più numerabile.
	\end{solution}
	
	\begin{theorem} (della permanenza del segno)
		Data $(x_n) \subseteq \RR$ tale che $x_n \tendston L > 0$, allora
		$(x_n)$ è strettamente positiva definitivamente. Analogamente, se $L < 0$,
		$(x_n)$ è negativa definitivamente.
	\end{theorem}

	\begin{proof}
		Senza perdita di generalità si pone $L > 0$. Allora esiste sicuramente un intorno $I$ di $L$ tale che ogni suo elemento è positivo (e.g.~$I = [\frac{L}{2}, \frac{3L}{2}]$, se $L \in \RR$, altrimenti $[a, \infty]$ con $a > 0$ se $L = +\infty$). Dal momento che $x_n \tendston L$, $\exists n_k \mid n \geq n_k \implies x_n \in I$,
		ossia, in particolare, $n \geq n_k \implies x_n > 0$, da cui la tesi.
	\end{proof}

	\begin{proposition}
		Sia $f : X \to \RRbar$ e sia $\xbar$ un punto di accumulazione di $X$. Se $\lim_{x \to \xbar} f(x) = L > 0$,
		allora $\exists J$ intorno non vuoto di $\xbar$ tale che $f(x) > 0$ $\forall x \in J \cap X \setminus \{\xbar\}$.
	\end{proposition}

	\begin{proof}
		Analogamente a come visto per il teorema del segno, si pone $L > 0$. Allora esiste sicuramente un intorno $I$ di $L$ tale che ogni suo elemento è positivo. Poiché $\lim_{x \to \xbar} f(x) = L > 0$, deve esistere un intorno $J$ di
		$\xbar$ tale che $f(J \cap X \setminus \{\xbar\}) \subseteq I$. In particolare, $J \cap X \setminus \{\xbar\}$ non
		è mai vuoto, dal momento che $\xbar$ è un punto di accumulazione di $X$, e vale che $f(x) > 0$ $\forall x \in J \cap X \setminus \{\xbar\}$ (dal momento che $f(x) \in I$, che ha tutti elementi positivi), da cui la tesi.
	\end{proof}
	
	\begin{theorem} (degli zeri) Dati $I = [a, b]$ e
		$f : I \to \RRbar$ continua tale che $f(a) f(b) < 0$ (i.e.~sono discordi), allora $\exists c \in (a, b) \mid f(c) = 0$.
	\end{theorem}

	\begin{proof}
		Senza alcuna perdita di generalità si pone $f(a) < 0 < f(b)$ (il caso $f(a) > 0 > f(b)$ è
		infine dimostrato considerando $g(x) = -f(x)$). Si definisce allora l'insieme $E$ in modo tale che:
		
		\[ E = \{ a \in I \mid f(a) < 0 \}. \]
		
		\vskip 0.05in
		
		Si osserva che $E \neq \emptyset$, dacché $a \in E$. Per la completezza dei numeri reali,
		$E$ ammette un estremo superiore $\xbar := \sup E$. Sia $(x_n) \subseteq E$ una successione
		tale che $x_n \tendston \xbar$: poiché $f$ è continua in $\xbar$, $\lim_{x \to \xbar} f(x) = f(\xbar) \implies
		f(x_n) \tendston f(\xbar)$. Allora, poiché $f(x_n) < 0$ $\forall n \in \NN$, $f(\xbar) \leq 0$ (se così non fosse
		$f(x_n)$ dovrebbe essere definitivamente positiva per il teorema della permanenza del segno, ma questo
		è assurdo dacché $x_n \in E$ $\forall n \in \NN$, \Lightning). \\
		
		Sia ora $(y_n) \in I$ una successione tale che $y_n \tendston \xbar$ e che $y_n > \xbar$ $\forall n \in \NN$ (questo
		è sempre possibile dal momento che $\xbar \neq b \impliedby f(\xbar) \leq 0$). Allora,
		poiché $y_n > \xbar = \sup E$, $y_n$ non appartiene ad $E$, e quindi deve valere che $f(y_n) > 0$. Si conclude
		allora, per il teorema della permanenza del segno, che $f(\xbar) \geq 0$, e quindi che $f(\xbar) = 0$, da cui
		la tesi.
	\end{proof}

	\begin{proof}[Dimostrazione alternativa] (metodo di bisezione per la ricerca degli zeri)
		Come prima, senza alcuna perdita di generalità, si pone $f(a) < 0 < f(b)$. Si ponga $x_0 = \frac{a+b}{2}$,
		$I_0 = (a, b)$.
		Se $f(x_0) = 0$, allora il teorema è dimostrato. Altrimenti, $f(x_0) > 0$ o $f(x_0) < 0$. Nel primo caso,
		si consideri $I_1 = (a, x_0)$, altrimenti si ponga $I_1 = (x_0, b)$. Si riapplichi allora l'algoritmo con $a := \inf I_1$ e $b := \sup I_1$, definendo la successione $(x_n)$ e gli intervalli $I_n$ per ogni passo $n$ dell'algoritmo. \\
		
		Se la successione $(x_n)$ è finita, allora $\exists n \mid f(x_n) = 0$, e quindi il teorema è dimostrato. Altrimenti,
		si osservi che la successione degli intervalli è decrescente, e che $\abs{I_n} = \frac{b-a}{2^n} \tendston 0$:
		allora, poiché $x_n \in I_n$ $\forall n \in \NN$, $(x_n)$ ammette limite. In particolare, $I_n \tendston \{c\}$,
		e quindi $x_n \tendston c \in I_0$. Siano $a_n$, $b_n$ le successioni tali che $I_n = (a_n, b_n)$ $\forall n \in \NN$.
		Vale in particolare che $a_n, b_n \tendston c$. Allora, per la continuità di $f$ su $(a, b)$, vale che
		$\lim_{n \to \infty} f(a_n) = f(c)$ e che $\lim_{n \to \infty} f(b_n) = f(c)$: poiché ogni elemento di $(a_n)$ è
		per costruzione tale che $f(a_n) < 0$, deve valere che $f(x) \leq 0$ per il teorema della permanenza del segno; analogamente deve valere per costruzione di $(b_n)$ che $f(c) \geq 0$. Si conclude allora che $f(c) = 0$, da cui
		la tesi.
	\end{proof}

	\begin{corollary} (dei valori intermedi) Dati $I = (a, b)$ e
		$f : I \to \RRbar$ continua, allora $y_1$, $y_2 \in f(I) \implies
		[y_1, y_2] \subseteq f(I)$ (ossia $f$ assume tutti i valori
		compresi tra $y_1$ e $y_2$).
	\end{corollary}

	\begin{proof}
		Supponiamo $y_1 < y_2$: poiché $y_1$, $y_2$ appartengono già a $f(I)$, è sufficiente mostrare che anche ogni $y \in (y_1, y_2)$ appartiene a $f(I)$. Dal momento che $y_1$, $y_2 \in f(I)$, $\exists x_1$, $x_2 \in I \mid f(x_1) = y_1$ e $f(x_2) = y_2$. Si consideri allora $g : I \to \RRbar$ tale che
		$g(x) = f(x) - y$. Allora $g(x_1) = y_1 - y < 0$, mentre $g(x_2) = y_2 - y > 0$. Pertanto, per il teorema
		degli zeri, $\exists \xbar \in (x_1, x_2) \mid g(\xbar) = 0 \implies f(\xbar) = y$. Si conclude allora che anche
		$y \in f(I)$, da cui la tesi. 
	\end{proof}

	\begin{remark}
		In realtà, la dimostrazione del teorema dei valori intermedi
		si basa sul fatto che gli unici insiemi convessi di $\RR$ sono
		gli intervalli (da sopra segue infatti che $f(I)$ è un intervallo).
		%TODO: dimostrare.
	\end{remark}

	\begin{theorem} (di Weierstrass) Sia $I$ un intervallo chiuso\footnote{In realtà è sufficiente che $I$ sia chiuso, ossia che contenga i suoi punti
	di accumulazione} e sia
		$f : I \to \RRbar$ continua. Allora esistono $x_m$ e $x_M$ punti
		di massimo e minimo assoluti.
	\end{theorem}

	\begin{proof}
		Ci si limita a dimostrare l'esistenza del minimo, dacché l'esistenza
		del massimo segue dal considerare $g = -f$. Sia $m := \inf f(I)$.
		Esiste allora una successione $(y_n) \subseteq f(I)$ tale che
		$y_n \tendston m$. Poiché $y_n \in f(I)$, $\exists x_n \in I \mid
		y_n = f(x_n)$. Per il teorema di Bolzano-Weierstrass, $\exists \, (x_{n_k})
		\subseteq I$ sottosuccessione convergente, ossia tale che
		$x_{n_k} \to \xbar \in \RRbar$. In particolare vale che
		$\xbar \in I$, dal momento che $I$ è un intervallo chiuso. %TODO: approfondire
		Per la continuità di $f$ (in particolare in $\xbar$), allora $f(x_{n_k}) \tendston f(\xbar)$.
		Essendo $f(x_{n_k})$ una sottosuccessione di $(y_n)$, che è
		convergente, deve valere che $f(\xbar) = m$, ossia
		$m \in f(I)$, da cui si ricava che $f(I)$ ammette un minimo,
		ovvverosia la tesi.
	\end{proof}
	
	\begin{remark} (algoritmo max. e min.) Si consideri $\tilde{f} : \tilde{I} \to \RRbar$. Allora, poiché $\tilde{f}$ è continua ed è definita su
		un intervallo chiuso, per Weierstrass ammette un massimo e un
		minimo. Preso per esempio il minimo, esso potrebbe essere un
		estremo di $\tilde{I}$, oppure è un punto derivabile (e quindi è
			stazionario), oppure non è derivabile.
	\end{remark}
	
\end{document}
