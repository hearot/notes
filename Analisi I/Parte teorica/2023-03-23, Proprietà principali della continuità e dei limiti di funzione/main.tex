\documentclass[11pt]{article}
\usepackage{personal_commands}
\usepackage[italian]{babel}

\title{\textbf{Note del corso di Analisi Matematica 1}}
\author{Gabriel Antonio Videtta}
\date{21 marzo 2023}

\begin{document}
	
	\maketitle
	
	\begin{center}
		\Large \textbf{Proprietà principali della continuità e dei limiti di funzione}
	\end{center}
	
	\begin{note} Nel corso del documento, per un insieme $X$, qualora non
		specificato, si intenderà sempre un sottoinsieme generico dell'insieme
		dei numeri reali esteso $\RRbar$. Analogamente per $f$ si intenderà
		sempre una funzione $f : X \to \RRbar$.
	\end{note}
	
	\begin{proposition}
		Dati $f : X \to \RRbar$, $\xbar$ punto di accumulazione di $X$
		tale che $\forall \, (x_n) \subseteq X \setminus \{\xbar\} \mid x_n \tendston \xbar$ vale che
		$f(x_n)$ converge. Allora il limite di $f(x_n)$ è sempre lo stesso, indipendentemente
		dalla scelta di $(x_n)$.
	\end{proposition}

	\begin{proof}
		Siano per assurdo $(x_n), (y_n) \subseteq X \setminus \{\xbar\}$ due successioni tali che
		$x_n, y_n \tendston \xbar$ e che $f(x_n) \tendston L$ e $f(y_n) \tendston G$ con $L \neq G$. Si
		costruisce allora la successione $(z_n) \subseteq X \setminus \{\xbar\}$ nel seguente modo:
		
		\[ z_n = \system{x_{\frac{n}{2}} & \text{se } n \text{ è pari}, \\ y_{\frac{n-1}{2}} & \text{altrimenti},} \]
		
		\vskip 0.05in
		
		ossia unendo le due successioni $(x_n)$ e $(y_n)$ in modo tale che agli indici pari corrispondano gli
		elementi di $x_n$ e a quelli dispari quelli di $y_n$. \\
		
		Si mostra che $z_n \tendston \xbar$. Sia $I$ un intorno di $\xbar$. Allora, dal momento che
		$(x_n), (y_n) \tendston \xbar$, esistono sicuramente due
		$n_x, n_y \in \NN$ tali che $n \geq n_x \implies x_n \in I$ e $n \geq n_y \implies y_n \in I$. Pertanto,
		detto $n_k = \max\{n_x, n_y\}$, $n \geq n_k \implies x_n, y_n \in I$, ossia che per $n \geq 2 n_k$,
		$z_n \in I$. Si conclude allora che $(z_n) \tendston \xbar$. \\
		
		Tuttavia $f(z_n)$ non può convergere a nessun limite, dal momento che le due sottosuccessioni
		$f(x_n)$ e $f(y_n)$ convergono a valori distinti ed il limite deve essere unico. L'esistenza di
		tale successione contraddice allora l'ipotesi, \Lightning.
	\end{proof}

	\begin{proposition}
		Data $(x_n) \subseteq \RR$, definisco $f : \NN \to \RRbar$ tale
		che $f(n) := x_n$, $\forall n \in \NN$. Allora $f(n) \tendston L \iff x_n \tendston L$.
	\end{proposition}

	\begin{proof} Si dimostrano le due implicazioni separatamente. \\
		
		\rightproof Sia $I$ un intorno di $L$. Allora, poiché $f(n) \tendston L$,
		esiste un intorno $J = [a, \infty]$ tale che $f(J \cap \NN \setminus \{\infty\}) \subseteq I$.
		Poiché $\infty$ è un punto di accumulazione di $\NN$, $A = J \cap \NN \setminus \{\infty\}$ non è mai
		vuoto. Inoltre, poiché $A \subseteq \NN$, $A$ ammette un minimo\footnote{Non è in realtà necessario che
		si consideri il minimo di tale insieme, occorre semplicemente che $A$ sia non vuoto.}, detto $m$.
		Vale in particolare che
		$f(n) \in I$, $\forall n \geq m$, e quindi che $x_n \in I$, $\forall n \geq m$, ossia che $x_n \tendston L$. \\
		
		\leftproof Sia $I$ un intorno di $L$. Dal momento che $x_n \tendston L$, $\exists n_k \in \NN \mid n \geq n_k \implies
		x_n \in I$. Allora, detto $J = [n_k, \infty]$, vale che $f(J \cap \NN \setminus \{\infty\}) \subseteq I$, ossia
		che $f(n) \tendston L$.
	\end{proof}

	\begin{proposition}
		Siano $f : X \to \RRbar$, $\xbar \in X$ punto di accumulazione
		di $X$. Allora sono fatti equivalenti i seguenti:
		
		\begin{enumerate}[(i)]
			\item $f(x) \tendsto{\xbar} f(\xbar)$,
			\item $f$ è continua in $\xbar$.
		\end{enumerate}
	\end{proposition}

	\begin{proof}
		Sia $I$ un intorno di $f(\xbar)$. Dal momento che $\xbar$ è un punto di accumulazione, si ricava allora da
		entrambe le ipotesi che esiste un intorno $J$ di $f(\xbar)$ tale che
		$f(J \cap X \setminus \{\xbar\}) \subseteq I$, e quindi, per definizione, la tesi.
	\end{proof}

	\begin{remark}
		Se $\xbar$ è un punto isolato di $X$, allora $f$ è continua
		in $\xbar$. Pertanto per rendere la proposizione precedente
		vera, è necessario ipotizzare che $\xbar$ sia un punto
		di accumulazione (infatti il limite in un punto isolato
		non esiste per definizione, mentre in tale punto $f$ è
		continua).
	\end{remark}

	\begin{proposition}
		Siano $f : X \to \RR$  e $\xbar$ punto di accumulazione di $X$.
		Siano $L \in \RRbar$ e $\tilde{f} : X \cup \{\xbar\} \to \RRbar$ tale
		che:
		
		\[ \tilde{f}(x) = \begin{cases}
			L & \text{se } x = \xbar, \\
			f(x) & \text{altrimenti}.
		\end{cases} \]
	
		\vskip 0.05in
	
		Allora $f(x) \tendsto{\xbar} L \iff \tilde{f}$ è continua in $\xbar$.
	\end{proposition}

	\begin{proof}
		Si dimostrano le due implicazioni separatamente. \\
		
		\rightproof Sia $I$ un intorno di $L$. Si ricava allora dalle ipotesi che esiste sempre un intorno
		$J$ di $\xbar$ tale che $f(\underbrace{J \cap X \setminus \{\xbar\}}_{A}) \subseteq I$. Dal momento che $\xbar
		\notin A$, si deduce che $f(J \cap X \setminus \{\xbar\}) = \tilde{f}(J \cap X \setminus \{\xbar\}) \subseteq I$,
		ossia che $\tilde{f}$ è continua in $\xbar$. \\
		
		\leftproof Sia $I$ un intorno di $L$. Poiché $\tilde{f}$ è continua in $\xbar$, esiste un intorno $J$ di $\xbar$
		tale che $\tilde{f}(\underbrace{J \cap (X \cup \{\xbar\}) \setminus \{\xbar\}}_{A}) \subseteq I$. Poiché $\xbar \notin A$ e $\xbar$ è punto di accumulazione, si deduce che $I \supseteq \tilde{f}(J \cap (X \cup \{\xbar\}) \setminus \{\xbar\})
		= f(J \cap (X \cup \{\xbar\}) \setminus \{\xbar\}) \supseteq f(J \cap X \setminus \{\xbar\})$, e quindi che
		$f(x) \tendsto{\xbar} L$.
	\end{proof}

	\begin{remark}
		Tutte le funzioni elementari (e.g.~$\sin(x)$, $\cos(x)$, $\exp(x)$, $\ln(x)$, $\abs{x}$, $x^a$) sono funzioni continue nel loro insieme
		di definizione.
	\end{remark}

	\begin{proposition}
		Siano $f : X \to Y \subseteq \RRbar$ e $g : Y \to \RRbar$ e sia $\xbar \in X$. Sia
		$f$ continua in $\xbar$ e sia $g$ continua in $f(\xbar)$. Allora
		$g \circ f$ è continua in $\xbar$.
	\end{proposition}

	\begin{proof}
		Sia $I$ un intorno di $z = g(f(\xbar))$. Allora, poiché $g$ è continua
		in $f(\xbar)$, $\exists J$ intorno di $f(\xbar)$ $\mid g(J \cap Y \setminus \{f(\xbar)\}) \subseteq
		I$. Tuttavia, poiché $f$ è continua in $\xbar$, $\exists K$ intorno
		di $\xbar$ $\mid f(K \cap X \setminus \{\xbar\}) \subseteq J$, da cui si conclude che
		$g(f(K \cap X \setminus \{\xbar\})) \subseteq I$, dacché $\forall x \in K \cap X \setminus \{\xbar\}$,
		o $f(x) = f(\xbar)$, e quindi $g(f(x)) = z$ chiaramente appartiene a $I$, o altrimenti
		$f(x) \in J \cap Y \setminus \{f(\xbar)\} \implies g(f(x)) \in g(J \cap Y \setminus \{f(\xbar)\}) \subseteq I$.
	\end{proof}

	\begin{theorem}
		Sia $f : X \to Y \subseteq \RRbar$, sia $\xbar$ punto di
		accumulazione di $X$ tale che $f(x) \tendsto{\xbar} \ybar$.
		Se $\ybar$ è un punto di accumulazione di $Y$ e $g : Y \to \RRbar$
		è tale che $\ybar \in Y \implies
		g$ continua in $\ybar$ e $g(y) \tendstoy{\ybar} \zbar$, allora
		$g(f(x)) \tendsto{\xbar} \zbar$.
	\end{theorem}

	\begin{proof}
		Siano $\tilde{f} : X \cup \{\xbar\}$, $\tilde{g} : Y \cup \{\ybar\}$ due funzioni costruite nel seguente
		modo:
		
		\[ \tilde{f}(x) = \begin{cases}
			\ybar & \text{se } x = \xbar, \\
			f(x) & \text{altrimenti},
		\end{cases} \qquad
			\tilde{g}(y) = \begin{cases}
				\zbar & \text{se } y = \ybar, \\
				g(y) & \text{altrimenti}.
			\end{cases} \]
		
		Poiché $f(x) \tendsto{\xbar} \ybar$ e $\xbar$ è un punto di accumulazione di $X$, per una proposizione precedente, $\tilde{f}$ è continua in $\xbar$. Analogamente $\tilde{g}$ è continua in $\ybar$. Dal momento che
		vale che $\tilde{f}(\xbar) = \ybar$, per la proposizione precedente $\tilde{g} \circ \tilde{f}$ è continua in
		$\xbar$, e dunque $\lim_{x \to \xbar} \tilde{g}(\tilde{f}(x)) = \tilde{g}(\tilde{f}(\xbar)) = \zbar$. \\
		
		Si consideri adesso la funzione $\widetilde{g \circ f} : X \to \RRbar$ definita nel seguente modo:
		
		\[ \widetilde{g \circ f}(x) = \begin{cases}
			\zbar & \text{se } x = \xbar, \\
			g(f(x)) & \text{altrimenti}.
		\end{cases} \]
	
		Si mostra che $\widetilde{g \circ f} = \tilde{g} \circ \tilde{f}$. Se $x = \xbar$, chiaramente
		$\widetilde{g \circ f}(x) = \zbar = \tilde{g}(\tilde{f}(\xbar))$. Se $x \neq \xbar$, si
		considera il caso in cui $\tilde{f}(x) = f(x)$ è uguale a $\ybar$ ed il caso in cui non vi è
		uguale. \\
		
		Se $\tilde{f}(x) \neq \ybar$, $\tilde{g}(\tilde{f}(x)) = \tilde{g}(f(x)) \overbrace{=}^{f(x) \neq \ybar} g(f(x)) = \widetilde{g \circ f}(x)$. Se invece
		$\tilde{f}(x) = \ybar$, $\ybar \in Y$, e quindi $g$ è continua in $\ybar$, da cui necessariamente
		deriva che $g(\ybar) = \zbar$. Allora $\widetilde{g \circ f}(x) = g(f(x)) = g(\ybar) = \zbar = \tilde{g}(\tilde{f}(\xbar))$. \\ 
		
		Si conclude allora che $\widetilde{g \circ f} = \tilde{g} \circ \tilde{f}$, e
		quindi che $\widetilde{g \circ f}$ è continua in $\xbar$. Pertanto,
		dalla proposizione precedente, $g(f(x)) \tendsto{\xbar} \zbar$.
	\end{proof}

	\begin{exercise}
		Mostrare che tutte le ipotesi della proposizione precedente sono necessarie, fornendo alcuni controesempi.
	\end{exercise}

	\begin{proposition}
		Date $f_1, f_2 : X \to \RR$ continue in $\xbar$. Allora:
		
		\begin{enumerate}[(i)]
			\item $f_1 + f_2$ è continua in $\xbar$,
			\item $f_1 f_2$ è continua in $\xbar$.
		\end{enumerate}
	\end{proposition}

	\begin{proof}
		Sia $f := f_1 + f_2$.
		
		\begin{enumerate}[(i)]
			\item Poiché $f_1, f_2$ sono continue in $\xbar$,
			$\forall \eps > 0$, $\exists \delta > 0 \mid \abs{x - \xbar} < \delta
			\implies \abs{f_1(x) - f_1(\xbar)}, \abs{f_2(x) - f_2(\xbar)} \leq \eps$ (per ogni $\eps > 0$, si prende $\delta = \min\{\delta_1, \delta_2\}$, ossia il minimo delle semilunghezze degli intorni
			di $\xbar$). Allora $\abs{f(x) - f(\xbar)} \leq
			\abs{f_1(x) - f_1(\xbar)} + \abs{f_2(x) - f_2(\xbar)} \leq 2\eps$.
			Si conclude dunque che $\forall \eps > 0$, $\exists \delta > 0
			\mid \abs{f(x) - f(\xbar)} \leq 2\eps$, e quindi, poiché
			$2\eps \tends{\eps \to 0} 0$, che $f$ è continua in $\xbar$.
		\end{enumerate}
	\end{proof}

	\begin{proposition}
		Date $f_1, f_2 : X \to \RRbar$, $\xbar$ punto di accumulazione
		di $X$. Se $\lim_{x \to \xbar} f_1(x) = L_1 \in \RR$ e
		$\lim_{x \to \xbar} f_2(x) = L_2 \in \RR$, allora valgono
		i seguenti risultati:
		
		\begin{enumerate}[(i)]
			\item $f_1(x) + f_2(x) \tendsto{\xbar} L_1 + L_2$,
			\item $f_1(x) f_2(x) \tendsto{\xbar} L_1 L_2$.
		\end{enumerate}
	\end{proposition}

\end{document}
