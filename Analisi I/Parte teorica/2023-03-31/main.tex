\documentclass[11pt]{article}
\usepackage{personal_commands}
\usepackage[italian]{babel}

\title{\textbf{Note del corso di Analisi Matematica 1}}
\author{Gabriel Antonio Videtta}
\date{\today}

\begin{document}
	
	\maketitle
	
	\begin{center}
		\Large \textbf{Teoria sulle derivate}
	\end{center}

	\begin{definition}
		Sia $f : X \subseteq \RR \to \RR$. Si definisce allora \textbf{derivata}
		di $f$ in $\xbar \in X$ punto di accumulazione, se esiste, il seguente limite:
		
		\[f'(\xbar) = \lim_{h \to 0} \frac{f(\xbar + h) - f(\xbar)}{h} = \lim_{x \to \xbar} \frac{f(x) - f(\xbar)}{x - \xbar}.\]
		
		Si definisce anche $f' : D \subseteq X \to \RR$ come la funzione derivata,
		la quale associa ogni punto in cui la derivata di $f$ esiste a
		tale derivata, dove $D$ è proprio l'insieme dei punti in cui questa esiste.
	\end{definition}

	%TODO: spiegare il perché dei domini
	
	\begin{definition}
		$\xbar \in X$ si dice \textbf{derivabile} se e solo se $f'(\xbar)$ esiste ed è finito.
	\end{definition}
	
	\begin{remark}\nl
		\li L'insieme $D$ può essere vuoto. \\
		\li Si definisce $f^{(n)}(\xbar)$ come la derivata $n$-esima
		di $f$ in $\xbar$. \\
		\li Si definisce $f^{(0)}(x) = f(x)$. \\
		\li L'operazione di derivata è un operatore lineare. \\
		\li Si può definire la derivata sinistra e destra.
	\end{remark}

	\begin{definition}
		Si dice che $f : X \to \RR$ è derivabile se è derivabile in ogni
		suo punto.
	\end{definition}
	
	\begin{definition}
		Si dice che $f \in \cc^1$ se è derivabile e la sua
		funzione derivata è continua. In generale, si dice che $f \in \cc^n$ se
		è derivabile $n$ volte e ogni sua derivata, fino alla $n$-esima,
		è continua. Si pone $f \in \cc^\infty$ se $f$ è derivabile per un
		numero arbitrario di volte e ogni sua derivata è continua.
	\end{definition}

	\begin{proposition}
		Sia $f : X \to \RR$ e sia $\xbar \in X$ un punto di accumulazione di $X$. Allora:
		
		\begin{enumerate}[(i)]
			\item $f$ derivabile in $\xbar$ $\implies$ $f(\xbar + h) = f(\xbar) + f'(\xbar) h + o(h)$.
			\item Se esiste $a$ tale che $f(\xbar + h) = f(\xbar) + ah + o(h)$,
			allora $f$ è derivabile in $\xbar$ e $f'(\xbar) = a$.
		\end{enumerate}
	\end{proposition}

	\begin{proof}
		Se $f$ è derivabile in $\xbar$, allora $\lim_{h \to 0} \frac{f(\xbar + h) - f(\xbar) - f'(\xbar) h}{h} = \lim_{h \to 0} \frac{f(\xbar + h) - f(\xbar)}{h} - f'(\xbar) = 0$, da cui la prima tesi. \\
		
		Inoltre, se esiste $a$ come nelle ipotesi, $\lim_{h \to 0} \frac{f(\xbar + h) - f(\xbar)}{h} =\lim_{h \to 0} \frac{ah + o(h)}{h} = 0$, quindi $f$ è derivabile in $\xbar$ e $f'(\xbar) = a$.
	\end{proof}

	\begin{corollary}
		Se $f$ è derivabile in $\xbar$, allora è anche continua in $\xbar$.
	\end{corollary}

	\begin{proof}
		Infatti, poiché $f(x) = f(\xbar) + f'(\xbar) (x - \xbar) + o(x-\xbar)$,
		$\lim_{x \to \xbar} f(x) = f(\xbar)$, e quindi $f$ è continua in $\xbar$. %TODO: trovare esempio di derivabilità infinita e non continuità
	\end{proof}

	\begin{proposition}
		Siano $f_1$, $f_2 : X \to \RR$ entrambe derivabili in
		$\xbar$. Allora:
		
		\begin{enumerate}[(i)]
			\item $(f_1 + f_2)'(\xbar) = f_1'(\xbar) + f_2'(\xbar)$,
			\item $(f_1f_2)'(\xbar) f_1(\xbar) f_2'(\xbar) + f_1'(\xbar) f_2(\xbar)$.
		\end{enumerate}
	\end{proposition}

	\begin{proof}
		\begin{enumerate}[(i)]
			\item $\lim_{h \to 0} \frac{(f_1 + f_2)'(\xbar + h) - (f_1 + f_2)'(\xbar)(\xbar)}{h} = \lim_{h \to 0} \frac{f_1(x+h) - f_1(x)}{h} + \lim_{h \to 0} \frac{f_2(x+h) - f_2(x)}{h} =
			f_1'(\xbar) + f_2'(\xbar)$.
			\item Poiché $f_1$ ed $f_2$ sono derivabili in $\xbar$,
			$f_1(\xbar + h) = f_1(\xbar) + f_1'(\xbar) h + o(h)$ e
			$f_2(\xbar + h) = f_2(\xbar) + f_2'(\xbar) h + o(h)$,
			da cui $(f_1 f_2)(\xbar + h) = (f_1f_2)(\xbar) + (f_1f_2'(\xbar) +
			f_1'(\xbar) f_2(\xbar))h + o(h) \implies (f_1 f_2)'(\xbar) = (f_1f_2'(\xbar) +
			f_1'(\xbar) f_2(\xbar)$.
		\end{enumerate}
	\end{proof}

	\begin{proposition}
		Siano $f : X \to Y$ e $g : Y \to \RR$, con $f$ derivabile in $\xbar$ e $g$ tale che
		sia derivabile in $\ybar = f(\xbar)$. Allora $g \circ f$ è
		derivabile in $\xbar$ e $(g \circ f)'(\xbar) = f'(\xbar) g'(\ybar)$.
	\end{proposition}

	\begin{proof}
		Vale che $f(\xbar + h) = \ybar + f'(\xbar) h + o(h)$, e quindi
		che $g(f(\xbar + h)) = g(\ybar + f'(\xbar) h + o(h))$. In particolare,
		$g(\ybar + h) = g(\ybar) + g'(\ybar) h + o(h)$, e quindi
		$g(f(\xbar + h)) = g(\ybar) + g'(\ybar) (f'(\xbar)h + o(h)) +
		o(f'(\xbar) h + o(h)) = g(\ybar) + g'(\ybar) + g'(\ybar) f'(\xbar) h + o(h) \implies (g \circ f)'(\xbar) = g'(\ybar) f'(\xbar)$.
	\end{proof}

	\begin{proposition}
		Sia $f : X \to Y$ con inversa $g : Y \to X$. Sia $f$ derivabile
		in $\xbar$ con $f'(\xbar) \neq 0$. Sia $g$ continua in $\ybar = f(\xbar)$. Allora:
		
		\begin{enumerate}[(i)]
			\item $\ybar$ è un punto di accumulazione di $Y$,
			\item $g$ è derivabile in $\ybar$,
			\item $g'(\ybar) = \frac{1}{f'(\xbar)}$.
		\end{enumerate}
	\end{proposition}

	\begin{proof}\nl
		\begin{enumerate}[(i)]
			\item Poichè $f$ è derivabile in $\xbar$, $f$ è continua
			in $\xbar$. Quindi per ogni intorno $I$ di $\ybar$, esiste
			un intorno $J$ di $\xbar$ tale per cui $f(I \cap X \setminus \{ \xbar \}) \subseteq J$, e poiché $I \cap X \setminus \{\xbar\}$ non
			è mai vuoto perché $\xbar$ è un punto di accumulazione di $X$ a causa della derivabilità di $f$ in $\xbar$, $J$ contiene in particolare un immagine di $f$ in esso, e quindi un punto di $Y$;
			inoltre, tale punto è diverso da $\ybar$ dacché $f$ è
			iniettiva. Quindi $\ybar$ è un punto di accumulazione.
			\item e (iii) Vale\footnote{Nel dire che $h \to 0$, si è usato che $g$ è
			continua in $\ybar$.} che $\ybar + k = f(g(\ybar + k)) = f(g(\ybar) + (\underbrace{g(\ybar + k) - g(\ybar)}_h)) = f(\xbar + h) =
			f(\xbar) + f'(\xbar) h + o(h) = \ybar + f'(\xbar) h + o(h)$. Quindi $k = f'(\xbar) h + o(h)$. Dal momento che $f'(\xbar) \neq 0$
			per ipotesi, $h \sim \frac{k}{f'(\xbar)}$. Quindi
			$\lim_{k \to 0} \frac{g(\ybar + k) - g(\ybar)}{k} = \lim_{k \to 0} \frac{h}{k} = \frac{1}{f'(\xbar)}$. Quindi la derivata esiste
			ed è proprio come desiderata nella tesi.
 		\end{enumerate}
	\end{proof}

	\begin{example}
		La continuità è necessaria nelle scorse ipotesi. Si può costruire
		infatti una funzione del tipo:
		
		\[ f(x) = \system{x & \se x \geq 0, \\ -(x+2) & \se -2 < x \leq -1.} \]
		
		dove $f'(0) = 1$, $f$ è invertibile, ma la derivata di $g$ in $0$ non
		esiste ($D_+ g(0) = 1)$, ma $D_- g(0) = +\infty$).
	\end{example}

	\begin{theorem} (di Fermat)
		Sia $I$ intervallo, $f : I \to \RR$, $\xbar$ interno a $I$ punto
		di massimo o minimo locale con $f$ derivabile in $\xbar$, allora
		$f'(\xbar) = 0$.
	\end{theorem}

	\begin{example}
		Dimostrare che la derivata sinistra è negativa, e che quella
		destra è positiva nei casi che hai capito.
	\end{example}
\end{document}
