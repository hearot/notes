\documentclass[11pt]{article}
\usepackage{personal_commands}
\usepackage[italian]{babel}

\title{\textbf{Note del corso di Analisi Matematica 1}}
\author{Gabriel Antonio Videtta}
\date{21 marzo 2023}

\begin{document}
	
	\maketitle
	
	\begin{center}
		\Large \textbf{Analogie tra i limiti di funzioni e i limiti di successioni}
	\end{center}
	
	\begin{note} Nel corso del documento, per un insieme $X$, qualora non
		specificato, si intenderà sempre un sottoinsieme generico dell'insieme
		dei numeri reali esteso $\RRbar$. Analogamente per $f$ si intenderà
		sempre una funzione $f : X \to \RRbar$.
	\end{note}
	
	\begin{exercise}
		Dati $f : X \to \RRbar$, $\xbar$ punto di accumulazione di $X$
		tale che $\forall \, (x_n) \subseteq X \setminus \{\xbar\}$ vale che
		$f(x_n)$ converge. Allora il limite di $f(x_n)$ è sempre lo stesso.
	\end{exercise}

	\begin{exercise}
		Data $(x_n) \subseteq \RR$, definisco $f : \NN \to \RRbar$ tale
		che $f(n) := x_n$, $\forall n \in \NN$. Allora $f(n) \tendston L \iff x_n \tendston L$.
	\end{exercise}

	\begin{proposition}
		Siano $f : X \to \RRbar$, $\xbar \in X$ punto di accumulazione
		di $X$. Allora sono fatti equivalenti i seguenti:
		
		\begin{enumerate}[(i)]
			\item $f(x) \tendsto{\xbar} L$,
			\item $f$ è continua in $\xbar$.
		\end{enumerate}
	\end{proposition}

	\begin{proof}
		Si dimostrano le due implicazioni separatamente.
	\end{proof}

	\begin{remark}
		Se $\xbar$ è un punto isolato di $X$, allora $f$ è continua
		in $\xbar$. Pertanto per rendere la proposizione precedente
		vera, è necessario ipotizzare che $\xbar$ sia un punto
		di accumulazione (infatti il limite in un punto isolato
		non esiste per definizione, mentre in tale punto $f$ è
		continua).
	\end{remark}

	\begin{exercise}
		Siano $f : X \to \RR$  e $\xbar$ punto di accumulazione di $X$.
		Siano $L \in \RRbar$ e $\tilde{f} : X \cup \{\xbar\} \to \RRbar$ tale
		che:
		
		\[ \tilde{f}(x) = \begin{cases}
			L & \text{se } x = \xbar, \\
			f(x) & \text{altrimenti}.
		\end{cases} \]
	
		\vskip 0.05in
	
		Allora $f(x) \tendsto{\xbar} L \iff \tilde{f}$ è continua in $\xbar$.
	\end{exercise}

	\begin{remark}
		Tutte le funzioni elementari (e.g.~$\sin(x)$, $\cos(x)$, $\exp(x)$, $\ln(x)$, $\abs{x}$, polinomi) sono funzioni continue nel loro insieme
		di definizione.
	\end{remark}

	\begin{proposition}
		Date $f : X \to Y \subseteq \RRbar$ e $g : Y \to \RRbar$. Sia
		$f$ continua in $\xbar$ e sia $g$ continua in $f(\xbar)$. Allora
		$g \circ f$ è continua in $\xbar$.
	\end{proposition}

	\begin{proof}
		Sia $I$ un intorno di $z = g(f(\xbar))$. Allora, poiché $g$ è continua
		in $f(\xbar)$, $\exists J$ intorno di $f(\xbar)$ $\mid g(J) \subseteq
		I$. Tuttavia, poiché $f$ è continua in $\xbar$, $\exists K$ intorno
		di $\xbar$ $\mid f(K) \subseteq J$, da cui si conclude che
		$g(f(K)) \subseteq g(J) \subseteq I$.
	\end{proof}

	\begin{proposition}
		Sia $f : X \to Y \subseteq \RRbar$, sia $\xbar$ punto di
		accumulazione di $X$ tale che $f(x) \tendsto{\xbar} \ybar$.
		Se $\ybar$ è un punto di accumulazione di $Y$, $\ybar \in Y \implies
		g$ continua in $\ybar$ e $g : Y \to \RRbar$
		è tale che $g(y) \tendstoy{\ybar} \zbar$, allora
		$g(f(x)) \tendsto{\xbar} \zbar$.
	\end{proposition}

	\begin{proof}
		
	\end{proof}

	\begin{exercise}
		Mostrare che tutte le ipotesi della proposizione precedente sono necessarie, fornendo alcuni controesempi.
	\end{exercise}

	\begin{proposition}
		Date $f_1, f_2 : X \to \RR$ continue in $\xbar$. Allora:
		
		\begin{enumerate}[(i)]
			\item $f_1 + f_2$ è continua in $\xbar$,
			\item $f_1 f_2$ è continua in $\xbar$.
		\end{enumerate}
	\end{proposition}

	\begin{proof}
		Sia $f := f_1 + f_2$.
		
		\begin{enumerate}[(i)]
			\item Poiché $f_1, f_2$ sono continue in $\xbar$,
			$\forall \eps > 0$, $\exists \delta > 0 \mid \abs{x - \xbar} < \delta
			\implies \abs{f_1(x) - f_1(\xbar)}, \abs{f_2(x) - f_2(\xbar)} \leq \eps$ (per ogni $\eps > 0$, si prende $\delta = \min\{\delta_1, \delta_2\}$, ossia il minimo delle semilunghezze degli intorni
			di $\xbar$). Allora $\abs{f(x) - f(\xbar)} \leq
			\abs{f_1(x) - f_1(\xbar)} + \abs{f_2(x) - f_2(\xbar)} \leq 2\eps$.
			Si conclude dunque che $\forall \eps > 0$, $\exists \delta > 0
			\mid \abs{f(x) - f(\xbar)} \leq 2\eps$, e quindi, poiché
			$2\eps \tends{\eps \to 0} 0$, che $f$ è continua in $\xbar$.
		\end{enumerate}
	\end{proof}

	\begin{proposition}
		Date $f_1, f_2 : X \to \RRbar$, $\xbar$ punto di accumulazione
		di $X$. Se $\lim_{x \to \xbar} f_1(x) = L_1 \in \RR$ e
		$\lim_{x \to \xbar} f_2(x) = L_2 \in \RR$, allora valgono
		i seguenti risultati:
		
		\begin{enumerate}[(i)]
			\item $f_1(x) + f_2(x) \tendsto{\xbar} L_1 + L_2$,
			\item $f_1(x) f_2(x) \tendsto{\xbar} L_1 L_2$.
		\end{enumerate}
	\end{proposition}

\end{document}
