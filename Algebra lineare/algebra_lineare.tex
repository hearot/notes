\documentclass[oneside]{book}

\usepackage{amsmath}
\usepackage{amssymb}
\usepackage{amsthm}
\usepackage{enumitem}
\usepackage[a4paper, total={6in, 8in}]{geometry}
\usepackage{hyperref}
\usepackage{mathtools}
\usepackage[italian]{babel}
\usepackage[utf8]{inputenc}
\usepackage[parfill]{parskip}
\usepackage{wrapfig}

\usepackage{pgfplots}
\pgfplotsset{compat=1.15}
\usepackage{mathrsfs}
\usetikzlibrary{arrows,angles,quotes}

\renewcommand\qedsymbol{$\blacksquare$}

\newcommand{\gfrac}[2]{\displaystyle \frac{#1}{#2}}
\newcommand{\abs}[1]{\lvert#1\rvert}
\newcommand{\norm}[1]{\lVert \vec{#1} \rVert}
\newcommand{\nnorm}[1]{\lVert #1 \rVert}

\newtheorem{axiom}{Assioma}[section]
\newtheorem{theorem}{Teorema}[section]
\newtheorem{corollary}{Corollario}[theorem]
\newtheorem{lemma}[theorem]{Lemma}

\theoremstyle{definition}
\newtheorem{definition}{Definizione}[section]

\begin{document}

\author{Gabriel Antonio Videtta}
\title{Appunti di Algebra lineare}

\maketitle
\newpage

\tableofcontents
\newpage

\chapter{Primo capitolo}

Lorem ipsum dolor sit amet, consectetur adipiscing elit.
Etiam laoreet venenatis ligula, et posuere est malesuada non.
In placerat rutrum felis, vel consectetur justo commodo tempus.
Etiam placerat mattis lectus, eget convallis ipsum convallis
sit amet. Nunc laoreet sapien sed accumsan aliquet. Vestibulum
justo purus, varius et dolor feugiat, viverra tincidunt diam.
Suspendisse maximus est augue, eget tincidunt turpis lobortis
eget. Vivamus placerat, elit a gravida sollicitudin, ante mauris
fermentum erat, accumsan mattis lectus justo quis massa. Cras
eleifend arcu vitae mauris efficitur, ut dapibus ligula fermentum.
Aliquam eget nisi congue, varius mi id, placerat ante. Duis at
egestas ligula. Morbi pulvinar dolor ut nibh auctor, quis congue
elit aliquam. Cras placerat lorem eros, et pretium nisi finibus
a. Integer dignissim mi nulla, id consectetur nisi blandit sed.
In in maximus erat. Aenean gravida nibh elit, at pellentesque
lorem porttitor eget.

\end{document}