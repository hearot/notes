%--------------------------------------------------------------------
\chapter{Probabilità sulla retta reale}
\setlength{\parindent}{2pt}

\begin{multicols*}{2}

Discutiamo in questa sezione la teoria della probabilità sulla
retta reale, uscendo dunque dal caso discreto.

Per restringere la $\sigma$-algebra su cui lavoreremo (ossia l'insieme degli
eventi interessanti), siamo costretti a limitarci a una $\sigma$-algebra molto più
piccola di $\PP(\RR)$, la $\sigma$-algebra dei boreliani, che ci permette di escludere
``casi meno interessanti''.

\section{Cenni di teoria della misura}

\subsection{La \texorpdfstring{$\sigma$}{σ}-algebra di Borel}

\begin{definition}[$\sigma$-algebra dei boreliani]
    Dato uno spazio metrico separabile\footnote{
        Si può generalizzare in modo naturale tale definizione a un qualsiasi spazio topologico.
        Dal momento che considereremo solo spazi metrici separabili (in particolare $X \subseteq \RR^d$), concentreremo
        le proprietà e le definizioni su questa classe di spazi topologici.} $X \neq \emptyset$
    si definisce la \textbf{$\sigma$-algebra $\BB(X)$ dei boreliani di $X$} (o
    $\sigma$-algebra di Borel)
    come la $\sigma$-algebra generata dai suoi aperti, ovverosia:
    \[
        \BB(X) \defeq \sigma \{ A \subseteq X \mid A \text{ aperto}\, \}.
    \]
\end{definition}

\begin{proposition}[Proprietà di $\BB(X)$]
    Sia $X \neq \emptyset$ uno spazio metrico separabile. Allora valgono
    le seguenti affermazioni:

    \begin{enumerate}[(i.)]
        \item $\BB(X)$ contiene tutti gli aperti e i chiusi di $X$ (infatti
        metrico e separabile implica II-numerabile),
        \item $\BB(X) = \sigma \{ A \subseteq X \mid A \text{ chiuso}\, \}$, ossia
        $\BB(X)$ è generata anche dai chiusi di $X$ (infatti $\BB(X)$ è chiuso per
        complementare),
        \item se $Y \subseteq X$, $Y \neq \emptyset$ ha metrica indotta da $X$, allora
        $\BB(Y) = \sigma \{ Y \cap B \mid B \in \BB(X) \} \subseteq \BB(X)$ (segue dal fatto che
        gli aperti di $Y$ sono tutti e solo gli aperti di $X$ intersecati a $Y$).
    \end{enumerate}
\end{proposition}

\begin{proposition}[Proprietà di $\BB(\RR^d)$]
    Valgono le seguenti affermazioni:
    \begin{enumerate}[(i.)]
        \item $\BB(\RR)$ contiene tutti gli intervalli e tutte le semirette (infatti si ammettono anche intersezioni infinite di aperti),
        \item $\BB(\RR)$ è generato dagli intervalli semiaperti, ovverosia $\BB(\RR) = \sigma \{ (a, b] \mid a, b \in \RR, a < b \}$,
        \item $\BB(\RR)$ è generato dalle semirette, ovverosia $\BB(\RR) = \sigma \{ (-\infty, a) \mid a \in \RR \}$,
        \item $\BB(\RR^d) = \sigma \{ (-\infty, a_1) \times \ldots \times (-\infty, a_n) \mid a_1, \ldots, a_n \in \RR \}$ (segue da (iii.)),
        \item $\BB(\RR^d) \neq \PP(\RR^d)$ (segue dal controesempio di Vitali, oltre che da considerazioni sulle cardinalità).
    \end{enumerate}
\end{proposition}

\subsection{Definizione di misura e misura di Lebesgue}

\begin{definition}[Misura]
    Dato $(\Omega, \FF)$ spazio misurabile, una misura $\mu$ su $(\Omega, \FF)$ è una
    funzione $\mu : \FF \to [0, \infty]$ con $\mu(\emptyset) = 0$ e per cui valga
    la $\sigma$-additività, ovverosia:
    \[
        \mu\left(\bigcupdot_{i \in \NN} A_i\right) = \sum_{i \in \NN} \mu(A_i), \quad A_i \in \FF.
    \]
\end{definition}

\begin{definition}[Insiemi $\mu$-trascurabili e proprietà $\mu$-quasi certe]
    Un insieme $A \in \FF$ si dice \textbf{$\mu$-trascurabile} se
    $\mu(A) = 0$. Una proprietà $M$ si dice che accade
    $\mu$-quasi certamente se esiste $A \in \FF$ $\mu$-trascurabile per cui
    $M$ accade per $A^c$.
\end{definition}


\end{multicols*}