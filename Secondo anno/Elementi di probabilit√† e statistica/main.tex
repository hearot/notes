\documentclass[10pt]{report}
\usepackage[top=1.5cm,bottom=1.5cm,left=1.5cm,right=1.5cm]{geometry}
\usepackage[utf8]{inputenc}
\usepackage[italian]{babel}
\usepackage{amsmath,amssymb,amsfonts,amsthm,stmaryrd}
\usepackage{mathrsfs} % per mathscr
\usepackage{graphicx}% ruota freccia per le azioni
\usepackage{marvosym}% per il \Lightning
\usepackage{array}
\usepackage{faktor} % per gli insiemi quoziente
\usepackage[colorlinks=false]{hyperref}
\usepackage{xparse} % Per nuovi comandi con tanti input opzionali
\usepackage{relsize} % per \mathlarger
\usepackage{tikz-cd}
\usepackage{multicol}
\usepackage{multirow}
\usepackage{cancel}
\usepackage{fourier}
\usepackage{enumerate}
\usepackage{soul}
\usepackage{nicefrac}
\usepackage{longtable}
\usepackage{pdflscape}

\newtheorem*{warn}{\warning \; Attenzione}

\newtheoremstyle{customth}
{\topsep}{\topsep}
{\itshape}{}{\bfseries}{.}{\newline}{}

\newtheoremstyle{customdef}
{\topsep}{\topsep}
{\normalfont}{}{\bfseries}{.}{\newline}{}

\newtheoremstyle{customrem}
{\topsep}{\topsep}
{\normalfont}{}{\itshape}{.}{\newline}{}

\usepackage{fourier}

\theoremstyle{customth}
\newtheorem{theorem}{Teorema}[chapter]
\newtheorem{lemma}[theorem]{Lemma}
\newtheorem{corollary}[theorem]{Corollario}
\newtheorem{proposition}[theorem]{Proposizione}
\newtheorem{fact}[theorem]{Fatto}
\newtheorem{application}[theorem]{Applicazione}
\theoremstyle{customrem}
\newtheorem{remark}[theorem]{Osservazione\,}
\theoremstyle{customdef}
\newtheorem{definition}[theorem]{Definizione}
\newtheorem{notation}[theorem]{Notazione}
\newtheorem{example}[theorem]{Esempio}

\DeclareMathOperator{\BinNeg}{BinNeg}
\DeclareMathOperator{\Geom}{Geom}
\DeclareMathOperator{\Poisson}{Poisson}

\makeatletter
\renewenvironment{proof}[1][\proofname]{\par
  \pushQED{\qed}%
  \normalfont \topsep6\p@\@plus6\p@\relax
  \trivlist
  \item[\hskip\labelsep
        \itshape
    #1\@addpunct{.}]\mbox{}\\*
}{%
  \popQED\endtrivlist\@endpefalse
}
\makeatother

%============ Simboli standard =================
\newcommand{\FF}{\mathcal{F}}
\newcommand{\PP}{\mathcal{P}}
\newcommand{\NN}{\mathbb{N}}
\newcommand{\RR}{\mathbb{R}}
\newcommand{\QQ}{\mathbb{Q}}

\newcommand{\pp}{\text{p\hspace{-0.7em}\raisebox{-3.4pt}{--}}\,\,}
\newcommand{\pbern}{\pp}


\newcommand{\defeq}{\overset{\mathrm{def}}{=}}
\newcommand{\deq}{\overset{\mathrm{(d)}}{=}}
\newcommand{\toprob}{\overset{\mathbb{P}}{\to}}
\DeclareMathOperator{\VA}{VA}
\DeclareMathOperator{\im}{im}
\DeclareMathOperator{\supp}{supp}
\DeclareMathOperator{\id}{id}
\DeclareMathOperator{\sgn}{sgn}

\DeclareMathOperator{\CI}{CI}
\newcommand{\eps}{\varepsilon}

\newcommand{\dx}{\mathop{dx}}

%\setcounter{secnumdepth}{1}

\newcommand{\groupto}{\rightrightarrows}

\newcommand{\restr}[2]{
	#1\arrowvert_{#2}
}

\makeatletter
\def\moverlay{\mathpalette\mov@rlay}
\def\mov@rlay#1#2{\leavevmode\vtop{%
   \baselineskip\z@skip \lineskiplimit-\maxdimen
   \ialign{\hfil$\m@th#1##$\hfil\cr#2\crcr}}}
\newcommand{\charfusion}[3][\mathord]{
    #1{\ifx#1\mathop\vphantom{#2}\fi
        \mathpalette\mov@rlay{#2\cr#3}
      }
    \ifx#1\mathop\expandafter\displaylimits\fi}
\makeatother

\newcommand{\cupdot}{\charfusion[\mathbin]{\cup}{\cdot}}
\newcommand{\bigcupdot}{\charfusion[\mathop]{\bigcup}{\cdot}}

\newcommand{\goesup}{\nearrow}
\newcommand{\goesdown}{\searrow}
\newcommand{\qc}{q.c.\ \!}
\newcommand{\qo}{q.o.\ \!}
\newcommand{\va}{v.a.\ \!}

\newcommand{\BB}{\mathcal{B}}
\newcommand{\EE}{\mathbb{E}}
\DeclareMathOperator{\Var}{Var}
\DeclareMathOperator{\Cov}{Cov}

\newcommand{\inv}{^{-1}}

\newcommand{\abs}[1]{\left\lvert #1 \right\rvert}
\newcommand{\norm}[1]{\left\lVert #1 \right\rVert}


\NeedsTeXFormat{LaTeX2e}
%\ProvidesPackage{quiver}[2021/01/11 quiver]

% `tikz-cd` is necessary to draw commutative diagrams.
\RequirePackage{tikz-cd}
% `amssymb` is necessary for `\lrcorner` and `\ulcorner`.
\RequirePackage{amssymb}
% `calc` is necessary to draw curved arrows.
\usetikzlibrary{calc}
% `pathmorphing` is necessary to draw squiggly arrows.
\usetikzlibrary{decorations.pathmorphing}

% A TikZ style for curved arrows of a fixed height, due to AndréC.
\tikzset{curve/.style={settings={#1},to path={(\tikztostart)
    .. controls ($(\tikztostart)!\pv{pos}!(\tikztotarget)!\pv{height}!270:(\tikztotarget)$)
    and ($(\tikztostart)!1-\pv{pos}!(\tikztotarget)!\pv{height}!270:(\tikztotarget)$)
    .. (\tikztotarget)\tikztonodes}},
    settings/.code={\tikzset{quiver/.cd,#1}
        \def\pv##1{\pgfkeysvalueof{/tikz/quiver/##1}}},
    quiver/.cd,pos/.initial=0.35,height/.initial=0}

% TikZ arrowhead/tail styles.
\tikzset{tail reversed/.code={\pgfsetarrowsstart{tikzcd to}}}
\tikzset{2tail/.code={\pgfsetarrowsstart{Implies[reversed]}}}
\tikzset{2tail reversed/.code={\pgfsetarrowsstart{Implies}}}
% TikZ arrow styles.
\tikzset{no body/.style={/tikz/dash pattern=on 0 off 1mm}}

%PER CAMBIARE I MARGINI
%\usepackage[margin=2cm]{geometry}

%----------- Setup stilistico ----------------
\renewcommand\thefootnote{\textcolor{blue}{\arabic{footnote}}}
\renewcommand{\chaptername}{Parte}
\addto\captionsitalian{\renewcommand{\chaptername}{Parte}}



\title{\Huge{Schede riassuntive di \\ \textit{Elementi di Probabilità e Statistica}}}
\date{A.A. 2023-2024 \\[0.6in] Ultimo aggiornamento: \today \\[1in] \href{https://eps.hearot.it}{\texttt{https://eps.hearot.it}}}
\author{A cura di Gabriel Antonio Videtta\footnote{Basato su un layout di \underline{Luca Lombardo} e di \underline{Francesco Sorce}.} \\ \href{mailto:g.videtta1@studenti.unipi.it}{\texttt{g.videtta1@studenti.unipi.it}} \\[0.3in] Testo basato sul contenuto del corso del prof. Maurelli \\ e del prof. Trevisan tenutosi presso l'Università di Pisa.}

\begin{document}
\maketitle

\begin{multicols*}{2}
    \tableofcontents
\end{multicols*}

\newpage
%--------------------------------------------------------------------
\chapter*{Notazioni impiegate}
\addcontentsline{toc}{chapter}{Notazioni impiegate}  
\setlength{\parindent}{2pt}

\begin{multicols*}{2}
    \section*{Algebra lineare}
    \addcontentsline{toc}{section}{Algebra lineare}

    \begin{itemize}
        \item $q_\varphi$ -- dato uno spazio vettoriale $V$ equipaggiato con un
        prodotto scalare $\varphi$, $q_\varphi$ è la forma quadratica associatogli, ovverosia
        $q_\varphi(v) = \varphi(v, v)$.
        \item $\norm{v}_{\varphi}$ -- dato uno spazio vettoriale reale $V$ equipaggiato con
        un prodotto scalare (semi)definito positivo $\varphi$, $\norm{\cdot}_{\varphi}$ è
        la (semi)norma indotta da $\varphi$, ovverosia $\norm{v}_{\varphi} = \sqrt{q_{\varphi}(v)} = \sqrt{\varphi(v, v)}$.
        \item vettore isotropo --
        vettore che annulla la forma quadratica.
        \item vettore anisotropo -- vettore non isotropo, vettore che non annulla la forma quadratica.
        \item $\cos_\varphi(v, w)$, $\cos(v, w)$ -- dati due vettori anisotropi $v$, $w$ su uno spazio vettoriale reale $V$ equipaggiato
        di un prodotto scalare semidefinito positivo $\varphi$, si definisce
        $\cos_\varphi(v, w)$ (o $\cos(v, w)$ se $\varphi$ è noto dal contesto) in modo tale che:
        \[
            \cos_\varphi(v, w) = \frac{\varphi(v, w)}{\norm{v}_\varphi \cdot \norm{w}_\varphi}.
        \]
        \item vettore $v$ ortogonale a $w$ per $\varphi$ -- Due vettori $v$, $w$ tali
        per cui $\varphi(v, w) = 0$.
        \item $V^\perp_{\varphi}$ -- Radicale del prodotto scalare (o hermitiano) $\varphi$
        sullo spazio $V$, ovverosia sottospazio dei vettori ortogonali ai vettori di tutto
        lo spazio.
        \item $\CI(\varphi)$ -- Sottoinsieme dei vettori di $V$ che annullano $q_{\varphi}$, ossia
        sottoinsieme dei vettori isotropi.
        \item $C_{\varphi}(v, w)$ -- coefficiente di Fourier di
        $v$ rispetto a $w$, ossia $C(v, w) \defeq \frac{\varphi(v, w)}{\varphi(v, v)}$.
    \end{itemize}

    \section*{Analisi matematica}
    \addcontentsline{toc}{section}{Analisi matematica}

    \begin{itemize}
        \item $f(A_i) \goesup x$ -- la successione $(f(A_i))_{i \in \NN}$ a valori
        in $\RR$ è crescente al crescere di $i$ e ha come limite $x$.
        \item $f(A_i) \goesdown x$ -- la successione $(f(A_i))_{i \in \NN}$ a valori
        in $\RR$ è decrescente al crescere di $i$ e ha come limite $x$.
        \item esponente coniugato di $p$ -- per $p > 1$, l'esponente coniugato
        $p'$ di $p$ è un numero reale $p' > 1$ tale per cui:
        \[
            \frac{1}{p} + \frac{1}{p'} = 1.
        \]
        \item $\norm{x}_p$ -- norma $p$-esima del vettore $x \in \RR^n$, ovverosia:
        \[
            \norm{x}_p = \left(\sum_{i \in [n]} \abs{x_i}^p\right)^\frac{1}{p}.
        \]
        Per $p = 2$, si scrive semplicemente $\norm{x}$, e coincide con la norma
        indotta dal prodotto scalare canonico di $\RR^n$.
        \item $f > g$ -- per una funzione $f$ a valori reali, come affermazione
        corrisponde a dire che per un qualsiasi punto del dominio $x$, $f(x) > g(x)$. Si estende naturalmente a $<$, $\geq$, $\leq$ (eventualmente con
        catene di disuguaglianze). Da non
        confondersi con l'insieme $f > g$.
        \item $a$ -- per una costante $a \in \RR$ la mappa costante $D \ni d \mapsto a \in \RR$;
        la sua interpretazione dipende dal contesto.
        \item $f^+$ -- parte positiva di una mappa $f$ a valori reali, ovverosia
        $f^+(a)$ è uguale a $f(a)$ se $f(a) \geq 0$ e $0$ altrimenti.
        \item $f^-$ -- parte negativa di una mappa $f$ a valori reali, ovverosia
        $f^-(a)$ è uguale a $-f(a)$ se $f(a) \leq 0$ e $0$ altrimenti. In questo
        modo $f = f^+ - f^-$.
        \item $\exp$ -- funzione esponenziale $e^x$.
        \item $\log \equiv \ln = \log_e$ -- logaritmo naturale, ossia logaritmo in base $e$.
        \item $C^n$, $C^n(\RR)$ -- classe delle funzioni derivabili $n$ volte con $n$-esima derivata continua. Per $n = 0$, classe di funzioni continue.
        \item $C^\infty$ -- classe delle funzioni derivabili un numero illimitato di volte.
        \item $C_b$, $C_b(\RR)$ -- classe delle funzioni reali, continue e limitate.
        \item $\Gamma(x) = \int_0^\infty t^{x-1} e^{-t} \dt$ -- funzione gamma. È tale per cui $\Gamma(n+1) = n!$ per ogni $n \in \NN$.
        \item $f * g$ -- convoluzione di funzioni; tale che $(f * g)(z)$ sia pari a $\int_\RR f(x) g(z-x) \dx$.
    \end{itemize}

    \section*{Combinatoria}
    \addcontentsline{toc}{section}{Combinatoria}

    \begin{itemize}
        \item $D_{n,k} = \frac{n!}{(n-k)!}$ -- numero di disposizioni ottenute prendendo
        $k$ elementi tra $n$ oggetti.
        \item $\binom{n}{k} = C_{n,k}$ -- il coefficiente binomiale $n$ su $k$,
        ovverosia il numero di combinazioni possibili prendendo $k$ elementi tra $n$ oggetti; equivale a $\frac{n!}{(n-k)!k!} = D_{n,k}/k!$. Alternativamente,
        il numero di sottoinsiemi di $k$ elementi in $[n]$.
        \item $S(I)$ -- gruppo simmetrico relativo a $I$, gruppo delle permutazioni
        di $I$.
        \item $S_n$ -- $n$-esimo gruppo simmetrico, gruppo delle permutazioni
        di $[n]$.
    \end{itemize}

    \section*{Teoria degli insiemi}
    \addcontentsline{toc}{section}{Teoria degli insiemi}

    \begin{itemize}
        \item $\PP(\Omega)$ -- insieme delle parti di $\Omega$, ossia insieme
        dei sottoinsiemi di $\Omega$.
        \item $\restr{f}{A}$ -- restrizione della funzione al dominio $A$.
        \item $A \cupdot B$ -- unione disgiunta di $A$ e $B$, ovverosia $A \cup B$ con
        l'ipotesi che $A \cap B = \emptyset$ (la notazione si estende naturalmente a
        una famiglia di insiemi a due a due disgiunti).
        \item $A \Delta B = A \setminus B \cupdot B \setminus A$ -- differenza simmetrica
        tra $A$ e $B$.
        \item $[n]$ -- l'insieme $\{1, \ldots, n\}$.
        \item $\prod_{i \in I} S_i$ con $S_i$ insieme e $I$ ordinato -- prodotto cartesiano degli $S_i$, ordinato secondo $I$.
        \item $[[n]]$ -- l'insieme $\{0, \ldots, n\} = \{0\} \cup [n]$. 
        \item $\#A$, $\abs{A}$ -- numero di elementi di $A$, o semplicemente la cardinalità di $A$.
        \item insieme finito -- insieme in bigezione con $[n]$ per qualche $n \in \NN$.
        \item insieme numerabile -- insieme in bigezione con $\NN$.
        \item $A_i \goesup A$ -- la famiglia $(A_i)_{i \in \NN}$ è crescente e ha
        come limite $A$, ovverosia $A_i \subseteq A_{i+1}$ per ogni $i \in \NN$ e
        $\bigcup_{i \in \NN} A_i = A$.
        \item $A_i \goesdown A$ -- la famiglia $(A_i)_{i \in \NN}$ è decrescente e ha
        come limite $A$, ovverosia $A_i \supseteq A_{i+1}$ per ogni $i \in \NN$ e
        $\bigcap_{i \in \NN} A_i = A$.
        \item $\omega_i$ -- $i$-esima coordinata di $\omega \in \Omega$, se
        $\Omega$ è un prodotto cartesiano di finiti termini o di un numero
        numerabile di termini.
        \item $A^1 \defeq A$ -- useremo questa notazione per comodità.
        \item $A^c$ -- il complementare di $A$ riferito a $\Omega$, quindi $\Omega \setminus A$, in modo tale che $\Omega = A \cupdot A^c$.
        \item $X\inv(A)$ -- controimmagine dell'insieme $A \subseteq C$ in riferimento
        alla funzione $X : D \to C$, ovverosia $X\inv(A) = \{\omega \in D \mid X(\omega) \in A\}$.
        \item $S_X$, $\im X$ -- immagine della funzione $X$.
        \item $\supp X$ -- supporto di $X$, ovverosia sottoinsieme del
        dominio degli elementi che non annullano $X$.
        \item $1_A$, $I_A$ -- funzione indicatrice di $A$, ovverosia la
        funzione $1_A : B \to [[1]] \subseteq \RR$ riferita ad $A \subseteq B$
        tale per cui:
        \[
            1_A(b) = \begin{cases}
                1 & \text{se } b \in A, \\
                0 & \text{altrimenti}.
            \end{cases}
        \]
        \item $1_{\texttt{exp}}$ -- $1$ se $\texttt{exp}$ è vera, $0$ altrimenti.
        \item $\groupto$ -- simbolo utilizzato al posto $\to$ quando si elencano
        più funzioni che condividono o lo stesso dominio o lo stesso codominio (e.g.~$f$, $g : A$, $B \groupto C$ elenca una funzione $f : A \to C$ e una $g : B \to C$; $f$, $g : A \groupto B$, $C$ elenca una funzione
        $f : A \to B$ e una $g : A \to C$).
    \end{itemize}

    \section*{Topologia generale}
    \addcontentsline{toc}{section}{Topologia generale}

    \begin{itemize}
        \item $\tau(X)$ -- dato $X$ spazio metrico, insieme degli aperti di $X$, ossia topologia di $X$.
        \item spazio separabile -- spazio topologico contenente un denso, ossia un insieme la cui chiusura è tutto lo spazio (e.g.~$\QQ$ per $\RR$).
        \item spazio II-numerabile -- spazio topologico che ammette una base numerabile.
    \end{itemize}

    \section*{Probabilità e teoria della misura}
    \addcontentsline{toc}{section}{Probabilità e teoria della misura}

    \begin{itemize}
        \item $\Omega$ -- spazio campionario, l'insieme di tutti i possibili esiti dell'esperimento aleatorio considerato.
        \item $\sigma(\tau)$ -- $\sigma$-algebra generata dalla famiglia $\tau \subseteq \PP(\Omega)$.
        \item $\sigma\{A_1, \ldots, A_n\}$ -- $\sigma$-algebra generata dalla famiglia
        $\tau = \{A_1, \ldots, A_n\} \subseteq \PP(\Omega)$.
        \item $\BB(X)$ -- $\sigma$-algebra dei boreliani, ossia $\sigma$-algebra generata dagli aperti di $X$ spazio metrico separabile.
        \item $\FF$ -- $\sigma$-algebra relativa a $\Omega$, ossia l'insieme dei possibili eventi.
        \item $(\Omega, \FF)$ -- spazio misurabile.
        \item $\pi$-sistema -- insieme $I \subseteq \FF$, $I \neq \emptyset$ con $(\Omega, \FF)$ spazio misurabile, $\sigma(I) = \FF$ e $I$ chiuso per intersezioni.
        \item $\mu$ -- misura su uno spazio misurabile.
        \item $m$ -- misura di Lebesgue sullo spazio misurabile $(\RR, \BB(\RR))$. È tale per cui $m([a, b]) = b-a$ per $b > a$.
        \item $m$ -- misura di Lebesgue sullo spazio misurabile $\left(\RR^d, \BB\left(\RR^d\right)\right)$ con $d \geq 1$. È tale per cui
        $m\left([a_1, b_1] \times \cdots \times [a_d, b_d]\right) = (b_1 - a_1) \cdots (b_d - a_d)$ con $a_i$, $b_i \in \RR$ e
        $b_i > a_i$ per $1 \leq i \leq d$. Non si distingue generalmente la notazione dal caso unidimensionale.
        \item $P$ -- misura di probabilità su uno spazio misurabile.
        \item $(\Omega, \FF, P)$ -- spazio di probabilità.
        \item \qc -- quasi certo/quasi certamente.
        \item \qo -- quasi ovunque.
        \item $p$ -- per $\Omega$ discreto, funzione di densità discreta; per una probabilità discreta $P$, la densità discreta della probabilità
        ristretta all'insieme $\Omega_0$ su cui è concentrata $P$ o, con abuso di notazione, la mappa $x \mapsto P(\{x\})$ (che coincide
        sui termini di $\Omega_0$ con $p$ e che è $0$ negli altri punti).
        \item $\delta_a$ -- delta di Dirac; dato uno spazio misurabile $(\Omega, \FF)$ e $a \in \Omega$, probabilità tale per cui
        $\delta_a(A) = 1$ se $a \in A$ e $0$ altrimenti (tale probabilità è concentrata in $\{a\}$ ed è dunque
        discreta).
        \item f.d.r.~-- funzione di ripartizione, rispetto a una probabilità reale.
        \item $F$, $F_P$ -- per una probabilità reale, funzione di ripartizione.
        \item $f$ -- densità (in senso reale) della probabilità.
        \item AC -- assolutamente continua, riferito a una probabilità.
        \item \va -- variabile aleatoria.
        \item $P^X$ -- legge della v.a.~$X$ rispetto a $P$.
        \item $p_X$ -- densità della legge della v.a.~$X$, rispetto a $P$.
        \item $X \in A$ -- per una \va $X : \Omega \to S$,
        $X \in A$ è l'insieme $X\inv(A)$. Si estende naturalmente
        al caso $\notin$.
        \item $X = a$ -- per una \va $X : \Omega \to S$,
        $X = a$ è l'insieme $X\inv(a)$. Si estende naturalmente
        al caso $\neq$.
        \item $X = Y$ -- per due \va $X, Y : \Omega \groupto S$
        l'insieme $\{ \omega \in \Omega \mid X(\omega) = Y(\omega) \}$.
        Si estende naturalmente al caso $\neq$ e in modo analogo a $>$, $<$, $\leq$, $\geq$.
        \item $X > a$ -- per una \va reale $X : \Omega \to \RR$,
        $X > a$ è l'insieme $X\inv((a, \infty))$; per una \va discreta
        $X : \Omega \to \RR$ è l'insieme $X\inv(\{m \in \NN \mid m > a\})$.
        Si estende naturalmente ai casi $<$, $\leq$, $\geq$ (eventualmente
        anche con una catena di disuguaglianze). Da non confondersi con
        l'affermazione $X > a$ per $X$ a valori reali.
        \item $\varphi(X)$ -- per una \va, la composizione $\varphi \circ X$.
        \item $\deq$, $\sim$ -- per due v.a.~$X, Y : \Omega_1, \Omega_2 \groupto S$
        indica l'uguaglianza di legge, ovverosia $P_{\Omega_1}^X = P_{\Omega_2}^Y$.
        \item i.d.~-- identicamente distribuite; utilizzato in relazione a un gruppo
        di v.a.~che condividono la stessa legge (spesso rispetto a uno stesso $\Omega$).
        \item i.i.d.~-- indipendenti e identicamente distribuite; utilizzato in relazione
        a un gruppo di v.a.~indipendenti che condividono la stessa legge (spesso rispetto
        a uno stesso $\Omega$).
        \item $(X_i)_{i \in I}$ -- famiglia di v.a., oppure v.a.~congiunta.
        \item $(X_1, \ldots, X_n)$ -- per una famiglia $(X_i : \Omega \to S_i)_{i \in [n]}$ di
        v.a.~indica la v.a.~congiunta (multivariata) $(X_1, \ldots, X_n) : \Omega \to \prod_{i \in [n]} S_i$, $\omega \mapsto (X_1(\omega), \ldots, X_n(\omega))$. Se la
        famiglia è composta da due variabili, si dice anche \textit{coppia bivariata}.
        \item $P(A, B) \defeq P(A \cap B)$ -- notazione introdotta per scrivere
        più comodamente $P(X = x, Y = y)$ in luogo di $P((X = x) \cap (Y = y))$. Si
        generalizza in modo naturale a più eventi.
        \item $L(A, B) \defeq \frac{P(A \mid B)}{P(A)}$ -- rapporto di influenza tra
        $A$ e $B$.
        \item $\bigotimes_{i \in [n]} P_i = P_1 \otimes \cdots \otimes P_n$ --
        Date $P_i$ probabilità su $S_i$ discreto, $P_1 \otimes \cdots \otimes P_n \defeq P$ è la misura di probabilità naturale su $\prod_{i \in [n]} S_i$ tale per cui
        le proiezioni $\pi_i$ siano v.a.~discrete indipendenti e per cui
        $P(\pi_i = x_i) = p_i(x_i)$ per ogni $x_i \in S_i$, $i \in [n]$.
        \item $\EE[X]$ -- valore atteso di $X$.
        \item $\EE[X \mid A] = \defeq \frac{\EE[X \cdot 1_A]}{P(A)}$ -- valore atteso di $X$
        condizionato a $A$.
        \item $\Cov(X, Y) \defeq \EE[(X - \EE[X])(Y - \EE[Y])]$ -- covarianza di $X$ e $Y$.
        \item $\Var(X) \defeq \Cov(X, X)$ -- varianza di $X$.
        \item $\sigma(X) \defeq \sqrt{\Var(X)}$ -- deviazione standard di $X$.
        \item $\rho(X, Y)$ -- coefficiente
        di correlazione di Pearson, ovverosia
        $\cos_{\Cov}(X, Y) = \frac{\Cov(X, Y)}{\sigma(X) \cdot \sigma(Y)}$.
        \item $a^*$, $b^*$ -- date due
        v.a.~$X$, $Y$, $a^*$ e $b^*$ sono
        i parametri della retta di
        regressione $y = a^*x + b^*$.
        \item $I(t)$ -- trasformata di Cramer.
        \item LGN - Legge dei Grandi Numeri.
        \item TCL, TLC - Teorema Centrale del Limite.
        \item $m$, $\sigma$ -- spesso nel contesto
        della LGN e del TCL si usa $m$ per
        indicare $\EE[X_1]$ e $\sigma$ per
        indicare $\sigma(X_1)$.
    \end{itemize}
\end{multicols*}

%--------------------------------------------------------------------
\chapter*{Prerequisiti matematici}
\addcontentsline{toc}{chapter}{Prerequisiti matematici}  
\setlength{\parindent}{2pt}

\begin{multicols*}{2}

\section*{Algebra lineare}

\begin{itemize}
    \item \textbf{Disuguaglianza di Cauchy-Schwarz} -- Se $\varphi(\cdot, \cdot)$
    è un prodotto scalare (o hermitiano) definito positivo su uno spazio vettoriale $V$, allora vale la seguente disuguaglianza:
    \[
        \varphi(v, v) \varphi(w, w) \geq \abs{\varphi(v, w)}^2 , \quad \forall v, w \in V.
    \]
    Inoltre vale l'uguaglianza se e solo se $v$ è multiplo di $w$, o viceversa. Per
    prodotti semidefiniti positivi la disuguaglianza vale ugualmente, ma in
    tal caso $v$ si scrive come somma di un vettore del cono isotropo e del prodotto di $w$ per uno scalare. 
    \item \textbf{Proprietà di $\cos(v, w)$} -- Vale che $\cos(v, w) \in [-1, 1]$ per
    ogni $v$, $w \in V$ in spazi vettoriali reali dove $\cos$ è ben definito. Segue
    dalla disuguaglianza di Cauchy-Schwarz.
\end{itemize}

\section*{Analisi matematica}

\begin{itemize}
    \item \textbf{Limite delle successioni monotone} -- Se una successione
    $(a_i)_{i \in \NN}$ è monotona, allora ammette limite. Se $(a_i)_{i \in \NN}$
    è crescente, allora $a_i \to \sup\{a_i \mid i \in \NN\}$ per $i \to \infty$ (e
    dunque converge se la successione è limitata dall'alto); se
    $(a_i)_{i \in \NN}$ è decrescente, allora $a_i \to \inf\{a_i \mid i \in \NN\}$ per $i \to \infty$ (e dunque converge se la successione è limitata dal basso).
    \item \textbf{Convergenza delle serie a termini positivi} -- Se una serie è
    a termini positivi, allora la successione delle somme parziali è crescente,
    e dunque la serie ammette come valore un valore reale o $\infty$.
    \item \textbf{Convergenza assoluta} -- Se una serie $\sum_{i \in \NN} \abs{a_i}$ converge
    (l'unica altra opzione è che diverga, per la proprietà sopracitata), allora
    $\sum_{i \in \NN} a_i$ converge. Non è vero il viceversa in generale.
    \item \textbf{Disuguaglianza di Jensen} -- Sia $f : \RR \supseteq S \to \RR$ una funzione convessa a
    valori reali. Allora vale che:
    \[
        f\left(\sum_{i \in [n]} a_i x_i\right) \leq \sum_{i \in [n]} a_i f(x_i), \quad \sum_{i \in [n]} a_i = 1, x_i.
    \]
    Se invece $f$ è concava, vale la disuguaglianza con $\geq$ al posto di $\leq$.
    \item \textbf{Disuguaglianza di Young} -- Sia $p \geq 1$ e sia $p'$ il
    suo esponente coniugato. Allora vale che:
    \[
        ab \leq \frac{a^p}{p} + \frac{b^p}{p}, \forall a, b > 0.
    \]
    Segue dalla disuguaglianza di Jensen applicata a $e^x$, che è convessa.
    \item \textbf{Disuguaglianza di Hölder} -- Sia $p > 1$ e sia $p'$ il
    suo esponente coniugato. Allora vale che:
    \[
        \sum_{i \in [n]} \abs{x_i y_i} \leq \norm{x}_p \norm{y}_p, \quad \forall x, y \in \RR^n, \forall n \in \NN.
    \]
    Per $p = 2$, è equivalente alla disuguaglianza di Cauchy-Schwarz sul
    prodotto scalare canonico di $\RR^n$. Segue dalla disuguaglianza di Young.
    \item \textbf{Disuguaglianza sulle potenze} -- Siano $x$, $y \in \RR$ e sia
    $p \geq 1$. Allora vale che:
    \[
        \abs{x+y}^p \leq 2^{p-1} (\abs{x}^p + \abs{y}^p).
    \]
    Segue dalla disuguaglianza di Jensen applicata a $f(t) = t^p$ per
    $\abs{x}$ e $\abs{y}$ ($t^p$ è convessa per $t \geq 0$).
\end{itemize}

\section*{Combinatoria}

\begin{itemize}
    \item \textbf{Principio di \textit{double counting}} -- Principio di dimostrazione per il quale
    se vi sono due modi diversi, ma equivalenti, di contare lo stesso numero di scelte
    di un qualsiasi sistema, allora le formule ricavate dai due modi devono
    essere identicamente uguali.
    \item \textbf{Principio di inclusione-esclusione} -- Teorema da cui discende che per $(A_i)_{i \in [n]}$ vale che: \[\abs{\bigcup_{i \in [n]} A_i} = \sum_{j \in [n]} (-1)^{j+1} \sum_{1 \leq i_1 < \cdots < i_j \leq n} \abs{\bigcap_{k \in [j]} A_{i_k}}.\]
    Inoltre vale che $\abs{\bigcup_{i \in [n]} A_i} = \sum_{i \in [n]} \abs{A_i}$ se e solo se gli $A_i$ sono a due a due disgiunti. Per $n = 2$,
    $\abs{A \cup B} = \abs{A} + \abs{B} - \abs{A \cap B}$.
    \item \textbf{Principio della piccionaia} (\textit{Pigeonhole principle}) -- Teorema che
    asserisce che per ogni funzione $f : [n+1] \to [n]$ esistono $i$, $j \in [n+1]$
    tali per cui $f(i) = f(j)$. Più informalmente, se si hanno $n+1$ oggetti da
    posizionare in $n$ buchi, esiste per forza un buco con due oggetti.
    \item \textbf{Principio della piccionaia generalizzato} -- Teorema che asserisce che
    per ogni funzione $f : [kn+1] \to [n]$ esistono $k+1$ elementi di $[kn+1]$ che
    condividono la stessa immagine. Più informalmente, se si hanno $kn+1$ oggetti
    da posizionare in $n$ buchi, esiste per forza un buco con $k+1$ oggetti. Segue per
    induzione dal Principio della piccionaia.
    \item \textbf{Principio moltiplicativo} -- Se una scelta può essere fatta in $N$
    passi e all'$i$-esimo passo corrispondono $n_i$ scelte, allora la scelta globale
    può essere fatta in $\prod_{i \in [N]} n_i$ modi.
    \item \textbf{Permutazioni di $n$ oggetti} -- Dati $n$ oggetti, esistono
    $n!$ modi di permutarli. Segue dal Principio moltiplicativo.
    \item \textbf{Disposizioni semplici di $n$ oggetti in $k$ posti} -- Dati $n$ oggetti
    e $k$ posti, allora esistono $D_{n,k}$ modi di disporre gli $n$ oggetti nei
    $k$ posti se $k \leq n$. Se $k = n$, ci si riduce a contare le permutazioni.
    \item \textbf{Disposizioni con ripetizione di $n$ oggetti in $k$ posti} -- Dati
    $n$ oggetti e $k$ posti, allora esistono $n^k$ modi di disporre con ripetizione gli $n$ oggetti
    nei $k$ posti. Segue dal Principio moltiplicativo.
    \item \textbf{Combinazioni di $n$ oggetti in $k$ posti} -- Dati $n$ oggetti
    e $k$ posti, allora esistono $C_{n,k} = \binom{n}{k} = \frac{n!}{(n-k)!k!}$ modi di disporre gli $n$ oggetti nei
    $k$ posti non facendo contare l'ordine, se $k \leq n$. Segue dal Principio
    moltiplicativo.
    \item \textbf{Combinazioni con ripetizione di $n$ oggetti in $k$ buchi} -- Data
    l'equazione $x_1 + \ldots + x_k = n$ con $x_i \in \NN$, esistono esattamente
    $\binom{n+k-1}{k-1}$ soluzioni. Alternativamente, data la disequazione
    $x_1 + \ldots + x_k \leq n$ con $x_i \in \NN$, esistono esattamente
    $\binom{n+k}{k}$ soluzioni (dacché ha le stesse soluzioni di
    $x_1 + \ldots + x_k + y = n$, dove $y \in \NN$). È un'applicazione di una
    tecnica combinatorica standard denominata \textit{stars and bars}.
    \item \textbf{Numero di scelte possibili per un'estrazione di $n$ palline rosse e nere da un insieme di $N_1$ palline rosse unito a un insieme di $N-N_1$ palline nere} -- Se $k$ è il numero di palline rosse estratte, le scelte possibili sono
    $\binom{N_1}{k} \binom{N - N_1}{n-k}$. Si può generalizzare il problema a
    un insieme di $N$ palline divise in $m$ gruppi da $N_i$ palline ciascuno
    (e dunque $\sum_{i \in [m]} N_i = N$) dove se ne estrae $n$ e $k_i$ è il
    numero di palline estratte dall'$i$-esimo gruppo (dunque $\sum_{i \in [m]} k_i = n$;
    in tal caso le scelte possibili sono $\prod_{i \in [m]} \binom{N_i}{k_i}$. Segue
    dal Principio moltiplicativo.
    \item \textbf{Identità sulle cardinalità}
\begin{itemize}
    \item $\#\{(a_1, \ldots, a_n) \in [k]^n \mid a_1 < a_2 < \ldots < a_n\} = \binom{n}{k}$ se $k \leq n$ -- Infatti data una classe di disposizione, esiste un unica lista ordinata
    in tale classe.
    \item $\#\{(a_1, \ldots, a_n) \in [k]^n \mid a_1 \leq a_2 \leq \ldots \leq a_n\} = \binom{n + k - 1}{k - 1}$. -- È sufficiente osservare che si sta
    contando esattamente le combinazioni con ripetizione in perfetta analogia con la precedente
    cardinalità.
\end{itemize}
\end{itemize}

\section*{Teoria degli insiemi}

\begin{itemize}
    \item \textbf{Leggi di De Morgan} -- Se $A$ e $B$ sono insiemi, allora
    $(A \cup B)^c = A^c \cap B^c$ e $(A \cap B)^c = A^c \cup B^c$.
    \item \textbf{Operazioni con $X\inv$ controimmagine} -- Se $X : D \to C$ è
    una funzione e $\FF = (A_i)_{i \in I}$ è una famiglia di sottoinsiemi di $C$, allora vale che $X\inv(\bigcup_{i \in I} A_i) = \bigcup_{i \in I} X\inv(A_i)$,
    $X\inv(\bigcap_{i \in I} A_i) = \bigcap_{i \in I} X\inv(A_i)$,
    $X\inv(A_i^c) = X\inv(A_i)^c$, ovverosia $X\inv$ commuta con unioni ($\cup$),
    intersezioni ($\cap$) e complementare ($^c$). $X\inv(\emptyset) = \emptyset$, e dunque $A_i \cap A_j = \emptyset \implies X\inv(A_i) \cap X\inv(A_j) = \emptyset$.
    Inoltre per $Y : C \to C'$ vale che $(Y \circ X)\inv(A) = X\inv(Y\inv(A))$,
    per $A \subseteq C'$.
\end{itemize}

\end{multicols*}
%--------------------------------------------------------------------
\chapter*{Lista delle identità sulle sommatorie}
\addcontentsline{toc}{chapter}{Lista delle identità sulle sommatorie}  
\setlength{\parindent}{2pt}

\section*{Identità sulle sommatorie}

\begin{itemize}
    \item $\binom{n}{k} = \binom{n}{n-k}$ -- ogni scelta di $k$ oggetti corrisponde
    a non sceglierne $n-k$, e dunque vi è un principio di ``dualità''.
    \item $\binom{n}{k} = \binom{n-1}{k-1} + \binom{n-1}{k}$ -- le combinazioni
    di $n$ oggetti in $k$ posizioni si ottengono facendo la somma delle combinazioni
    ottenute fissando un oggetto e combinando gli altri $n-1$ oggetti sui $k-1$
    posti rimanenti, e delle combinazioni ottenute ignorando lo stesso oggetto,
    ossia combinando gli altri $n-1$ oggetti su tutti e $k$ i posti.
    \item $(1 + x)^n = \sum_{i = 0}^n \binom{n}{i} x^i$ -- Teorema del binomio di Newton.
    \item $2^n = \sum_{i = 0}^n \binom{n}{i}$ -- Segue immediatamente dal Teorema del binomio di Newton; è coerente col fatto che si stanno contando le parti di $[n]$.
    \item $\sum_{i = 0}^n (-1)^i \binom{n}{i} = 0$ -- Segue immediatamente dal Teorema del binomio
    di Newton (infatti $(1-1)^n = 0$).
    \item $\sum_{i = 0}^n i \binom{n}{i} = n 2^{n-1}$ -- Segue derivando rispetto a $x$ l'identità
    del Teorema del binomio di Newton.
    \item $\sum_{i = 0}^n \binom{n}{i} \pp^i (1 - \pp)^{n-i} = 1$ per $\pp \in [0, 1]$ -- Segue dal Teorema del binomio di Newton.
    \item $\sum_{i=0}^n \binom{n}{i}^2 = \sum_{i=0}^n \binom{n}{i} \binom{n}{n-i} = \binom{2n}{n}$ -- Dato un gruppo di $n$ maschi e di $n$ femmine, si vuole
    contare quanti team di $n$ persone si possono costruire prendendo persone
    da entrambi i gruppi. Chiaramente la risposta è $\binom{2n}{n}$, ma si
    può contare lo stesso numero di scelte fissando a ogni passo l'indice
    $i$, che conta il numero di maschi nel team, a cui corrispondono
    $\binom{n}{i} \binom{n}{n-i}$ scelte. L'identità segue dunque dal Principio
    del \textit{double counting}.
    \item $\sum_{i=r}^n \binom{i}{r} = \binom{n+1}{r+1}$ -- Dato un gruppo di $r$
    persone distinguibili e di $n$ bastoni indistinguibili, per contare le possibili
    distribuzioni con cui si possono affidare gli $n$ bastoni è sufficiente applicare
    la combinazione con ripetizione, ottenendo $\binom{n+r-1}{r-1}$; un altro modo
    di far ciò è fissare $i$ bastoni da affidare a una persona fissata in precedenza
    e distribuire gli $n-i$ bastoni rimanenti tra gli altri, che a ogni $i$ si può fare in
    $\binom{n-i+k-2}{k-2}$ modi. L'identità segue dunque dal Principio del \textit{double
    counting} riparametrizzando la somma ottenuta.
    \item $\sum_{i=1}^n i = \frac{n(n+1)}{2}$ -- Somma dei numeri da $1$ a $n$.
    \item $\sum_{i=1}^n i^2 = \frac{n(n+1)(2n+1)}{6}$ -- Somma dei quadrati da $1$ a $n$.
    \item $\sum_{i=1}^n i^3 = \left[ \sum_{i=1}^n i \right]^2 = \frac{n^2(n+1)^2}{4}$ -- Somma
    dei cubi da $1$ a $n$.
    \item $\sum_{i=0}^n a^i = \frac{a^{n+1}-1}{a-1}$ per $a \neq 1$, $n$ altrimenti -- Somma
    delle potenze di $a$ con esponente da $0$
    a $n$.
    \item $\sum_{i=0}^n i a^i = \frac{a}{(1-a)^2} \left[1 - (n+1)a^n + na^{n+1} \right]$ -- Segue derivando la somma delle potenze.
    \item $\sum_{i=0}^n i^2 a^i = \frac{a}{(1-a)^3} \left[ (1+a) - (n+1)^2 a^n + (2n^2 + 2n-1)a^{n+1} - n^2 a^{n+2} \right]$ -- Segue
    derivando due volte la somma delle potenze.
    \item $\sum_{i=0}^\infty x^i = \frac{1}{1-x}$ per $\abs{x} < 1$ -- Serie geometrica. Deriva
    prendendo il limite per $n \to \infty$ della
    somma di potenze.
    \item $\sum_{i=0}^\infty i x^i = \frac{x}{(1-x)^2}$ per $\abs{x} < 1$ -- Segue derivando la serie geometrica.
    \item $\sum_{i=0}^\infty i^2 x^i = \frac{x(x+1)}{(1-x)^3}$ per $\abs{x} < 1$ -- Segue derivando due volte
    la serie geometrica.
\end{itemize}

%--------------------------------------------------------------------
\chapter{Spazi di probabilità in generale}
\setlength{\parindent}{2pt}

\begin{multicols*}{2}
    \section{Definizioni preliminari}

    \subsection{Esperimento aleatorio, spazi campionari}

    \begin{definition}[Esperimento aleatorio]
        Si dice \textbf{esperimento aleatorio} un fenomeno il cui esito
        non è determinabile a priori.
    \end{definition}
    
    \begin{definition}[Spazio campionario]
        Si definisce \textbf{spazio campionario}, spesso indicato con
        $\Omega$, un insieme non vuoto che contiene gli
        esiti di un esperimento aleatorio.
    \end{definition}

    \subsection{\texorpdfstring{$\sigma$}{σ}-algebre, spazi e funzioni misurabili}

    \begin{definition}[$\sigma$-algebra]
        Una $\sigma$-algebra $\FF$ di $\Omega$ è un sottoinsieme $\FF \subseteq \PP(\Omega)$ tale per cui:

        \begin{enumerate}[(i.)]
            \item $\Omega \in \FF$,
            \item $A \in \FF \implies A^c \in \FF$,
            \item per $(A_i)_{i \in \NN}$ famiglia numerabile di insiemi
                in $\FF$, $\bigcup_{i \in \NN} A_i \in \FF$ ($\FF$ è chiuso per unioni numerabili).
        \end{enumerate}
    \end{definition}

    Una $\sigma$-algebra $\FF$ di uno spazio campionario $\Omega$ rappresenta l'insieme degli
    \textbf{eventi accettabili}. In particolare:

    \begin{definition}[Spazio misurabile]
        Si definisce \textbf{spazio misurabile} una coppia
        $(\Omega, \FF)$, dove $\FF$ è una $\sigma$-algebra
        di $\Omega$. Gli elementi di $\FF$ sono detti
        \textbf{insiemi misurabili} (e nel caso della probabilità,
        \textbf{eventi}).
    \end{definition}

    \begin{definition}[Funzione misurabile]
        Data una funzione $f$ dallo spazio misurabile $(X, \FF)$ allo spazio
        $(Y, \cS)$ si dice \textbf{misurabile} se $f\inv(A) \in \FF$ per ogni
        $A \in \cS$, ovverosia se la controimmagine di un insieme misurabile è
        misurabile.
    \end{definition}

    \begin{remark}
        Se $\mathcal{G}$ genera $\cS$, allora è sufficiente verifica che
        $f\inv(A) \in \FF$ per ogni $A \in \mathcal{G}$ affinché
        $f$ sia misurabile.
    \end{remark}

    \begin{remark}
        Un insieme $A$ è misurabile in $(\Omega, \FF)$ se e solo se
        $1_A$ è misurabile rispetto a $\{0,1\}$ e le sue parti (infatti
        $1_A\inv(1) = A$ e $1_A\inv(0) = A^c$).
    \end{remark}

    \subsection{Insiemi discreti e \texorpdfstring{$\sigma$}{σ}-algebra naturale}

    In alcuni casi la scelta della $\sigma$-algebra $\FF$ è
    naturale, come nel caso in cui si considera uno spazio
    campionario discreto:

    \begin{definition}[Insieme discreto]
        Diciamo che un insieme $\Omega$ è discreto se è finito o numerabile.
        Se non viene esplicitato altrimenti, per $\Omega$ si considererà
        sempre la $\sigma$-algebra naturale $\PP(\Omega)$.
    \end{definition}

    \subsection{Proprietà di una \texorpdfstring{$\sigma$}{σ}-algebra e \texorpdfstring{$\sigma$}{σ}-algebra generata}

    In casi non discreti, è invece più naturale considerare
    $\sigma$-algebre molto meno grandi dell'insieme delle
    parti; in particolare, come vedremo nella \textit{Parte 3},
    sarà naturale chiedersi qual è la $\sigma$-algebra più
    piccola che contiene una certa famiglia di insiemi:

    \begin{definition}[$\sigma$-algebra generata da una famiglia di insiemi]
        Sia $\tau$ una famiglia di sottoinsiemi di $\PP(\Omega)$. Allora
        si definisce la $\sigma$-algebra
        generata da $\tau$, detta $\sigma(\tau)$, come la più
        piccola $\sigma$-algebra contenente $\tau$. Equivalentemente:
        \[
            \sigma(\tau) = \bigcap_{\substack{\FF \subseteq \PP(\Omega) \\ \tau \subseteq \FF \\ \FF \; \sigma\text{-alg.}}} \FF.
        \]
    \end{definition}

    \begin{remark}
        La definizione data è una buona definizione dal momento che si
        verifica facilmente che l'intersezione di $\sigma$-algebre è ancora
        una $\sigma$-algebra.
    \end{remark}

    \begin{proposition}[Proprietà di $\FF$] Se $\FF$ è una $\sigma$-algebra
    di $\Omega$, allora:
        \begin{enumerate}[(i.)]
            \item $\emptyset \in \FF$,
            \item per $(A_i)_{i \in \NN}$ famiglia numerabile di insiemi
                in $\FF$, $\bigcap_{i \in \NN} A_i \in \FF$ ($\FF$ è chiuso per intersezioni numerabili),
            \item $A \setminus B = A \cap B^c \in \FF \impliedby A$, $B \in \FF$.
        \end{enumerate}
    \end{proposition}

    \section{Corrispondenze logiche e relazionali tra eventi}

    \begin{remark}[Corrispondenze affermazioni ed eventi]
        Ad alcune affermazioni logiche su $A$ e $B$ eventi di $\FF$ corrispondono degli eventi ben precisi o delle
        relazioni:
        \begin{itemize}
            \item ``Si verificano $A$ e $B$'' corrisponde a $A \cap B$,
            \item ``Si verifica $A$ o $B$'' corrisponde a $A \cup B$,
            \item ``Si verifica esattamente uno tra $A$ e $B$'' corrisponde a $A \setminus B \cupdot B \setminus A = A \Delta B$ (differenza simmetrica),
            \item ``Non si verifica $A$'' corrisponde a $A^c$,
            \item ``Si verifica qualcosa'' corrisponde a $\Omega$,
            \item ``Non si verifica niente'' corrisponde a $\emptyset$,
            \item ``Se succede $A$, allora succede $B$'' corrisponde a $A \subseteq B$,
            \item ``Non succedono $A$ e $B$ contemporaneamente'' corrisponde a
                $A \cap B = \emptyset$.
        \end{itemize}
    \end{remark}

    \section{Misure di probabilità}

    \subsection{La probabilità \texorpdfstring{$P$}{P} su \texorpdfstring{$\Omega$}{Ω} e spazi di probabilità}

    \begin{definition}[Probabilità \texorpdfstring{$P$}{P} su $(\Omega, \FF)$ secondo Kolmogorov]
        Dato $(\Omega, \FF)$ spazio misurabile, una \textbf{misura
        di probabilità} $P$, detta semplicemente \textit{probabilità},
        è una funzione $P : \FF \to \RR$ tale per cui:

        \begin{enumerate}[(i.)]
            \item $P(\Omega) = 1$,
            \item $0 \leq P(A) \leq 1$ per ogni $A \in \FF$ (ossia $P$ può restringersi su $[0, 1]$ al codominio),
            \item $P(\bigcupdot_{i \in \NN} A_i) = \sum_{i \in \NN} P(A_i)$ ($\sigma$-additività).
        \end{enumerate}

        In particolare $P$ è una misura per cui $P(\Omega) = 1$.
    \end{definition}

    \begin{definition}[Spazio di probabilità]
        Si dice \textbf{spazio di probabilità} una tripla
        ($\Omega$, $\FF$, $P$) dove ($\Omega$, $\FF$) è
        uno spazio misurabile e $P$ è una
        probabilità su ($\Omega$, $\FF$).
    \end{definition}

    \subsection{Proprietà della probabilità \texorpdfstring{$P$}{P}}

    \begin{proposition}[Proprietà di $P$]
        Se $P$ è una probabilità su ($\Omega$, $\FF$), allora:

        \begin{enumerate}[(i.)]
            \item $P(\emptyset) = 0$,
            \item $P(\bigcupdot_{i \in [n]} A_i) = \sum_{i \in [n]} P(A_i)$ ($\sigma$-additività finita),
            \item $P(A) + P(A^c) = 1$,
            \item $A \subseteq B \implies P(A) \leq P(B)$ e $P(B \setminus A) = P(B) - P(A)$ (segue da (iii.)),
            \item $P(B \setminus A) = P(B) - P(A \cap B)$ (segue da (iv) considerando che $B \setminus A = B \setminus (A \cap B)$),
            \item $P(A \cup B) = P(A \Delta B \cupdot A \cap B) = P(A) + P(B) - P(A \cap B)$ (segue da (v.)),
            \item $P(\bigcup_{i \in [n]} A_i) = \sum_{j \in [n]} (-1)^{j+1} \sum_{1 \leq i_1 < \cdots < i_j \leq n} P(\bigcap_{k \in [j]} A_{i_{k}})$ (segue da (vi.) per induzione, Principio di inclusione-esclusione ``probabilistico''),
            \item $P(\bigcup_{i \in \NN} A_i) \leq \sum_{i \in \NN} P(A_i)$ ($\sigma$-subadditività).
        \end{enumerate}
    \end{proposition}

    \begin{remark}
        Per $\Omega$ finito, la $\sigma$-additività finita implica la $\sigma$-additività per il Principio della piccionaia.
    \end{remark}

    \begin{proposition}[Comportamento di $P$ al limite]
        Sia $(A_i)_{i \in \NN}$ una famiglia numerabile di
        eventi in $\FF$ sullo spazio di probabilità
        $(\Omega, \FF, P)$. Allora:

        \begin{enumerate}[(i.)]
            \item $A_i \goesup A \implies P(A_i) \goesup P(A)$,
            \item $A_i \goesdown A \implies P(A_i) \goesdown P(A)$.
        \end{enumerate}
    \end{proposition}

    \subsection{Eventi incompatibili, quasi certi e trascurabili, proprietà che accadono q.c.}

    \begin{definition}[Eventi trascurabili e quasi certi]
        Sia $A \in \FF$. Allora $A$ si dice \textbf{trascurabile} se
        $P(A) = 0$; si dice \textbf{quasi certo} se $P(A) = 1$.
    \end{definition}

    \begin{definition}[Eventi incompatibili]
        Due eventi $A$, $B \in \FF$ si dicono \textbf{incompatibili} se
        $A \cap B = \emptyset$.
    \end{definition}

    \begin{definition}[$q$ accade \qc]
        Si dice che una proprietà $q$ \textbf{accade quasi certamente (\qc)}
        se esiste $A \in \FF$ quasi certo che soddisfa
        $q$.
    \end{definition}

    \begin{remark}
        Si osserva che la nozione di proprietà che accade \qc è perfettamente
        coerente con la nozione di proprietà che accade \qc riferita a
        $P$ come misura (e non specificatamente come misura di probabilità) su $\RR$, ovverosia $q$ accade \qc se esiste
        $A \in \FF$ trascurabile tale per cui $A^c$ soddisfi $q$.
    \end{remark}

    \section{Probabilità condizionata}

    \subsection{Definizione di \texorpdfstring{$P(\cdot \mid B)$}{P(•|B)}}

    \begin{definition}[Probabilità condizionata su $B$]
        Dato $B \in \FF$ evento non trascurabile (i.e.~$P(B) \neq 0$),
        la \textbf{probabilità condizionata} su $B$ è la misura
        di probabilità $P(\cdot \mid B)$ sullo stesso spazio misurabile
        tale per cui:
        \[
            P(A \mid B) = \frac{P(A \cap B)}{P(B)}, \quad \forall A \in \FF.
        \]
    \end{definition}

    \begin{proposition}
        $P(\cdot \mid B)$ è una misura di probabilità su $(\Omega, \FF)$.
    \end{proposition}

    \begin{remark}
        La probabilità condizionata su $\Omega$ coincide con $P$.
    \end{remark}

    \begin{remark}
        In generale $P(A \mid \cdot)$ non è una probabilità, dacché
        per $\Omega$ si ricava che $P(A \mid \Omega) = P(A)$, che
        potrebbe non essere $1$.
    \end{remark}

    \subsection{Regola della catena, formula delle probabilità totali e Teorema di Bayes}

    \begin{lemma}[Regola della catena, o della torre]
        Dati $(A_i)_{i \in [n]}$ con $P(\bigcap_{i \in [n]} A_i) > 0$, allora vale che
        $P(\bigcap_{i \in [j]} A_i) > 0$ per ogni $j \leq n$. Inoltre vale che:
        \[ P\left(\bigcap_{i \in [n]} A_i\right) = \left(\prod_{j \in [n-1]} P\left(A_j \,\middle\vert\, \bigcap_{i=j+1}^{n} A_i\right)\right) P(A_n), \]

        che segue per induzione applicando $P(A \cap B) = P(A \mid B) P(B)$.
    \end{lemma}

    \begin{remark}
        Per esempio, la regola della catena per $A$, $B$ e $C$ si riduce
        a:
        \[
            P(A \cap B \cap C) = P(A \mid B \cap C) P(B \mid C) P(C).
        \]
    \end{remark}

    \begin{definition}[Sistema di alternative]
        Una famiglia $(B_i)_{i \in I}$ con $I = \NN$ o
        $I = [n]$ si dice \textbf{sistema di alternative}
        per $\Omega$ se $\Omega = \bigcupdot_{i \in I} B_i$
        e $P(B_i) > 0$ per ogni $i \in I$ (ovverosia
        $B_i$ non è mai trascurabile).
    \end{definition}

    Un sistema di alternative permette di calcolare più agevolmente
    la probabilità di un evento riducendosi alle probabilità
    condizionate, come mostra il:

    \begin{lemma}[Formula delle probabilità totali, o formula della partizione]
        Sia $(B_i)_{i \in I}$ un sistema di alternative per $\Omega$. Allora vale
        che:
        \[ 
            P(A) = \sum_{i \in I} P(A \cap B_i) = \sum_{i \in I} P(A \mid B_i) P(B_i).
        \]
    \end{lemma}

    Nella maggior parte dei casi è possibile ``invertire'' una probabilità
    condizionata, ovverosia ricavare una probabilità tra $P(A \mid B)$,
    $P(B \mid A)$, $P(A)$ e $P(B)$ conoscendone tre, a patto che
    $A$ e $B$ non siano trascurabili, come mostra il:

    \begin{theorem}[di Bayes]
        Siano $A$ e $B$ due eventi non trascurabili. Allora vale che:
        \[
            P(A \mid B) = \frac{P(B \mid A) P(A)}{P(B)}.
        \]
        Segue considerando le due scritture possibili di $P(A \cap B)$.
    \end{theorem}

    \begin{remark}
        Applicando il Teorema di Bayes e la formula delle probabilità totali,
        si ricava che per un sistema di alternative $(B_i)_{i \in I}$ e
        $A$ non trascurabile vale che:

        \[
            P(B_i \mid A) = \frac{P(A \mid B_i) P(B_i)}{\sum_{j \in I} P(A \mid B_j) P(B_j)}, \quad \forall i \in I.
        \]
    \end{remark}

    \begin{remark}
        Applicando la regola della catena, la formula delle probabilità totali
        e il Teorema di Bayes è possibile calcolare agevolmente la probabilità
        di un'intersezione di eventi cononoscendone l'albero di sviluppo probabilistico.
        In particolare, per calcolare la probabilità di un nodo è sufficiente
        moltiplicare le probabilità dei rami facenti parte del percorso dal nodo
        alla radice.
    \end{remark}

    \subsection{Rapporto di influenza, correlazione positiva e negativa}

    \begin{definition}[Rapporto di influenza]
        Siano $A$ e $B$ due eventi non trascurabili. Allora
        il \textbf{rapporto di influenza} di $A$ e $B$
        (o più brevemente, la loro \textit{influenza}) è
        il parametro:
        \[
            L(A, B) \defeq \frac{P(A\mid B)}{P(A)},
        \]
        ed è tale per cui:
        \[
            P(A \mid B) = L(A, B) P(A).
        \]
    \end{definition}

    \begin{proposition}
        $L(\cdot, \cdot)$ è simmetrica, ovverosia $L(A, B) = L(B, A)$ per
        ogni evento $A$ e $B$. Segue dal Teorema di Bayes.
    \end{proposition}

    \begin{definition}[Correlazione positiva e negativa tra $A$ e $B$]
        Se $A$ e $B$ sono due eventi non trascurabili, si dice
        che $A$ è \textbf{positivamente correlato} a $B$ (o che
        si \textit{dilata probabilisticamente} rispetto a $B$) se
        $P(A \mid B) \geq P(A)$ (ovverosia se $L(A, B) > 1$). \smallskip
        
        Analogamente
        si dice che $A$ è \textbf{negativamente correlato} a $B$
        (o che si \textit{contrae probabilisticamente} rispetto a $B$) se
        $P(A \mid B) \leq P(A)$ (ovverosia se $L(A, B) < 1$).
    \end{definition}

    \begin{remark}
        Il caso in cui $L(A, B) = 1$ è discusso nella sezione \textit{\nameref{sec:indipendenza}} e corrisponde all'indipendenza
        tra $A$ e $B$.
    \end{remark}

    \begin{remark}
        Si può parlare più generalmente di correlazione tra $A$ e $B$
        senza scegliere un evento ``rispetto'' a cui analizzarla, dacché
        $L(\cdot, \cdot)$ è simmetrica per il Teorema di Bayes. Infatti,
        se $P(A \mid B) \leq P(A)$, anche $P(B \mid A) \leq P(B)$, cioè
        $A$ è correlato positivamente a $B$ se e solo se $B$ è correlato
        positivamente ad $A$. \smallskip


        Una correlazione positiva tra $A$ e $B$ indica che, accadendo $B$,
        si amplifica la probabilità che accada $A$; viceversa, una correlazione
        negativa inficia ridimensionando in contrazione la probabilità che accada $A$
        se accade $B$.
    \end{remark}

    \section{Indipendenza stocastica tra eventi}
    \label{sec:indipendenza}

    \begin{definition}[Famiglia di eventi indipendenti]
        Una famiglia $(A_i)_{i \in I}$ di eventi si dice \textbf{stocasticamente
        indipendente}, o più semplicemente indipendente, se
        per ogni $J \subseteq I$ finito vale che:
        \[
            P(\cap_{j \in J} A_j) = \prod_{j \in J} P(A_j).
        \]

        Nel caso di due eventi questo si riduce a verificare
        che $P(A \cap B) = P(A) P(B)$. Si dice che gli $A_i$ sono
        \textbf{collettivamente indipendenti}.
    \end{definition}

    \begin{remark}
        Generalmente non è sufficiente verificare che ogni coppia di eventi distinti è
        indipendente per verificare che la famiglia è globalmente indipendente.
        Infatti, il significato dell'indipendenza in termini puramente probabilistici
        è che una famiglia $\FF$ è indipendente se e solo se il ``verificarsi'' di
        alcuni eventi della famiglia non influenza il ``verificarsi'' degli altri.
    \end{remark}

    \begin{remark}
        Se $(A_i)_{i \in I}$ è una famiglia di eventi indipendenti, allora
        per $J \subseteq I$, $(A_j)_{j \in J}$ è ancora una famiglia di
        eventi indipendenti (l'indipendenza si tramanda per restrizione).
    \end{remark}

    \begin{proposition}
        Se $P(B) > 0$, allora $A$ e $B$ sono indipendenti se
        e solo se $P(A \mid B) = P(A)$. Inoltre, se
        $(A_j)_{j \in J} \cup \{A\}$ è una famiglia finita di eventi
        non trascurabili (eccetto eventualmente per $A$)
        indipendenti tra loro, allora
        $P(\bigcap_{j \in J} A_j) \neq 0$ e
        $P(A \mid \bigcap_{j \in J} A_j) = P(A)$.
    \end{proposition}

    \begin{proposition}
        Se $A$ e $B$ sono indipendenti, allora anche
        $A^c$ e $B$ sono indipendenti. Analogamente
        lo sono $A$ e $B^c$, così come
        $A^c$ e $B^c$.


        Da ciò segue che se $(A_i)_{i \in I}$ è una famiglia di eventi
        indipendenti, allora $(A_i^{\alpha_i})_{i \in I}$ è una famiglia
        di eventi indipendenti per qualsiasi scelta di $\alpha_i$ in
        $\{1, c\}$.
    \end{proposition}

    \begin{proposition}
        Sia $(A_i)_{i \in I}$ una famiglia di eventi indipendenti. Allora,
        se $I$ è partizionato dagli $I_j$, ovverosia $I = \bigcupdot_{j \in J} I_j$,
        allora $(\bigcap_{i \in I_j} A_{i})_{j \in J})$ è ancora una famiglia
        di eventi indipendenti (ossia intersecando alcuni elementi della famiglia
        e lasciandone invariati altri, la famiglia ottenuta è ancora indipendente).
    \end{proposition}

    \begin{theorem}
        Sia $(A_i)_{i \in I}$ una famiglia di eventi indipendenti. Allora,
        ogni operazione di unione, intersecazione o complementare di alcuni elementi della famiglia restituisce una famiglia ancora indipendente. \smallskip

        Segue dalle due proposizioni precedenti (infatti $A \cup B = (A^c \cap B^c)^c$).
    \end{theorem}

    \begin{example}
        Per esempio, se $A$, $B$ e $C$ sono indipendenti, anche $A \cup B$, $C^c$
        è indipendente. Se $A$, $B$, $C$ e $D$ sono indipendenti, anche
        $(A \cap B) \cup C^c$ e $D^c$ lo sono.
    \end{example}

    \begin{remark}
        Un'evento $A$ è indipendente da ogni evento $B \in \FF$, incluso
        sé stesso, se e solo se $P(A) \in \{0, 1\}$, ovvero se e solo
        se $A$ è trascurabile o quasi certo (infatti si avrebbe che
        $P(A) = P(A \cap A) = P(A)^2$).
    \end{remark}

    \begin{remark}
        Due eventi incompatibili $A$ e $B$ sono indipendenti se e solo se
        uno dei due è trascurabile.
    \end{remark}
\end{multicols*}

%--------------------------------------------------------------------
\chapter{Probabilità discreta}
\setlength{\parindent}{2pt}

\begin{multicols*}{2}

Consideriamo in questa sezione soltanto i casi in cui $\Omega$ è
un insieme discreto, cioè finito o numerabile. Gli associamo
in modo naturale la $\sigma$-algebra $\PP(\Omega)$.

\section{Funzione di densità discreta}

\subsection{Definizione per il caso discreto}

\begin{definition}[Funzione di densità discreta]
    Per una probabilità $P$ su $\Omega$ si definisce
    \textbf{funzione di densità discreta} (o di massa, o
    più brevemenete di densità)
    la funzione $p : \Omega \to \RR$ tale per cui:
    \[ p(\omega) = P(\{\omega\}), \quad \forall \omega \in \Omega. \]
\end{definition}

\begin{proposition}[$P$ è univocamente determinata da $p$]
    Sia $p : \Omega \to \RR$ una funzione tale per cui:
    \begin{enumerate}[(i.)]
        \item $\sum_{\omega \in \Omega} p(\omega) = 1$,
        \item $p(\omega) \geq 0$ per ogni $\omega \in \Omega$.
    \end{enumerate}
    Allora esiste un'unica probabilità $P$ la cui funzione di densità
    è $p$, e vale che:
    \[
        P(A) = \sum_{a \in A} p(a).
    \]
\end{proposition}

\subsection{Range di una probabilità discreta e restrizione}

\begin{definition}[Range di $P$]
    Sia $P$ una probabilità su $\Omega$ discreto e
    sia $p$ la sua funzione di densità. Si
    definisce allora \textbf{range} $R_P$ di $P$ il
    supporto di $p$, ovverosia:
    \[ R_P \defeq \supp p = \{ \omega \in \Omega \mid p(\omega) > 0\} \subseteq \Omega. \]
\end{definition}

\begin{definition}[Restrizione di $P$ sul range]
    Data $P$ probabilità su $\Omega$ discreto, si
    definisce \textbf{probabilità ristretta sul range $R_P$}
    la misura di probabilità $\restr{P}{R_P} : \PP(R_P)$ tale
    per cui:
    \[
        \restr{P}{R_P}(A) = P(A).
    \]
\end{definition}

\begin{remark}
    La definizione data è una buona definizione dal momento che
    $P(R_P) = 1$.
\end{remark}

\begin{proposition}[Proprietà della restrizione di $P$ sul range]
    Sia $P$ una probabilità su $\Omega$ discreto e sia $p$ la
    sua funzione di densità. Allora vale che $P(A) = \restr{P}{R_P}(A \cap R_P)$.
\end{proposition}

\subsection{Misure di probabilità discrete su spazi campionari non discreti e discretizzazione}
\label{sec:discretizzazione}

\begin{definition}[Probabilità discreta su spazio campionario non discreto]
    Dato $(\Omega, \FF, P)$ spazio di probabilità con $\{\omega\} \in \FF$ per
    ogni $\omega \in \Omega$, la probabilità $P$ si dice \textbf{discreta} su
    $\Omega$ se esiste $\Omega_0 \in \FF$ discreto e quasi certo ($P(\Omega_0) = 1$).
    In tal caso si dice che $P$ si \textit{concentra} su $\Omega_0$.
\end{definition}

\begin{definition}[Discretizzazione di $P$ discreta su $\Omega$]
    Se $P$ è una probabilità discreta su $\Omega$ concentrata su $\Omega_0$,
    si definisce \textbf{discretizzazione di $P$} la misura di probabilità $P_0$
    su $(\Omega_0, \PP(\Omega_0))$ la cui funzione di densità discreta
    è la mappa $p$ per la quale $\Omega_0 \ni \omega_0 \mapsto P(\{\omega_0\})$. Equivalentemente
    vale che:
    \[
        P_0(A) = \sum_{a \in A} p(a) = P(A), \quad \forall A \in \PP(\Omega_0).
    \]
\end{definition}

\begin{proposition}[Proprietà della discretizzazione di $P$]
    Se $P$ è una probabilità discreta su $\Omega$ concentrata su $\Omega_0$, allora
    vale che:
    \[ 
        P(A) = P(A \cap \Omega_0) = P_0(A \cap \Omega_0) = \sum_{a \in A \cap \Omega_0} p(a),
    \]
    dove $p$ è la funzione di densità di $P_0$. Segue dall'identità $P(A \cup \Omega_0) = 1$ e dalla definizione di discretizzazione.
\end{proposition}

\begin{remark}
    In perfetta analogia al caso totalmente discreto, la discretizzazione
    di $P$ discreta su $\Omega$ e concentrata su $\Omega_0$ è univocamente
    determinata da $p$.
\end{remark}

\begin{remark}
    Se $\Omega$ è discreto, allora si può sempre discretizzare
    $P$ al suo range $R_P$.
\end{remark}

\begin{remark}
    \label{remark:identità_discreta_dirac}
    Se $P$ è una probabilità discreta e, per $a \in \Omega$, $\delta_a$ è il \textbf{delta di Dirac}, ovverosia
    la probabilità per cui $\delta_a(A) = 1$ se $a \in A$ e $\delta_a(A) = 0$ se $a \notin A$, allora vale
    la seguente identità:
    \[
        P = \sum_{\omega \in R_P} p(\omega) \, \delta_{\omega},
    \]
    dove si osserva che $R_P$ è numerabile (dacché $P$ è discreta).
\end{remark}

\section{Variabili aleatorie discrete}

\subsection{Definizione di v.a.~discreta e composizione}

\begin{definition}[Variabile aleatoria discreta]
    Dato $S \neq \emptyset$, si definisce \textbf{variabile
    aleatoria} (discreta) su $\Omega$ discreto, abbreviata \va, una funzione
    $X : \Omega \to S$. $X$ si dice \textbf{variabile aleatoria reale}
    (v.a.~reale) se $S \subseteq \RR$ o \textbf{variabile aleatoria vettoriale}
    (v.a.~vettoriale, o \textit{vettore aleatorio}) se $S \subseteq \RR^n$ per
    qualche $n \in \NN$. \smallskip


    Dato $S \neq \emptyset$, definiamo $\VA(\Omega, S)$ come l'insieme
    delle v.a.~discrete di $\Omega$ che hanno $S$ per codominio.
\end{definition}

\begin{remark}
    Si può dotare $\VA(\Omega, \RR)$ di una struttura di algebra, oltre che di
    spazio vettoriale, dove le operazioni di somma vettoriale, di prodotto
    esterno e di prodotto tra vettori sono completamente naturali. \medskip


    Se $\Omega$ è finito, allora $\VA(\Omega, \RR)$ è naturalmente isomorfo
    a $\RR^{\# \Omega}$ come spazio vettoriale, mentre
    nel caso di $\Omega$ numerabile $\VA(\Omega, \RR)$ ammette una base non numerabile.
\end{remark}

\begin{definition}[Composizione di v.a.~discrete]
    Data $X \in \VA(\Omega, S)$ e una funzione $F : S \to S'$,
    si definisce la \textbf{composizione di $X$ tramite $F$}
    come $F(X) = F \circ X \in \VA(\Omega, S')$.
\end{definition}

\subsection{Legge di una v.a.~\texorpdfstring{$X$}{X} e costruzione canonica}

Nel caso di $\Omega$ discreto, $S_X$, ossia l'immagine della v.a.~$X$, è
ancora un insieme discreto. Questo ci porta alla:

\begin{proposition}
    Sia $X : \Omega \to S$ una v.a.~discreta di $\Omega$.
    Sia $P'$ la misura di probabilità sullo spazio misurabile
    $(S, \PP(S))$ tale per cui:
    \[
        P'(A) = P(X \in A) = P(X\inv(A)).
    \]
    Allora $P'$ si concentra su $S_X$ e dunque vale che:
    \[
        P'(A) = P'(A \cap S_X).
    \]
\end{proposition}
    
\begin{definition}[Legge di $X$]
    Data una v.a.~$X : \Omega \to S$, si definisce \textbf{legge di $X$} (o \textit{distribuzione
    di $X$}) la discretizzazione $P^X = \restr{P'}{S_X}$ che
    agisce sullo spazio misurabile $(S_X, \PP(S_X))$, dove
    $P'$ è tale per cui $P'(A) = P(X \in A) = P(X\inv(A))$.
    Equivalentemente vale che:
    \[
        P^X : \PP(S_X) \ni A \mapsto P(X \in A) = P(X\inv(A)).
    \]


    Si indica con $p_X$ la funzione di densità discreta di $P^X$.
    Per $P^X(A)$ con $A \subseteq S$ si intenderà
    $P^X(A \cap S_X)$, e analogamente $p_X(x)$ si estende in modo
    tale che valga $0$ per $x \notin S_X$.
\end{definition}

\begin{remark}
    Dalla definizione della legge di $X$ si ricava immediatamente che:
    \[
        P(X \in A) = P^X(A) = \sum_{x \in A} p_X(x) = \sum_{x \in A} P(X = x),
    \]
    dove si osserva che $X \in A = \bigcupdot_{x \in A} (X = x)$.
\end{remark}

\begin{remark}
    Il range di $P^X$ è:
    \[ R_X \defeq R_{P^X} = \{x \in S \mid p_X(x) = P(X = x) > 0\}, \]
    ovverosia $R_{P^X}$ è composto dagli elementi di $S$ le cui
    controimmagini non siano trascurabili rispetto a $P$.
\end{remark}

\begin{remark}
    Dato uno spazio di probabilità $(S, \PP(S), Q)$ con
    $\Omega$ discreto è sempre possibile trovare uno
    spazio di probabilità $(\Omega, \PP(\Omega), P)$ e una
    v.a.~$X : \Omega \to S$ tale per cui $P^X = Q$. \smallskip

    È sufficiente porre $\Omega = S$, $P = Q$ e $X = \id_{S}$
    (\textbf{costruzione canonica}). Infatti vale che:
    \[
        P^X(A) = P(X \in A)) = P(A) = Q(A).
    \]
\end{remark}

\begin{proposition}
    Data una v.a.~$X : \Omega \to S$ e una funzione $f : S \to E$,
    vale la seguente identità:
    \[
        p_{f(X)}(e) = \sum_{x \in f\inv(e)} p_X(x).
    \]
    Equivalentemente vale che:
    \[
        P(f(X) = e) = \sum_{x \in f\inv(e)} P(X = x).
    \]
    Segue dal fatto che $(f(X) = e) = (X \in f\inv(e))$.
\end{proposition}

\subsection{Uguaglianza q.c., medesima legge e stabilità per composizione}
\label{sec:uguaglianza_qc}

\begin{definition}[Uguaglianza quasi certa tra v.a.]
    Date $X$, $Y \in \VA(\Omega, S)$, si dice che
    \textbf{$X$ è uguale a $Y$ quasi certamente} ($X = Y$ q.c.\footnote{
        Nella definizione compare due volte la scrittura $X = Y$: la prima
        volta si intende dire che la v.a.~$X$ è uguale a quella $Y$ q.c.,
        mentre dove compare la seconda volta si intende l'insieme $(X=Y) \subseteq \Omega$.
    }) rispetto
    alla probabilità $P$ se
    l'insieme $(X = Y) = \{\omega \in \Omega \mid X(\omega) = Y(\omega)\}$
    è quasi certo rispetto a $P$.
\end{definition}

\begin{proposition}[Comportamento delle uguaglianze q.c.~sulla composizione]
    Sia $F : S \to S'$. Siano $X$, $Y \in \VA(\Omega, S)$. Allora se
    $X = Y$ q.c., $F(X) = F(Y)$ q.c. \smallskip

    Segue considerando la seguente relazioni di insiemi: $(X = Y) \subseteq (F(X) = F(Y))$.
\end{proposition}

\begin{definition}[Uguaglianza di leggi tra v.a.]
    Data $X \in \VA(\Omega_1, S)$ e $Y \in \VA(\Omega_2, S)$,
    si dice che \textbf{$X$ e $Y$ hanno la stessa legge},
    e si scrive che $X \deq Y$ o che $X \sim Y$, se
    $P_{\Omega_1}^X \equiv P_{\Omega_2}^Y$.
\end{definition}

\begin{definition}[Variabili aleatorie i.d.]
    Si dice che una famiglia di v.a.~sono \textbf{identicamente distribuite (i.d.)}
    se condividono la stessa legge. \smallskip


    Spesso sottintenderemo che tali v.a.~sono costruite sullo stesso $\Omega$.
\end{definition}

\begin{proposition}
    Se $X = Y$ q.c., allora $X \deq Y$. Segue considerando che
    $P$ è concentrata sull'insieme $X=Y$, e quindi ci si può sempre
    restringere su questo insieme, interscambiando eventualmente
    le v.a.
\end{proposition}

\begin{remark}
    Per $X$, $Y \in \VA(\Omega, S)$ v.a. non è generalmente vero che
    $X \deq Y$ implica $X = Y$ q.c. 
\end{remark}

\begin{proposition}[Comportamento delle uguaglianze di legge sulla composizione]
    Sia $F : S \to S'$. Siano $X$, $Y : \Omega_1, \Omega_2 \groupto S$ v.a. Allora
    se $X \deq Y$, $F(X) \deq F(Y)$.
\end{proposition}

\subsection{Variabile aleatoria multivariata, leggi congiunte e marginali}

\begin{definition}[Variabile aleatoria multivariata, o congiunta]
    Data una famiglia $(X_i : \Omega \to S_i)_{i \in I}$ di
    v.a.~discrete di $\Omega$ con $I$ ordinato, si definisce la \textbf{v.a.~congiunta} (o
    \textit{blocco multivariato}) la variabile discreta $(X_i)_{i \in I}$ tale per cui:
    \[
        (X_i)_{i \in I} : \Omega \ni \omega \mapsto (X_i(\omega))_{i \in I} \in \prod_{i \in I} S_i.
    \]
    Se $I = [n]$, scriviamo $(X_1, \ldots, X_n)$ al posto di $(X_i)_{i \in I}$.
    Sottintenderemo sempre che $I$ è ordinato quando si nomina una famiglia
    di v.a.~discrete.
\end{definition}

\begin{definition}[Legge e densità congiunta]
    Data una famiglia $(X_i : \Omega \to S_i)_{i \in I}$ di
    v.a.~discrete di $\Omega$ e $P$ probabilità su $\Omega$ discreto,
    si dice \textbf{legge congiunta} delle $X_i$
    la legge relativa alla loro v.a.~congiunta, ovverosia
    $P^{(X_i)_{i \in I}}$. Analogamente, con il
    termine \textbf{densità congiunta} ci si riferirà
    alla densità discreta della legge congiunta.
\end{definition}

\begin{definition}[Leggi e densità marginali]
    Data una famiglia $(X_i : \Omega \to S_i)_{i \in I}$ di
    v.a.~discrete di $\Omega$ e $P$ probabilità su $\Omega$ discreto,
    ci si riferisce con il termine di \textbf{legge marginale} a una qualsiasi
    legge $P^{X_i}$ e con il termine di \textbf{densità marginale} alla relativa
    funzione di densità discreta.
\end{definition}

\begin{remark}
    La legge congiunta restituisce \textit{sempre} più informazioni rispetto
    all'insieme delle leggi marginali. Infatti, si può sempre ricostruire una
    legge marginale data la legge congiunta, ma non è sempre vero il
    viceversa. \medskip
\end{remark}

\begin{remark}
    Si osserva che vale la seguente identità:
    \[
        P^{(X_i)_{i \in I}}\left(\prod_{i \in I} A_i\right) = P\left(\bigcap_{i \in I} (X_i \in A_i)\right), \quad \forall A_i \subseteq S_i.
    \]
    Pertanto, nel caso finito vale che:
    \[
        P^{(X_1, \ldots, X_n)}\left(\prod_{i \in I} A_i\right) = P\left(X_1 \in A_1, \ldots, X_n \in A_n\right), \quad \forall A_i \subseteq S_i.
    \]
\end{remark}

\begin{proposition}
    Ogni densità marginale è univocamente determinata dalla densità
    congiunta. In particolare nel caso finito vale che:
    \[
        p_{X_i}(x_i) = \sum_{\substack{x_j \in S_j \\ j \neq i}} p_{(X_1, \ldots, X_n)}(x_1, \ldots, x_n).
    \]
\end{proposition}

\subsection{Indipendenza di variabili aleatorie discrete e stabilità per congiunzione e composizione}

\begin{definition}[Indipendenza tra v.a.~discrete]
    Sia $(X_i : \Omega \to S_i)_{i \in I}$ una famiglia di v.a.~discrete. Si dice che tale famiglia di v.a.~è \textbf{indipendente} se per ogni $n$ e ogni famiglia finita di
    indici distinti $(i_j)_{j \in [n]} \subseteq I$ vale che:
    \[
        P(X_{i_1} \in A_{i_1}, \ldots, X_{i_n} \in A_{i_n}) = \prod_{j \in [n]} P(X_{i_j} \in A_{i_j}), \quad \forall A_{i_j} \subseteq S_{i_j}.
    \]
    Equivalentemente tale famiglia è indipendente se:
    \[
        P^{(X_{i_1}, \ldots, X_{i_n})}(A_{i_1} \times \cdots \times A_{i_n}) = \prod_{j \in [n]} P^{X_{i_j}}(A_{i_j}), \quad \forall A_{i_j} \subseteq S_{i_j}.
    \]
\end{definition}

\begin{definition}[Variabili aleatorie i.i.d.]
    Data una famiglia di variabili aleatorie, si dice che
    queste sono \textbf{indipendenti e identicamente distribuite (i.i.d.)}
    se formano una famiglia di v.a.~indipendenti e se condividono
    la stessa legge. \smallskip

    Spesso sottintenderemo che tali v.a.~sono costruite sullo stesso $\Omega$.
\end{definition}

\begin{remark}
    La definizione è equivalente a richiedere che per ogni scelta di $A_{i_j} \subseteq S_{i_j}$,
    $X_{i_1} \in A_{i_1}$, ..., $X_{i_n} \in A_{i_n}$ formino una famiglia di eventi
    collettivamente indipendenti. Pertanto è possibile sfruttare tutte
    le proposizioni viste nella sottosezione \textit{\nameref{sec:indipendenza}}. \smallskip

    Inoltre, se la famiglia $(X_i)_{i \in I}$ è indipendente, lo è
    chiaramente anche $(X_{\sigma(i)})_{i \in I}$ per ogni $\sigma \in S(I)$
    (in riferimento in particolare alla seconda identità presente nella definizione
    di indipendenza tra v.a.).
\end{remark}

\begin{remark}
    Una v.a.~costante è sempre indipendente con altre v.a., dal momento che
    le sue uniche controimmagini sono $\Omega$ e $\emptyset$, che sono indipendenti
    da ogni evento.
\end{remark}

\begin{remark}
    Si osserva che vale la seguente identità:
    \[
        P(X_1 \in A_1, \ldots, X_n \in A_n) = \sum_{x_i \in A_i} P(X_1 = x_1, \ldots, X_n = x_n).
    \]
\end{remark}

\begin{proposition}
    Sia $(X_i : \Omega \to S_i)_{i \in I}$ una famiglia di v.a.~discrete. Allora
    tale famiglia è indipendente se per ogni $n$ e ogni famiglia finita di
    indici distinti $(i_j)_{j \in [n]} \subseteq I$ vale che:
    \[
        P(X_{i_1} = x_{i_1}, \ldots, X_{i_n} = x_{i_n}) = \prod_{j \in [n]} P(X_{i_j} = x_{i_j}), \quad \forall x_{i_j} \in S_{i_j}.
    \]
    Equivalentemente, sono indipendenti se e solo se:
    \[
        p_{(X_{i_1}, \ldots, X_{i_n})}(x_{i_1}, \ldots, x_{i_n}) = \prod_{j \in [n]} p_{X_{i_j}}(x_{i_j}), \quad \forall x_{i_j} \in S_{i_j}.
    \]
    Segue dalla precedente osservazione.
\end{proposition}

\begin{proposition}
    Sia $(A_i)_{i \in I}$ una famiglia di eventi. Allora tale famiglia
    è indipendente se e solo se la famiglia di v.a.~$(1_{A_i})_{i \in I}$ è
    indipendente. \smallskip


    Segue dalla precedente proposizione; infatti $(1_{A_i} = 1) = A_i$ e
    $(1_{A_i} = 0) = A_i^c$.
\end{proposition}

\begin{proposition}
    \label{prop:indipendenza_composizione}
    Sia $(X_i : \Omega \to S_i)_{i \in I}$ una famiglia di v.a.~discrete e
    sia $(f_i : S_i \to S_{i}')_{i \in I}$ una famiglia di funzioni. Allora
    se $(X_i)_{i \in I}$ è una famiglia di v.a.~indipendenti, anche
    $(f_i(X_i))_{i \in I}$ lo è. \smallskip


    Segue dal fatto che $(f_i(X_i) \in A_i) = (X_i \in f\inv(A_i))$.
\end{proposition}

\begin{proposition}
    \label{prop:indipendenza_partizione}
    Sia $(X_i : \Omega \to S_i)_{i \in I}$ una famiglia di v.a.~discrete e
    sia $I$ partizionato dagli $I_j$, ovverosia $I = \bigcupdot_{j \in J} I_j$.
    Allora se $(X_i)_{i \in I}$ è una famiglia di v.a.~indipendenti, anche
    $((X_i)_{i \in I_j})_{j \in J}$ è una famiglia di v.a.~indipendenti. \smallskip


    Segue applicando la definizione.
\end{proposition}

\begin{remark}
    Le ultime due proposizioni permettono di ricavare molto velocemente l'indipendenza
    di una certa famiglia di v.a.~discrete. Per esempio, se
    $X_1$, $X_2$, $X_3$, $X_4$, $X_5 \in \VA(\Omega, \RR)$ sono indipendenti,
    si ricava immediatamente che $X_1$, $X_2 + X_3$ e $\max(X_4, X_5)$ sono
    indipendenti a partire dal seguente albero, dove ogni colonna è una famiglia
    di v.a.~indipendenti:

    \[\begin{tikzcd}[cramped,column sep=scriptsize,row sep=tiny]
    	{X_1} && {X_1} && {X_1} \\
    	{X_2} && {(X_2, X_3)} && {X_2+X_3} \\
    	{X_3} && {(X_4, X_5)} && {\max(X_4, X_5)} \\
    	{X_4} \\
    	{X_5}
    	\arrow[squiggly, from=1-1, to=1-3]
    	\arrow[squiggly, from=2-1, to=2-3]
    	\arrow[curve={height=6pt}, squiggly, from=3-1, to=2-3]
    	\arrow[curve={height=6pt}, squiggly, from=5-1, to=3-3]
    	\arrow[squiggly, from=4-1, to=3-3]
    	\arrow["{\operatorname{id}}", from=1-3, to=1-5]
    	\arrow["{+}", from=2-3, to=2-5]
    	\arrow["\max", from=3-3, to=3-5]
    \end{tikzcd}\]

    Infatti la prima operazione restituisce una famiglia indipendente
    per la \textit{Proposizione \ref{prop:indipendenza_partizione}}, e la seconda fa lo stesso
    per la \textit{Proposizione \ref{prop:indipendenza_composizione}}.
\end{remark}

\begin{remark}
    Data una famiglia di probabilità $(P_i)_{i \in [n]}$ su spazi misurabili discreti
    $(S_i, \PP(S_i))$ è sempre possibile costruire uno
    spazio discreto di probabilità $(\Omega, \PP(\Omega), P)$ equipaggiato di
    una famiglia di v.a.~$(X_i : \Omega \to S_i)_{i \in [n]}$ tale per cui
    \begin{enumerate}
        \item la famiglia $(X_i)_{i \in [n]}$ è una famiglia di v.a.~indipendenti,
        \item $P^{X_i} \equiv P_i$.
    \end{enumerate}
    È infatti sufficiente porre $\Omega = \prod_{i \in [n]} S_i$ (il prodotto finito di discreti è discreto), $X_i = \pi_i$ (la
    proiezione dal prodotto cartesiano all'insieme $S_i$) con $P$ probabilità
    univocamente determinata dalla relazione:
    \[
        p(x_1, \ldots, x_n) = \prod_{i \in [n]} p_i(x_i).
    \]
    Infatti in tal caso varrebbe che:
    \[
        P(X_1 = x_1, \ldots, X_n = x_n) =
        p(x_1, \ldots, x_n) = \prod_{i \in [n]} P(X_i = x_i).
    \]
    Tale costruzione si indica come $P \defeq \bigotimes_{i \in [n]} P_i =
    P_1 \otimes \cdots \otimes P_n$.
\end{remark}

\section{Valore atteso e momenti}

\subsection{Valore atteso su v.a.~integrabili e/o non negative}

\begin{definition}[Variabile aleatoria integrabile]
    Sia $X$ v.a.~reale. Si dice che $X$ è \textbf{integrabile} (in senso discreto)
    se:
    \[
        \EE[\abs{X}] \defeq \sum_{\omega \in \Omega} \abs{X(\omega)} p(\omega) < \infty,
    \]
    ovverosia se $\EE[\abs{X}]$, detto il \textbf{momento primo assoluto},
    converge (l'unica altra possibilità è che diverga, dacché
    è una serie a termini positivi).    
\end{definition}

\begin{definition}[Valore atteso di una v.a.]
    Sia $X$ v.a.~reale. Se $X$ è integrabile si definisce
    il \textbf{valore atteso} di $X$ (o \textit{momento primo}) come:
    \[
        \EE[X] \defeq \sum_{\omega \in \Omega} X(\omega) p(\omega) \in \RR,
    \]
    dove l'ultima appartenenza è data proprio dal fatto che $\EE[\abs{X}] < \infty$ (e
    dunque vi è convergenza assoluta, dacché $p(\omega) \geq 0$). \smallskip

    Se $X \geq 0$ q.c.~, si definisce allora stesso modo $\EE[X]$, che però può assumere come
    valore anche $\infty$; e così per $X \leq 0$ q.c.~si pone
    $\EE[X] \defeq -\EE[X^-]$. In questo modo ammettiamo eventualmente i valori
    di $\infty$ o $-\infty$. \smallskip

    Diciamo che $X$ \textbf{ha valore atteso}, se esiste un $\EE[X]$ associatogli.
\end{definition}

\begin{remark}
    Il valore atteso è da associarsi a un ``baricentro'' della distribuzione di
    $X$, ovverosia, su una popolazione $\Omega$, misura quanto vale in media
    la caratteristica data da $X$.
\end{remark}

\begin{remark}
    Per la v.a.~$1_A$ con $A \subseteq \Omega$ vale che
    $\EE[1_A] = 1 \cdot P(1_A = 1) + 0 \cdot P (1_A = 0) = P(A)$.
\end{remark}

\begin{remark}
    Per $X$ tale per cui $\EE[X^+]$, $\EE[X^-] < \infty$ vale che:
    \[
        \EE[X] = \EE[X^+] - \EE[X^-].
    \]
    Come vedremo, questo è un caso particolare della linearità di $\EE[\cdot]$
    (infatti $X = X^+ - X^-$).
\end{remark}

\begin{lemma}[Valore atteso tramite la legge]
    Per $X$ con valore atteso vale la seguente identità:
    \[
        \EE[X] = \sum_{x \in R_X} x \cdot p_X(x) = \sum_{x \in R_X} x \cdot P(X = x).
    \]
    Segue dal fatto che $\EE[X] = \sum_{x \in R_X} \sum_{s \in X\inv(x)} x \cdot p(s)$.
\end{lemma}

Questa proposizione può estendersi facilmente alla:

\begin{proposition}[Valore atteso della composizione tramite la legge]
    Sia $X : \Omega \to S$ v.a.~discreta e sia $\varphi : S \to \RR$. Allora vale che:
    \begin{enumerate}[(i.)]
        \item $\varphi(X)$ è integrabile se e solo se $\sum_{x \in R_X} \abs{\varphi(x)} P(X = x) < \infty$,
        \item se $\varphi(X)$ ha valore atteso, allora:
        \[
            \EE[\varphi(X)] = \sum_{x \in R_X} \varphi(x) \cdot p_X(x) = \sum_{x \in R_X} \varphi(x) \cdot P(X = x).
        \]
    \end{enumerate}
    Segue dal fatto che $\EE[\varphi(X)] = \sum_{x \in R_X} \sum_{s \in X\inv(x)} \varphi(x) \cdot p(s)$.
\end{proposition}

\begin{remark}[Uguaglianza di valori attesi per leggi uguali]
    Dal momento che $\EE[\varphi(X)]$ dipende soltanto dalla legge di $p_X$,
    $X \deq Y \implies \EE[\varphi(X)] = \EE[\varphi(Y)]$.
\end{remark}

\subsection{Proprietà del valore atteso e moltiplicatività per v.a.~indipendenti}

\begin{proposition}
    \label{prop:prop_valore_atteso}
    Siano $X$ e $Y$ due v.a.~reali con valore atteso. Allora vale che:
    \begin{enumerate}[(i.)]
        \item Se $X=c$ q.c., allora $\EE[X] = c$,
        \item Se $X \geq 0$ q.c./integrabile, allora per $a \in \RR^+$, $aX \geq 0$ q.c./integrabile,
        \item Se $X$ ha valore atteso, allora per $a \in \RR$ pure $aX$ lo ha e $\EE[aX] = a \, \EE[X]$\footnote{
            Si assume la convenzione per cui $0 \cdot \infty = 0$, $a \cdot \infty = \sgn(a) \infty$ per
            $a \neq 0$.
        }
        \item Se $X \geq 0$ q.c.~o $X \leq 0$ q.c.~e $\EE[X] = 0$, allora $X = 0$ q.c.,
        \item Se $X \leq Y$ q.c.~, allora $E[X] \leq E[Y]$,
        \item Se $X$ e $Y$ hanno valore atteso e non sono uno $\infty$ e l'altro
        $-\infty$, allora $\EE[X + Y] = \EE[X] + \EE[Y]$.
    \end{enumerate}
\end{proposition}

\begin{proposition}
    Siano $X$, $Y : \Omega \groupto S$, $S'$, due v.a.~indipendenti. Se $g$, $h : S$, $S' \groupto \RR$ sono funzioni e $g(X)$ e $h(Y)$ ammettono valore atteso\footnote{
        Si ammette in questo caso la convenzione per cui $\infty \cdot \infty = \infty$ e
        che $-\infty \cdot \infty = -\infty$.
    }, allora vale che:
    \[
        \EE[g(X)h(Y)] = \EE[g(X)] \cdot \EE[h(Y)].
    \]
    Usando che $\EE[g(X)h(Y)] = \sum_{(x, y) \in R_{(X, Y)}} g(x) h(y) P(X = x, Y = y)$, segue, per
    l'indipendenza di $X$ e $Y$, dal fatto che $R_{(X, Y)} = R_X \times R_Y$ e che $P(X = x, Y = y) = P(X = x) P(Y = y)$.
\end{proposition}

\begin{remark}
    \label{remark:indipendenza_valore_atteso}
    In particolare, per v.a.~reali $X$, $Y$ indipendenti che ammettono valore atteso
    vale che:
    \[
        \EE[XY] = \EE[X] \cdot \EE[Y].
    \]
\end{remark}

\begin{remark}
    Dalla \textit{Proposizione \ref{prop:prop_valore_atteso}} si deduce che
    $\EE[\cdot]$ è un funzionale di $\VA(\Omega, \RR)$ (ovverosia
    $\EE[\cdot] \in \VA(\Omega, \RR)^*$).
\end{remark}

\begin{proposition}
    Sia $X$ una v.a.~reale che assume valori naturali quasi certamente.
    Allora vale che:
    \[
        \EE[X] = \sum_{n \in \NN} P(X > n).
    \]
    In generale se $X$ è una v.a.~reale che assume valori positivi il cui
    range ordinato è $(x_i)_{i \in I}$ (con $I = \NN^+$ o $I = [k]$),
    allora, posto $x_0 = 0$, vale che:
    \[
        \EE[X] = \sum_{n \in \NN} (x_{n+1} - x_n) P(X > x_n).
    \]
\end{proposition}

\subsection{Valore atteso condizionale}

\begin{definition}[Valore atteso condizionale]
    Sia $X$ una v.a.~reale. Dato allora un evento
    $A \in \PP(\Omega)$, si definisce il \textbf{valore atteso
    condizionale} $\EE[X \mid A]$ in modo tale che:
    \[
        \EE[X \mid A] \defeq \frac{\EE[X \cdot 1_A]}{P(A)} = \sum_{\omega \in A} X(\omega) \cdot P(\{\omega\} \mid A).
    \]
    Alternativamente vale che:
    \[
        \EE[X \mid A] = \sum_{x \in R_X} x \cdot \frac{P((X = x) \cap A)}{P(A)} = \sum_{x \in R_X} x \cdot P(X=x \mid A).
    \]
\end{definition}

Il valore atteso condizionale rimodula il valore atteso in modo
tale da considerare solamente le immagini di $X$ possibili sotto
l'ipotesi che sia accaduto l'evento $A$. Pertanto è naturale
aspettarsi il seguente:

\begin{lemma}[Formula dei valori attesi totali, o formula della partizione dei valori attesi]
    Sia $X$ una v.a.~reale e sia $(A_i)_{i \in [n]}$ un sistema di alternative
    finito per $\Omega$. Allora vale che:
    \[
        \EE[X] = \sum_{i \in [n]} \EE[X \mid A_i] P(A_i).
    \]
    Segue considerando che $X = X \cdot (\sum_{i \in [n]} 1_{A_i})$.
\end{lemma}

\subsection{Momenti (assoluti) \texorpdfstring{$n$}{n}-esimi}

\begin{definition}[Momento $n$-esimo assoluto]
    Data $X$ v.a.~reale e $n \in \RR^+$, definiamo il
    \textbf{momento assoluto di ordine $n$} (\textit{momento
    $n$-esimo assoluto}, se esiste, $\EE[\abs{X}^n]$. \smallskip

    Generalmente si pone più attenzione ai momenti $n$-esimi assoluti
    con $n$ intero positivo.
\end{definition}

\begin{definition}[Momento $n$-esimo]
    Data $X$ v.a.~reale e $n \in \RR^+$, se $X$ ammette
    momento $n$-esimo assoluto, allora $X^n$ ammette
    $\EE[X^n]$, che viene detto \textbf{momento $n$-esimo di $X^n$}.
\end{definition}

\begin{lemma}
    Data $X$ v.a.~reale e $1 \leq p \leq q$ in $\RR$,
    se $\EE[\abs{X}^q] < \infty$ allora
    $\EE[\abs{X}^p] < \infty$. \smallskip

    Segue dal fatto che $\EE[\abs{X}^p]$ è uguale
    a $\EE[\abs{X}^p \cdot 1_{{\abs{X}> 1}} + \abs{X}^p \cdot 1_{{\abs{X} \leq 1}}]$;
    applicando la linearità di $\EE[\cdot]$ e che $x^p \leq x^q$ per $x \geq 1$, si
    ricava così che $\EE[\abs{X}^p] \leq \EE[\abs{X}^q] + 1$.
\end{lemma}

\begin{remark}
    Se $X$ è limitata quasi certamente ($\abs{X} \leq M$ q.c.~con $M > 0$),
    allora $X$ ammette momento $n$-esimo assoluto per ogni $n \in \RR^+$
    (segue dal fatto che $\EE[\abs{X}^n] \leq M^m$).
\end{remark}

\begin{remark}
    La disuguaglianza impiegata nello scorso lemma ha una generalizzazione
    più ampia, che non dimostriamo, ma che segue dalla \textit{Disuguaglianza di Hölder}:
    \[
        \EE[\abs{X}^p]^{\frac{1}{p}} \leq \EE[\abs{X}^q]^{\frac{1}{q}}, \quad 1 < p < q.
    \]
\end{remark}

\begin{lemma}
    Se $\EE[\abs{X}^p]$, $\EE[\abs{X}^p] < \infty$, allora
    $\EE[\abs{aX+Y}^p] < \infty$ per ogni $a$, $b \in \RR$. \smallskip

    Segue dal fatto che $\abs{aX+Y}^p \leq 2^{p-1} (\abs{a}^p \abs{X}^p + \abs{Y}^p)$.
\end{lemma}

\subsection{Disuguaglianza di Markov, di Hölder, di Cauchy-Schwarz e di Jensen}

\begin{proposition}[Disuguaglianza di Markov]
    Sia $X \geq 0$ v.a.~reale. Allora $\forall a > 0$ vale che:
    \[
        P(X \geq a) \leq \frac{\EE[X]}{a}.
    \]
    Segue considerando che $X \geq a \cdot 1_{X \geq a}$,
    e dunque $\EE[X] \geq a \cdot \EE[1_{X \geq a}] = a \cdot P(X \geq a)$.
\end{proposition}

\begin{corollary}
    Sia $X$ v.a.~reale. Allora $\forall a \neq 0$, $\forall p > 0$ vale che:
    \[
        P(\abs{X} \geq \abs{a}) \leq \frac{\EE[\abs{X}^p]}{\abs{a}^p}.
    \]
    Segue dalla disuguaglianza di Markov.
\end{corollary}

In generale la disuguaglianza di Markov si può esprimere per composizione
con funzioni crescenti:

\begin{corollary}
    Sia $X$ v.a.~reale. Allora, se $f : \RR \to [0, \infty)$ è crescente, $\forall a \in \supp f$ (i.e.~$f(a) \neq 0$) vale che:
    \[
        P(X \geq a) \leq \frac{\EE[f(X)]}{f(a)}.
    \]
    Segue dalla disuguaglianza di Markov. Si osserva in particolare che non si è richiesto
    che $X$ fosse t.c.~$X \geq 0$.
\end{corollary}

\begin{proposition}[Disuguaglianza di Hölder]
    Siano $X$, $Y$ v.a.~reali. Siano $p$, $q > 1$ coniugati (ossia t.c.~$\frac{1}{p} + \frac{1}{q} = 1$). Allora, se $X$ ammette momento $p$-esimo assoluto e $Y$ ammette momento
    $q$-esimo assoluto, entrambi finiti, vale che:
    \[
        \EE[\abs{XY}] \leq \EE[\abs{X}^p]^{\frac{1}{p}} \cdot \EE[\abs{Y}^q]^{\frac{1}{q}}.
    \]
    Segue dalla usuale disuguaglianza di Hölder in analisi.
\end{proposition}

\begin{proposition}[Disuguaglianza di Cauchy-Schwarz]
    Siano $X$, $Y$ v.a.~reali. Allora, se $X$ e $Y$ ammettono momento secondo assoluto
    finito, vale che:
    \[
        \EE[\abs{XY}] \leq \EE[\abs{X}^2]^{\frac{1}{2}} \cdot \EE[\abs{Y}^2]^{\frac{1}{2}}.
    \]
    Segue dalla usuale disuguaglianza di Cauchy-Schwarz in analisi o dalla disuguaglianza
    di Hölder per $p = q = \frac{1}{2}$.
\end{proposition}

\begin{proposition}[Disuguaglianza di Jensen]
    Sia $X$ una v.a.~reale che ammette valore atteso.
    Allora, se $g : \RR \to \RR$ è una funzione
    convessa che ammette valore atteso vale che:
    \[
        g(\EE[X]) \leq \EE[g(X)].
    \]
    Equivalentemente, se $g$ è concava vale la disuguaglianza con
    $\geq$ al posto di $\leq$. Segue dall'usuale disuguaglianza di Jensen.
\end{proposition}

\section{Altri indici di centralità: moda e mediana}

Il valore atteso $\EE[X]$ è considerato un \textbf{indice di centralità} dacché
fornisce un'idea del baricentro della distribuzione di $X$. Di seguito
sono definiti altri due indici di centralità celebri.

\begin{definition}[Moda]
    Data una v.a.~reale $X$, si dice che $x \in S_X$ è una \textbf{moda}
    se $x$ è un massimo per $P_X$. Una distribuzione in generale può avere
    più mode.
\end{definition}

\begin{definition}[Mediana]
    Data una v.a.~reale $X$, si dice che $x \in S_X$ è una \textbf{mediana}
    se $P(X \leq x) \geq \frac{1}{2}$ e $P(X \geq x) \geq \frac{1}{2}$.
\end{definition}

\begin{proposition}
    Esistono sempre almeno una moda e almeno una mediana
    per $X$ v.a.~reale.
\end{proposition}

\section{Indici di dispersione: covarianza, varianza, dev.~standard e coeff.~di correlazione}

\subsection{Definizioni e covarianza come forma bilineare simmetrica}

\begin{definition}[Covarianza e v.a.~scorrelate]
    Date due v.a.~reali $X$, $Y$ con momento secondo finito,
    si definisce \textbf{covarianza di $X$ e $Y$} il termine:
    \[
        \Cov(X, Y) \defeq \EE[(X - \EE[X])(Y - \EE[Y])].
    \]
    Si dice che $X$ e $Y$ sono \textbf{scorrelate} se $\Cov(X, Y) = 0$.
\end{definition}

\begin{definition}[Varianza]
    Data una v.a.~reale $X$ con momento secondo finito, si
    definisce \textbf{varianza di $X$} il termine:
    \[
        \Var(X) \defeq \Cov(X, X) = \EE[(X -\EE[X])^2] \geq 0,
    \]
    dove la non negatività segue dal fatto che $(X - \EE[X])^2 \geq 0$.
\end{definition}

\begin{proposition}
    $\EE[X]$ è il termine che sostituito a $m$ minimizza il valore $\EE[(X - m)^2]$.
\end{proposition}

\begin{definition}[Deviazione standard]
    Data una v.a.~reale $X$ che ammette varianza, si definisce
    \textbf{deviazione standard di $X$} il termine:
    \[
        \sigma(X) \defeq \sqrt{\Var(X)}.
    \]
\end{definition}

\begin{remark}
    La deviazione standard misura quanto $X$ si discosta mediamente da
    $\EE[X]$, se esiste.
\end{remark}

\begin{remark}
    La varianza e la deviazione standard sono
    detti \textbf{indici di dispersione} della distribuzione
    di $X$, dacché misurano
    quanto le immagini di $X$ distano mediamente dal valore
    atteso $\EE[X]$.
\end{remark}

\begin{proposition}
    \label{prop:cono_isotropo}
    Sia $X$ una v.a.~reale che ammette varianza. Allora
    $\Var(X) = 0$ se e solo se $X$ è costante q.c. \smallskip


    Segue dal fatto che $\EE[(X -\EE[X])^2] = 0$ se e solo se
    $\EE[X] = X$ q.c., ovverosia se e solo se $X$ è una costante.
\end{proposition}

\subsection{Identità sulla (co)varianza e disuguaglianza di Chebyshev}

\begin{proposition}
    \label{prop:indipendenza_cov}
    $\Cov(\cdot, \cdot)$ è una funzione simmetrica e
    lineare in ogni suo argomento. In particolare per
    $X$ e $Y$ con momento secondo finito vale che:
    \[
        \Cov(X, Y) = \EE[XY] - \EE[X] \EE[Y].
    \]
    Pertanto due v.a.~indipendenti hanno covarianza nulla (i.e.~sono scorrelate)
    per l'\textit{Osservazione \ref{remark:indipendenza_valore_atteso}}.
    In particolare, la covarianza tra una qualsiasi costante q.c.~e
    un'altra v.a.~reale è nulla.
\end{proposition}

\begin{remark}
    La precedente proposizione mette ancora in luce come sia determinante la
    legge congiunta $p_{(X, Y)}$, usata per calcolare $\EE[XY]$, che
    in generale le leggi $p_X$ e $p_Y$, che pure si usano per calcolare
    $\EE[X]$ e $\EE[Y]$, non riescono a ricostruire.
\end{remark}

\begin{remark}
    A partire dalla precedente proposizione si ricava che per $X$ v.a.~reale
    con momento secondo finito vale che:
    \[
        \Var(X) = \EE[X^2] - \EE[X]^2.
    \]
\end{remark}

\begin{remark}
    Viste le proprietà discusse nella precedente proposizione
    si può concludere che la covarianza sul sottospazio di $\VA(\Omega, \RR)$
    delle v.a.~con momento secondo finito
    corrisponde a una forma bilineare simmetrica semidefinita positivo,
    ovverosia a un prodotto scalare. \smallskip


    Due v.a.~indipendenti sono ortogonali tramite $\Cov$ per la
    \textit{Proposizione \ref{prop:indipendenza_cov}}. \smallskip

    Al cono isotropo e al radicale di questo prodotto appartengono solo le costanti per la
    \textit{Proposizione \ref{prop:cono_isotropo}}. \smallskip


    Se $\varphi \defeq \Cov$, vale che $q_\varphi \equiv \Var$ e $\norm{\cdot}_\varphi \equiv \sigma$,
    ovverosia la varianza $\Var$ è la forma quadratica associata alla covarianza $\Cov$,
    mentre $\sigma$ ne è la norma.
\end{remark}

\begin{lemma}
    Siano $X_1$, ..., $X_n$ v.a.~reali con momento secondo finito. Allora vale che:
    \[
        \Var(X_1 + \ldots + X_n) = \sum_{i \in [n]} \Var(X_i) + 2 \sum_{1 \leq i < j \leq n} \Cov(X_i, X_j).
    \]
    In particolare, se $(X_i)_{i \in [n]}$ è una famiglia di v.a.~scorrelate a due a due (e.g.~indipendenti) vale che:
    \[
        \Var(X_1 + \ldots + X_n) = \sum_{i \in [n]} \Var(X_i).
    \]
\end{lemma}

\begin{lemma}
    Sia $aX + b$ una v.a.~reale con $X$ che ammette momento secondo finito. Allora
    vale che:
    \[
        \Var(aX + b) = a^2 \Var(X).
    \]
    Segue dal fatto che $aX$ e $b$ sono indipendenti, che $\Var(b) = 0$ e che
    $\Var$ è la forma quadratica di $\Cov$.
\end{lemma}

\begin{proposition}[Disuguaglianza di Chebyshev]
    Sia $X$ v.a.~reale con momento secondo finito. Allora $\forall a > 0$ vale
    che:
    \[
        P(\abs{X - \EE[X]} > a) \leq \frac{\Var(X)}{a^2}.
    \]
    Segue dall'immediata applicazione della disuguaglianza di Markov.
\end{proposition}

\subsection{Coeff.~di correlazione e retta di regressione lineare}

\begin{definition}[Coefficiente di correlazione di Pearson, PCC]
    Date $X$, $Y$ v.a.~reali non costanti q.c.\footnote{
        Infatti il coseno è definito solo per coppie di vettori anisotropi
        ed il cono isotropo di $\Cov$ è costituito dalle sole costanti q.c.
    }~e con momento secondo finito si definisce il \textbf{coefficiente di correlazione
    di Pearson} (PCC) $\rho(X, Y)$, o più brevemente \textit{coefficiente di correlazione}, 
     come il coseno di $X$ e $Y$ rispetto a $\Cov$, ovverosia:
    \[
        \rho(X, Y) \defeq \cos_{\Cov}(X, Y) = \frac{\Cov(X, Y)}{\sigma(X) \cdot \sigma(Y)}.
    \]
\end{definition}

\begin{lemma}
    Date $X$, $Y$ v.a.~reali non costanti q.c.~e con momento secondo finito vale che:
    \begin{enumerate}[(i.)]
        \item $\abs{\rho(X, Y)} \leq 1$ (per la disuguaglianza di Cauchy-Schwarz),
        \item $\rho(aX + b, cX + d) = \rho(X, Y)$ (per verifica diretta).
    \end{enumerate}
\end{lemma}

\begin{theorem}
    Siano $X$, $Y$ v.a.~reali con momento secondo finito e non costanti q.c. Allora
    la funzione:
    \[
        \RR^2 \ni (a, b) \mapsto \EE[(Y - (aX + b))^2] \in \RR
    \]
    è ben definita e ammette un unico punto di minimo $(a^*, b^*)$, dove:
    \[
        a^* = C_{\Cov}(X, Y) = \frac{\Cov(X, Y)}{\Var(X)}, \quad b^* = \EE[Y] - a^* \EE[X].
    \]
    Inoltre il valore di tale minimo è:
    \[
        \EE[(Y - (a^* X + b^*))^2] = \Var(Y) \cdot (1 - \rho(X, Y)^2).
    \]
\end{theorem}

\begin{definition}[Retta di regressione (lineare)]
    Date $X$, $Y$ v.a.~reali con momento secondo finito e non costanti q.c.
    si definisce \textbf{retta di regressione} (lineare) la retta $y = a^*x + b^*$.
\end{definition}

\begin{remark}
    Dal precedente teorema si può ottenere una caratterizzazione della
    correlazione lineare tra due v.a.~reali $X$ e $Y$ non costanti q.c.~e con
    momento secondo finito. Infatti vale che:
    \begin{itemize}
        \item la retta di regressione di $X$ e $Y$ rappresenta la migliore approssimazione
        lineare di $Y$ tramite $X$,
        
        \item $\rho(X, Y) \approx 0$ ($X$, $Y$ quasi scorrelate) $\implies$ poca correlazione lineare ($\EE[(Y - (a^* X + b^*))^2]$ assume approsimativamente il valore massimo possibile e dunque $Y$
        dista mediamente tanto da ogni retta di $X$),
        \item $\rho(X, Y) \approx 1 \implies$ forte correlazione lineare (infatti se
        $\rho = 1$, $\EE[(Y - (a^* X + b^*))^2] = 0$, e dunque $Y = a^* X + b^*$ q.c.).
    \end{itemize}
    Si osserva inoltre che $\sgn(a^*) = \sgn(\rho(X, Y))$.
\end{remark}

\section{Legge dei grandi numeri (LGN), media campionaria e limite in senso probabilistico}

\subsection{Definizioni ed enunciato}

\begin{definition}[Media campionaria $n$-esima]
    Data una famiglia di v.a.~reali $(X_i)_{i \in \NN}$ i.i.d.~dotate di momento secondo
    finito\footnote{
        Dal momento che le $X_i$ sono i.i.d.~è sufficiente che $X_1$ sia dotata di
        momento secondo finito.
    } si definisce \textbf{media campionaria $n$-esima} il termine:
    \[
        \overline{X_n} \defeq \frac{1}{n} \sum_{i \in [n]} X_i, 
    \]
    ovverosia la media aritmetica delle prime $n$ v.a.~della famiglia.
\end{definition}

\begin{definition}[Limite probabilistico]
    Data una successione di v.a.~reali $(Y_i : \Omega \to \RR)_{i \in \NN}$ e data
    una v.a.~reale $Y : \Omega \to \RR$ si
    dice che $Y_n$ tende (probabilisticamente) a $Y$ ($Y_n \toprob Y$) per $n \to \infty$
    se:
    \[
        \lim_{n \to \infty} P(\abs{Y_n - Y} > \eps) = 0, \quad \forall \eps > 0.
    \]
\end{definition}

\begin{remark}
    Una successione di v.a.~reali $(Y_i)_{i \in \NN}$ tende a $Y$ se si può
    sempre scegliere un $n$ arbitrariamente grande tale per cui la probabilità che $Y_i$
    sia pari a $Y$ (eccetto per un errore assoluto $\eps$ fissato) è certa entro un
    errore arbitrario.
\end{remark}

\begin{theorem}[Legge (debole) dei grandi numeri, LGN]
    Sia $(X_i)_{i \in \NN}$ una famiglia di v.a.~reali scorrelate e i.d.~(e.g.~i.i.d.) dotate di momento secondo
    finito, ovverosia con $\EE[X_1^2] < \infty$. Allora vale che:
    \[
        \overline{X_n} \toprob \EE[X_1], \quad \text{per } n \to \infty.
    \]
\end{theorem}

\begin{proof}
    Si osserva che $\EE[\overline{X_n}] = \EE[X_1]$ e che
    $\Var(\overline{X_n}) = \frac{1}{n} \Var(X_1)$. Allora, se $\eps > 0$,
    per la disuguaglianza di Chebyshev vale che:
    \[
        P\left(\abs{\overline{X_n} - \EE[X_1]} > \eps\right) \leq \frac{\Var(\overline{X_n})}{\eps^2} =
        \frac{\Var(X_1)}{\eps^2 n}.
    \]
    Dal momento che $\frac{\Var(X_1)}{\eps^2 n} \to 0$ per $n \to \infty$, si ottiene
    la tesi.
\end{proof}

\begin{remark}
    In alcune occasioni, ovverosia quando $\Var(\overline{X_n}) \to 0$
    per $n \to \infty$, è ancora possibile applicare la LGN seguendo la stessa
    dimostrazione.
\end{remark}

\begin{remark}
    La legge dei grandi numeri ci permette di ricondurre la definizione
    assiomatica di Kolmogorov di probabilità a quella frequentista. Se
    infatti fissiamo una probabilità $P$ e costruiamo un modello di prove
    ripetute (come definito successivamente) il cui successo è dipeso
    da se accade l'evento $A$, considerando come famiglia di
    v.a.~i.i.d.~la famiglia $(1_{A_i})_{i \in \NN}$, dove $A_i$ è l'evento di successo di $A$ nella prova
    $i$-esima, per la legge dei grandi numeri si ottiene che per $n \to \infty$ vale che:
    \[
        \overline{1_{A_n}} = \frac{\text{numero di volte che accade $A$}}{\text{numero di prove}} \toprob \EE[1_{A_1}] = P(A).
    \]
\end{remark}

\subsection{Trasformata di Cramer per l'ottimizzazione della stima}

Cerchiamo in questa sezione di ottenere, utilizzando la funzione
esponenziale, una stima ottimale per
$P(\overline{X_n} - m > \eps)$ con $\eps > 0$, $(X_i)_{i \in \NN}$ famiglia
di v.a.~i.i.d.~e $m = \EE[X_1]$ finito. \smallskip

Dacché $\exp : \RR \to (0, \infty)$ è crescente, vale che, per $\lambda > 0$:
\begin{multline*}
    P(\overline{X_n} - m > \eps) = P\left(\lambda \sum_{i \in [n]} (X_i - m) > \lambda n \eps\right) = \\ = P\left(\exp\left(\lambda \sum_{i \in [n]} (X_i - m)\right) > \exp(\lambda n \eps)\right).
\end{multline*}

Applicando la disuguaglianza di Markov si ottiene che:
\begin{multline*}
    P(\overline{X_n} - m > \eps) \leq \frac{1}{e^{\lambda n \eps}} \EE\left[\exp\left(\lambda \sum_{i \in [n]} (X_i - m)\right)\right] = \\
    = \frac{1}{e^{\lambda n \eps}} \EE[\exp(\lambda(X_1 - m))]^n = \\
    = \exp\left(-n\left(\lambda \eps - \log \, \EE\left[e^{\lambda(X_1-m)}\right]\right)\right).
\end{multline*}
dove si è utilizzato che le v.a.~sono indipendenti e identicamente distribuite.

\begin{definition}[Trasformata di Cramer]
    Dato $\eps > 0$, $(X_i)_{i \in \NN}$ famiglia
di v.a.~i.i.d.~e $m = \EE[X_1]$ finito, si definisce \textbf{trasformata di Cramer}
    il valore:
    \[
        I(t) = \sup_{\lambda > 0} \, \left(\lambda t - \log \, \EE\left[e^{\lambda(X_1-m)}\right]\right).
    \]
\end{definition}

Ottimizzando dunque in $\lambda$, la precedente disuguaglianza di scrive come:
\[
    P(\overline{X_n} - m > \eps) \leq e^{-n \cdot I(\eps)}.
\]
Se dunque esiste $\lambda > 0$ per cui $\EE\left[e^{\lambda(X_1-m)}\right]$ è finito, allora $I(\eps) > 0$, e dunque $P(\overline{X_n} - m > \eps)$ tende esponenzialmente a $0$
per $n \to \infty$.

\section{Teorema centrale del limite (TCL, o TLC)}

\subsection{Intuizione del TCL: \textit{zoom-in} e \textit{scaling}}
Per la legge dei grandi numeri sappiamo già che
$\overline{X_n} - m \toprob 0$ per $m = \EE[X_1]$, $n \to \infty$ e
$(X_i)_{i \in [n]}$ famiglia di v.a.~i.i.d. Ciò è dipeso, come illustrato dalla dimostrazione, dal fatto che è presente un fattore $\frac{1}{n}$ in $\Var(\overline{X_n})$.
\smallskip


Se $\alpha > 0$ e consieriamo lo \textit{scaling} (o \textit{zoom-in}) $n^\alpha (\overline{X_n} - m)$
vale che:
\[
    \Var(n^\alpha (\overline{X_n} - m)) = n^{2\alpha} \Var(\overline{X_n}) = n^{2\alpha - 1} \Var(X_1). 
\]
Pertanto, riapplicando la disuguaglianza di Chebyshev:
\[
    P\left(n^\alpha \abs{\overline{X_n} - m} > \eps\right) \leq \frac{1}{\eps^2} n^{2\alpha - 1} \Var(X_1).
\]
Per $\alpha < \frac{1}{2}$ si riottiene una tesi analoga a quella della LGN. È
lecito dunque aspettarsi che per $\alpha = \frac{1}{2}$ possa accadere qualcosa
di diverso, da cui l'intuizione del TCL.

\subsection{Enunciato del TCL e Teorema di De Moivre-Laplace per la distr.~binomiale}
\begin{theorem}[Teorema centrale del limite, TCL; oppure Teorema del limite centrale, TLC]
    Sia $(X_i)_{i \in \NN}$ una famiglia di v.a.~i.i.d dotate di momento secondo
    finito ($\EE[X_1^2] < \infty$) e non costanti q.c.~($\Var(X_1) > 0$). Sia
    $\sigma = \sigma(X_1)$ e sia $m = \EE[X_1]$. Allora per ogni scelta di $a$, $b$
    tali per cui $-\infty \leq a \leq b \leq \infty$\footnote{
        Si ammettono dunque anche i casi $\pm \infty$.
    } vale che per $n \to \infty$:
    \[
        P\left(a \leq \frac{\sqrt{n}}{\sigma} \left(\overline{X_n} - m\right) \leq b\right) \to \frac{1}{\sqrt{2\pi}}\int_a^b e^{-\frac{x^2}{2}} \dx. 
    \]
    Equivalentemente vale che:
    \[
        P\left(a \leq \frac{1}{\sqrt{n}\sigma} \left[\left(\sum_{i \in [n]} X_i\right) - nm\right] \leq b\right) \to \frac{1}{\sqrt{2\pi}}\int_a^b e^{-\frac{x^2}{2}} \dx. 
    \]
\end{theorem}

\begin{warn}
    Per il calcolo di $\frac{1}{\sqrt{2\pi}}\int_a^b e^{-\nicefrac{x^2}{2}} \dx$ mediante
    la funzione $\Phi(x)$ si rimanda
    alla \textit{Tabella \ref{tab:phi}} allegata nelle ultime pagine di queste schede riassuntive.
\end{warn}

\begin{corollary}[Teorema di De Moivre-Laplace]
    Sia $Y_n \sim B(n, \pp)$. Allora per ogni scelta di $a$, $b$ tali per cui
    $-\infty \leq a \leq b \leq \infty$ vale che per $n \to \infty$:
    \begin{multline*}
        P\left(n\pp + \sqrt{n \pp (1- \pp)} a  \leq Y_n \leq n\pp + \sqrt{n \pp (1 - \pp)} b\right) \\
        \to \frac{1}{\sqrt{2\pi}}\int_a^b e^{-\frac{x^2}{2}} \dx.
    \end{multline*}
\end{corollary}

\begin{proof}
    Segue dal TCL dal momento che $Y_n$ è somma di $n$ v.a.~$X_i$ i.i.d. con $X_i \sim B(\pp)$. In particolare $m = \EE[X_1] = \pp$ e $\sigma = \sigma(X_1) = \sqrt{\EE[X_1^2] - \EE[X_1]^2} = \sqrt{\pp (1-\pp)}$.
\end{proof}

\section{Modelli probabilistici classici}

\subsection{Probabilità uniforme}

\begin{definition}[Probabilità uniforme]
    Dato $\Omega$ finito, si definisce
    \textbf{probabilità uniforme} l'unica probabilità
    $P : \FF \to \RR$ la cui funzione di densità
    è costante (\textit{equiprobabile}). Equivalentemente è la probabilità
    $P$ tale per cui:
    \[
        P(A) = \frac{\#A}{\#\Omega}.
    \]
\end{definition}

\begin{remark}
    Non è possibile dotare $\Omega$ numerabile di una probabilità
    uniforme. Infatti, se l'unica immagine della funzione $p : \Omega \to \RR$ è
    $c$, $\sum_{\omega \in \Omega} p(\omega) = c \sum_{\omega \in \Omega} 1$, che
    può valere solo $0$ o $\infty$, e dunque non $1$ (e pertanto non può indurre
    una probabilità).
\end{remark}

\subsection{Sequenze di esperimenti e modello delle prove ripetute di Bernoulli}

    Cerchiamo di modellare una sequenza ordinata (e potenzialmente infinita,
    ma al più numerabile)
    di esperimenti. Data una famiglia $(\Omega_i)_{i \in I}$, con $I = \NN$ o
    $I = [n]$, dove ciascuno $\Omega_i$ indica l'$i$-esimo esperimento, definiamo
    in tal caso:
    \[ 
        \Omega = \left\{ (\omega_1, \omega_2, \ldots) \,\middle\vert\, \omega_1 \in \Omega_1, \omega_2 \in \Omega_2^{(\omega_1)}, \omega_3 \in \Omega_3^{(\omega_1, \omega_2)}, \ldots\right\},
    \]

    dove la notazione $\Omega_i^{(\omega_j)_{j \in [i-1]}}$ indica il sottoinsieme
    di $\Omega_i$ degli esiti dell'esperimento possibili una volta che nei precedenti
    esperimenti sono successi $\omega_1$, \ldots, $\omega_{i-1}$. Se i precedenti
    esperimenti non condizionano gli esiti dei successivi, allora
    $\Omega = \prod_{i \in I} \Omega_i$. \medskip


    Riduciamoci al caso di una sequenza (finita o infinita) di esperimenti tra di
    loro non condizionati, ciascuno
    con esito successo ($1$) o insuccesso ($0$). Un tale esperimento è
    detto \textbf{prova di Bernoulli}. In tal caso $\Omega = \prod_{i \in I} [[1]]$. \medskip
    
    
    Sia $A_i$ l'evento ``successo all''$i$-esima prova'', ossia:
    \[
        A_i = \{ \omega \in \Omega \mid \omega_i = 1 \}.
    \]

    Sia $p_i : [[1]] \to \RR$ la funzione di densità associata alla misura
    di probabilità dell'esperimento $\Omega_i$. Associamo allora ad $\Omega$ la $\sigma$-algebra $\FF = \sigma(A_i)_{i \in I}$ generata
    dagli $A_i$ (che è al più numerabile). Se $I$ è finito, $\FF = \PP(\Omega)$.

    \begin{definition}[Modello della sequenza di prove]
        Si definisce \textbf{probabilità del modello della sequenza di prove}
        l'unica probabilità $P$ sullo spazio misurabile $(\Omega, \FF)$ tale
        per cui $(A_i)_{i \in I}$ è una famiglia di eventi indipendenti e
        per la quale $P(A_i) = p_i(1)$.
    \end{definition}

    \begin{remark}
        Tale probabilità è univocamente determinata dal momento che
        gli $A_i$ generano $\FF$ e che sono indipendenti.
    \end{remark}

    \begin{definition}[Modello delle prove ripetute]
        Se $P$ è una probabilità del modello della sequenza di prove e
        $p_i(1) = p_j(1)$ per ogni coppia $i$, $j$, allora il modello
        prende il nome di \textbf{modello delle prove ripetute} e si dice
        che $\pbern \defeq p_1(1)$ è il \textbf{parametro di Bernoulli}.
    \end{definition}

    A partire dal modello delle prove ripetute si possono formalizzare
    numerose distribuzioni, come quelle della sezione delle
    \textit{\hyperref[tab:distr_discrete]{Distribuzioni discrete}}.
\end{multicols*}
%--------------------------------------------------------------------
\chapter{Probabilità sulla retta reale}
\setlength{\parindent}{2pt}

\begin{multicols*}{2}

Discutiamo in questa sezione la teoria della probabilità sulla
retta reale, uscendo dunque dal caso discreto. \smallskip

Per restringere la $\sigma$-algebra su cui lavoreremo (ossia l'insieme degli
eventi interessanti), siamo costretti a limitarci a una $\sigma$-algebra molto più
piccola di $\PP(\RR)$, la $\sigma$-algebra dei boreliani, che ci permette di escludere
``casi meno interessanti''. \smallskip

\begin{warn}
    Eccetto che nella prima sezione, assumeremo se non detto altrimenti
    di star lavorando sullo spazio misurabile
    $(\RR, \BB(\RR))$ dotato eventualmente della misura di Lebesgue $m$. $\BB(\RR)$ ed
    $m$ sono definiti nella sezione seguente.
\end{warn} 

\section{Cenni di teoria della misura}

\subsection{La \texorpdfstring{$\sigma$}{σ}-algebra di Borel e funzioni boreliane}

\begin{definition}[$\sigma$-algebra dei boreliani]
    Dato uno spazio metrico separabile\footnote{
        Si può generalizzare in modo naturale tale definizione a un qualsiasi spazio topologico.
        Dal momento che considereremo solo spazi metrici separabili (in particolare $X \subseteq \RR^d$), concentreremo
        le proprietà e le definizioni su questa classe di spazi topologici.} $X \neq \emptyset$
    si definisce la \textbf{$\sigma$-algebra $\BB(X)$ dei boreliani di $X$} (o
    $\sigma$-algebra di Borel)
    come la $\sigma$-algebra generata dai suoi aperti, ovverosia:
    \[
        \BB(X) \defeq \sigma \{ A \subseteq X \mid A \text{ aperto}\, \}.
    \]
    Gli elementi della $\sigma$-algebra di Borel sono detti \textit{boreliani}.
\end{definition}

\begin{proposition}[Proprietà di $\BB(X)$]
    Sia $X \neq \emptyset$ uno spazio metrico separabile. Allora valgono
    le seguenti affermazioni:

    \begin{enumerate}[(i.)]
        \item $\BB(X)$ contiene tutti gli aperti e i chiusi di $X$ (infatti
        metrico e separabile implica II-numerabile), pertanto se $\tau(X)$ è la
        topologia di $X$ vale che $\tau(X) \subseteq \BB(X)$,
        \item $\BB(X) = \sigma \{ A \subseteq X \mid A \text{ chiuso}\, \}$, ossia
        $\BB(X)$ è generata anche dai chiusi di $X$ (infatti $\BB(X)$ è chiuso per
        complementare),
        \item se $Y \subseteq X$, $Y \neq \emptyset$ ha metrica indotta da $X$, allora
        $\BB(Y) = \sigma \{ Y \cap B \mid B \in \BB(X) \} \subseteq \BB(X)$ (segue dal fatto che
        gli aperti di $Y$ sono tutti e solo gli aperti di $X$ intersecati a $Y$).
    \end{enumerate}
\end{proposition}

\begin{proposition}[Proprietà di $\BB(\RR^d)$]
    Valgono le seguenti affermazioni:
    \begin{enumerate}[(i.)]
        \item $\BB(\RR)$ contiene tutti gli intervalli e tutte le semirette (infatti si ammettono anche intersezioni infinite di aperti),
        \item $\BB(\RR)$ è generato dagli intervalli semiaperti, ovverosia $\BB(\RR) = \sigma \{ (a, b] \mid a, b \in \RR, a < b \}$,
        \item $\BB(\RR)$ è generato dalle semirette, ovverosia $\BB(\RR) = \sigma \{ (-\infty, a) \mid a \in \RR \}$,
        \item $\BB(\RR^d) = \sigma \{ (-\infty, a_1) \times \ldots \times (-\infty, a_n) \mid a_1, \ldots, a_n \in \RR \}$ (segue da (iii.)),
        \item $\BB(\RR^d) \neq \PP(\RR^d)$ (segue dal controesempio di Vitali, oltre che da considerazioni sulle cardinalità).
    \end{enumerate}
\end{proposition}

\begin{definition}
    Data una funzione $f : X \to Y$ con $X$ e $Y$ spazi metrici separabili, si dice che
    $f$ è una \textbf{funzione boreliana} se $f\inv(A)$ è boreliano per ogni
    $A$ boreliano di $Y$. Equivalentemente $f$ è boreliana se la controimmagine di ogni
    boreliano è un boreliano.
\end{definition}

\begin{proposition}
    Sia $f : X \to Y$ con $X$ e $Y$ spazi metrici separabili una funzione continua. Allora
    $f$ è boreliana. \smallskip


    Segue dal fatto che $\BB(Y)$ è generato dagli aperti di $Y$, le cui controimmagini sono
    aperte, e dunque boreliane.
\end{proposition}

\subsection{Definizione e proprietà di misura, \texorpdfstring{$\pi$}{π}-sistemi per \texorpdfstring{$\sigma$}{σ}-algebre}

\begin{definition}[Misura]
    Dato $(\Omega, \FF)$ spazio misurabile, una misura $\mu$ su $(\Omega, \FF)$ è una
    funzione $\mu : \FF \to [0, \infty]$ con $\mu(\emptyset) = 0$ e per cui valga
    la $\sigma$-additività, ovverosia:
    \[
        \mu\left(\bigcupdot_{i \in \NN} A_i\right) = \sum_{i \in \NN} \mu(A_i), \quad A_i \in \FF.
    \]
\end{definition}

\begin{remark}[Proprietà basilari di una misura]
    Dal momento che si richiede per una misura valga $\mu(\emptyset) = 0$, si verifica
    facilemente che vale la $\sigma$-additività finita. \smallskip


    Inoltre, se $A \subseteq B$, allora $\mu(B) = \mu(B \setminus A \cupdot A) = \mu(B \setminus A) + \mu(A)$, e
    dunque vale sempre che $\mu(A) \leq \mu(B)$. Vale inoltre ancora la $\sigma$-subadditività, con la stessa
    dimostrazione data per la probabilità, e dunque:
    \[
        \mu\left(\bigcup_{i \in \NN} A_i\right) \leq \sum_{i \in \NN} \mu(A_i).
    \]
\end{remark}

\begin{remark}[Comportamento di $\mu$ al limite]
    Se $(A_i)_{i \in \NN}$ è una famiglia numerabile di
    insiemi in $\FF$, allora, seguendo la stessa dimostrazione
    data per le misure di probabilità, che:

    \begin{enumerate}[(i.)]
        \item $A_i \goesup A \implies \mu(A_i) \goesup \mu(A)$,
        \item $A_i \goesdown A \implies \mu(A_i) \goesdown \mu(A)$.
    \end{enumerate}
\end{remark}

\begin{definition}
    Una misura $\mu$ su $(\Omega, \FF)$ si dice \textbf{misura finita} se $\mu(\Omega)$ è finito.
\end{definition}

\begin{proposition}[Proprietà di una misura finita $\mu$]
    Sia $\mu$ una misura finita su $(\Omega, \FF)$. Allora valgono le seguenti affermazioni:
    \begin{enumerate}[(i.)]
        \item $P(A) = \frac{\mu(A)}{\mu(\Omega)}$ è una misura di probabilità,
        \item $\mu(A)$ è sempre finito e $\mu(\Omega) = \mu(A) + \mu(A^c)$,
        \item $A \subseteq B \implies \mu(B) = \mu(B \setminus A) + \mu(A)$,
        \item $\mu(B \setminus A) = \mu(B) - \mu(A \cap B)$,
        \item $\mu(A \cup B) = \mu(A \Delta B \cupdot A \cap B) = \mu(A) + \mu(B) - \mu(A \cap B)$,
        \item $\mu\left(\bigcup_{i \in [n]} A_i\right) = \sum_{j \in [n]} (-1)^{j+1} \sum_{1 \leq i_1 < \cdots < i_j \leq n} \mu\left(\bigcap_{k \in [j]} A_{i_{k}}\right)$ (Principio di inclusione-esclusione per le misure finite).
    \end{enumerate}

    Tutte le affermazioni seguono immediatamente dalla prima.
\end{proposition}

\begin{definition}[Insiemi $\mu$-trascurabili e proprietà che accadono $\mu$-quasi ovunque]
    Un insieme $A \in \FF$ si dice \textbf{$\mu$-trascurabile} se
    $\mu(A) = 0$. Una proprietà $M$ si dice che accade
    $\mu$-quasi ovunque ($\mu$-q.o.) se esiste $A \in \FF$ $\mu$-trascurabile per cui
    $M$ accade per $A^c$.
\end{definition}

\begin{definition}[\texorpdfstring{$\pi$}{π}-sistema di una $\sigma$-algebra]
    Sia $(\Omega, \FF)$ uno spazio misurabile. Allora un sottoinsieme $\mathcal{C} \subseteq \FF$
    si dice \textbf{$\pi$-sistema di $\FF$} se:
    \begin{enumerate}[(i.)]
        \item $A$, $B \in \mathcal{C} \implies A \cap B \in \mathcal{C}$ (chiusura per intersezioni),
        \item $\sigma(C) = \FF$ (genera $\FF$).
    \end{enumerate}
\end{definition}

\begin{remark}
    Un $\pi$-sistema di una $\sigma$-algebra svolge lo ``stesso ruolo'' che una
    base svolge per una topologia.
\end{remark}

\begin{lemma}[di Dynkin, versione probabilistica]
    Sia $(\Omega, \FF)$ uno spazio misurabile e sia $\mathcal{C}$ un suo $\pi$-sistema. Siano
    $P$ e $Q$ due probabilità sullo spazio misurabile di $\Omega$. Se $P$ e $Q$ coincidono su
    $\mathcal{C}$, allora $P \equiv Q$.
\end{lemma}

\begin{example}
    Alcuni esempi di $\pi$-sistemi per $(\RR, \BB(\RR))$ sono:
    \begin{itemize}
        \item gli aperti, ovverosia $\mathcal{C} = \{ A \in \FF \mid A \text{ aperto}\, \}$ (oppure i chiusi),
        \item le semirette (a sinistra), ovverosia $\mathcal{C} = \{ (-\infty, a] \mid a \in \RR \}$ (oppure le semirette a destra),
        \item gli intervalli semiaperti (a sinistra), ovverosia $\mathcal{C} = \{ (a, b] \mid a, b \in \RR, b > a \}$ (oppure semiaperti a destra).
    \end{itemize}
\end{example}

\subsection{La misura di Lebesgue}

\begin{theorem}[Esistenza e unicità della misura di Lebesgue]
    Esiste ed è unica la misura $m$ su $(\RR, \BB(\RR))$ tale per cui
    $m([a, b]) = b-a$ per ogni $a$, $b \in \RR$ con $b > a$. Tale misura
    è detta \textbf{misura di Lebesgue} e corrisponde al concetto ``primitivo'' di
    \textit{lunghezza}. \smallskip


    L'unicità segue dall'enunciato generale del lemma di Dynkin.
\end{theorem}

\begin{remark}
    Dal momento che $m([0, 1]) = 1$,
    la misura $\restr{m}{[0,1]}$ è una misura di probabilità su $([0,1], \BB([0, 1]))$,
    detta \textit{probabilità uniforme su $[0,1]$}. Analogamente per $a$, $b \in \RR$
    con $b > a$, $m([a, b]) = b-a$ e
    dunque $P = \frac{1}{b-a} \restr{m}{[a,b]}$ è una misura di probabilità (detta
    \textit{probabilità uniforme su $[a,b]$}). \smallskip


    Assumendo l'assioma della scelta si può dimostrare che \underline{non} si può estendere in modo coerente
    $\restr{m}{[0,1]}$ a $([0, 1], \PP([0, 1]))$.
\end{remark}

\section{Probabilità reale, funzione di ripartizione (f.d.r.) e proprietà}

\subsection{Definizioni e proprietà della f.d.r.}

\begin{definition}
    Si dice \textbf{probabilità reale} una qualsiasi
    probabilità $P$ su $(\RR, \BB(\RR))$.
\end{definition}

\begin{definition}[Funzione di ripartizione di $P$]
    Data una probabilità reale $P$ si definisce
    allora la sua \textbf{funzione di ripartizione (f.d.r.)}
    come la funzione $F : \RR \to [0, 1]$ tale per cui:
    \[
        F(x) = P((-\infty, x]), \quad \forall x \in \RR.
    \]
    Si definisce inoltre $F(\pm\infty) \defeq \lim_{x \to \pm\infty} F(x)$.
    Indicheremo $F$ come $F_P$, e quando $P$ sarà nota dal contesto
    ci limiteremo a scrivere $F$.
\end{definition}

\begin{proposition}[Proprietà della f.d.r.]
    Sia $P$ una probabilità reale. Allora, se $F$ è la
    sua f.d.r. vale che:
    \begin{enumerate}[(i.)]
        \item $F$ è crescente, ovvero $F(x) \geq F(y) \impliedby x \geq y$ (infatti $(-\infty, x] \supseteq (-\infty, y]$),
        \item $F$ è continua a destra, ovverosia per ogni $\tilde{x} \in \RR$ vale che $\lim_{x \to \tilde{x}^+} F(x) = F(\tilde{x})$,
        \item $F(-\infty) = 0 \impliedby ((-\infty, -i])_{i \in \NN} \goesdown \emptyset$,
        \item $F(\infty) = 1 \impliedby ((-\infty, i])_{i \in \NN} \goesup \RR$.
    \end{enumerate}


    L'affermazione (ii.)~segue dal fatto che per ogni successione decrecente da destra $(x_i)_{i \in \NN} \goesdown \tilde{x}$,
    che esclude $\tilde{x}$, è
    tale per cui $((-\infty, x_i])_{i \in \NN} \goesdown (-\infty, \tilde{x}]$, e dunque
    $(F(x_i))_{i \in \NN} = (P((-\infty, x_i]))_{i \in \NN} \goesdown P((-\infty, \tilde{x}]) = F(\tilde{x})$.
\end{proposition}

\subsection{Corrispondenza tra f.d.r.~e probabilità, calcolo di \texorpdfstring{$P$}{P} tramite \texorpdfstring{$F$}{F} e probabilità continue}

\begin{proposition}[$P$ è univocamente determinata da $F$]
    \label{prop:unicita_fdr}
    Sia $F : \RR \to \RR$ una funzione tale per cui:
    \begin{enumerate}[(i.)]
        \item $F$ è crescente,
        \item $F$ è continua a destra,
        \item $\lim_{x \to \infty} F(x) = 1$,
        \item $\lim_{x \to -\infty} F(x) = 0$.
    \end{enumerate}
    Allora $0 \leq F \leq 1$ ed esiste un'unica probabilità reale $P$ avente
    $F$ come funzione di ripartizione. \smallskip


    L'unicità segue dal lemma di Dynkin.
\end{proposition}

\begin{remark}
    La continuità a sinistra della f.d.r.~non è invece garantita dacché per ogni successione da sinistra crescente
    $(x_i)_{i \in \NN} \goesup \tilde{x}$, che esclude $\tilde{x}$,
    vale che $((-\infty, x_i])_{i \in \NN} \goesup (-\infty, \tilde{x})$, e non
    $(-\infty, \tilde{x}]$. Dunque $\lim_{x \to \tilde{x}^-} F(x)$ esiste ed è $P((-\infty, x))$, indicato
    comunemente come $F(x^-)$, che può non coincidere con $F(x)$. \smallskip

    Dal momento che:
    \[
        P(\{x\}) = P((-\infty, x] \setminus (-\infty, x)) = F(x) - F(x^-),
    \]
    si deduce che $F$ è continua se e solo se $P(\{x\}) = 0$ (ossia se e solo se
    $F(x) = F(x^-)$).
\end{remark}

\begin{remark}
    Si deducono immediatamente dalla precedente osservazione le seguenti identità:
    \begin{itemize}
        \item $P([a, b]) = F(b) - F(a^-)$,
        \item $P((a, b)) = F(b^-) - F(a)$,
        \item $P([a, b)) = F(b^-) - F(a^-)$,
        \item $P((a, b]) = F(b) - F(a)$.
    \end{itemize} 
\end{remark}

\begin{definition}[$P$ continua]
    Si dice che una probabilità reale $P$ è \textbf{continua} se
    la sua f.d.r.~$F$ lo è, ossia se $P(\{a\}) = 0$ per ogni $a \in \RR$
    (quest'ultima equivalenza deriva dalla penultima osservazione). 
\end{definition}

\begin{remark}
    Per una probabilità $P$ continua la misura di un intervallo con estremi $a$ e $b$ è semplificata
    a $F(b) - F(a)$ in tutti i casi (infatti $F(a^-) = F(a)$ e $F(b^-) = F(b)$).  
\end{remark}

\section{Classi principali di probabilità reale}

Esistono due classi importanti, ma non esaustive, di probabilità reale: le
probabilità discrete e quelle assolutamente
continue, contenute tra quelle continue. Le classi di probabilità reali
si dividono dunque secondo il seguente schema:

\begin{center}
    \tikzset{every picture/.style={line width=0.75pt}} %set default line width to 0.75pt        

    \begin{tikzpicture}[x=0.75pt,y=0.75pt,yscale=-1,xscale=1,scale=0.8]
    %uncomment if require: \path (0,300); %set diagram left start at 0, and has height of 300
    
    %Shape: Rectangle [id:dp7885262489349896] 
    \draw   (220.19,65.28) -- (466.19,65.28) -- (466.19,196.28) -- (220.19,196.28) -- cycle ;
    %Shape: Ellipse [id:dp8572353674963273] 
    \draw   (229.69,95.79) .. controls (236.04,78.39) and (264.48,72.78) .. (293.2,83.27) .. controls (321.92,93.76) and (340.05,116.37) .. (333.7,133.77) .. controls (327.34,151.17) and (298.91,156.78) .. (270.19,146.29) .. controls (241.47,135.8) and (223.33,113.19) .. (229.69,95.79) -- cycle ;
    %Shape: Ellipse [id:dp3583963732297444] 
    \draw   (342.58,156.41) .. controls (332.85,131.07) and (350.75,100.63) .. (382.58,88.41) .. controls (414.4,76.19) and (448.08,86.82) .. (457.81,112.15) .. controls (467.54,137.48) and (449.63,167.93) .. (417.81,180.15) .. controls (385.99,192.37) and (352.31,181.74) .. (342.58,156.41) -- cycle ;
    %Shape: Ellipse [id:dp5413476700533957] 
    \draw   (393.91,107.98) .. controls (397.1,99.22) and (408.98,95.52) .. (420.44,99.7) .. controls (431.9,103.89) and (438.6,114.38) .. (435.4,123.13) .. controls (432.21,131.89) and (420.33,135.59) .. (408.87,131.41) .. controls (397.41,127.22) and (390.71,116.73) .. (393.91,107.98) -- cycle ;
    
    % Text Node
    \draw (252,103) node [anchor=north west][inner sep=0.75pt]   [align=left] {discrete};
    % Text Node
    \draw (358,145) node [anchor=north west][inner sep=0.75pt]   [align=left] {continue};
    % Text Node
    \draw (403,105) node [anchor=north west][inner sep=0.75pt]   [align=left] {AC};
    \end{tikzpicture}
\end{center}

\begin{example}[Esempio di probabilità né discreta né continua]
    Sia $F : \RR \to \RR$ tale per cui:
    \[
        F(x) = \begin{cases}
            0 & \text{se } x < 0, \\
            x + \frac{1}{2} & \text{se } 0 \leq x \leq \frac{1}{2}, \\
            1 & \text{altrimenti}.
        \end{cases}
    \]
    Allora $F$ è crescente, continua a destra e tale per cui
    $\lim_{x \to -\infty} F(x) = 0$, $\lim_{x \to \infty} F(x) = 1$.
    Pertanto esiste un'unica probabilità $P$ avente $F$ come f.d.r. per la
    \textit{Proposizione \ref{prop:unicita_fdr}}. \smallskip


    Dal momento che $F$ non è continua a sinistra in $0$, $F$ non è continua, e dunque
    $P$ non è continua. Inoltre $F$ non induce una probabilità discreta dacché
    non è costante a tratti in $[0, \nicefrac{1}{2}]$. Pertanto $P$ non è né
    continua né discreta.
\end{example}

\section{Probabilità discreta e rappresentazione della f.d.r.}

Come già discusso nella sezione della \textit{\hyperref[sec:discretizzazione]{Discretizzazione}},
una probabilità reale $P$ si dice \textit{discreta} se esiste $\Omega_0 \subseteq \RR$
discreto per cui $P$ è concentrata su $\Omega_0$. In tal caso, come già visto,
$P(A) = P(A \cap \Omega_0)$ per ogni $A \in \BB(\RR)$, e dunque $P$ è univocamente determinata
dalla densità discreta di $\restr{P}{\PP(\Omega_0)}$, che chiameremo $p$. \smallskip


In questo caso il range $R_P$ è dunque numerabile e, se $F$ è la f.d.r.~di $P$, vale che:
\[
    F(x) = P((-\infty, x]) = \sum_{\substack{y \in R_P \\ y \leq x}} p(y).
\]

\begin{remark}
    Se $P$ è discreta, come già osservato nella sezione della \textit{\hyperref[remark:identità_discreta_dirac]{Discretizzazione}},
    allora vale che:
    \[
        P = \sum_{x \in R_P} p(x) \, \delta_x.
    \]
\end{remark}

\begin{remark}
    Se $R_P$ non ha punti di accumulazione, allora $F$ è costante a tratti con salti
    negli $y \in R_P$ di ampiezza $p(y)$. \smallskip


    Al contrario, presa una successione $(p_r)_{r \in \QQ}$ con $\sum_{r \in \QQ} p_r = 1$,
    la probabilità $P = \sum_{r \in \QQ} p_r \, \delta_r$ è una probabilità discreta con
    f.d.r.~non costante a tratti (infatti tutti i punti di $\QQ$ sono punti di accumulazione). 
\end{remark}

Pertanto, se una probabilità reale è discreta, ci si può effettivamente restringere a tutti
i risultati della \textit{Parte 2}.

\section{Probabilità assolutamente continue (AC)}

\begin{warn}
    Si ricorda che con il simbolo $\int$ si intende l'integrale
    secondo Lebesgue e che si assume di star lavorando sempre con la
    misura di Lebesgue $m$.
\end{warn}

\subsection{Probabilità AC e funzione di densità}

\begin{definition}[Probabilità assolutamente continua (AC) e densità]
    Una probabilità $P$ si dice \textbf{assolutamente continua (AC)}
    se esiste una funzione
    boreliana $f : \RR \to \RR$ tale per cui:
    \[
        P(A) = \int_A f(x) \dx,
    \]
    dove si impiega l'integrale secondo Lebesgue. Tale funzione $f$ è
    detta \textbf{densità} di $P$. \smallskip


    Si assume implicitamente che $\int_\RR \abs{f(x)} \dx$ sia finito.
\end{definition}

\begin{remark}
    Se $P$ è AC, allora la sua f.d.r. $F$ è in particolare
    assolutamente continua, e dunque anche continua.
\end{remark}

\begin{remark}
    Nella pratica l'integrale $\int_A f(x) \dx$ si riduce in molti casi
    al più semplice integrale di Riemann, eventualmente improprio.
\end{remark}

\subsection{Proprietà e caratterizzazione della densità}

\begin{proposition}[Unicità della densità a meno di $m$-trascurabilità]
    Se $P$ è una probabilità AC con densità $f$ e $g$, allora
    $f = g$ q.o.~(e dunque $m(f \neq g) = 0$, ossia l'insieme
    $f \neq g$ è $m$-trascurabile).
\end{proposition}

\begin{remark}
    Si osserva che se $P$ è una probabilità AC con densità
    $f$, allora $f \geq 0$ q.o.~per continuità (altrimenti $P$ potrebbe
    assumere valori negativi) e $\int_\RR f(x) \dx = P(\RR) = 1$.
\end{remark}

\begin{proposition}[La densità determina univocamente la probabilità]
    Sia $f : \RR \to \RR$ una funzione boreliana tale per cui:
    \begin{enumerate}[(i.)]
        \item $f \geq 0$,
        \item $\int_\RR f(x) \dx = 1$.
    \end{enumerate}
    Allora esiste un'unica probabilità reale $P$ avente $f$ come densità.
\end{proposition}

\begin{proposition}[Relazioni tra la densità e la f.d.r.]
    Sia $P$ una probabilità reale con f.d.r. $F$. Allora valgono le seguenti affermazioni:
    \begin{enumerate}[(i.)]
        \item Se $P$ è AC con densità $f$, allora $F(x) = \int_{-\infty}^x f(y) \dy$. Viceversa
        se esiste $f$ per cui $F(x) = \int_{-\infty}^x f(y) \dy$, allora $P$ è AC con densità
        $f$.
        \item Se $F$ è continua e $C^1$ a tratti (ovverosia si restringe a una funzione $C^1$ eccetto che per un insieme di punti isolati),
        allora $P$ è AC con densità $f$ t.c.~$f = F'$ dove è definibile $F'$ e $f = 0$ altrimenti (segue dal Teorema fondamentale del calcolo integrale).
    \end{enumerate}
\end{proposition}

\begin{remark}
    Se $P$ è AC con densità $f$, allora $P(f = 0) = \int_{f = 0} f(x) \dx = 0$ e dunque
    l'insieme $f = 0$ è trascurabile rispetto a $P$. Dunque, ristringendo il range si
    ottiene che:
    \[
        P(A) = P(A \cap (f > 0)).
    \]
\end{remark}

\end{multicols*}
%--------------------------------------------------------------------
\begin{landscape}

\chapter*{Tabella e proprietà delle distribuzioni discrete}
\addcontentsline{toc}{chapter}{Tabella e proprietà delle distribuzioni discrete}

\vskip -0.3in

\begin{table}[htb]
\scalebox{0.74}{
\begin{tabular}{|l|l|l|l|l|l|l|}
\hline
Nome distribuzione                                                                                             & Caso di utilizzo                                                                                                                                                                                                                                           & Parametri                                                                                                                                                     & Densità discreta                                                                                                                                 & Valore atteso                                                                         & Momento secondo                   & Varianza                                                                                                   \\ \hline
\begin{tabular}[c]{@{}l@{}}Distr. di Bernoulli\\ $X \sim B(\pp)$\end{tabular}                                  & \begin{tabular}[c]{@{}l@{}}Esperimento con esito\\ di successo ($1$) o\\ insuccesso ($0$).\end{tabular}                                                                                                                                                    & $\pp$ -- probabilità di successo.                                                                                                                             & $P(X=1) = \pp$, $P(X=0) = 1-\pp$                                                                                                                     & $\EE[X] = \pp$                                                                          & $\EE[X^2] = \pp$                    & $\Var(X) = \pp(1-\pp)$                                                                                         \\ \hline
\begin{tabular}[c]{@{}l@{}}Distr. binomiale\\ $X \sim B(n, \pp)$\end{tabular}                                  & \begin{tabular}[c]{@{}l@{}}In una serie di $n$ esperimenti\\ col modello delle prove ripetute,\\ $X$ conta il numero di successi.\\ $X$ è in particolare somma di $n$\\ v.a.~i.i.d.~distribuite come $B(\pp)$.\end{tabular}                      & \begin{tabular}[c]{@{}l@{}}$n$ -- numero di esperimenti\\ $\pp$ -- probabilità di successo\\ dell'$i$-esimo esperimento\end{tabular}                          & \begin{tabular}[c]{@{}l@{}}$P(X=k) = \binom{n}{k}{\pp}^k (1-\pp)^{n-k}$\\ per $0 \leq k \leq n$ e $0$ altrimenti.\end{tabular}                         & \begin{tabular}[c]{@{}l@{}}$\EE[X] = n \pp$\\ (è somma di $n$ Bernoulliane)\end{tabular} & $\EE[X^2] = n \pp + n(n-1)\pp^2$       & \begin{tabular}[c]{@{}l@{}}$\Var(X) = n \pp(1-\pp)$\\ (è somma di $n$\\ Bernoulliane indipendenti)\end{tabular} \\ \hline
\begin{tabular}[c]{@{}l@{}}Distr. binomiale\\ negativa\\ $X \sim \BinNeg(h, \pp)$\end{tabular}                 & \begin{tabular}[c]{@{}l@{}}In una serie di infiniti esperimenti\\ col modello delle prove ripetute,\\ $X$ conta l'esperimento in cui si\\ ha l'$h$-esimo successo.\end{tabular}                                                                            & \begin{tabular}[c]{@{}l@{}}$h$ -- numero dei successi da misurare\\ $\pp \in (0, 1)$ -- probabilità di successo\\ dell'$i$-esimo esperimento\end{tabular}                & \begin{tabular}[c]{@{}l@{}}$P(X=k) = \binom{k-1}{h-1} \pp^h (1-\pp)^{k-h}$\\ laddove definibile e $0$ altrimenti.\end{tabular}                   & \begin{tabular}[c]{@{}l@{}}$\EE[X] = \frac{h}{\pp}$\\ (è somma di $h$ Geometriche)\end{tabular}                                                                & $\EE[X^2] = \frac{h(1+h-\pp)}{\pp^2}$ & \begin{tabular}[c]{@{}l@{}}$\Var(X) = \frac{h(1-\pp)}{\pp^2}$ \\ (è somma di $h$\\Geometriche indipendenti)\end{tabular}                                                                             \\ \hline
\begin{tabular}[c]{@{}l@{}}Distr. geometrica\\ $X \sim \Geom(\pp)$\end{tabular}                                & \begin{tabular}[c]{@{}l@{}}In una serie di infiniti esperimenti\\ col modello delle prove ripetute,\\ $X$ conta l'esperimento in cui si\\ ha il primo successo. È pari a\\ $\BinNeg(1, \pp)$\end{tabular}                                                  & \begin{tabular}[c]{@{}l@{}}$\pp \in (0, 1)$ -- probabilità di successo\\ all'$i$-esimo esperimento.\end{tabular}                                                         & \begin{tabular}[c]{@{}l@{}}$P(X=k) = \pp (1-\pp)^{k-1}$ per\\ $k \geq 1$ e $0$ per $k=0$.\end{tabular}                                               & $\EE[X] = \frac{1}{\pp}$                                                                & $\EE[X^2] = \frac{2-\pp}{\pp^2}$      & $\Var(X) = \frac{1-\pp}{\pp^2}$                                                                                \\ \hline
\begin{tabular}[c]{@{}l@{}}Distr. ipergeometrica\\ $X \sim H(N, N_1, n)$\end{tabular}                          & \begin{tabular}[c]{@{}l@{}}In un'estrazione di $n$ palline in\\ un'urna di $N$ palline, di cui\\ $N_1$ sono rosse, $X$ conta\\ il numero di palline rosse estratte.\end{tabular}                                                                           & \begin{tabular}[c]{@{}l@{}}$N$ -- numero di palline nell'urna\\ $N_1$ -- numero di palline rosse\\ nell'urna\\ $n$ -- numero di palline estratte\end{tabular} & \begin{tabular}[c]{@{}l@{}}$P(X=k) = \frac{\binom{N_1}{k} \binom{N-N_1}{n-k}}{\binom{N}{n}}$\\ laddove definibile e $0$ altrimenti.\end{tabular} &                                                                                       &                                   &                                                                                                            \\ \hline
\begin{tabular}[c]{@{}l@{}}Distri.di Poisson\\ (o degli eventi rari)\\ $X \sim \Poisson(\lambda)$\end{tabular} & \begin{tabular}[c]{@{}l@{}}In una sequenza di $n \gg 1$\\ esperimenti di parametro $\pp \ll 1$\\ con $n \pp \approx \lambda$,\\ $X$ misura il numero di successi.\\ Si può studiare come distribuzione\\ limite della distribuzione binomiale.\end{tabular} & $\lambda$ -- tasso di successo.                                                                                                                               & $P(X=k) = \frac{\lambda^k}{k!} e^{-\lambda}$                                                                                                           & $\EE[X] = \lambda$                                                                    & $\EE[X^2] = \lambda(\lambda + 1)$             & $\Var(X) = \lambda$                                                                                        \\ \hline
\end{tabular}}
\end{table}

Valgono inoltre le seguenti altre proprietà:

\small
\begin{itemize}
    \item Una somma di $n$ v.a.~i.i.d.~distribuite come $B(\pp)$ si
    distribuisce come $B(n, \pp)$.
    \item Se $X \sim B(n, \pp)$ e $Y \sim B(m, \pp)$ sono indipendenti, $X + Y$ si distribuisce come $B(n + m, \pp)$.
    \item Se $X \sim \Poisson(\lambda)$ e $Y \sim \Poisson(\mu)$ sono indipendenti,
    $X + Y$ si distribuisce come $\Poisson(\lambda + \mu)$.
    \item Se $X \sim \Geom(\pp)$, allora $P(X = \infty) = 0$\footnote{
        Ovverosia la probabilità che non vi siano mai successi è nulla.
    }. Da ciò si deduce che $P(X > k) = (1-\pp)^k$.
    \item Una $X \sim \BinNeg(h, \pp)$ è
    somma di $h$ v.a.~i.i.d.~distribuite
    come $\Geom(\pp)$.
    \item Una v.a.~$X$ sui numeri naturali si dice che ha la \textit{proprietà di perdita di memoria} se $P(X > n + k \mid X > k) = P(X > n)$. Una v.a.~ha
    la proprietà di perdita della memoria se e solo se è distribuita come
    la distribuzione geometrica.
\end{itemize}

\end{landscape}
%--------------------------------------------------------------------
\begin{landscape}

    \chapter*{Tabella e proprietà delle distribuzioni assolutamente continue}
    \addcontentsline{toc}{chapter}{Tabella e proprietà delle distribuzioni assolutamente continue}
    
    \vskip -0.3in

    \begin{center}
        \begin{table}[htb]
            \scalebox{1.1}{
                \begin{tabular}{|l|l|l|l|l|l|}
                \hline
                Nome distribuzione                                                                 & Caso di utilizzo                                                                                         & Parametri                                                                              & Densità                                                                        & Funzione di ripartizione                                                                                                                                                                                                                          & Probabilità                             \\ \hline
                \begin{tabular}[c]{@{}l@{}}Distr. uniforme\\ $X \sim U(B)$\end{tabular}            & \begin{tabular}[c]{@{}l@{}}Estrazione di un\\ punto reale a caso\\ su $B$ senza preferenze.\end{tabular} & \begin{tabular}[c]{@{}l@{}}$B \in \BB(\RR)$ -- insieme\\ da cui estrarre.\end{tabular} & $f(x) = \frac{1}{m(B)} 1_B(x)$                                                 & $F(x) = \frac{m((-\infty, x] \cap B)}{m(B)}$                                                                                                                                                                                    & $P(X \in A) = \frac{m(A \cap B)}{m(B)}$ \\ \hline
                \begin{tabular}[c]{@{}l@{}}Distr. esponenz.\\ $X \sim \Exp(\lambda)$\end{tabular}  & \begin{tabular}[c]{@{}l@{}}Processo di Poisson\\ in senso continuo.\end{tabular}                         & \begin{tabular}[c]{@{}l@{}}$\lambda > 0$ -- parametro\\ di Poisson.\end{tabular}       & $f(x) = \lambda e^{-\lambda x} 1_{(0, \infty)} (x)$                            & \begin{tabular}[c]{@{}l@{}}$F(x) = 1-e^{-\lambda x}$\\ per $x \geq 0$, $0$ altrimenti.\end{tabular}                                                                                                                             &                                         \\ \hline
                \begin{tabular}[c]{@{}l@{}}Distr. gamma\\ $X \sim \Gamma(r, \lambda)$\end{tabular} & \begin{tabular}[c]{@{}l@{}}Estensione della distr.\\ binomiale in senso\\ continuo.\end{tabular}         & $r > 0$, $\lambda > 0$.                                                                & $f(x) = \frac{\lambda^r}{\Gamma(r)} x^{r-1} e^{-\lambda x} 1_{(0, \infty)}(x)$ &                                                                                                                                                                                                                                 &                                         \\ \hline
                \begin{tabular}[c]{@{}l@{}}Distr. normale\\ $X \sim N(m, \sigma^2)$\end{tabular} &                                                                                                          & \begin{tabular}[c]{@{}l@{}}$m$ -- media.\\ $\sigma^2 > 0$ -- varianza.\end{tabular}  & $f(x) = \frac{1}{\sqrt{2\pi \sigma^2}} e^{-\nicefrac{(x-m)^2}{2\sigma^2}}$   & \begin{tabular}[c]{@{}l@{}}$\Phi(x) = \frac{1}{\sqrt{2\pi}} \int_{-\infty}^x e^{-\nicefrac{t^2}{2}} \dt$\\ per $N(0, 1)$ e si standardizza\\ le altre distr. con il cambio\\ di var. $z = \nicefrac{(x-m)}{\sigma}$.\end{tabular} &                                         \\ \hline
                \end{tabular}
            }
            \end{table}
    \end{center}

    Si ricorda che la funzione $\Gamma$ è definita in modo tale che $\Gamma(x) = \int_0^\infty t^{x-1} e^{-t} \dt$; si tratta
    di un'estensione della nozione di fattoriale ai valori reali (infatti, $\Gamma(n+1) = n!$ per $n \in \NN$).
    
    % Valgono inoltre le seguenti altre proprietà:
    %
    % \small
    % \begin{itemize}
    %     \item
    % \end{itemize}
    
    \end{landscape}
%--------------------------------------------------------------------
\chapter*{Tabella e proprietà della f.d.r.~\texorpdfstring{$\Phi(x)$}{Φ(x)} di una normale standard}
\addcontentsline{toc}{chapter}{Tabella e proprietà della f.d.r.~\texorpdfstring{$\Phi(x)$}{Φ(x)} di una normale standard}

Per $a \in \RR$ si definisce la funzione $\Phi(a) = \int_{-\infty}^a e^{-\nicefrac{x^2}{2}} \dx$. L'integrale $\Phi(\infty) \defeq \int_{-\infty}^a e^{-\nicefrac{x^2}{2}} \dx$ vale
esattamente $1$, mentre $\Phi(-\infty) \defeq 0$. Per la parità di $e^{-\nicefrac{x^2}{2}}$ vale che $\Phi(a) = 1 - \Phi(-a)$ (\textbf{simmetria}). A partire da questa funzione si può calcolare
$\int_{a}^b e^{-\nicefrac{x^2}{2}} \dx$, che risulta essere $\Phi(b) - \Phi(a)$. Se
$a > 0$, allora $\int_{-a}^a e^{-\nicefrac{x^2}{2}} \dx = \Phi(a) - \Phi(-a) = 2\Phi(a) - 1$.


\begin{longtable}{|l|l|l|l|l|l|l|l|l|l|l|}
\caption{Tabella $z$ di alcuni valori di $\Phi(x)$ per $x$ \textit{non negativo}. Per $x$ \textit{negativo} utilizzare \textbf{simmetria}. Si prendono le cifre fino al decimo e si legge la riga corrispondente, in base al centesimo si individua poi l'approssimazione da usare.} \label{tab:phi} \\

\hline
\endfirsthead

\endhead

\hline \multicolumn{3}{|r|}{{Continued on next page}} \\ \hline
\endfoot

\hline
\endlastfoot
z   & 0       & 0,01    & 0,02    & 0,03    & 0,04    & 0,05    & 0,06    & 0,07    & 0,08    & 0,09    \\
0   & 0,5     & 0,50399 & 0,50798 & 0,51197 & 0,51595 & 0,51994 & 0,52392 & 0,5279  & 0,53188 & 0,53586 \\
0,1 & 0,53983 & 0,5438  & 0,54776 & 0,55172 & 0,55567 & 0,55962 & 0,56356 & 0,56749 & 0,57142 & 0,57535 \\
0,2 & 0,57926 & 0,58317 & 0,58706 & 0,59095 & 0,59483 & 0,59871 & 0,60257 & 0,60642 & 0,61026 & 0,61409 \\
0,3 & 0,61791 & 0,62172 & 0,62552 & 0,6293  & 0,63307 & 0,63683 & 0,64058 & 0,64431 & 0,64803 & 0,65173 \\
0,4 & 0,65542 & 0,6591  & 0,66276 & 0,6664  & 0,67003 & 0,67364 & 0,67724 & 0,68082 & 0,68439 & 0,68793 \\
0,5 & 0,69146 & 0,69497 & 0,69847 & 0,70194 & 0,7054  & 0,70884 & 0,71226 & 0,71566 & 0,71904 & 0,7224  \\
0,6 & 0,72575 & 0,72907 & 0,73237 & 0,73565 & 0,73891 & 0,74215 & 0,74537 & 0,74857 & 0,75175 & 0,7549  \\
0,7 & 0,75804 & 0,76115 & 0,76424 & 0,7673  & 0,77035 & 0,77337 & 0,77637 & 0,77935 & 0,7823  & 0,78524 \\
0,8 & 0,78814 & 0,79103 & 0,79389 & 0,79673 & 0,79955 & 0,80234 & 0,80511 & 0,80785 & 0,81057 & 0,81327 \\
0,9 & 0,81594 & 0,81859 & 0,82121 & 0,82381 & 0,82639 & 0,82894 & 0,83147 & 0,83398 & 0,83646 & 0,83891 \\
1   & 0,84134 & 0,84375 & 0,84614 & 0,84849 & 0,85083 & 0,85314 & 0,85543 & 0,85769 & 0,85993 & 0,86214 \\
1,1 & 0,86433 & 0,8665  & 0,86864 & 0,87076 & 0,87286 & 0,87493 & 0,87698 & 0,879   & 0,881   & 0,88298 \\
1,2 & 0,88493 & 0,88686 & 0,88877 & 0,89065 & 0,89251 & 0,89435 & 0,89617 & 0,89796 & 0,89973 & 0,90147 \\
1,3 & 0,9032  & 0,9049  & 0,90658 & 0,90824 & 0,90988 & 0,91149 & 0,91309 & 0,91466 & 0,91621 & 0,91774 \\
1,4 & 0,91924 & 0,92073 & 0,9222  & 0,92364 & 0,92507 & 0,92647 & 0,92785 & 0,92922 & 0,93056 & 0,93189 \\
1,5 & 0,93319 & 0,93448 & 0,93574 & 0,93699 & 0,93822 & 0,93943 & 0,94062 & 0,94179 & 0,94295 & 0,94408 \\
1,6 & 0,9452  & 0,9463  & 0,94738 & 0,94845 & 0,9495  & 0,95053 & 0,95154 & 0,95254 & 0,95352 & 0,95449 \\
1,7 & 0,95543 & 0,95637 & 0,95728 & 0,95818 & 0,95907 & 0,95994 & 0,9608  & 0,96164 & 0,96246 & 0,96327 \\
1,8 & 0,96407 & 0,96485 & 0,96562 & 0,96638 & 0,96712 & 0,96784 & 0,96856 & 0,96926 & 0,96995 & 0,97062 \\
1,9 & 0,97128 & 0,97193 & 0,97257 & 0,9732  & 0,97381 & 0,97441 & 0,975   & 0,97558 & 0,97615 & 0,9767  \\
2   & 0,97725 & 0,97778 & 0,97831 & 0,97882 & 0,97932 & 0,97982 & 0,9803  & 0,98077 & 0,98124 & 0,98169 \\
2,1 & 0,98214 & 0,98257 & 0,983   & 0,98341 & 0,98382 & 0,98422 & 0,98461 & 0,985   & 0,98537 & 0,98574 \\
2,2 & 0,9861  & 0,98645 & 0,98679 & 0,98713 & 0,98745 & 0,98778 & 0,98809 & 0,9884  & 0,9887  & 0,98899 \\
2,3 & 0,98928 & 0,98956 & 0,98983 & 0,9901  & 0,99036 & 0,99061 & 0,99086 & 0,99111 & 0,99134 & 0,99158 \\
2,4 & 0,9918  & 0,99202 & 0,99224 & 0,99245 & 0,99266 & 0,99286 & 0,99305 & 0,99324 & 0,99343 & 0,99361 \\
2,5 & 0,99379 & 0,99396 & 0,99413 & 0,9943  & 0,99446 & 0,99461 & 0,99477 & 0,99492 & 0,99506 & 0,9952  \\
2,6 & 0,99534 & 0,99547 & 0,9956  & 0,99573 & 0,99585 & 0,99598 & 0,99609 & 0,99621 & 0,99632 & 0,99643 \\
2,7 & 0,99653 & 0,99664 & 0,99674 & 0,99683 & 0,99693 & 0,99702 & 0,99711 & 0,9972  & 0,99728 & 0,99736 \\
2,8 & 0,99744 & 0,99752 & 0,9976  & 0,99767 & 0,99774 & 0,99781 & 0,99788 & 0,99795 & 0,99801 & 0,99807 \\
2,9 & 0,99813 & 0,99819 & 0,99825 & 0,99831 & 0,99836 & 0,99841 & 0,99846 & 0,99851 & 0,99856 & 0,99861 \\
3   & 0,99865 & 0,99869 & 0,99874 & 0,99878 & 0,99882 & 0,99886 & 0,99889 & 0,99893 & 0,99896 & 0,999
\end{longtable}

\end{document}
