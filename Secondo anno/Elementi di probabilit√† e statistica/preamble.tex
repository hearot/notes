\usepackage[top=1.5cm,bottom=1.5cm,left=1.5cm,right=1.5cm]{geometry}
\usepackage[utf8]{inputenc}
\usepackage[italian]{babel}
\usepackage{amsmath,amssymb,amsfonts,amsthm,stmaryrd}
\usepackage{mathrsfs} % per mathscr
\usepackage{graphicx}% ruota freccia per le azioni
\usepackage{marvosym}% per il \Lightning
\usepackage{array}
\usepackage{faktor} % per gli insiemi quoziente
\usepackage[colorlinks=false]{hyperref}
\usepackage{xparse} % Per nuovi comandi con tanti input opzionali
\usepackage{relsize} % per \mathlarger
\usepackage{tikz-cd}
\usepackage{multicol}
\usepackage{multirow}
\usepackage{cancel}
\usepackage{fourier}
\usepackage{enumerate}
\usepackage{soul}
\usepackage{nicefrac}
\usepackage{longtable}
\usepackage{pdflscape}

\newtheorem*{warn}{\warning \; Attenzione}

\newtheoremstyle{customth}
{\topsep}{\topsep}
{\itshape}{}{\bfseries}{.}{\newline}{}

\newtheoremstyle{customdef}
{\topsep}{\topsep}
{\normalfont}{}{\bfseries}{.}{\newline}{}

\newtheoremstyle{customrem}
{\topsep}{\topsep}
{\normalfont}{}{\itshape}{.}{\newline}{}

\usepackage{fourier}

\theoremstyle{customth}
\newtheorem{theorem}{Teorema}[chapter]
\newtheorem{lemma}[theorem]{Lemma}
\newtheorem{corollary}[theorem]{Corollario}
\newtheorem{proposition}[theorem]{Proposizione}
\newtheorem{fact}[theorem]{Fatto}
\newtheorem{application}[theorem]{Applicazione}
\theoremstyle{customrem}
\newtheorem{remark}[theorem]{Osservazione\,}
\theoremstyle{customdef}
\newtheorem{definition}[theorem]{Definizione}
\newtheorem{notation}[theorem]{Notazione}
\newtheorem{example}[theorem]{Esempio}

\DeclareMathOperator{\BinNeg}{BinNeg}
\DeclareMathOperator{\Geom}{Geom}
\DeclareMathOperator{\Poisson}{Poisson}

\makeatletter
\renewenvironment{proof}[1][\proofname]{\par
  \pushQED{\qed}%
  \normalfont \topsep6\p@\@plus6\p@\relax
  \trivlist
  \item[\hskip\labelsep
        \itshape
    #1\@addpunct{.}]\mbox{}\\*
}{%
  \popQED\endtrivlist\@endpefalse
}
\makeatother

%============ Simboli standard =================
\newcommand{\FF}{\mathcal{F}}
\newcommand{\PP}{\mathcal{P}}
\newcommand{\NN}{\mathbb{N}}
\newcommand{\RR}{\mathbb{R}}

\newcommand{\pp}{\text{p\hspace{-0.7em}\raisebox{-3.4pt}{--}}\,\,}
\newcommand{\pbern}{\pp}


\newcommand{\defeq}{\overset{\mathrm{def}}{=}}
\newcommand{\deq}{\overset{\mathrm{(d)}}{=}}
\newcommand{\toprob}{\overset{\mathbb{P}}{\to}}
\DeclareMathOperator{\VA}{VA}
\DeclareMathOperator{\im}{im}
\DeclareMathOperator{\supp}{supp}
\DeclareMathOperator{\id}{id}
\DeclareMathOperator{\sgn}{sgn}

\DeclareMathOperator{\CI}{CI}
\newcommand{\eps}{\varepsilon}

\newcommand{\dx}{\mathop{dx}}

%\setcounter{secnumdepth}{1}

\newcommand{\groupto}{\rightrightarrows}

\newcommand{\restr}[2]{
	#1\arrowvert_{#2}
}

\makeatletter
\def\moverlay{\mathpalette\mov@rlay}
\def\mov@rlay#1#2{\leavevmode\vtop{%
   \baselineskip\z@skip \lineskiplimit-\maxdimen
   \ialign{\hfil$\m@th#1##$\hfil\cr#2\crcr}}}
\newcommand{\charfusion}[3][\mathord]{
    #1{\ifx#1\mathop\vphantom{#2}\fi
        \mathpalette\mov@rlay{#2\cr#3}
      }
    \ifx#1\mathop\expandafter\displaylimits\fi}
\makeatother

\newcommand{\cupdot}{\charfusion[\mathbin]{\cup}{\cdot}}
\newcommand{\bigcupdot}{\charfusion[\mathop]{\bigcup}{\cdot}}

\newcommand{\goesup}{\nearrow}
\newcommand{\goesdown}{\searrow}
\newcommand{\qc}{q.c.\ \!}
\newcommand{\va}{v.a.\ \!}

\newcommand{\BB}{\mathcal{B}}
\newcommand{\EE}{\mathbb{E}}
\DeclareMathOperator{\Var}{Var}
\DeclareMathOperator{\Cov}{Cov}

\newcommand{\inv}{^{-1}}

\newcommand{\abs}[1]{\left\lvert #1 \right\rvert}
\newcommand{\norm}[1]{\left\lVert #1 \right\rVert}


\NeedsTeXFormat{LaTeX2e}
%\ProvidesPackage{quiver}[2021/01/11 quiver]

% `tikz-cd` is necessary to draw commutative diagrams.
\RequirePackage{tikz-cd}
% `amssymb` is necessary for `\lrcorner` and `\ulcorner`.
\RequirePackage{amssymb}
% `calc` is necessary to draw curved arrows.
\usetikzlibrary{calc}
% `pathmorphing` is necessary to draw squiggly arrows.
\usetikzlibrary{decorations.pathmorphing}

% A TikZ style for curved arrows of a fixed height, due to AndréC.
\tikzset{curve/.style={settings={#1},to path={(\tikztostart)
    .. controls ($(\tikztostart)!\pv{pos}!(\tikztotarget)!\pv{height}!270:(\tikztotarget)$)
    and ($(\tikztostart)!1-\pv{pos}!(\tikztotarget)!\pv{height}!270:(\tikztotarget)$)
    .. (\tikztotarget)\tikztonodes}},
    settings/.code={\tikzset{quiver/.cd,#1}
        \def\pv##1{\pgfkeysvalueof{/tikz/quiver/##1}}},
    quiver/.cd,pos/.initial=0.35,height/.initial=0}

% TikZ arrowhead/tail styles.
\tikzset{tail reversed/.code={\pgfsetarrowsstart{tikzcd to}}}
\tikzset{2tail/.code={\pgfsetarrowsstart{Implies[reversed]}}}
\tikzset{2tail reversed/.code={\pgfsetarrowsstart{Implies}}}
% TikZ arrow styles.
\tikzset{no body/.style={/tikz/dash pattern=on 0 off 1mm}}
