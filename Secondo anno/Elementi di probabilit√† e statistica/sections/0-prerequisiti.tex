%--------------------------------------------------------------------
\chapter*{Prerequisiti matematici}
\addcontentsline{toc}{chapter}{Prerequisiti matematici}  
\setlength{\parindent}{2pt}

\begin{multicols*}{2}

\section*{Algebra lineare}

\begin{itemize}
    \item \textbf{Disuguaglianza di Cauchy-Schwarz} -- Se $\varphi(\cdot, \cdot)$
    è un prodotto scalare (o hermitiano) definito positivo su uno spazio vettoriale $V$, allora vale la seguente disuguaglianza:
    \[
        \varphi(v, v) \varphi(w, w) \geq \abs{\varphi(v, w)}^2 , \quad \forall v, w \in V.
    \]
    Inoltre vale l'uguaglianza se e solo se $v$ è multiplo di $w$, o viceversa. Per
    prodotti semidefiniti positivi la disuguaglianza vale ugualmente, ma in
    tal caso $v$ si scrive come somma di un vettore del cono isotropo e del prodotto di $w$ per uno scalare. 
    \item \textbf{Proprietà di $\cos(v, w)$} -- Vale che $\cos(v, w) \in [-1, 1]$ per
    ogni $v$, $w \in V$ in spazi vettoriali reali dove $\cos$ è ben definito. Segue
    dalla disuguaglianza di Cauchy-Schwarz.
\end{itemize}

\section*{Analisi matematica}

\begin{itemize}
    \item \textbf{Limite delle successioni monotone} -- Se una successione
    $(a_i)_{i \in \NN}$ è monotona, allora ammette limite. Se $(a_i)_{i \in \NN}$
    è crescente, allora $a_i \to \sup\{a_i \mid i \in \NN\}$ per $i \to \infty$ (e
    dunque converge se la successione è limitata dall'alto); se
    $(a_i)_{i \in \NN}$ è decrescente, allora $a_i \to \inf\{a_i \mid i \in \NN\}$ per $i \to \infty$ (e dunque converge se la successione è limitata dal basso).
    \item \textbf{Convergenza delle serie a termini positivi} -- Se una serie è
    a termini positivi, allora la successione delle somme parziali è crescente,
    e dunque la serie ammette come valore un valore reale o $\infty$.
    \item \textbf{Convergenza assoluta} -- Se una serie $\sum_{i \in \NN} \abs{a_i}$ converge
    (l'unica altra opzione è che diverga, per la proprietà sopracitata), allora
    $\sum_{i \in \NN} a_i$ converge. Non è vero il viceversa in generale.
    \item \textbf{Disuguaglianza di Jensen} -- Sia $f : \RR \supseteq S \to \RR$ una funzione convessa a
    valori reali. Allora vale che:
    \[
        f\left(\sum_{i \in [n]} a_i x_i\right) \leq \sum_{i \in [n]} a_i f(x_i), \quad \sum_{i \in [n]} a_i = 1, x_i.
    \]
    Se invece $f$ è concava, vale la disuguaglianza con $\geq$ al posto di $\leq$.
    \item \textbf{Disuguaglianza di Young} -- Sia $p \geq 1$ e sia $p'$ il
    suo esponente coniugato. Allora vale che:
    \[
        ab \leq \frac{a^p}{p} + \frac{b^p}{p}, \forall a, b > 0.
    \]
    Segue dalla disuguaglianza di Jensen applicata a $e^x$, che è convessa.
    \item \textbf{Disuguaglianza di Hölder} -- Sia $p > 1$ e sia $p'$ il
    suo esponente coniugato. Allora vale che:
    \[
        \sum_{i \in [n]} \abs{x_i y_i} \leq \norm{x}_p \norm{y}_p, \quad \forall x, y \in \RR^n, \forall n \in \NN.
    \]
    Per $p = 2$, è equivalente alla disuguaglianza di Cauchy-Schwarz sul
    prodotto scalare canonico di $\RR^n$. Segue dalla disuguaglianza di Young.
    \item \textbf{Disuguaglianza sulle potenze} -- Siano $x$, $y \in \RR$ e sia
    $p \geq 1$. Allora vale che:
    \[
        \abs{x+y}^p \leq 2^{p-1} (\abs{x}^p + \abs{y}^p).
    \]
    Segue dalla disuguaglianza di Jensen applicata a $f(t) = t^p$ per
    $\abs{x}$ e $\abs{y}$ ($t^p$ è convessa per $t \geq 0$).
\end{itemize}

\section*{Combinatoria}

\begin{itemize}
    \item \textbf{Principio di \textit{double counting}} -- Principio di dimostrazione per il quale
    se vi sono due modi diversi, ma equivalenti, di contare lo stesso numero di scelte
    di un qualsiasi sistema, allora le formule ricavate dai due modi devono
    essere identicamente uguali.
    \item \textbf{Principio di inclusione-esclusione} -- Teorema da cui discende che per $(A_i)_{i \in [n]}$ vale che: \[\abs{\bigcup_{i \in [n]} A_i} = \sum_{j \in [n]} (-1)^{j+1} \sum_{1 \leq i_1 < \cdots < i_j \leq n} \abs{\bigcap_{k \in [j]} A_{i_k}}.\]
    Inoltre vale che $\abs{\bigcup_{i \in [n]} A_i} = \sum_{i \in [n]} \abs{A_i}$ se e solo se gli $A_i$ sono a due a due disgiunti. Per $n = 2$,
    $\abs{A \cup B} = \abs{A} + \abs{B} - \abs{A \cap B}$.
    \item \textbf{Principio della piccionaia} (\textit{Pigeonhole principle}) -- Teorema che
    asserisce che per ogni funzione $f : [n+1] \to [n]$ esistono $i$, $j \in [n+1]$
    tali per cui $f(i) = f(j)$. Più informalmente, se si hanno $n+1$ oggetti da
    posizionare in $n$ buchi, esiste per forza un buco con due oggetti.
    \item \textbf{Principio della piccionaia generalizzato} -- Teorema che asserisce che
    per ogni funzione $f : [kn+1] \to [n]$ esistono $k+1$ elementi di $[kn+1]$ che
    condividono la stessa immagine. Più informalmente, se si hanno $kn+1$ oggetti
    da posizionare in $n$ buchi, esiste per forza un buco con $k+1$ oggetti. Segue per
    induzione dal Principio della piccionaia.
    \item \textbf{Principio moltiplicativo} -- Se una scelta può essere fatta in $N$
    passi e all'$i$-esimo passo corrispondono $n_i$ scelte, allora la scelta globale
    può essere fatta in $\prod_{i \in [N]} n_i$ modi.
    \item \textbf{Permutazioni di $n$ oggetti} -- Dati $n$ oggetti, esistono
    $n!$ modi di permutarli. Segue dal Principio moltiplicativo.
    \item \textbf{Disposizioni semplici di $n$ oggetti in $k$ posti} -- Dati $n$ oggetti
    e $k$ posti, allora esistono $D_{n,k}$ modi di disporre gli $n$ oggetti nei
    $k$ posti se $k \leq n$. Se $k = n$, ci si riduce a contare le permutazioni.
    \item \textbf{Disposizioni con ripetizione di $n$ oggetti in $k$ posti} -- Dati
    $n$ oggetti e $k$ posti, allora esistono $n^k$ modi di disporre con ripetizione gli $n$ oggetti
    nei $k$ posti. Segue dal Principio moltiplicativo.
    \item \textbf{Combinazioni di $n$ oggetti in $k$ posti} -- Dati $n$ oggetti
    e $k$ posti, allora esistono $C_{n,k} = \binom{n}{k} = \frac{n!}{(n-k)!k!}$ modi di disporre gli $n$ oggetti nei
    $k$ posti non facendo contare l'ordine, se $k \leq n$. Segue dal Principio
    moltiplicativo.
    \item \textbf{Combinazioni con ripetizione di $n$ oggetti in $k$ buchi} -- Data
    l'equazione $x_1 + \ldots + x_k = n$ con $x_i \in \NN$, esistono esattamente
    $\binom{n+k-1}{k-1}$ soluzioni. Alternativamente, data la disequazione
    $x_1 + \ldots + x_k \leq n$ con $x_i \in \NN$, esistono esattamente
    $\binom{n+k}{k}$ soluzioni (dacché ha le stesse soluzioni di
    $x_1 + \ldots + x_k + y = n$, dove $y \in \NN$). È un'applicazione di una
    tecnica combinatorica standard denominata \textit{stars and bars}.
    \item \textbf{Numero di scelte possibili per un'estrazione di $n$ palline rosse e nere da un insieme di $N_1$ palline rosse unito a un insieme di $N-N_1$ palline nere} -- Se $k$ è il numero di palline rosse estratte, le scelte possibili sono
    $\binom{N_1}{k} \binom{N - N_1}{n-k}$. Si può generalizzare il problema a
    un insieme di $N$ palline divise in $m$ gruppi da $N_i$ palline ciascuno
    (e dunque $\sum_{i \in [m]} N_i = N$) dove se ne estrae $n$ e $k_i$ è il
    numero di palline estratte dall'$i$-esimo gruppo (dunque $\sum_{i \in [m]} k_i = n$;
    in tal caso le scelte possibili sono $\prod_{i \in [m]} \binom{N_i}{k_i}$. Segue
    dal Principio moltiplicativo.
    \item \textbf{Identità sulle cardinalità}
\begin{itemize}
    \item $\#\{(a_1, \ldots, a_n) \in [k]^n \mid a_1 < a_2 < \ldots < a_n\} = \binom{n}{k}$ se $k \leq n$ -- Infatti data una classe di disposizione, esiste un unica lista ordinata
    in tale classe.
    \item $\#\{(a_1, \ldots, a_n) \in [k]^n \mid a_1 \leq a_2 \leq \ldots \leq a_n\} = \binom{n + k - 1}{k - 1}$. -- È sufficiente osservare che si sta
    contando esattamente le combinazioni con ripetizione in perfetta analogia con la precedente
    cardinalità.
\end{itemize}
\end{itemize}

\section*{Teoria degli insiemi}

\begin{itemize}
    \item \textbf{Leggi di De Morgan} -- Se $A$ e $B$ sono insiemi, allora
    $(A \cup B)^c = A^c \cap B^c$ e $(A \cap B)^c = A^c \cup B^c$.
    \item \textbf{Operazioni con $X\inv$ controimmagine} -- Se $X : D \to C$ è
    una funzione e $\FF = (A_i)_{i \in I}$ è una famiglia di sottoinsiemi di $C$, allora vale che $X\inv(\bigcup_{i \in I} A_i) = \bigcup_{i \in I} X\inv(A_i)$,
    $X\inv(\bigcap_{i \in I} A_i) = \bigcap_{i \in I} X\inv(A_i)$,
    $X\inv(A_i^c) = X\inv(A_i)^c$, ovverosia $X\inv$ commuta con unioni ($\cup$),
    intersezioni ($\cap$) e complementare ($^c$). $X\inv(\emptyset) = \emptyset$, e dunque $A_i \cap A_j = \emptyset \implies X\inv(A_i) \cap X\inv(A_j) = \emptyset$.
    Inoltre per $Y : C \to C'$ vale che $(Y \circ X)\inv(A) = X\inv(Y\inv(A))$,
    per $A \subseteq C'$.
\end{itemize}

\end{multicols*}