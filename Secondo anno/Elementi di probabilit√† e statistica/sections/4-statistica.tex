%--------------------------------------------------------------------
\chapter{Statistica inferenziale}
\setlength{\parindent}{2pt}

\begin{multicols*}{2}

Lo scopo della statistica inferenziale è quello di ottenere informazioni
riguardanti la distribuzione di probabilità di un esperimento a partire
dagli esiti di $n$ ripetizioni di quest'ultimo. \smallskip

Nel caso di questo corso, studieremo situazioni di statistica inferenziale
\textit{parametrica}, ovverosia situazioni in cui è conosciuto il modello
di probabilità del singolo esperimento a meno di un singolo parametro
(e.g.~l'esperimento $X$ è in legge uguale a $B(p)$, ma $p$ non è noto).

\section{Definizioni preliminari}

Si considerino dei dati statistici $x_1$, ...,
$x_n \in \RR$. Si consideri come spazio di probabilità
lo spazio discreto relativo a $[n]$ con distribuzione
uniforme. \smallskip

Si definisca su tale spazio la v.a.~$X : [n] \to \RR$ tale per cui
$i \mapsto x_i$. Si osserva facilmente che $X$ ha
range $r_x = \{x_1, ..., x_n\}$, e dunque il calcolo di tutti
i suoi indici può essere ristretto a $r_x$. \smallskip

Analogamente definiamo per dei dati $y_1$, ..., $y_n \in \RR$ la v.a.~$Y$.

\subsection{Indici di centralità e di dispersione sui singoli dati}

\begin{definition}[Media campionaria]
    Si definisce \textbf{media campionaria} il seguente
    indice di centralità:
    \[
        \overline{x} \defeq \frac{1}{n} \sum_{i=1}^n x_i. 
    \]
    Tale media coincide con il valore atteso di $X$.
\end{definition}

\begin{definition}[Mediana campionaria]
    Si definisce \textbf{mediana campionaria} il seguente
    indice di centralità:
    \[
        m_x \defeq \begin{cases}
            x_{\nicefrac{(n+1)}{2}} & \mbox{se $n$ dispari}, \\
            \nicefrac{\left(x_{\nicefrac{n}{2}} + x_{\nicefrac{(n+2)}{2}}\right)}{2} & \mbox{se $n$ pari}.
        \end{cases}
    \]
    Tale indice è una mediana per $X$.
\end{definition}

\begin{definition}[Varianza campionaria \textit{corretta}]
    Si definisce \textbf{varianza campionaria (corretta)} il seguente
    indice di dispersione:
    \[
        s^2 = s_x^2 = \sigma_x^2 \defeq \frac{1}{n-1} \sum_{i=1}^n (x_i - \overline{x})^2.
    \]
\end{definition}

\begin{warn}
    A differenza della media e della mediana, la varianza campionaria appena
    descritta \underline{non} coincide con la varianza che si calcolerebbe
    sulla v.a.~$X$. Infatti vale che:
    \[
        \Var(X) = \EE\left[(X - \EE[X])^2\right] = \frac{1}{n} \sum_{i=1}^n (x_i - \overline{x})^2,
    \]
    e dunque:
    \[
        s^2 = \frac{n}{n-1} \Var(X).
    \]
\end{warn}

\subsection{Indici su coppie di dati}

\begin{definition}[Coeff.~di correlazione campionario]
    Date delle coppie di dati $(x_i, y_i)_{i \in [n]}$, si definisce
    il \textbf{coefficiente di correlazione campionario} come:
    \[
        r \defeq \frac{\sum_{i=1}^n \left(x_i - \overline{x}\right)\left(y_i - \overline{y}\right)}{\sqrt{\sum_{i=1}^n \left(x_i - \overline{x}\right)^2 \cdot \sum_{i=1}^n \left(y_i - \overline{y}\right)^2}}.
    \]
    Tale valore coincide con l'usuale coefficiente di correlazione lineare di Bearson su
    $X$ e $Y$, ovverosia:
    \[
        r = \cos_{\Cov}(X, Y) = \frac{\Cov(X, Y)}{\sqrt{\Var(X) \Var(Y)}},
    \]
    che, per la disuguaglianza di Cauchy-Schwarz, appartiene all'intervallo $[-1, 1]$.
\end{definition}

\section{Modello statistico}

Come già osservato, la statistica inferenziale parametrica studia situazioni in cui
è necessario ricavare o stimare un singolo parametro su un dato modello di probabilità al fine
di ricavare la distribuzione di probabilità dei dati $x_1$, ..., $x_n$.

\begin{notation}[Parametri $\theta$ e probabilità $Q_\theta$]
    Denotiamo con $\Theta$ l'insieme dei possibili parametri $\theta$ per la distribuzione
    di probabilità sui dati $x_1$, ..., $x_n$. \smallskip

    Denotiamo con $Q_\theta$ la probabilità che si otterrebbe utilizzando il parametro $\sigma$
    nel modello di probabilità noto a meno di parametro.
\end{notation}

\begin{definition}
    Si definisce \textbf{modello statistico parametrico} una terna $(S, \cS, (Q_\theta)_{\theta \in \Theta})$,
    dove $(S, \cS)$ è uno spazio misurabile e $(Q_\theta)_{\theta \in \Theta}$ è una famiglia di
    misure di probabilità.
\end{definition}

\begin{example}
    Supponiamo di star cercando di ricavare la probabilità $p$ con cui esce testa per una data moneta. Allora
    un modello statistico che possiamo associare a questo problema è dato da $S = [1]$, $\cS = \PP([1])$ e
    $Q_\theta \sim B(\theta)$, con $\Theta = [0, 1]$, dove $1$ identifica la testa e $0$ la croce.
\end{example}

\section{Teoria degli stimatori su campioni di taglia \texorpdfstring{$n$}{n}}

\subsection{Campione, statistica e stimatore}

D'ora in avanti, sottintenderemo di star lavorando sul modello
statistico $(S, \cS, (Q_\theta)_{\theta \in \Theta})$.

\begin{definition}[Campione i.i.d.~di taglia $n$]
    Dato un modello statistico, si dice
    che una famiglia di v.a.~$(X_i : \Omega \to S)_{i \in [n]}$ i.i.d.~è un \textbf{campione i.i.d.~di taglia $n$}
    se per ogni $\sigma \in \Sigma$ esiste uno spazio di probabilità $(\Omega, \FF, P_\sigma)$ tale per cui
    $(P_\sigma)^{X_i}$ è uguale in legge a $Q_\theta$.
\end{definition}

Dato un campione di taglia $n$, useremo $P_\sigma$ per riferirci alla misura di probabilità
su $(\Omega, \FF)$ appena descritta. Scriveremo
come apice $\sigma$ per indicare di star lavorando nello spazio
di probabilità $(\Omega, \FF, P_\theta)$ (e.g.~$\EE^\sigma$ è riferito
a $P_\theta$).

\begin{definition}[Statistica e stimatore]
    Dato un campione i.i.d.~$(X_i)_{i \in [n]}$, si dice \textbf{statistica}
    una v.a.~dipendente dalle v.a.~$X_i$ ed eventualmente dal parametro $\sigma$.
    Si dice \textbf{stimatore} una statistica non dipendente direttamente da $\sigma$.
\end{definition}

\subsection{Correttezza di uno stimatore}

\begin{definition}[Stimatore corretto]
    Si dice che uno stimatore $U$ è \textbf{corretto} (o \textit{non distorto}) rispetto
    a $h : \Sigma \to \RR$ se per ogni $\sigma \in \Sigma$ vale che:
    \begin{enumerate}[(i.)]
        \item $U$ è $P_\sigma$-integrabile (i.e.~ammette valore atteso),
        \item $\EE^\sigma[U] = h(\sigma)$.
    \end{enumerate}
\end{definition}

\begin{remark}
    La media campionaria è uno stimatore corretto del valore atteso ($h : \sigma \mapsto \EE^\sigma[X_1]$). Infatti:
    \[
        \EE^\sigma\!\left[\overline{X}\right] = \EE^\sigma[X_1].
    \]
\end{remark}

\begin{remark}
    La varianza campionaria è uno stimatore corretto della varianza ($h : \sigma \mapsto \Var^\sigma(X_1)$). Infatti:
    \[
        \EE^\sigma[S^2] = \frac{1}{n-1} \left( n \EE^\sigma[X_1^2] - \EE^\sigma[X_1^2] - (n-1) \EE^\sigma[X_1]^2 \right) = \Var^\sigma(X_1).
    \]
    Si verifica analogamente che il coeff.~di correlazione campionario è uno stimatore corretto del
    coeff.~di correlazione tra $X_i$ e $X_j$.
\end{remark}

\subsection{Consistenza e non distorsione di una successione di stimatori}

\begin{definition}[Successione non distorta di stimatori]
    Una successione di stimatori $(U_k)_{k \in \NN^+}$ di $h(\sigma)$ si dice
    \textbf{asintoticamente non distorta} se $U_k$ è $P_\sigma$-integrabile
    (i.e.~ammette valore atteso) e:
    \[
        \lim_{k \to \infty} \EE^\sigma[U_k] = h(\sigma).
    \]
\end{definition}

\begin{definition}[Successione consistente di stimatori]
    Una successione di stimatori $(U_k)_{k \in \NN^+}$ di $h(\sigma)$ si dice
    \textbf{consistente} se:
    \[
        \lim_{k \to \infty} P_\sigma(\abs{U_k - h(\sigma)} > \eps) = 0, \quad \forall \eps > 0,
    \]
    ovverosia se $U_k$ converge in $P_\sigma$-probabilità a $h(\sigma)$.
\end{definition}

\begin{remark}
    La successione di stimatori $(\overline{X_n})_{n \in \NN^+}$, corretti per
    il valore atteso, è sia consistente che
    asintoticamente non distorta, per la LGN.
\end{remark}

\begin{remark}
    La successione di stimatori $(S^2_n)_{n \in \NN^+}$, corretti per la
    varianza, consistente, sempre per la LGN.
\end{remark}

\subsection{Rischio quadratico e preferibilità}

\begin{definition}[Rischio quadratico di uno stimatore]
    Dato uno stimatore $U$ di $h : \Sigma \to \RR$, si definisce
    \textbf{rischio quadratico} di $U$ per $\sigma$ il seguente valore:
    \[
        R_\sigma(U) = \EE[(U - h(\sigma))^2].
    \]
\end{definition}

\begin{remark}
    Se $U$ è uno stimatore corretto di $h$, allora
    $R_\sigma(U) = \Var^\sigma(U)$.
\end{remark}

\begin{definition}[Preferibilità]
    Dati due stimatori $U$, $V$ di $h : \Sigma \to \RR$, si dice
    che $U$ è \textbf{preferibile} rispetto a $V$ se
    $R_\sigma(U) \leq R_\sigma(V)$ per ogni $\sigma \in \Sigma$.
\end{definition}

\subsection{Stimatore di massima verosomiglianza}

D'ora in avanti sottintenderemo di star lavorando sullo
spazio misurabile $(\RR, \BB(\RR))$.

\begin{notation}
    Data la famiglia di probabilità $(Q_\sigma)_{\sigma \in \Sigma})$, usiamo
    scrivere $m_\sigma$ per riferirci alla densità discreta $q_\sigma$ (o $p_\sigma$)
    di $Q_\sigma$, qualora sia discreta, oppure alla sua funzione di densità
    $f_\sigma$, qualora $Q_\sigma$ sia assolutamente continua.
\end{notation}

\begin{definition}[Funzione di verosomiglianza]
    Dato un campione $(X_i)_{i \in [n]}$ i.i.d.~, si definisce
    \textbf{funzione di verosomiglianza} la funzione $L : \Sigma \times \RR^n$
    tale per cui:
    \[
        (\sigma, (x_i)_{i \in [n]}) \xmapsto{L} L_\sigma(x_1, \ldots, x_n) \defeq m_\sigma(x_1) \cdots m_\sigma(x_n).
    \]
    Equivalentemente, $L_\sigma(x_1, \ldots, x_n)$ rappresenta la densità congiunta su $Q_\sigma$
    di $x_1$, ..., $x_n$.
\end{definition}

\begin{notation}
    Scriveremo $L_U(X_1, \ldots, X_n)$ con $U$ v.a. e
    $(X_i)_{i \in [n]}$ famiglia di v.a.~reali sottintendendo
    l'insieme $L_{U(\omega)}(X_1(\omega), \ldots, X_n(\omega))$,
    assumendo $U(\omega) \in \Sigma$.
\end{notation}

\begin{definition}[Stimatore di massima verosomiglianza di $\sigma$]
    Si dice che uno stimatore $U$ è di \textbf{massima verosomiglianza di $\sigma$}
    su un campione i.i.d.~$(X_i)_{i \in [n]}$ se:
    \[
        L_U(X_1, \ldots, X_n) = \sup_{\theta \in \Theta} L_\theta(X_1, \ldots, X_n), \quad \forall \omega \in S.
    \]
    In altre parole, uno stimatore $U$ è di massima verosomiglianza su un campione se
    per dei dati $x_1$, ..., $x_n$ restituisce il parametro $\theta$ che massimizza
    $L_\theta(x_1, \ldots, x_n)$, ovverosia la densità consiunta dei dati
    $x_1$, ..., $x_n$ (i.e.~la probabilità che si ottenga $x_1$, ..., $x_n$).
\end{definition}

\begin{example}[Prova di Bernoulli]
    Sia $Q_\theta \sim B(\theta)$. Dati gli esiti $x_1$, ..., $x_n$ di $n$ prove,
    ricaviamo che:
    \[
        L_\theta(x_1, \ldots, x_n) = \theta^{\sum_i x_i} (1 - \theta)^{n - \theta^{\sum_i x_i}},
    \]
    da cui:
    \[
        \log L_\theta(x_1, \ldots, x_n) = n \overline{x} \log(\theta) + n (1 - \overline{x}) \log(1 - \theta).
    \]
    Tale funzione ha massimo per $\theta = \overline{x}$, e dunque
    $\overline{X}$ è uno stimatore di massima verosomiglianza di $\theta$. \smallskip

    In altre parole, la migliore stima di $\sigma$ data una sequenza di $n$ prove di Bernoulli è
    la frequenza relativa di successi.
\end{example}

\begin{example}
    Sia $Q_\theta \sim U([0, \theta])$ con $\theta > 0$. Dati gli esiti $x_1$, ..., $x_n$ ricaviamo che:
    \[
        L_\theta(x_1, \ldots, x_n) = \frac{1}{\theta^n} \prod_i 1_{[0, \theta]}(x_i) =
        \frac{1}{\theta^n} 1_{0 \leq \min_i x_i \leq \max_i x_i \leq \theta},
    \]
    che ha massimo per $\theta = \max_i x_i$. Pertanto $\max\{X_1, \ldots, X_n\}$ è uno stimatore
    di massima somiglianza di $\theta$. \smallskip

    In altre parole, dati degli esiti $x_1$, ..., $x_n$, una delle migliori stime che possiamo fare
    su $\theta$ è $\max_i x_i$.
\end{example}

\end{multicols*}
