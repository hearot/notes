%--------------------------------------------------------------------
\chapter{Statistica inferenziale}
\setlength{\parindent}{2pt}

\begin{multicols*}{2}

Lo scopo della statistica inferenziale è quello di ottenere informazioni
riguardanti la distribuzione di probabilità di un esperimento a partire
dagli esiti di $n$ ripetizioni di quest'ultimo. \smallskip

Nel caso di questo corso, studieremo situazioni di statistica inferenziale
\textit{parametrica}, ovverosia situazioni in cui è conosciuto il modello
di probabilità del singolo esperimento a meno di un singolo parametro
(e.g.~l'esperimento $X$ è in legge uguale a $B(p)$, ma $p$ non è noto).

\section{Definizioni preliminari}

Si considerino dei dati statistici $x_1$, ...,
$x_n \in \RR$. Si consideri come spazio di probabilità
lo spazio discreto relativo a $[n]$ con distribuzione
uniforme. \smallskip

Si definisca su tale spazio la v.a.~$X : [n] \to \RR$ tale per cui
$i \mapsto x_i$. Si osserva facilmente che $X$ ha
range $r_x = \{x_1, ..., x_n\}$, e dunque il calcolo di tutti
i suoi indici può essere ristretto a $r_x$. \smallskip

Analogamente definiamo per dei dati $y_1$, ..., $y_n \in \RR$ la v.a.~$Y$.

\subsection{Indici di centralità e di dispersione sui singoli dati}

\begin{definition}[Media campionaria]
    Si definisce \textbf{media campionaria} il seguente
    indice di centralità:
    \[
        \overline{x} \defeq \frac{1}{n} \sum_{i=1}^n x_i. 
    \]
    Tale media coincide con il valore atteso di $X$.
\end{definition}

\begin{definition}[Mediana campionaria]
    Si definisce \textbf{mediana campionaria} il seguente
    indice di centralità:
    \[
        m_x \defeq \begin{cases}
            x_{\nicefrac{(n+1)}{2}} & \mbox{se $n$ dispari}, \\
            \nicefrac{\left(x_{\nicefrac{n}{2}} + x_{\nicefrac{(n+2)}{2}}\right)}{2} & \mbox{se $n$ pari}.
        \end{cases}
    \]
    Tale indice è una mediana per $X$.
\end{definition}

\begin{definition}[Varianza campionaria \textit{corretta}]
    Si definisce \textbf{varianza campionaria (corretta)} il seguente
    indice di dispersione:
    \[
        s^2 = s_x^2 = \sigma_x^2 \defeq \frac{1}{n-1} \sum_{i=1}^n (x_i - \overline{x})^2.
    \]
\end{definition}

\begin{warn}
    A differenza della media e della mediana, la varianza campionaria appena
    descritta \underline{non} coincide con la varianza che si calcolerebbe
    sulla v.a.~$X$. Infatti vale che:
    \[
        \Var(X) = \EE\left[(X - \EE[X])^2\right] = \frac{1}{n} \sum_{i=1}^n (x_i - \overline{x})^2,
    \]
    e dunque:
    \[
        s^2 = \frac{n}{n-1} \Var(X).
    \]
\end{warn}

\subsection{Indici su coppie di dati}

\begin{definition}[Coeff.~di correlazione campionario]
    Date delle coppie di dati $(x_i, y_i)_{i \in [n]}$, si definisce
    il \textbf{coefficiente di correlazione campionario} come:
    \[
        r \defeq \frac{\sum_{i=1}^n \left(x_i - \overline{x}\right)\left(y_i - \overline{y}\right)}{\sqrt{\sum_{i=1}^n \left(x_i - \overline{x}\right)^2 \cdot \sum_{i=1}^n \left(y_i - \overline{y}\right)^2}}.
    \]
    Tale valore coincide con l'usuale coefficiente di correlazione lineare di Bearson su
    $X$ e $Y$, ovverosia:
    \[
        r = \cos_{\Cov}(X, Y) = \frac{\Cov(X, Y)}{\sqrt{\Var(X) \Var(Y)}},
    \]
    che, per la disuguaglianza di Cauchy-Schwarz, appartiene all'intervallo $[-1, 1]$.
\end{definition}

\section{Modello statistico}

Come già osservato, la statistica inferenziale parametrica studia situazioni in cui
è necessario ricavare o stimare un singolo parametro su un dato modello di probabilità al fine
di ricavare la distribuzione di probabilità dei dati $x_1$, ..., $x_n$.

\begin{notation}[Parametri $\theta$ e probabilità $Q_\theta$]
    Denotiamo con $\Theta$ l'insieme dei possibili parametri $\theta$ per la distribuzione
    di probabilità sui dati $x_1$, ..., $x_n$. \smallskip

    Denotiamo con $Q_\theta$ la probabilità che si otterrebbe utilizzando il parametro $\theta$
    nel modello di probabilità noto a meno di parametro.
\end{notation}

\begin{definition}
    Si definisce \textbf{modello statistico parametrico} una terna $(S, \cS, (Q_\theta)_{\theta \in \Theta})$,
    dove $(S, \cS)$ è uno spazio misurabile e $(Q_\theta)_{\theta \in \Theta}$ è una famiglia di
    misure di probabilità.
\end{definition}

\begin{example}
    Supponiamo di star cercando di ricavare la probabilità $p$ con cui esce testa per una data moneta. Allora
    un modello statistico che possiamo associare a questo problema è dato da $S = [1]$, $\cS = \PP([1])$ e
    $Q_\theta \sim B(\theta)$, con $\Theta = [0, 1]$, dove $1$ identifica la testa e $0$ la croce.
\end{example}

\section{Teoria degli stimatori su campioni di taglia \texorpdfstring{$n$}{n}}

\subsection{Campione, statistica e stimatore}

D'ora in avanti, sottintenderemo di star lavorando sul modello
statistico $(S, \cS, (Q_\theta)_{\theta \in \Theta})$.

\begin{definition}[Campione i.i.d.~di taglia $n$]
    Dato un modello statistico, si dice
    che una famiglia di v.a.~$(X_i : \Omega \to S)_{i \in [n]}$ i.i.d.~è un \textbf{campione i.i.d.~di taglia $n$}
    se esiste uno spazio misurabile $(\Omega, \FF)$ tale per cui,
    per ogni $\theta \in \Theta$, esiste una probabilità $P_\theta$ su $(\Omega, \FF)$ tale per cui
    $(P_\theta)^{X_i}$ è uguale in legge a $Q_\theta$. Un campione rappresenta generalmente il risultato di
    $n$ esiti di un esperimento aleatorio.
\end{definition}

Dato un campione di taglia $n$, useremo $P_\theta$ per riferirci alla misura di probabilità
su $(\Omega, \FF)$ appena descritta. Scriveremo
come apice $\theta$ per indicare di star lavorando nello spazio
di probabilità $(\Omega, \FF, P_\theta)$ (e.g.~$\EE^\theta$ è riferito
a $P_\theta$).

\begin{definition}[Statistica e stimatore]
    Dato un campione i.i.d.~$(X_i)_{i \in [n]}$, si dice \textbf{statistica}
    una v.a.~dipendente dalle v.a.~$X_i$ ed eventualmente dal parametro $\theta$.
    Si dice \textbf{stimatore} una statistica non dipendente direttamente da $\theta$.
\end{definition}

\subsection{Correttezza di uno stimatore}

\begin{definition}[Stimatore corretto]
    Si dice che uno stimatore $U$ è \textbf{corretto} (o \textit{non distorto}) rispetto
    a $h : \Theta \to \RR$ se per ogni $\theta \in \Theta$ vale che:
    \begin{enumerate}[(i.)]
        \item $U$ è $P_\theta$-integrabile (i.e.~ammette valore atteso),
        \item $\EE^\theta[U] = h(\theta)$.
    \end{enumerate}
\end{definition}

\begin{remark}
    La media campionaria è uno stimatore corretto del valore atteso ($h : \theta \mapsto \EE^\theta[X_1]$). Infatti:
    \[
        \EE^\theta\!\left[\overline{X}\right] = \EE^\theta[X_1].
    \]
\end{remark}

\begin{remark}
    La varianza campionaria è uno stimatore corretto della varianza ($h : \theta \mapsto \Var^\theta(X_1)$). Infatti:
    \[
        \EE^\theta[S^2] = \frac{1}{n-1} \left( n \EE^\theta[X_1^2] - \EE^\theta[X_1^2] - (n-1) \EE^\theta[X_1]^2 \right) = \Var^\theta(X_1).
    \]
    Si verifica analogamente che il coeff.~di correlazione campionario è uno stimatore corretto del
    coeff.~di correlazione tra $X_i$ e $X_j$.
\end{remark}

\subsection{Consistenza e non distorsione di una successione di stimatori}

\begin{definition}[Successione non distorta di stimatori]
    Una successione di stimatori $(U_k)_{k \in \NN^+}$ di $h(\theta)$ si dice
    \textbf{asintoticamente non distorta} se $U_k$ è $P_\theta$-integrabile
    (i.e.~ammette valore atteso) e:
    \[
        \lim_{k \to \infty} \EE^\theta[U_k] = h(\theta).
    \]
\end{definition}

\begin{definition}[Successione consistente di stimatori]
    Una successione di stimatori $(U_k)_{k \in \NN^+}$ di $h(\theta)$ si dice
    \textbf{consistente} se:
    \[
        \lim_{k \to \infty} P_\theta(\abs{U_k - h(\theta)} > \eps) = 0, \quad \forall \eps > 0,
    \]
    ovverosia se $U_k$ converge in $P_\theta$-probabilità a $h(\theta)$.
\end{definition}

\begin{remark}
    La successione di stimatori $(\overline{X_n})_{n \in \NN^+}$, corretti per
    il valore atteso, è sia consistente che
    asintoticamente non distorta, per la LGN.
\end{remark}

\begin{remark}
    La successione di stimatori $(S^2_n)_{n \in \NN^+}$, corretti per la
    varianza, consistente, sempre per la LGN.
\end{remark}

\subsection{Rischio quadratico e preferibilità}

\begin{definition}[Rischio quadratico di uno stimatore]
    Dato uno stimatore $U$ di $h : \Theta \to \RR$, si definisce
    \textbf{rischio quadratico} di $U$ per $\theta$ il seguente valore:
    \[
        R_\theta(U) = \EE[(U - h(\theta))^2].
    \]
\end{definition}

\begin{remark}
    Se $U$ è uno stimatore corretto di $h$, allora
    $R_\theta(U) = \Var^\theta(U)$.
\end{remark}

\begin{definition}[Preferibilità]
    Dati due stimatori $U$, $V$ di $h : \Theta \to \RR$, si dice
    che $U$ è \textbf{preferibile} rispetto a $V$ se
    $R_\theta(U) \leq R_\theta(V)$ per ogni $\theta \in \Theta$.
\end{definition}

\subsection{Stimatore di massima verosomiglianza (MLE)}

D'ora in avanti sottintenderemo di star lavorando sullo
spazio misurabile $(\RR, \BB(\RR))$.

\begin{notation}
    Data la famiglia di probabilità $(Q_\theta)_{\theta \in \Theta})$, usiamo
    scrivere $m_\theta$ per riferirci alla densità discreta $q_\theta$ (o $p_\theta$)
    di $Q_\theta$, qualora sia discreta, oppure alla sua funzione di densità
    $f_\theta$, qualora $Q_\theta$ sia assolutamente continua.
\end{notation}

\begin{definition}[Funzione di verosomiglianza]
    Dato un campione $(X_i)_{i \in [n]}$ i.i.d.~, si definisce
    \textbf{funzione di verosomiglianza} la funzione $L : \Theta \times \RR^n$
    tale per cui:
    \[
        (\theta, (x_i)_{i \in [n]}) \xmapsto{L} L_\theta(x_1, \ldots, x_n) \defeq m_\theta(x_1) \cdots m_\theta(x_n).
    \]
    Equivalentemente, $L_\theta(x_1, \ldots, x_n)$ rappresenta la densità congiunta su $Q_\theta$
    di $x_1$, ..., $x_n$.
\end{definition}

\begin{notation}
    Scriveremo $L_U(X_1, \ldots, X_n)$ con $U$ v.a. e
    $(X_i)_{i \in [n]}$ famiglia di v.a.~reali sottintendendo
    l'insieme $L_{U(\omega)}(X_1(\omega), \ldots, X_n(\omega))$,
    assumendo $U(\omega) \in \Theta$.
\end{notation}

\begin{definition}[Stimatore di massima verosomiglianza di $\theta$]
    Si dice che uno stimatore $U$ è di \textbf{massima verosomiglianza di $\theta$} (MLE, da
    \textit{maximum likelihood estimator})
    su un campione i.i.d.~$(X_i)_{i \in [n]}$ se:
    \[
        L_U(X_1, \ldots, X_n) = \sup_{\theta \in \Theta} L_\theta(X_1, \ldots, X_n), \quad \forall \omega \in S.
    \]
    In altre parole, uno stimatore $U$ è di massima verosomiglianza su un campione se
    per dei dati $x_1$, ..., $x_n$ restituisce il parametro $\theta$ che massimizza
    $L_\theta(x_1, \ldots, x_n)$, ovverosia la densità consiunta dei dati
    $x_1$, ..., $x_n$ (i.e.~la probabilità che si ottenga $x_1$, ..., $x_n$).
\end{definition}

\begin{example}[Prova di Bernoulli]
    Sia $Q_\theta \sim B(\theta)$. Dati gli esiti $x_1$, ..., $x_n$ di $n$ prove,
    ricaviamo che:
    \[
        L_\theta(x_1, \ldots, x_n) = \theta^{\sum_i x_i} (1 - \theta)^{n - \theta^{\sum_i x_i}},
    \]
    da cui:
    \[
        \log L_\theta(x_1, \ldots, x_n) = n \overline{x} \log(\theta) + n (1 - \overline{x}) \log(1 - \theta).
    \]
    Tale funzione ha massimo per $\theta = \overline{x}$, e dunque
    $\overline{X}$ è uno stimatore di massima verosomiglianza di $\theta$. \smallskip

    In altre parole, la migliore stima di $\theta$ data una sequenza di $n$ prove di Bernoulli è
    la frequenza relativa di successi.
\end{example}

\begin{example}
    Sia $Q_\theta \sim U([0, \theta])$ con $\theta > 0$. Dati gli esiti $x_1$, ..., $x_n$ ricaviamo che:
    \[
        L_\theta(x_1, \ldots, x_n) = \frac{1}{\theta^n} \prod_i 1_{[0, \theta]}(x_i) =
        \frac{1}{\theta^n} 1_{0 \leq \min_i x_i \leq \max_i x_i \leq \theta},
    \]
    che ha massimo per $\theta = \max_i x_i$. Pertanto $\max\{X_1, \ldots, X_n\}$ è uno stimatore
    di massima somiglianza di $\theta$. \smallskip

    In altre parole, dati degli esiti $x_1$, ..., $x_n$, una delle migliori stime che possiamo fare
    su $\theta$ è $\max_i x_i$.
\end{example}

\section{Modello esponenziale, unicità e consistenza dello stimatore MLE}

\begin{definition}[Modello statistico esponenziale]
    Dato un modello statistico $(S, \cS, (Q_\theta)_{\theta \in \Theta})$, si dice che
    tale modello è \textbf{esponenziale} nei seguenti due casi:
    
    \begin{enumerate}[(i.)]
        \item[\scriptsize (caso discreto)] data $Q_\theta$ discreta, allora esistono
    $g$, $T : \NN \to \RR$ e $c_\theta : \Theta \to \RR$ per cui
    $p_\theta(k) = c_\theta g(k) e^{\theta T(k)}$ e tali che
    $g$, $T$ dipendano solo da $k$ e $c_\theta$ solo da $\theta$.
        \item[\scriptsize (caso ass.~cont.)] data $Q_\theta$ AC, allora esistono
        $g$, $T : \RR \to \RR$ boreliane e $c_\theta : \Theta \to \RR$ per cui
        $f_\theta(x) = c_\theta g(x) e^{\theta T(x)}$ e tali che
        $g$, $T$ dipendano solo da $x$ e $c_\theta$ solo da $\theta$.
    \end{enumerate}
\end{definition}

Per i modelli esponenziali valgono i seguenti fondamentali teoremi:

\begin{theorem}[Unicità e consistenza dello stimatore MLE per densità discrete]
    Si consideri il modello $(\RR, \BB(\RR), (Q_\theta)_{\theta \in \Theta})$ tale per cui:
    \begin{itemize}
        \item $\theta_1 \neq \theta_2 \implies Q_{\theta_1} \neq Q_{\theta_2}$,
        \item $\Theta \subseteq \RR$ è un intervallo aperto,
        \item $Q_\theta$ è esponenziale discreta di densità
        $p_\theta(k) = c_\theta g(k) e^{\theta T(k)}$,
        \item $\sum_{i \in \NN} g(k) T^2(k) e^{\theta T(k)^+} < \infty$ per ogni $\theta \in \Theta$.
    \end{itemize}
    Premesso ciò, se $(X_i)_{i \in [n]}$ è un campione i.i.d.~di taglia $n$ ed esiste uno stimatore
    $U$ di massima verosomiglianza di $\theta$ rispetto a tale campione, allora, sempre rispetto
    a $(X_i)_{i \in [n]}$,
    $U$ è l'unico stimatore di massima verosomiglianza di $\theta$ ed è consistente rispetto a $\theta$. \smallskip

    In particolare, fissati i dati $x_1$, ..., $x_n$, lo stimatore di massima verosomiglianza $\hat\theta$ risolve la seguente equazione:
    \[
        \frac{d \left[- \log(c_\theta)\right]}{d\theta} \left(\hat\theta\right) = \sum_{i \in [n]} T(x_i).
    \]
\end{theorem}

\begin{theorem}[Unicità e consistenza dello stimatore MLE per densità AC]
    Si consideri il modello $(\RR, \BB(\RR), (Q_\theta)_{\theta \in \Theta})$ tale per cui:
    \begin{itemize}
        \item $\theta_1 \neq \theta_2 \implies Q_{\theta_1} \neq Q_{\theta_2}$,
        \item $\Theta \subseteq \RR$ è un intervallo aperto,
        \item $Q_\theta$ è esponenziale assolutamente continua di densità
        $f_\theta(x) = c_\theta g(x) e^{\theta T(x)}$,
        \item $h : x \mapsto g(x) T^2(x) e^{\theta T(x)^+}$ è integrabile per ogni $\theta \in \Theta$.
    \end{itemize}
    Premesso ciò, se $(X_i)_{i \in [n]}$ è un campione i.i.d.~di taglia $n$ ed esiste uno stimatore
    $U$ di massima verosomiglianza di $\theta$ rispetto a tale campione, allora, sempre rispetto
    a $(X_i)_{i \in [n]}$,
    $U$ è l'unico stimatore di massima verosomiglianza di $\theta$ ed è consistente rispetto a $\theta$.
\end{theorem}

\begin{remark}
    L'enunciato precedente può essere generalizzato ad aperti $\Theta$ convessi in $\RR^d$
    con funzione $T : \RR \to \RR^d$ boreliana, ponendo:
    \[
        f_\theta(x) = c_\theta g(x) \exp\left(\theta^\top T(x)\right).
    \]
\end{remark}

\begin{remark}
    A partire al precedente teorema si può dunque dimostrare che:
    \begin{itemize}
        \item $(\overline{X}, \frac{n-1}{n} S^2)$ è l'unico stimatore di massima verosomiglianza per $(m, \sigma^2)$ sul
        modello $N(m, \sigma^2)$,
        \item se $\sigma^2$ è nota, $\overline{X}$ è l'unico stimatore di massima verosomiglianza per
        $m$ sul modello $N(m, \sigma^2)$,
        \item se $m$ è nota, $\frac{n-1}{n} S^2$ è l'unico stimatore di massima verosomiglianza per
        $\sigma^2$ sul modello $N(m, \sigma^2)$.
    \end{itemize}
\end{remark}

\section{Intervalli di fiducia}

\subsection{Regione di fiducia}

\begin{definition}
    Dato il modello statistico $(S, \cS, (Q_\theta)_{\theta \in \Theta})$ con campione
    i.i.d.~$(X_i)_{i \in \NN}$, si definisce \textbf{regione di fiducia a livello $1-\alpha$}
    per il parametro $\theta$ una mappa $D : \Theta \to \PP(\Omega)$, detta \textit{insieme aleatorio}, tale per cui:
    \[
        P_\theta(\theta \in D) \geq 1 - \alpha, \quad \forall \theta \in \Theta,
    \]
    dove $P_\theta$ è la probabilità relativa allo spazio misurabile del campione i.i.d.~e
    $\{\theta \in D\} = \{\omega \in \Omega \mid \theta \in D(\omega)\} \in \FF$.
\end{definition}

\subsection{Quantili e distribuzione gaussiana}

\begin{definition}
    Data una probabilità $P$ su $(\RR, \BB(\RR))$, con funzione di ripartizione $F$ si
    definisce \textbf{quantile di ordine $\beta$} con $\beta \in (0, 1)$ il valore:
    \[ 
        r_\beta = \inf \{ x \in \RR \mid F(x) \geq \beta \}.
    \]
    In altre parole, $r_\beta$ è l'estremo inferiore dell'insieme degli $x$ tali per cui
    $P((-\infty, x)) \geq \beta$, ossia ``il primo valore'' per cui si supera la probabilità
    $\beta$. \smallskip


    Se $P$ si distribuisce come $N(0, 1)$, si denota $r_\beta$ come $q_\beta$.
\end{definition}

\begin{remark}
    Per simmetria della f.d.r.~$\Phi$, vale che $q_{1-\beta} = -q_\beta$. Inoltre vale che:
    \[
        P(-q_{1-\alpha/2} \leq Z \leq q_{1-\alpha/2}) = 1-\alpha,
    \]
    dove $Z \sim N(0, 1)$ e $\alpha \in (0, 1)$. \smallskip

    Queste due proprietà valgono in generale se la legge considerata ha densità pari, o ancora
    più generalmente se ha la stessa legge al suo opposto.
\end{remark}

\subsection{Intervalli di fiducia per la media in una popolazione normale}

Consideriamo il modello $(\RR, \BB(\RR), (Q_\theta)_{\theta \in \Theta})$ con $Q_\theta \sim N(m, \sigma^2)$, dove il
parametro da ricercare è la media $m$. Sia $(X_i)_{i \in [n]}$ un campione i.i.d.~con $X_i \sim N(m, \sigma^2)$.
Dal momento che $\overline{X}$ è uno stimatore
di $m$, un intervallo di fiducia per il livello $1-\alpha$ è intuitivamente della forma $D = [\overline{X} \pm d]$ con $d \in \RR$.
Dacché $\EE[\overline{X}] = m$ e $\Var(\overline{X}) = \sigma^2/n$, per riproducibilità delle variabili gaussiane si
ricava che $\overline{X} \sim N(m, \sigma^2/n)$, ovverosia $\frac{\sqrt{n}}{\sigma}(\overline{X} - m) \sim N(0, 1)$ per
standardizzazione. \smallskip

Pertanto vale che:
\[ P_m(m \in D) = P_m\left(\abs{\overline{X} - m} \leq d\right) = 2 \Phi\left(\frac{\sqrt{n}}{\sigma} d\right) - 1, \]
e dunque, ponendo $P_m(m \in D) = 1-\alpha$, si ottiene che:
\[
    d = \frac{\sigma}{\sqrt{n}} q_{1-\alpha/2}.
\]

\end{multicols*}
