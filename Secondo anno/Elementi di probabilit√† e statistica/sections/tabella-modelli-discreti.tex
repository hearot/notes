%--------------------------------------------------------------------
\begin{landscape}

\chapter*{Tabella e proprietà delle distribuzioni discrete}
\addcontentsline{toc}{chapter}{Tabella e proprietà delle distribuzioni discrete}
\label{tab:distr_discrete}

\vskip -0.45in

\begin{table}[htb]
\scalebox{0.85}{
    \begin{tabular}{|l|l|l|l|l|l|}
        \hline
        Nome distribuzione                                                                                             & Caso di utilizzo                                                                                                                                                                                                                                            & Parametri                                                                                                                                                     & Densità discreta                                                                                                                                 & Valore atteso e momenti                                                                                                                   & Varianza                                                                                                                 \\ \hline
        \begin{tabular}[c]{@{}l@{}}Distr. di Bernoulli\\ $X \sim B(\pp)$\end{tabular}                                  & \begin{tabular}[c]{@{}l@{}}Esperimento con esito\\ di successo ($1$) o\\ insuccesso ($0$).\end{tabular}                                                                                                                                                     & $\pp$ -- probabilità di successo.                                                                                                                             & $P(X=1) = \pp$, $P(X=0) = 1-\pp$                                                                                                                 & $\EE[X] = \EE[X^2] = \pp$                                                                                                                 & $\Var(X) = \pp(1-\pp)$                                                                                                   \\ \hline
        \begin{tabular}[c]{@{}l@{}}Distr. binomiale\\ $X \sim B(n, \pp)$\end{tabular}                                  & \begin{tabular}[c]{@{}l@{}}In una serie di $n$ esperimenti\\ col modello delle prove ripetute,\\ $X$ conta il numero di successi.\\ $X$ è in particolare somma di $n$\\ v.a.~i.i.d.~distribuite come $B(\pp)$.\end{tabular}                                 & \begin{tabular}[c]{@{}l@{}}$n$ -- numero di esperimenti\\ $\pp$ -- probabilità di successo\\ dell'$i$-esimo esperimento\end{tabular}                          & \begin{tabular}[c]{@{}l@{}}$P(X=k) = \binom{n}{k}{\pp}^k (1-\pp)^{n-k}$\\ per $0 \leq k \leq n$ e $0$ altrimenti.\end{tabular}                   & \begin{tabular}[c]{@{}l@{}}$\EE[X] = n \pp$\\ (è somma di $n$ Bernoulliane) \\ $\EE[X^2] = n \pp + n(n-1)\pp^2$ \end{tabular}             & \begin{tabular}[c]{@{}l@{}}$\Var(X) = n \pp(1-\pp)$\\ (è somma di $n$\\ Bernoulliane indipendenti)\end{tabular}          \\ \hline
        \begin{tabular}[c]{@{}l@{}}Distr. binomiale\\ negativa\\ $X \sim \BinNeg(h, \pp)$\end{tabular}                 & \begin{tabular}[c]{@{}l@{}}In una serie di infiniti esperimenti\\ col modello delle prove ripetute,\\ $X$ conta l'esperimento in cui si\\ ha l'$h$-esimo successo.\end{tabular}                                                                             & \begin{tabular}[c]{@{}l@{}}$h$ -- numero dei successi da misurare\\ $\pp \in (0, 1)$ -- probabilità di successo\\ dell'$i$-esimo esperimento\end{tabular}     & \begin{tabular}[c]{@{}l@{}}$P(X=k) = \binom{k-1}{h-1} \pp^h (1-\pp)^{k-h}$\\ laddove definibile e $0$ altrimenti.\end{tabular}                   & \begin{tabular}[c]{@{}l@{}}$\EE[X] = \frac{h}{\pp}$\\ (è somma di $h$ Geometriche) \\ $\EE[X^2] = \frac{h(1+h-\pp)}{\pp^2}$ \end{tabular} & \begin{tabular}[c]{@{}l@{}}$\Var(X) = \frac{h(1-\pp)}{\pp^2}$ \\ (è somma di $h$\\Geometriche indipendenti)\end{tabular} \\ \hline
        \begin{tabular}[c]{@{}l@{}}Distr. geometrica\\ $X \sim \Geom(\pp)$\end{tabular}                                & \begin{tabular}[c]{@{}l@{}}In una serie di infiniti esperimenti\\ col modello delle prove ripetute,\\ $X$ conta l'esperimento in cui si\\ ha il primo successo. È pari a\\ $\BinNeg(1, \pp)$\end{tabular}                                                   & \begin{tabular}[c]{@{}l@{}}$\pp \in (0, 1)$ -- probabilità di successo\\ all'$i$-esimo esperimento.\end{tabular}                                              & \begin{tabular}[c]{@{}l@{}}$P(X=k) = \pp (1-\pp)^{k-1}$ per\\ $k \geq 1$ e $0$ per $k=0$.\end{tabular}                                           & \begin{tabular}[c]{@{}l@{}}$\EE[X] = \frac{1}{\pp}$\\ $\EE[X^2] = \frac{2-\pp}{\pp^2}$\end{tabular}                                       & $\Var(X) = \frac{1-\pp}{\pp^2}$                                                                                          \\ \hline
        \begin{tabular}[c]{@{}l@{}}Distr. ipergeometrica\\ $X \sim H(N, N_1, n)$\end{tabular}                          & \begin{tabular}[c]{@{}l@{}}In un'estrazione di $n$ palline in\\ un'urna di $N$ palline, di cui\\ $N_1$ sono rosse, $X$ conta\\ il numero di palline rosse estratte.\end{tabular}                                                                            & \begin{tabular}[c]{@{}l@{}}$N$ -- numero di palline nell'urna\\ $N_1$ -- numero di palline rosse\\ nell'urna\\ $n$ -- numero di palline estratte\end{tabular} & \begin{tabular}[c]{@{}l@{}}$P(X=k) = \frac{\binom{N_1}{k} \binom{N-N_1}{n-k}}{\binom{N}{n}}$\\ laddove definibile e $0$ altrimenti.\end{tabular} &                                                                                                                                           &                                                                                                                          \\ \hline
        \begin{tabular}[c]{@{}l@{}}Distri.di Poisson\\ (o degli eventi rari)\\ $X \sim \Poisson(\lambda)$\end{tabular} & \begin{tabular}[c]{@{}l@{}}In una sequenza di $n \gg 1$\\ esperimenti di parametro $\pp \ll 1$\\ con $n \pp \approx \lambda$,\\ $X$ misura il numero di successi.\\ Si può studiare come distribuzione\\ limite della distribuzione binomiale.\end{tabular} & $\lambda$ -- tasso di successo.                                                                                                                               & $P(X=k) = \frac{\lambda^k}{k!} e^{-\lambda}$                                                                                                     & \begin{tabular}[c]{@{}l@{}}$\EE[X] = \lambda$\\ $\EE[X^2] = \lambda(\lambda + 1)$\end{tabular}                                            & $\Var(X) = \lambda$                                                                                                      \\ \hline
    \end{tabular}
}
\end{table}

Valgono inoltre le seguenti altre proprietà:

\small
\begin{itemize}
    \item Una somma di $n$ v.a.~i.i.d.~distribuite come $B(\pp)$ si
    distribuisce come $B(n, \pp)$.
    \item Se $X \sim B(n, \pp)$ e $Y \sim B(m, \pp)$ sono indipendenti, $X + Y$ si distribuisce come $B(n + m, \pp)$.
    \item Se $X \sim \Poisson(\lambda)$ e $Y \sim \Poisson(\mu)$ sono indipendenti,
    $X + Y$ si distribuisce come $\Poisson(\lambda + \mu)$.
    \item Se $X \sim \Geom(\pp)$, allora $P(X = \infty) = 0$\footnote{
        Ovverosia la probabilità che non vi siano mai successi è nulla.
    }. Da ciò si deduce che $P(X > k) = (1-\pp)^k$.
    \item Una $X \sim \BinNeg(h, \pp)$ è
    somma di $h$ v.a.~i.i.d.~distribuite
    come $\Geom(\pp)$.
    \item Una v.a.~$X$ sui numeri naturali si dice che ha la \textit{proprietà di perdita di memoria} se $P(X > n + k \mid X > k) = P(X > n)$. Una v.a.~ha
    la proprietà di perdita della memoria se e solo se è distribuita come
    la distribuzione geometrica.
\end{itemize}

\end{landscape}