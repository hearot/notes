\documentclass[10pt,landscape]{article}
\usepackage{amssymb,amsmath,amsthm,amsfonts}
\usepackage{multicol,multirow}
\usepackage{marvosym}
\usepackage{calc}
\usepackage{ifthen}
\usepackage[landscape]{geometry}
\usepackage[colorlinks=true,citecolor=blue,linkcolor=blue]{hyperref}
\usepackage{notes_2023}

\setlength{\extrarowheight}{0pt}

\ifthenelse{\lengthtest { \paperwidth = 11in}}
{ \geometry{top=.5in,left=.5in,right=.5in,bottom=.5in} }
{\ifthenelse{ \lengthtest{ \paperwidth = 297mm}}
	{\geometry{top=1cm,left=1cm,right=1cm,bottom=1cm} }
	{\geometry{top=1cm,left=1cm,right=1cm,bottom=1cm} }
}
%\pagestyle{empty}
\makeatletter
\renewcommand{\section}{\@startsection{section}{1}{0mm}%
	{-1ex plus -.5ex minus -.2ex}%
	{0.5ex plus .2ex}%x
	{\normalfont\large\bfseries}}
\renewcommand{\subsection}{\@startsection{subsection}{2}{0mm}%
	{-1explus -.5ex minus -.2ex}%
	{0.5ex plus .2ex}%
	{\normalfont\normalsize\bfseries}}
\renewcommand{\subsubsection}{\@startsection{subsubsection}{3}{0mm}%
	{-1ex plus -.5ex minus -.2ex}%
	{1ex plus .2ex}%
	{\normalfont\small\bfseries}}
\makeatother
\setcounter{secnumdepth}{0}
\setlength{\parindent}{0pt}
\setlength{\parskip}{0pt plus 0.5ex}
% -----------------------------------------------------------------------

\title{Scheda riassuntiva di Geometria 2}

\begin{document}
	
	\parskip=0.7ex
	
	\raggedright
	\footnotesize
	
	\begin{center}
		\Large{\textbf{Scheda riassuntiva di Geometria 2}} \\
	\end{center}
	\begin{multicols}{3}
		\setlength{\premulticols}{1pt}
		\setlength{\postmulticols}{1pt}
		\setlength{\multicolsep}{1pt}
		\setlength{\columnsep}{2pt}
		
		\section{Geometria proiettiva}
		
		\subsection{Spazi e trasformazioni proiettive}

		Sia $\KK$ un campo e sia $V$ uno spazio proiettivo. Sia $\sim$ la seguente
		relazione di equivalenza su $V \setminus \zerovecset$ tale per cui
		\[ \v \sim \w \defiff \exists \lambda \in \KK^* \mid \v = \lambda \w. \]
		Allora si definisce lo \textbf{spazio proiettivo} associata a $V$, denotato
		con $\PP(V)$, come:
		\[ \PP(V) = V \setminus \zerovecset \quot \sim. \]
		In particolare esiste una bigezione tra gli elementi dello spazio proiettivo
		e le rette di $V$ (i.e.~i sottospazi di $V$ con dimensione $1$). Si definisce
		la \textit{dimensione} di $\PP(V)$ come:
		\[ \dim \PP(V) := \dim V - 1. \]
		Gli spazi proiettivi di dimensione $1$ sono detti \textit{rette proiettive},
		mentre quelli di dimensione $2$ \textit{piani}. Si dice
		\textbf{spazio proiettivo standard di dimensione $n$} lo spazio proiettivo
		associato a $\KK^{n+1}$, e viene denotato come $\PP^n(\KK) := \PP(\KK^{n+1})$. \medskip
		
		
		Sia $W$ uno spazio vettoriale. Una mappa $f : \PP(V) \to \PP(W)$ si dice
		\textbf{trasformazione proiettiva} se è tale per cui esiste un'applicazione
		lineare $\varphi \in \Ll(V, W)$ che soddisfa la seguente identità:
		\[ f([\v]) = [\varphi(\w)], \]
		dove con $[\cdot]$ si denota la classe di equivalenza in $\PP(V)$.
		Una trasformazione proiettiva invertibile da $\PP(V)$ in $\PP(W)$
		si dice \textbf{isomorfismo proiettivo}. Una
		trasformazione proiettiva da $\PP(V)$ in $\PP(V)$ si dice
		\textbf{proiettività}.
		
		\begin{itemize}
			\item Se $f$ è una trasformazione proiettiva, allora $\varphi$ è necessariamente
				iniettiva (altrimenti l'identità non sussisterebbe, dacché $[\vec 0]$ non
				esiste -- la relazione d'equivalenza $\sim$ è infatti definita su $V \setminus
				\zerovecset$).
			\item Allo stesso tempo, un'applicazione lineare $\varphi$ iniettiva induce
				sempre una trasformazione proiettiva $f$,
			\item Se $f$ è una trasformazione proiettiva, allora $f$ è in particolare anche
				iniettiva (infatti $[\varphi(\v)] = [\varphi(\w)] \implies \exists \lambda \in \KK^* \mid \v = \lambda \w \implies \v \sim \w$),
			\item La composizione di due trasformazioni proiettive è ancora una
				trasformazione proiettiva ed è indotta dalla composizione delle app.
				lineari associate alle trasformazioni di partenza,
			\item L'identità $\Id$ è una proiettività di $\PP(V)$, ed è indotta
				dall'identità di $V$.
		\end{itemize}
		
		Poiché allora nelle proiettività di $V$ esiste un'identità, un inverso e vale
		l'associatività nella composizione, si definisce $\PPGL(V)$ come il gruppo delle
		proiettività di $V$ rispetto alla composizione.
		
		Sono inoltre equivalenti i seguenti fatti:
		
		\begin{enumerate}[(i)]
			\item $f$ è surgettiva,
			\item $f$ è bigettiva,
			\item $\dim \PP(V) = \dim \PP(W)$,
			\item $f$ è invertibile e $f\inv$ è una trasformazione proiettiva.
		\end{enumerate}
		
		In particolare $\varphi\inv$ induce esattamente $f\inv$.		
		\vfill
		\hrule
		~\\
		Ad opera di Gabriel Antonio Videtta, \url{https://poisson.phc.dm.unipi.it/~videtta/}.
		~\\Reperibile su
		\url{https://notes.hearot.it}, nella sezione \textit{Secondo anno $\to$ Geometria 2 $\to$ Scheda riassuntiva}.
	\end{multicols}
	
\end{document}
