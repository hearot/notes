\documentclass[10pt,landscape]{article}
\usepackage{amssymb,amsmath,amsthm,amsfonts}
\usepackage{multicol,multirow}
\usepackage{marvosym}
\usepackage{calc}
\usepackage{ifthen}
\usepackage[landscape]{geometry}
\usepackage[colorlinks=true,citecolor=blue,linkcolor=blue]{hyperref}
\usepackage{notes_2023}

\setlength{\extrarowheight}{0pt}

\ifthenelse{\lengthtest { \paperwidth = 11in}}
{ \geometry{top=.5in,left=.5in,right=.5in,bottom=.5in} }
{\ifthenelse{ \lengthtest{ \paperwidth = 297mm}}
	{\geometry{top=1cm,left=1cm,right=1cm,bottom=1cm} }
	{\geometry{top=1cm,left=1cm,right=1cm,bottom=1cm} }
}
%\pagestyle{empty}
\makeatletter
\renewcommand{\section}{\@startsection{section}{1}{0mm}%
	{-1ex plus -.5ex minus -.2ex}%
	{0.5ex plus .2ex}%x
	{\normalfont\large\bfseries}}
\renewcommand{\subsection}{\@startsection{subsection}{2}{0mm}%
	{-1explus -.5ex minus -.2ex}%
	{0.5ex plus .2ex}%
	{\normalfont\normalsize\bfseries}}
\renewcommand{\subsubsection}{\@startsection{subsubsection}{3}{0mm}%
	{-1ex plus -.5ex minus -.2ex}%
	{1ex plus .2ex}%
	{\normalfont\small\bfseries}}
\makeatother
\setcounter{secnumdepth}{0}
\setlength{\parindent}{0pt}
\setlength{\parskip}{0pt plus 0.5ex}
% -----------------------------------------------------------------------

\title{Scheda riassuntiva di Geometria 2}

\begin{document}
	
	\parskip=0.7ex
	
	\raggedright
	\footnotesize
	
	\begin{center}
		\Large{\textbf{Scheda riassuntiva di Geometria 2}} \\
	\end{center}
	\begin{multicols}{3}
		\setlength{\premulticols}{1pt}
		\setlength{\postmulticols}{1pt}
		\setlength{\multicolsep}{1pt}
		\setlength{\columnsep}{2pt}
		
		\section{Geometria proiettiva}
		
		\subsection{Spazi e trasformazioni proiettive}

		Sia $\KK$ un campo e sia $V$ uno spazio proiettivo. Sia $\sim$ la seguente
		relazione di equivalenza su $V \setminus \zerovecset$ tale per cui
		\[ \v \sim \w \defiff \exists \lambda \in \KK^* \mid \v = \lambda \w. \]
		Allora si definisce lo \textbf{spazio proiettivo} associata a $V$, denotato
		con $\PP(V)$, come:
		\[ \PP(V) = V \setminus \zerovecset \quot \sim. \]
		In particolare esiste una bigezione tra gli elementi dello spazio proiettivo
		e le rette di $V$ (i.e.~i sottospazi di $V$ con dimensione $1$). Si definisce
		la \textit{dimensione} di $\PP(V)$ come:
		\[ \dim \PP(V) := \dim V - 1. \]
		Gli spazi proiettivi di dimensione $1$ sono detti \textit{rette proiettive},
		mentre quelli di dimensione $2$ \textit{piani}. Si dice
		\textbf{spazio proiettivo standard di dimensione $n$} lo spazio proiettivo
		associato a $\KK^{n+1}$, e viene denotato come $\PP^n(\KK) := \PP(\KK^{n+1})$.
		Si indica con $\pi$ la proiezione al quoziente tramite $\sim$, ossia:
		\[ \pi(W) = \{ [\w] \mid \w \in W \}. \]
		
		
		Si dice \textbf{sottospazio proiettivo} un qualsiasi sottoinsieme $S$ di $\PP(V)$
		tale per cui esista un sottospazio vettoriale $W$ di $V$ tale per cui
		$S = \pi(W \setminus \zerovecset)$, e si scrive $S = \PP(W)$, con:
		\[ \dim S = \dim W - 1. \]
		In particolare, tramite $\pi$ si descrive una bigezione tra i sottospazi vettoriali di 
		$V$ e i sottospazi proiettivi di $\PP(V)$. \medskip
		
		
		L'intersezione di sottospazi proiettivi è ancora un sottospazio proiettivo ed
		è indotto dall'intersezione degli spazi vettoriali che generano i singoli
		sottospazi proiettivi. Pertanto, se $F \subseteq \PP(V)$, è ben definito
		il seguente sottospazio:
		\[ \displaystyle L(F) = \bigcap_{\substack{F \subseteq S_i \\ S_i \text{\;ssp. pr.}}} S_i, \]
		ossia l'intersezione di tutti i sottospazi proiettivi che contengono $F$.
		Si scrive $L(S_1, \ldots, S_n)$ per indicare $L(S_1 \cup \cdots \cup S_n)$.
		Se $S_1 = \PP(W_1)$, ..., $S_n = \PP(W_n)$, allora vale che:
		\[ L(S_1, \ldots, S_n) = \PP(W_1 + \ldots + W_n). \]
		
		
		Vale pertanto la \textbf{formula di Grassmann proiettiva}:
		\[ \dim L(S_1, S_2) = \dim S_1 + \dim S_2 - \dim (S_1 \cap S_2). \]
		Allora, se $\dim S_1 + \dim S_2 \geq \dim \PP(V)$ (si osservi che è
		$\geq$ e non $>$ come nel caso vettoriale, dacché un sottospazio di dimensione
		zero è comunque un punto in geometria proiettiva), vale necessariamente
		che:
		\[ S_1 \cap S_2 \neq \emptyset, \]
		infatti $\dim S_1 \cap S_2 = \dim S_1 + \dim S_2 - \dim L(S_1, S_2) \geq
		\dim S_1 + \dim S_2 - \dim \PP(V) \geq 0$. In particolare, in $\PP^2(\KK)$,
		questo implica che due rette proiettive distinte si incontrano sempre in un unico
		punto (infatti $1+1\geq2$).
		
		Sia $W$ uno spazio vettoriale. Una mappa $f : \PP(V) \to \PP(W)$ si dice
		\textbf{trasformazione proiettiva} se è tale per cui esiste un'applicazione
		lineare $\varphi \in \Ll(V, W)$ che soddisfa la seguente identità:
		\[ f([\v]) = [\varphi(\w)], \]
		dove con $[\cdot]$ si denota la classe di equivalenza in $\PP(V)$. Si scrive
		in questo caso che $[\varphi] = f$.
		Una trasformazione proiettiva invertibile da $\PP(V)$ in $\PP(W)$
		si dice \textbf{isomorfismo proiettivo}. Una
		trasformazione proiettiva da $\PP(V)$ in $\PP(V)$ si dice
		\textbf{proiettività}.
		
		\begin{itemize}
			\item Se $f$ è una trasformazione proiettiva, allora $\varphi$ è necessariamente
				iniettiva (altrimenti l'identità non sussisterebbe, dacché $[\vec 0]$ non
				esiste -- la relazione d'equivalenza $\sim$ è infatti definita su $V \setminus
				\zerovecset$).
			\item Allo stesso tempo, un'applicazione lineare $\varphi$ iniettiva induce
				sempre una trasformazione proiettiva $f$,
			\item Se $f$ è una trasformazione proiettiva, allora $f$ è in particolare anche
				iniettiva (infatti $[\varphi(\v)] = [\varphi(\w)] \implies \exists \lambda \in \KK^* \mid \v = \lambda \w \implies \v \sim \w$),
			\item La composizione di due trasformazioni proiettive è ancora una
				trasformazione proiettiva ed è indotta dalla composizione delle app.
				lineari associate alle trasformazioni di partenza,
			\item L'identità $\Id$ è una proiettività di $\PP(V)$, ed è indotta
				dall'identità di $V$.
		\end{itemize}
		
		Poiché allora nelle proiettività di $V$ esiste un'identità, un inverso e vale
		l'associatività nella composizione, si definisce $\PPGL(V)$ come il gruppo delle
		proiettività di $V$ rispetto alla composizione. In particolare si pone la
		seguente definizione
		\[ \PPGL_{n+1}(\KK) := \PPGL(\KK^{n+1}). \]
		
		Sono inoltre equivalenti i seguenti fatti:
		
		\begin{enumerate}[(i)]
			\item $f$ è surgettiva,
			\item $f$ è bigettiva,
			\item $\dim \PP(V) = \dim \PP(W)$,
			\item $f$ è invertibile e $f\inv$ è una trasformazione proiettiva.
		\end{enumerate}
		
		In particolare $\varphi\inv$ induce esattamente $f\inv$.
		
		\begin{itemize}
			\item I punti fissi di $f$ sono indotti esattamente dalle rette di autovettori
				di $\varphi$ (infatti $\varphi(\v) = \lambda \v \implies f([\v]) = [\v]$),
			\item In particolare, $f \in \PPGL(\PP^n(\RR))$ ammette sempre un punto
				fisso se $n$ è pari (il polinomio caratteristico di $\varphi$ ha grado
				dispari, e quindi ammette una radice in $\RR$),
			\item Se $\KK$ è algebricamente chiuso, $f$ ammette sempre un punto fisso
				(il polinomio caratteristico di $\varphi$ ha tutte le radici in $\KK$).
		\end{itemize}
		
		\subsection{Riferimenti proiettivi, teorema fondamentale della geometria proiettiva
			e coordinate omogenee}
		
		Più punti $P_1$, ..., $P_k$ si dicono \textbf{indipendenti} se e solo se
		i vettori delle loro classi di equivalenza sono tra di loro linearmente indipendenti.
		In particolare, $P_1$, ..., $P_k$ sono indipendenti se e solo se
		$\dim L(P_1, \ldots, P_k) = k-1$. Analogamente al caso vettoriale, se $\dim \PP(V) = n$,
		presi più di $n+1$ punti, questi sono sicuramente non indipendenti. \medskip
		
		
		
		Un insieme $\{P_1, \ldots, P_k\}$ si dice \textit{in posizione generale} se e solo se
		ogni suo sottoinsieme di $h \leq n+1$ punti è indipendente. Se $k \leq n+1$, un
		insieme è in posizione generale se e solo se è indipendente. Altrimenti, l'insieme
		è in posizione generale se ogni sottoinsieme di $n+1$ punti è indipendente. \medskip
		
		
		
		Si dice \textbf{riferimento proiettivo} una qualsiasi $(n+2)$-upla di punti
		$P_1$, ..., $P_{n+2}$ in posizione generale. In particolare, si dice che i punti
		$P_1$, ..., $P_{n+1}$ sono i \textbf{punti fondamentali} del riferimento, mentre
		$P_{n+2}$ è il \textbf{punto unità}. Una base $\basis = \{\vv 1, \ldots, \vv{n+1}\}$
		di $V$ si dice \textbf{base normalizzata} rispetto a $P_1$, ..., $P_{n+2}$ se:
		\[ P_i = [\vv i] \, \forall i \leq n+1 \qquad P_{n+2} = [\vv 1 + \ldots + \vv n]. \]
		
		Una base normalizzata per $R$ esiste sempre ed
		è unica a meno di \textit{riscalamento simultaneo}
		(ossia a meno di moltiplicare ogni vettore della base per uno stesso $\lambda \in \KK^*$). In particolare, se $P_i = [\vv i]$ con $i \leq n+1$ e
		$P_{n+2} = [\v]$, dacché $\{\vv 1, \ldots, \vv {n+1}\}$ è una base di $V$
		esistono $\alpha_i \in \KK$ per cui:
		\[ \v = \alpha_1 \vv 1 + \ldots + \alpha_{n+1} \vv{n+1}, \]
		con $\alpha_i \neq 0$ (altrimenti si avrebbero $n+1$ vettori linearmente
		dipendenti, contraddicendo la posizione generale). Allora
		$\{\alpha_1 \vv 1, \ldots, \alpha_{n+1} \vv {n+1}\}$ è una base normalizzata
		per il riferimento proiettivo. \medskip
		
		
		Sia d'ora in poi $R = \{P_1, \ldots, P_{n+2}\}$ un riferimento proiettivo e
		$\basis = \{\vv 1, \ldots, \vv{n+1}\}$ una base normalizzata rispetto ad $R$.
		Se $f = [\varphi]$, $g = [\psi]$ sono trasformazioni da $\PP(V)$ in $\PP(W)$, sono equivalenti i seguenti fatti:
		
		\begin{itemize}
			\item $\varphi = \lambda \psi$ per $\lambda \in \KK^*$,
			\item $f = g$,
			\item $f(P_i) = g(P_i)$ per $1 \leq i \leq n+2$.
		\end{itemize}
		
		Come conseguenza di questo fatto, vale che:
		\[ \PPGL(V) \cong GL(V) \quot N, \]
		dove $N = \{ \lambda \Id_V \mid \lambda \in \KK^* \}$ (è sufficiente
		considerare l'omomorfismo $\zeta : GL(V) \to \PPGL(V)$ tale per cui
		$f \xmapsto{\zeta} [f]$).
		
		Il \textbf{teorema fondamentale della geometria proiettiva}
		asserisce che se $R = \{P_1, \ldots, P_{n+2}\}$ e $R' = \{Q_1, \ldots, Q_{m+2}\}$ sono
		due riferimenti proiettivi di $V$ e $W$ e vale che $\dim \PP(W) \geq \dim \PP(V)$,
		allora, per ogni scelta di $n+2$ punti $Q_1'$, ..., $Q_{n+2}'$ da $R'$, esiste
		un'unica trasformazione proiettiva tale per cui:
		\[ f(P_i) = Q_i' \, \forall 1 \leq i \leq n+2. \]
		Se $n=m$, il teorema asserisce semplicemente che esiste un'unica trasformazione
		che mappa ordinatamente $R$ in $R'$. \medskip
		
		
		Si può costruire su $R$ un sistema di coordinate, dette \textbf{coordinate omogenee},
		per cui $P = [a_1, \ldots, a_n] = [a_1 : \cdots : a_n]$ se e solo se
		$P = [a_1 \vv 1 + \ldots + a_{n+1} \vv n]$ dove $\basis = \{\vv 1, \ldots, \vv{n+1}\}$
		è una base normalizzata associata a $R$. Per $\PP^n(\KK)$, si definisce il
		\textit{riferimento standard} come il riferimento dato da
		$[\e1]$, ..., $[\e{n+1}]$ e $[\e1 + \ldots + \e{n+1}]$. In tal caso vale
		la seguente identità:
		\[ [a_1, \ldots, a_n] = [(a_1, \ldots, a_n)]. \]
		Si osserva che $[0, \ldots, 0]$ non è mai associato a nessun punto e che due punti
		hanno le stesse coordinate in un riferimento proiettivo a meno di riscalamento
		di tutte le coordinate per uno stesso $\lambda \in \KK^*$.
		
		\vfill
		\hrule
		~\\
		Ad opera di Gabriel Antonio Videtta, \url{https://poisson.phc.dm.unipi.it/~videtta/}.
		~\\Reperibile su
		\url{https://notes.hearot.it}, nella sezione \textit{Secondo anno $\to$ Geometria 2 $\to$ Scheda riassuntiva}.
	\end{multicols}
	
\end{document}
