\documentclass[12pt]{scrartcl}
\usepackage{notes_2023}

\begin{document}
	\title{Il teorema di Cauchy}
	\maketitle
	
	\begin{note}
		Nel corso del documento per $(G, \cdot)$ si intenderà un qualsiasi gruppo.
	\end{note}
	
	Si dimostra in questo documento, per ben due volte, un inverso parziale
	del teorema di Lagrange, il celebre teorema di Cauchy. Tale teorema
	asserisce che se $p$ è un numero primo che divide l'ordine di $G$,
	allora esiste un elemento di $G$ di ordine $p$. \medskip
	
	
	Si mostra innanzitutto che il teorema vale per gruppi abeliani.
	
	\begin{theorem}[di Cauchy per gruppi abeliani]
		Sia $G$ un gruppo abeliano finito. Se un numero primo $p$ divide $\abs{G}$, allora
		esiste $g \in G$ tale per cui $o(g) = p$.
	\end{theorem}
	
	\begin{proof}
		Sia $\abs{G} = pn$ con $n \in \NN^+$. Si dimostra per induzione su $n$ la
		validità della tesi. Se $n = 1$, allora $G$ è ciclico, e quindi ammette un
		elemento di ordine $p$, completando il passo base. \medskip
		
		
		Sia allora $n > 1$ e si ipotizzi allora che tutti i gruppi tali che $\abs{G} = pk$ con $k < n$, $k \in \NN^+$ ammettano un elemento di ordine $p$. Sia $h \in G$, $h \neq e$ (questo $h$ sicuramente esiste, dal momento che $p > 1$). Se $p \mid o(h)$, allora
		$h^{\nicefrac{o(h)}{p}}$ è un elemento di $G$ di ordine $p$. Altrimenti,
		si consideri $H = \Cyc{h}$. \medskip
		
		
		Dal momento che $G$ è abeliano, $H$ è normale, e dunque si può considerare il gruppo quoziente $G \quot H$. Poiché $p \nmid o(h) = \abs H$ e $p$ divide $\abs G$,
		$p$ divide anche $\abs{G \quot H}$ per il teorema di Lagrange. Inoltre, poiché
		$o(h) > 1$ (infatti $h \neq e$), $\abs{G \quot H} < \abs{G}$. Per l'ipotesi
		induttiva, allora, esiste un elemento $tH$ di ordine $p$ in $G \quot H$. \medskip
		
		
		Si mostra adesso che $p \mid o(t)$. Si consideri la proiezione al quoziente $\pi : G \to G \quot H$ tale per cui:
		\[ g \xmapsto{\pi} gH. \]
		Allora $p = o(tH) \mid o(t)$, dal momento che $eH=\pi(t^{o(t)})=(tH)^{o(t)}$.
		Pertanto, come prima, $t^{\nicefrac{o(t)}{p}}$ è un elemento di ordine $p$,
		concludendo il passo induttivo.
	\end{proof} \bigskip
	
	Di seguito si dimostra il teorema di Cauchy in generale.
	
	\begin{theorem}[di Cauchy]
		Sia $G$ un gruppo finito. Se un numero primo $p$ divide $\abs{G}$, allora
		esiste $g \in G$ tale per cui $o(g) = p$.
	\end{theorem}
	
	\begin{proof}
		Sia $\abs{G} = pn$ con $n \in \NN^+$. Si dimostra la tesi per induzione.
		Se $n = 1$, $G$ è ciclico e dunque ammette un generatore di ordine $p$,
		completando il passo base. Sia ora $n > 1$ e si assuma che ogni gruppo di ordine $pk$ con
		$k < n$ ammetta un elemento di ordine $p$. \medskip
		
		
		Sia $\mathcal{R}$ è un insieme dei rappresentanti delle classi di coniugio
		di $G$. Se esiste $g \in \mathcal{R}$ tale per cui $p$ divida
		$\abs{Z_G(g)}$, allora esiste un elemento di ordine $p$ in
		$Z_G(g)$ per ipotesi induttiva (infatti $Z_G(g) \neq G$, altrimenti
		$g$ apparterrebbe al centro di $G$). Altrimenti si consideri la formula delle classi
		di coniugio:
		\[
			\abs G = \abs{Z(G)} + \sum_{g \in \mathcal{R} \setminus Z(G)} \frac{\abs G}{\abs{Z_G(g)}}.
		\]
		Poiché $p$ non divide $\abs{Z_G(g)}$ per ogni $g \in \mathcal{R} \setminus Z(G)$,
		$p$ divide ancora $\nicefrac{\abs{G}}{\abs{Z_G(g)}}$ (e quindi il secondo termine
		del secondo membro). Allora, prendendo
		l'identità modulo $p$, si deduce che:
		\[ \abs{Z(G)} \equiv 0 \pod p. \]		
		Poiché allora $p$ divide $\abs{Z(G)}$ e $Z(G)$ è un gruppo abeliano,
		il passo induttivo segue dal Teorema di Cauchy per gruppi abeliani,
		da cui la tesi.
	\end{proof} \smallskip
	
	
	Si mostra infine una dimostrazione alternativa del teorema di Cauchy (più immediata
	e facile da ricordare), basata su una particolare costruzione.
	
	\begin{proof}[Dimostrazione alternativa]
		Si\footnote{
			Riadattando opportunamente questa dimostrazione, si può fornire un'ulteriore dimostrazione del Teorema di Eulero di teoria dei numeri.
		} consideri l'insieme $S$, dove:
		\[ S = \{ (a_1, \ldots, a_p) \in G^p \mid a_1 \cdots a_p = e \}. \]
		Dimostrando che esiste un elemento $h \in G$ diverso dall'identità tale
		per cui $(h, \ldots, h) \in S$, si mostra che $h^p = e$, e dunque che
		$o(h) = p$ (infatti $h \neq e$), dimostrando la tesi. \medskip
		
		
		Si consideri l'azione $\varphi$
		di $\ZZ \quot p\ZZ$ su $S$ univocamente determinata\footnote{$\ZZ \quot p\ZZ$ è infatti generato da $1$.} dalla relazione:
		\[ 1 \xmapsto{\varphi} \left[ (a_1, a_2, \ldots, a_p) \mapsto (a_2, \ldots, a_p, a_1) \right]. \]
		In particolare $m \cdot (a_1, \ldots, a_p)$ restituisce una $p$-upla ottenuta
		``ciclando a sinistra'' la $p$-upla iniziale di $m$ posizioni. Si consideri la
		somma data dal teorema orbita-stabilizzatore:
		\[ \abs{S} = \sum_{x \in S} \frac{p}{\abs{\Stab(x)}} = 1 + N + \sum_{x \in S \setminus (\{(e,\ldots,e)\} \cup H)} \frac{p}{\abs{\Stab(x)}}, \]
		dove $H$ è l'insieme degli elementi $h \neq e$ tali per cui $h^p = e$ (ossia
		le $p$-uple con coordinate identiche tra loro) e $N = \abs H$.
		Poiché $\Stab(x) \leq \ZZ \quot p\ZZ$, gli unici ordini di $\Stab(x)$ possono
		essere $1$ e $p$. Se tuttavia, per $x \in S \setminus (\{(e,\ldots,e)\} \cup H)$,
		valesse $\Stab(x) = \ZZ \quot p\ZZ$, $x$ avrebbe coordinate tutte uguali,
		e quindi, per ipotesi, $x$ apparterrebbe ad $H$ o sarebbe l'identità, \Lightning. Quindi la somma del secondo membro vale esattamente $pk$, dove $k = \abs{S \setminus (\{(e,\ldots,e)\} \cup H)}$. \medskip
		
		
		Si osserva adesso che $\abs S = n^{p-1}$, dove $n = \abs G$. Infatti è sufficiente
		determinare le prime $p-1$ coordinate, per le quali vi sono $n$ scelte, per determinare
		anche l'ultima coordinata tramite la relazione $a_1 \cdots a_n = e$. Prendendo
		allora la precedente identità modulo $p$, si ottiene che\footnote{
			Questa dimostrazione fornisce quindi anche un risultato sul numero di elementi
			con ordine primo in $G$, ossia esso è congruo a $-1$ in modulo $p$.
		}:
		\[ N \equiv -1 \pod p, \]
		e quindi in particolare esiste almeno un elemento di ordine $p$ diverso dall'identità.
	\end{proof}
\end{document}