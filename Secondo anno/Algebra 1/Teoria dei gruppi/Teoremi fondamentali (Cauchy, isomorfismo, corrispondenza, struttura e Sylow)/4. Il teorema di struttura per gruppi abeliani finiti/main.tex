\documentclass[12pt]{scrartcl}
\usepackage{notes_2023}

\begin{document}
	\title{Il teorema di struttura per gruppi abeliani finiti}
	\maketitle
		
	\begin{note}
		Nel corso del documento con $G$ si indicherà un qualsiasi gruppo.
	\end{note}
	
	\begin{theorem}[di struttura per gruppi abeliani finiti]
		Sia $G$ un gruppo abeliano finito. Allora esistono unici
		$n_1$, \ldots, $n_s \in \NN$ tali per cui:
		\[ G \cong \ZZmod{n_1} \times \cdots \times \ZZmod{n_s}, \qquad n_s \mid n_{s-1} \mid \cdots \mid n_2 \mid n_1. \]
	\end{theorem}
	
	\begin{definition}[$p$-componente]
		Si definisce \textbf{$p$-componente} $G(p)$ (o $p$-torsione)
		di $G$ il sottogruppo di $G$ tale per cui:
		\[ G(p) = \{ x \in G \mid \ord(x) = p^k \text{ per qualche } k \}. \]
	\end{definition}
	
	\begin{remark}
		Si dimostra facilmente che $G(p)$ è un sottogruppo. Chiaramente
		$G(p) \subseteq G$; inoltre $e$ chiaramente appartiene
		a $G(p)$. Se poi $x$, $y \in G(p)$, allora
		$\ord(xy) \mid \mcm(\ord(x), \ord(y))$, e quindi
		$\ord(xy) = p^k$ per qualche $k$. Pertanto anche
		$xy \in G(p)$. Dal momento che $G(p)$ è finito,
		questo dimostra che $G(p)$ è un sottogruppo.
	\end{remark}
	
	\begin{remark}
		$G(p)$ è un sottogruppo caratteristico di $G$. Infatti
		$\varphi \in \Aut(G)$ lascia invariato l'ordine di
		un elemento di $G(p)$, e quindi $\varphi(G(p)) = G(p)$.
	\end{remark}
	
	\begin{remark}
		La dimostrazione del teorema di struttura segue
		il seguente schema:
		
		\begin{itemize}
			\item Se $G$ è abeliano con $\abs{G} = p_1^{e_1} \cdots p_r^{e_r}$, allora $G \cong G(p_1) \times \cdots \times G(p_r)$, ossia $G$ è isomorfo al prodotto diretto tra
			le sue $p$-componenti. Tale decomposizione di $G$
			come prodotto di $p$-gruppi di ordini tra loro
			coprimi è unica.
			\item Se $G$ è un $p$-gruppo abeliano. Allora esistono
			e sono univocamente determinati degli interi
			positivi $r_1 \geq \cdots \geq r_s$ tali che
			$G \cong \ZZmod{p_1^{r_1}} \times \cdots
			\times \ZZmod{p_1^{r_s}}$.
		\end{itemize}
	\end{remark}
	
	\begin{proof}[Dimostrazione a priori]
		Per il primo teorema, $G$ si può decomporre nelle sue
		$p$-componenti:
		\[ G \cong G(p_1) \times \cdots \times G(p_s). \]
		
		
		Allora, per il secondo teorema, ogni $G(p_i)$ può
		decomporsi come prodotto diretto di $\ZZmod{p_i^k}$,
		e quindi:
		\[ G \cong (\ZZmod{p_1}^{e_{1,1}} \times \cdots \times \ZZmod{p_1}^{e_{1,t_1}}) \times \cdots \times (\ZZmod{p_r}^{e_{r,1}} \times \cdots \times \ZZmod{p_1}^{e_{r,t_r}}). \]
		
		
		Sia $t = \max\{t_1, \ldots, t_r\}$. Posso allungare le
		fattorizzazioni di $G(p_i)$ fino ad ottenere $t$ fattori aggiungendo eventualmente dei gruppi banali nella
		fattorizzazione. \medskip
		
		
		Applicando allora il Teorema cinese del resto si ottiene
		l'esistenza della fattorizzazione secondo il Teorema
		di struttura per gruppi abeliani finiti. 
		
		
		L'unicità segue dal primo teorema riapplicando il Teorema
		cinese del resto al contrario. %TODO: estendi
	\end{proof}
	
	%TODO: esempio su Z_26 x Z_169 x Z_8 x Z_12
	
	Si dimostrano i due teoremi:
	
	\begin{theorem}
		Se $G$ è abeliano con $\abs{G} = p_1^{e_1} \cdots p_r^{e_r}$, allora $G \cong G(p_1) \times \cdots \times G(p_r)$, ossia $G$ è isomorfo al prodotto diretto tra
		le sue $p$-componenti. Tale decomposizione di $G$
		come prodotto di $p$-gruppi di ordini tra loro
		coprimi è unica.
	\end{theorem}
	
	\begin{proof}
		Si dimostra per induzione sul numero $s$ di primi distinti
		nella fattorizzazione di $\abs{G}$. Se $s=1$,
		$G$ coincide con l'unica sua $p$-componente. Sia
		ora $s \geq 2$. Se $\abs{G} = m m'$ con $m'>1$ e
		$\MCD(m, m') = 1$. Mostro che $G \cong mG \times m'G$. \medskip
		
		
		%TODO: continuare con Bezout
	\end{proof}
	
	\begin{theorem}
		Se $G$ è un $p$-gruppo abeliano. Allora esistono
		e sono univocamente determinati degli interi
		positivi $r_1 \geq \cdots \geq r_s$ tali che
		$G \cong \ZZmod{p_1^{r_1}} \times \cdots
		\times \ZZmod{p_1^{r_s}}$.
	\end{theorem}
\end{document}