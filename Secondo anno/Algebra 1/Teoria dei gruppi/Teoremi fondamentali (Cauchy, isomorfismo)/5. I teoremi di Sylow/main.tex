\documentclass[12pt]{scrartcl}
\usepackage{notes_2023}

\begin{document}
	\title{I teoremi di Sylow}
	\maketitle
	
	\begin{note}
		Nel corso del documento con $p$ si indicherà un numero
		primo, con $G$ si indicherà un qualsiasi gruppo finito di ordine $p^n m$ tale per cui $\MCD(p, m) = 1$
		(ossia $n$ è la valutazione $p$-adica di $\abs{G}$).
	\end{note}

	I teoremi di Sylow rappresentano, insieme al teorema di
	struttura per gruppi abeliani finiti, lo strumento più
	importante e applicabile dell'algebra elementare. Attraverso
	questi teoremi, lo studio e la classificazione dei gruppi
	finiti viene enormemente facilitata e ridotta ai suoi
	$p$-sottogruppi. \medskip
	
	
	Prima di illustrare gli enunciati e le dimostrazioni di questi
	teoremi, si definisce preliminarmente cos'è un $p$-sottogruppo
	di Sylow, detto poi semplicemente $p$-Sylow:
	
	\begin{definition}[$p$-Sylow]
		Sia $H \leq G$. Si dice che $H$ è un \textbf{$p$-Sylow}
		di $G$ se $\abs{H} = p^n$, ossia se $H$ è un $p$-sottogruppo
		di $H$ con valutazione $p$-adica massima.
	\end{definition}
	
	Si illustra adesso il Primo teorema di Sylow, che riguarda
	l'esistenza di $p$-sottogruppi di tutte le cardinalità
	possibili\footnote{
		A dire la verità il Primo teorema di Sylow si deduce
		anche solo mostrando l'esistenza di un $p$-Sylow. Infatti,
		per una proposizione nota sui $p$-gruppi, che discende
		direttamente dal Teorema di corrispondenza, in un
		$p$-gruppo esiste sempre
		una catena di $p$-sottogruppi normali che comprende
		$p$-sottogruppi di tutte le cardinalità. Dal momento
		però che la dimostrazione è molto istruttiva (e anche
		molto generale), si è preferito lasciare la generalizzazione.
	}\footnote{
		Si osserva che il Primo teorema di Sylow generalizza il
		Teorema di Cauchy alla sua massima estensione.
	} in $G$:
	
	\begin{theorem}[Primo teorema di Sylow, esistenza]
		Per ogni $i \in \NN$ tale per cui $0 \leq i \leq n$, esiste
		un sottogruppo $H \leq G$ tale per cui $\abs{H} = p^i$.
	\end{theorem}

	\begin{proof}
		Si consideri il sottoinsieme $\MM$
		di $\pset(G)$ dato da:
		\[ \MM = \{ X \subseteq G \mid \abs{X} = p^i \}. \]


		Allora vale che:
		\[ \abs{\MM} = \binom{p^n m}{p^i} = \frac{(p^n m)!}{(p^i)! (p^n m - p^i)!} = \frac{p^n m (p^n m - 1) \cdots (p^n m - p^i + 1)}{p^i (p^i - 1) \cdots 1}, \]
		ossia, equivalentemente, che:
		\[ \abs{\MM} = p^{n-i} m \prod_{j=1}^{p^i - 1} \frac{p^n m - j}{p^i - j}. \]


		Si osserva che $p^{n-i} \exactdiv \abs{M}$. Infatti,
		$p \nmid m$ perché $\MCD(p, m) = 1$ per ipotesi; inoltre,
		considerando il termine generico $a_j$ della produttoria,
		vale che\footnote{
			Infatti $j$ può valere al più $p^i - 1$.
		} $\nu_p(p^n m - j) = \nu_p(j) = \nu_p(p^i - j)$,
		e quindi che $\nu_p(a_j) = 0$. \medskip
		
		
		Dal momento che, dato $X \in \MM$, $g X$ appartiene ancora
		ad $\MM$ e $g X = h X \iff g = h$, $\forall$ $g$, $h \in G$,
		si può considerare l'azione $\phi : G \to S(\MM)$
		tale per cui $g \xmapsto{\phi} [X \mapsto gX]$.
		Dacché le orbite forniscono una partizione di $\MM$,
		vale che:
		\[ \abs{\MM} = \sum_{X \in \rotations} \frac{\abs{G}}{\abs{\Stab(X)}}, \]
		dove $\rotations$ è un insieme di rappresentanti delle
		orbite e dove si è applicato il Teorema orbita-stabilizzatore.
		Dal momento che $p^{n-i} \exactdiv \abs{M}$, esiste
		sicuramente un $X \in \rotations$ tale per cui
		$p^{n-i+1} \nmid \abs{\Orb(X)}$, da cui si deduce che
		$p^i \mid \abs{\Stab(X)}$. \medskip
		
		
		Sia $x \in X$ e si consideri ora la mappa $\tau : \Stab(X) \to X$ tale per cui $g \xmapsto{\tau} gx$. Tale mappa è
		sicuramente iniettiva (infatti $gx = hx \implies g = h$),
		e quindi $\abs{\Stab(X)} \leq \abs{X} = p^i$. Si deduce
		dunque che $\abs{\Stab(X)} = p^i$, da cui la tesi.
	\end{proof} \bigskip
	

	Si dimostra adesso il Secondo teorema di Sylow, che mostra
	che i $p$-Sylow sono tra loro coniugati e che dimostra l'esistenza
	di un'inclusione più generale tra i $p$-sottogruppi con
	i $p$-sottogruppi di cardinalità maggiore. Da questo
	teorema discenderà in particolare uno dei due risultati
	del Terzo teorema di Sylow sul numero di $p$-Sylow di
	un gruppo $G$.

	\begin{theorem}[Secondo teorema di Sylow, coniugio e inclusione]
		Tutti i $p$-Sylow di $G$ sono coniugati (e quindi isomorfi)
		tra loro. Inoltre, ogni $p$-sottogruppo di ordine
		$p^i$, se $i \neq n$, è contenuto
		in un $p$-sottogruppo di ordine $p^{i+1}$ (in particolare
		questi sottogruppi sono sottogruppi di un $p$-Sylow)\footnote{
			Il Secondo teorema di Sylow implica in particolare
			che se $H$ è un $p$-sottogruppo di ordine $p^i$,
			esiste sempre un $p$-sottogruppo $K$ di $G$ di
			ordine $p^j$ con $j \geq i$ tale per cui
			$H \leq K$.
		}.
	\end{theorem}
	
	\begin{proof}
		Sia\footnote{
			Tale $S$ esiste per il Primo teorema di Sylow.
		} $S$ un $p$-Sylow di $G$. Sia $H$ un $p$-sottogruppo
		di ordine $p^i$ e si consideri l'azione
		$\varphi : H \to S(X)$ su $X = G \quot S$ tale per
		cui $h \xmapsto{\varphi} [gS \mapsto hgS]$. Dal momento
		che $\abs{X} = [G : S] = m$, per
		il Teorema orbita-stabilizzatore vale allora che:
		\[ m = \sum_{gS \in \rotations} \frac{p^i}{\abs{\Stab(gS)}}, \]
		dove $\rotations$ è un insieme di rappresentanti delle
		orbite tramite $\varphi$. \medskip
		
		
		Dal momento che $p \nmid m$ per ipotesi, deve esistere
		$gS \in \rotations$ tale per cui $\abs{\Stab(gS)} = p^i$,
		da cui si deduce che $\Stab(gS) = H$. Pertanto vale che
		$hgS = gS$ $\forall h \in H$, e quindi
		$hg \in gS$, da cui si ricava infine che $h \in gSg\inv$.
		Allora $H \subseteq gSg\inv$. Se allora $H$ è un $p$-Sylow,
		$H = gSg\inv$ per cardinalità, e quindi tutti i
		$p$-Sylow sono coniugati tra loro, dimostrando la prima
		parte dell'enunciato. \medskip
		
		
		Sia ora $i \neq n$. Allora $H$ è un $p$-sottogruppo proprio
		di $P = gSg\inv$, che è un $p$-Sylow di $G$. Allora
		vale che $H \lneq N_P(H)$ dal momento che $P$ è un $p$-gruppo.
		Dacché $H \nsg N_P(H)$, $N_P(H) \quot H$ è un $p$-gruppo
		non banale. Allora, per il Teorema di Cauchy, esiste
		$x \in N_P(H)$ tale per cui $\ord(xH) = p$. Allora
		$\pi_H\inv(\gen{xH})$ è un sottogruppo di $N_P(H)$ di ordine
		$p \cdot p^i = p^{i+1}$ che contiene $H$, da cui
		la tesi.
	\end{proof} \bigskip
	
	
	Si dimostra infine il Terzo teorema di Sylow, che riguarda
	il numero di $p$-Sylow in $G$, indicato con $n_p$. Questo
	teorema, al di là del lato meramente computazionale, risulta
	spesso utile quando si cerca di dimostrare che un $p$-Sylow
	è caratteristico. Infatti è sufficiente verificare che
	$n_p$ sia esattamente $1$; in questo modo esiste un solo
	$p$-Sylow, e tale $p$-Sylow deve essere caratteristico, e
	quindi normale.
	
	\begin{theorem}[Terzo teorema di Sylow, numero]
		Sia $n_p$ il numero di $p$-Sylow di $G$. Allora vale
		che:
		
		\begin{itemize}
			\item $n_p = [G : N_G(S_p)]$, e dunque $n_p$ divide $\abs{G}$, dove
				$S_p$ è un $p$-Sylow,
			\item $n_p \equiv 1 \pod p$, e quindi\footnote{
				Poiché $n_p \mid \abs{G} = p^n m$, ma $n_p \equiv 1 \pod p$, $n_p$ è coprimo con $p^n$, e quindi
				$n_p$ deve dividere $m$.
			} $n_p \mid m$.
		\end{itemize}
	\end{theorem}
	
	\begin{proof}
		Poiché i coniugati di un $p$-Sylow $S$ hanno la stessa cardinalità di $S$, tali coniugati sono ancora
		$p$-Sylow. Similmente, per il Secondo teorema di Sylow, tutti
		i $p$-Sylow sono a loro volta coniugati di $S$. Pertanto,
		se $X$ è l'insieme dei $p$-Sylow di $G$, vale che
		$X$ è esattamente l'insieme dei coniugati di $S$. Allora,
		per il Teorema orbita-stabilizzatore, vale che:
		\[ n_p = \abs{X} = [G : N_G(S)], \]
		che chiaramente divide $\abs{G}$. \medskip
		
		
		Sia $\varphi : S \to S(X)$ l'azione su $X$ tale per cui
		$s \xmapsto{\varphi} [H \mapsto sHs\inv]$. Si mostra
		che $\Orb(S) = \{S\}$ è l'unica orbita banale. Se
		$\Orb(H)$ fosse banale, varebbe $sHs\inv = H$ $\forall s \in S$,
		e quindi varebbe $S \leq N_G(H)$. In tal caso esisterebbe
		il sottogruppo $HS$, che ha cardinalità:
		\[ \abs{HS} = \frac{p^n p^n}{p^i} = p^{2n-i}, \]
		dove $p^i$ è la cardinalità di $H \cap S$. Poiché
		$n$ è il massimo esponente di un $p$-sottogruppo di
		$G$, deve valere $2n-i \leq n \implies n \leq i$. Allo
		stesso tempo, anche il massimo esponente di $p$ in
		$H \cap S$, in quanto $p$-sottogruppo, deve essere
		minore o uguale a $n$, e quindi $n = i$. Pertanto
		$H = S$. \medskip
		
		
		Allora, se $\rotations$ è un insieme di rappresentanti
		delle orbite di $X$ tramite $\varphi$, vale che:
		\[ n_p = \abs{X} = \sum_{H \in \rotations} \frac{p^n}{\abs{\Stab(H)}} = 1 + \sum_{H \in \rotations \setminus \{S\}} \frac{p^n}{\abs{\Stab(H)}}. \]
		Poiché $p$ divide la somma del membro a destra (infatti
		le orbite sono non banali, e quindi $\abs{\Stab(H)} \neq p^n$),
		deve dunque valere $n_p \equiv 1 \pod 1$, da cui la tesi.
	\end{proof}
\end{document}