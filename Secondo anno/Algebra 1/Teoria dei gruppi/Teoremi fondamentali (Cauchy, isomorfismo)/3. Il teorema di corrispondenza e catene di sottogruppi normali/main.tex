\documentclass[12pt]{scrartcl}
\usepackage{notes_2023}

\begin{document}
	\title{Il teorema di corrispondenza e catene di sottogruppi normali}
	\maketitle

	Si illustra adesso un teorema che mette in corrispondenza
	i sottogruppi di $G \quot H$ con i sottogruppi di $G$
	che contengono $H$. Benché questo teorema possa sembrare
	a prima vista di poca utilità, in realtà svela alcune
	proprietà che hanno portato allo sviluppo della celebre
	teoria di Galois. Non solo, guardando anche nelle piccole
	applicazioni, il teorema di corrispondenza permette di
	contare molto facilmente i sottogruppi di $G \quot H$,
	nonché di dimostrare l'esistenza di una catena di
	$p$-sottogruppi normali contenente tutti gli ordini
	possibili per un $p$-gruppo.
	
	\begin{theorem}[di corrispondenza]
		Sia $H$ un sottogruppo normale di $G$. Allora
		la proiezione al quoziente $\pi_H : G \to G \quot H$
		induce una bigezione tra l'insieme
		\[ X = \{ K \leq G \mid H \subseteq K \} \]
		dei sottogruppi di $G$ che contengono $H$ e l'insieme
		\[ Y = \{ K' \leq G \quot H \} \]
		dei sottogruppi di $G \quot H$. Tale bigezione preserva
		la normalità di un gruppo e il suo indice, ossia:
		\begin{itemize}
			\item $K \nsgeq G \iff K' \nsgeq G \quot H$,
			\item $\left[ G : K \right] = \left[ G \quot H : K' \right]$,
		\end{itemize}
		dove $K \in X$ e $K' \in Y$ sono in corrispondenza biunivoca
		mediante $\pi_H$.
	\end{theorem}

	\begin{proof}
		Sia $\alpha : X \to Y$ definita nel seguente modo:
		\[ K \xmapsto{\alpha} \pi_H(K), \]
		dove si osserva che $\pi_H(K) = \{ kH \mid k \in K \} = K \quot H \leq G \quot H$.
		Si definisce analogamente $\beta : Y \to X$ in modo tale che:
		\[ K' \xmapsto{\beta} \pi_H\inv(K'). \]
		Le due mappe sono entrambe ben definite (infatti $\pi_H\inv(K')$ è sempre un sottogruppo di $G$ e contiene
		sempre $H$, dacché $H \in K'$, essendo l'identità di $G \quot H$).
		È dunque sufficiente mostrare che vale $\beta \circ \alpha = \Id_X$ e che $\alpha \circ \beta = \Id_Y$. \bigskip
		
		
		Siano quindi $K \in X$ e $K' \in Y$. Chiaramente
		$\pi_H(\pi_H\inv(K')) = K'$, dal momento che $\pi_H$ è
		surgettiva; dunque $\alpha \circ \beta = \Id_Y$. Inoltre
		$\pi_H \inv (\pi_H(K)) = \pi_H\inv (K \quot H) = \{ g \in G \mid gH \in K \quot H \} = K$\footnote{
			Infatti se $gH=kH$ con $k \in K$, esiste un $h \in H$ tale per cui $g=kh$.
			Dal momento che $H \subseteq K$, $g$ è dunque un elemento di
		$K$.}, da cui $\beta \circ \alpha = \Id_X$. Quindi $X$ e $Y$ sono in corrispondenza biunivoca
		tramite $\alpha$ e $\beta$. \bigskip
		
		
		Rimane da dimostrare che $\alpha$ e $\beta$ preservano
		la normalità e l'indice di sottogruppo. Se $K \nsgeq G$,
		allora chiaramente $K' = K \quot H \nsgeq G \quot H$
		(infatti $gH \, kH \, g\inv H = (gkg\inv) H$, dove
		$gkg\inv \in K$ per ipotesi di normalità). Sia
		ora $K' \nsgeq G \quot H$. Allora, se $k \in K$,
		$gH \, kH \, g\inv H = (gkg\inv) H$, e per ipotesi
		di normalità deve esistere $k' \in K$ tale per cui
		$(gkg\inv) H = k'H$, e quindi deve esistere
		$h \in H$ tale per cui $gkg\inv = k'h$. Dal momento
		che $H \subseteq K$, $gkg\inv \in K$, e quindi
		$K \nsgeq G$. \bigskip
		
		
		Per mostrare che l'indice di sottogruppo si preserva
		si dimostra che esiste lo stesso numero di classi
		laterali in $G \quot K$ e $(G \quot H) \quot (K \quot H)$.
		Pertanto è sufficiente mostrare che:
		\[
			xK = yK \iff xH (K \quot H) = yH (K \quot H), \qquad x, y \in G.
		\]
		Infatti, in tal caso vi sarebbero esattamente $[G : K]$
		classi laterali in $(G \quot H) \quot (K \quot H)$.
		Si consideri ora la classe laterale $xH(K \quot H)$:
		\[
			xH(K \quot H) = \{ xHkH \mid k \in K \} = \{ (xk) H \mid k \in K \},
		\]
		dove nell'ultima uguaglianza si è impiegata la normalità
		di $H$ in $G$ (altrimenti il prodotto non sarebbe ben
		definito).
		Analogamente $yH(K \quot H) = \{ (yk)H \mid k \in K \}$. 
		Quindi, se $xH (K \quot H) = yH (K \quot H)$, allora $xH = (yk)H$, con $k \in K$.
		Allora $x = ykh$ con $h \in H$. Poiché $H \subseteq K$, si deduce
		quindi che $xK = yK$. Infine, se $xK = yK$, esiste
		$k \in K$ tale per cui $x=yk$. Allora:
		\[ xH(K \quot H) = yH \, kH (K \quot H) = yH(K \quot H), \]
		da cui la tesi.
	\end{proof}
	
	
	\begin{example}[I sottogruppi di $\ZZmod n$]
		Attraverso il teorema di corrispondenza è facile
		contare i sottogruppi di $\ZZmod n$ senza ricorrere
		alla teoria sui gruppi ciclici finiti. Infatti, per
		il teorema di corrispondenza, i sottogruppi di
		$\ZZ \quot n \ZZ$ sono in esatta corrispondenza
		con i sottogruppi\footnote{
			Poiché $\ZZ$ è ciclico, ogni sottogruppo è
			della forma $m \ZZ$. In particolare, ogni sottogruppo
			di $\ZZ$ è anche un suo ideale, se si intende
			$\ZZ$ come anello, ed è dunque monogenerato in
			quanto $\ZZ$, essendo un anello euclideo, è
			anche un PID.
		} $m\ZZ$ di $\ZZ$ tali per cui $n \ZZ \subseteq m \ZZ$.
		In particolare, $n \ZZ \subseteq m \ZZ$ se e solo se
		$m \mid n$, e quindi vi sono $d(n)$ possibili sottogruppi, che, tramite il teorema di corrispondenza, sono esattamente
		i sottogruppi della forma $m\ZZ \quot n\ZZ \cong \ZZ \quot{\frac{n}{m}} \ZZ$.
	\end{example}
	\bigskip
	
	
	Si illustra allora il seguente fondamentale risultato sui $p$-gruppi,
	che è conseguenza del Teorema di corrispondenza e delle proprietà degli
	ordini di gruppi abeliani.
	
	\begin{proposition}
		Sia $G$ un $p$-gruppo di ordine $p^n$, con $n \in \NN^+$. Allora
		esiste una successione $H_1$, ..., $H_{n-1}$ di sottogruppi normali in $G$
		tali per cui:
		\[ \{ e \} < H_1 < H_2 < \cdots < H_{n-1} < G, \qquad \abs{H_i} = p^i. \]
	\end{proposition}
	
	\begin{proof}
		Si dimostra la tesi per induzione su $n$. Per $n=1$, la tesi è banale. Si
		ipotizzi allora che la tesi valga per $t<n$ con $t \in \NN^+$. Se
		$G$ è abeliano, allora $G$ ammette un sottogruppo $H_{n-1}$ di ordine
		$p^{n-1}$. Tale sottogruppo $H_{n-1}$ ammette per ipotesi induttiva
		una successione $H_1$, ..., $H_{n-2}$ di sottogruppi normali in
		$H_{n-1}$ come desiderato dalla tesi.
		Poiché $G$ è abeliano, tali sottogruppi sono
		normali anche in $G$, e quindi:
		\[ \{ e \} < H_1 < \cdots < H_{n-1} < G. \] \smallskip
		
		
		Sia adesso $G$ non abeliano. Allora $\abs{Z(G)} < \abs{G}$. Si può
		dunque considerare il gruppo quoziente $G \quot Z(G)$, di ordine
		strettamente inferiore a $p^n$. Per ipotesi induttiva
		esiste una catena di sottogruppi $\mathcal{H}_1$, ...,
		$\mathcal{H}_{k-1}$ normali in $G \quot Z(G)$ tale per cui:
		\[ \{e\} < \mathcal{H}_1 < \cdots < \mathcal{H}_{k-1} < G/Z(G), \qquad \abs{\mathcal{H}_i} = p^i, \]
		dove $\abs{Z(G)} = p^{n-k}$. \medskip
		
		
		Per il Teorema di corrispondenza, $\mathcal{H}_i$ corrisponde
		a un sottogruppo normale $H_{n-k+i}$ di $G$ contenente $Z(G)$ tale per cui
		$[G : H_{n-k+i}] = [G \quot Z(G) : \mathcal{H}_i]$. Allora vale che:
		\[ H_{n-k+i} = \abs{Z(G)} \abs{\mathcal{H}_i} = p^{n-k} p^i = p^{n-k+i}, \]
		e quindi $H_{n-k+i}$ copre tutti gli esponenti di $p$ da $n-k+1$ a $n-1$.
		Inoltre, tramite $\pi_{Z(G)} \inv$, vale anche che $H_{n-k+i} < H_{n-k+i+1}$\footnote{Segue dal fatto
		secondo cui $T \quot H < S \quot H \implies T < S$.}. Sempre per ipotesi
		induttiva (come nella costruzione di prima), $Z(G)$ ammette una catena
		di sottogruppi $H_1$, ..., $H_{n-k-1}$ normali in $Z(G)$ tale per cui:
		\[
			\{e\} < H_1 < \cdots < H_{n-k-1} < Z(G), \qquad \abs{H_i} = p^i.
		\]
		Poiché $Z(G)$ è il centro di $G$, tali sottogruppi sono normali anche in $G$.
		Ponendo allora $H_{n-k} := Z(G)$ si è costruita la catena desiderata:
		\[ \{e\} < H_1 < \cdots < H_{n-k} = Z(G) < H_{n-k+1} < \cdots < H_{n-1} < G. \]
	\end{proof}
\end{document}