\documentclass[12pt]{scrartcl}
\usepackage{notes_2023}

\begin{document}
	\title{Azione di coniugio e $p$-gruppi}
	\maketitle
	
	\begin{note}
		Nel corso del documento per $(G, \cdot)$ si intenderà un qualsiasi gruppo.
	\end{note}
	
	Si consideri l'omomorfismo $\zeta$ che associa ad ogni $g \in G$ l'automorfismo interno
	che induce. Questo omomorfismo induce la cosiddetta:
	
	\begin{definition}[azione di coniugio]
		Si definisce \textbf{azione di coniugio} l'azione di $G$ su sé stesso indotta da $\zeta : G \to \Aut(G)$ dove:
		\[ g \xmapsto{\zeta} \varphi_g = \left[ h \mapsto g h g\inv \right]. \]
	\end{definition}
	
	L'orbita di un elemento $g \in G$ prende in questo particolare caso il nome
	di \textbf{classe di coniugio} (e si indica come $\Cl(g)$), mentre il suo stabilizzatore viene detto \textbf{centralizzatore} (indicato con $Z_G(g)$). Si verifica facilmente
	che $Z_G(g)$ è composto da tutti gli elementi $h \in G$ che commutano con $g$, ossia
	tali che $gh = hg$. Allora vale in particolare che:
	\[ Z(G) = \Ker \zeta = \bigcap_{g \in G} Z_G(g). \] \medskip
	
	
	Si osserva inoltre che se $g \in Z(G)$, allora $\Cl(g) = \{g\}$ (infatti, per $h \in G$, si avrebbe $h g h\inv = h h\inv g = g$). Si può dunque riscrivere la somma data dal
	Teorema orbita-stabilizzatore nel seguente modo:
	\[ \abs{G} = \sum_{g \in \mathcal{R}} \frac{\abs{G}}{\abs{Z_G(g)}} = \sum_{g \in Z(G)} \underbrace{\abs{\Cl(g)}}_{=1} + \sum_{g \in \mathcal{R} \setminus Z(G)} \frac{\abs{G}}{\abs{Z_G(g)}} = (*), \]
	che riscritta ancora si risolve nella \textbf{formula delle classi di coniugio}:
	\[ (*) = \abs{Z(G)} + \sum_{g \in \mathcal{R} \setminus Z(G)} \frac{\abs{G}}{\abs{Z_G(g)}}, \]
	dove $\mathcal{R}$ è un insieme di rappresentanti delle orbite dell'azione di coniugio
	(si osserva che ogni elemento di $Z(G)$ è un rappresentante dacché l'orbita di un
	elemento del centro è banale). \medskip
	
	
	Utilizzando la nozione di centralizzatore, si può contare ``facilmente'' il numero
	di classi di coniugio di un gruppo. Infatti, si osserva crucialmente che
	$\Fix(g)$ (il numero di elementi di $G$ lasciati invariati sotto il coniugio di $g$)
	è lo stesso insieme $Z_G(g)$. Infatti vale che:
	\[ \Fix(g) = \{ h \in G \mid gh = hg \} = Z_G(g). \]
	Allora, per il lemma di Burnside, se $k(G)$ è il numero di classi di coniugio di $G$, vale che:
	\[ k(G) = \frac{1}{\abs{G}} \sum_{g \in G} \abs{Z_G(g)}. \] \bigskip
	
	
	La formula delle classi di coniugio risulta in particolare utile nella discussione
	dei $p$-gruppi, definiti di seguito.
	
	\begin{definition}[$p$-gruppo]
		Sia $G$ un gruppo finito. $G$ si dice allora \textbf{$p$-gruppo} se
		$\abs{G} = p^n$ per $n \in \NN^+$ e un numero primo $p \in \NN$.
	\end{definition}
	
	Infatti, grazie alla formula delle classi di coniugio, si osserva facilmente che il centro di un $p$-gruppo non è mai banale (ossia composto dalla sola identità), come mostra la:
	
	\begin{proposition}
		Sia $G$ un $p$-gruppo. Allora $\abs{Z(G)} > 1$. 
	\end{proposition}
	
	\begin{proof}
		Dalla formula delle classi di coniugio si ha che:
		\[ \abs{G} = \abs{Z(G)} + \sum_{g \in \mathcal{R} \setminus Z(G)} \frac{\abs G}{\abs{Z_G(g)}}. \]
		Si osserva in particolare che il secondo termine della somma a destra è divisibile
		per $p$. Infatti, poiché $g \notin Z(G)$ per ipotesi, $Z_G(g) \neq Z(G)$; da cui
		si deduce che $\abs{Z_G(g)}$ deve essere un divisore stretto di $p^n$, e dunque
		che $p \mid \nicefrac{\abs G}{\abs{Z_G(g)}}$. Prendendo l'identità di sopra modulo
		$p$, si deduce allora che:
		\[ \abs{Z(G)} \equiv 0 \pod p. \]
		Combinando questo risultato col fatto che $\abs{Z(G)} \geq 1$ (infatti $Z(G) \leq G$),
		si conclude che deve valere necessariamente la tesi.
	\end{proof} \medskip
	
	
	Quest'ultima proposizione spiana il terreno per un risultato interessante sui
	gruppi di ordine $p^2$, come mostra il:
	
	\begin{theorem}
		Ogni gruppo $G$ di ordine $p^2$ è abeliano.
	\end{theorem}
	
	\begin{proof}
		Dal momento che $G$ è un $p$-gruppo, per la precedente proposizione
		$\abs{Z(G)} > 1$. Allora $\abs{Z(G)}$ è pari a $p$ o $p^2$, per il
		Teorema di Lagrange. Se $\abs{Z(G)}$ fosse pari a $p$, allora
		$\abs{G \quot Z(G)} = \nicefrac{\abs G}{\abs{Z(G)}} = p$. Pertanto
		$G \quot Z(G)$ sarebbe ciclico, e dunque $G$ sarebbe abeliano; assurdo,
		dal momento che si era presupposto che $Z(G)$ fosse un sottogruppo proprio
		di $G$, \Lightning. Allora $Z(G)$ ha ordine $p^2$,
		e dunque $Z(G) = G$.
	\end{proof} \medskip
	
	
	\begin{example}
		Si mostra che\footnote{
			Il risultato è facilmente dimostrabile attraverso
			il Teorema di struttura dei gruppi abeliani
			finitamente generati.
		} $G$ è obbligatoriamente isomorfo a
		$\ZZ_{p^2}$ o a $\ZZ_p \times \ZZ_p$ se
		$\abs{G} = p^2$. \vskip 0.1in
		

		Se $G$ ammette un generatore,
		allora $G$ è ciclico e quindi isomorfo a $\ZZ_{p^2}$.
		Altrimenti, sia $g \in G$ un elemento di ordine\footnote{
			Questo elemento deve esistere obbligatoriamente, non
			solo per il Teorema di Cauchy, ma anche perché solo
			l'identità ammette ordine $1$ e perché si è supposto
			che nessun elemento abbia ordine $p^2$ (altrimenti
			il gruppo sarebbe ciclico).
		} $p$ e sia\footnote{
			Tale $h$ deve esistere, altrimenti $G$ sarebbe ciclico.
		} $h \in G$ tale che $h \notin \gen{g}$. Per il teorema
		precedente $G$ è abeliano, e quindi $\gen{g}\gen{h}$ è
		un sottogruppo di $G$. \medskip
		
		
		Inoltre $\gen{g} \cap \gen{h}$ è
		banale: se non lo fosse avrebbe ordine $p$, e quindi
		$\gen{g}$ e $\gen{h}$ coinciderebbero insiemisticamente,
		\Lightning. Pertanto $\gen{g}\gen{h} \cong \gen{g}
		\times \gen{h} \cong \ZZ_p \times \ZZ_p$. Infine, poiché
		$\abs{\gen{g} \gen{h}} = p^2$, vale anche che
		$G = \gen{g} \gen{h}$, da cui la tesi.
	\end{example}
\end{document}