\documentclass[12pt]{scrartcl}
\usepackage{notes_2023}

\begin{document}
	\title{Il gruppo delle permutazioni}
	\maketitle
	
	\begin{note}
		Nel corso del documento con $X_n$ si indicherà l'insieme
		$\{1, \ldots, n\}$ e con $G$ un qualsiasi gruppo.
	\end{note}

	Si definisce brevemente il \textbf{gruppo delle permutazioni} $S_n$ come il gruppo
	delle bigezioni su $G$, ossia $S(X_n)$. Si deduce facilmente che
	$\abs{S_n} = n!$ dal momento che vi sono esattamente $n!$ scelte possibili
	per costruire una bigezione da $X_n$ in $X_n$ stesso. \medskip
	
	
	Si definisce
	l'\textbf{azione naturale} di $S_n$ su $X_n$ come l'azione $\varphi : S_n \to S(X_n)$
	tale per cui $\sigma \xmapsto{\varphi} [n \mapsto \sigma(n)]$. In particolare,
	per $H \leq S_n$, si definisce la sua azione naturale come la restrizione dell'azione
	naturale di $S_n$ su $H$. Un sottogruppo $H$ si dice \textit{transitivo} se la
	sua azione naturale è transitiva. Si osserva che ogni tale azione naturale è fedele
	(infatti $\sigma \in S_n$ fissa tutto $X_n$ solo se è l'identità di $S_n$). Si illustra allora subito un risultato sui
	sottogruppi abeliani transitivi di $S_n$:
	
	\begin{proposition}
		Sia $H$ un sottogruppo abeliano transitivo di $S_n$. Allora
		$\abs{H} = n$.
	\end{proposition}
	
	\begin{proof}
		Dal Teorema orbita-stabilizzatore, $\abs{H} = \abs{\Stab(i)} \abs{\Orb(i)}$. Poiché
		$H$ è un sottogruppo transitivo, $\abs{\Orb(i)} = n$, e quindi è sufficiente
		verificare che $\Stab(i)$ sia banale. \medskip
		
		
		Ogni $\Stab(i)$ è coniugato
		ad ogni altro $\Stab(j)$, sempre per la transitività dell'azione; poiché allora $H$ è abeliano, in particolare $\Stab(i)$ coincide con
		ogni altro stabilizzatore. Pertanto $\sigma \in \Stab(i)$ se e solo se
		$\sigma$ appartiene al nucleo dell'azione naturale di $H$, ossia
		a $\cap_{x=1}^n \Stab(x)$, e quindi se e solo se $\sigma = e$. Si conclude dunque
		che $\Stab(i)$ è banale e quindi che $\abs{H} = n$.
	\end{proof}
	
	\begin{example}[Il gruppo di Klein $V_4$]
		In $S_4$, e in particolare in $A_4$, esiste un sottogruppo normale non banale
		molto particolare\footnote{
			Pertanto $A_4$ non è semplice.
		}, il cosiddetto\footnote{
			La lettere $V$ è dovuta al termine \textit{vier}, che in tedesco
			significa ``quattro''.
		} \textbf{gruppo di Klein} $V_4$, dove:
		\[ V_4 = \{ e, (1, 2)(3, 4), (1, 3)(2, 4), (1, 4)(2, 3) \}. \]
		Tale sottogruppo è abeliano e transitivo (e quindi, per il risultato di prima,
		$\abs{V_4} = 4$, come si osserva facilmente). Poiché ogni suo elemento ha
		ordine $2$ (e in particolare $V_4$ non è ciclico), $V_4$ deve necessariamente
		essere isomorfo a $\ZZmod2 \times \ZZmod2$. Pertanto $V_4$ è il più piccolo
		gruppo non ciclico per ordine (a meno di isomorfismo).
	\end{example}
	
	Come è noto, ogni $\sigma \in S_n$ può scriversi come prodotto di cicli
	disgiunti. Di seguito si introduce un modo formale per descrivere questi
	cicli. \medskip
	
	
	Si consideri l'azione naturale di $\gen{\sigma}$. Allora i cicli di $\sigma$ sono esattamente
	le orbite di $\sigma$ ordinate nel seguente modo:
	\[ \Orb(x) = \{ x, \sigma(x), \dots, \sigma^m(x) \}. \]

	
	Si osserva che in effetti tutti gli elementi di $X$ sono considerati nella
	scrittura delle orbite dal momento che tali orbite inducono una partizione
	di $X$ (infatti sono classi di equivalenza). Si definisce inoltre una
	permutazione \textit{ciclo} se esiste al più un'unica orbita di cardinalità diversa
	da $1$ e si dice \textit{lunghezza del ciclo} la cardinalità di tale orbita (o se non esiste, si dice che ha lunghezza unitaria). Due cicli si dicono disgiunti se almeno uno dei due è l'identità o se le loro uniche orbite non banali hanno intersezione nulla (e in entrambi i casi, commutano). Per ogni $k$-ciclo esistono esattamente $k$ scritture
	distinte (in funzione dell'elemento iniziale del ciclo). \medskip
	
	
	Pertanto si deduce facilmente che ogni permutazione $\sigma$ è prodotto
	di cicli disgiunti in modo unico (a meno della scelta del primo elemento
	dell'orbita). Poiché allora ogni $n$-ciclo è generato dalla composizione
	di $n-1$ trasposizioni ($2$-cicli) e ogni permutazione è prodotto di cicli,
	$S_n$ è generato dalle trasposizioni. Infatti:
	\[ (a_1, \dots, a_i) = (a_1, a_i) \circ (a_1, a_{i-1}) \circ \cdots \circ (a_1, a_2), \]
	o altrimenti:
	\[ (a_1, \dots, a_i) = (a_1, a_2) \circ (a_2, a_3) \circ \cdots \circ (a_{i-1}, a_i), \]
	da cui si deduce che la scrittura come prodotto di
	trasposizioni non è unica. Ciononostante viene sempre mantenuta la parità
	del numero di trasposizioni impiegate. \medskip
	

	Per questo motivo la mappa $\sgn : S_n \to \{\pm 1\}$ che vale $1$ sulle
	permutazioni con numero pari di trasposizioni impiegabili e $-1$ sul resto
	è ben definita. Inoltre questa mappa è un omomorfismo di gruppi, e si
	definisce $\An := \Ker \sgn$ come il sottogruppo di $S_n$ delle permutazioni
	pari, detto anche \textit{gruppo alterno}. La classe laterale $(1, 2) \An$
	rappresenta invece le permutazioni dispari. \medskip
	
	
	In particolare, se $\sigma_k$ è un $k$-ciclo, $\sgn(\sigma_k) = (-1)^{k-1}$ e $\ord(\sigma_k) = k$. Si osserva inoltre che vi sono esattamente $\binom{n}{k} \frac{k!}{k} =
	\binom{n}{k} (k-1)!$ $k$-cicli in $S_n$ e che in generale l'ordine
	di una permutazione è il minimo comune multiplo degli
	ordini dei suoi cicli. In particolare vale la seguente identità\footnote{
		Si verifica facilmente che il prodotto a destra fornisce un omomorfismo. Allora
		è sufficiente mostrare che è ben definito e che vale $-1$ sulle trasposizioni.
		Se si considera $\sigma = (a, b)$, per $i$ e $j$ tali per cui
		$\{i, j\} \cap \{a, b\} = \emptyset$ il termine della produttoria è unitario;
		per $\{i, j\} = \{a, b\}$ il termine è $-1$ e per un'intersezione di un solo
		termine si osserva che vi sono due termini del prodotto che valgono $-1$ e
		che moltiplicati si annullano nell'unità. Poiché $\sgn$ vale anch'esso $-1$ sulle trasposizioni, i due omomorfismi coincidono (infatti le trasposizioni generano $S_n$).
	}:
	\[ \sgn(\sigma) = \prod_{1 \leq i < j \leq n} \frac{\sigma(i) - \sigma(j)}{i - j}. \]

	Si definisce \textit{tipo} di una permutazione $\sigma$ la sua decomposizione
	in cicli disgiunti a meno degli elementi presenti nei cicli. Sia $\sigma$
	tale per cui:
	\[ \sigma = (a_1, a_2, \ldots, a_{k_1}) (b_1, \ldots, b_{k_2}) \cdots (c_1, \ldots, c_{k_i}), \]
	allora vale la seguente relazione sul coniugio:
	\[ \tau \sigma \tau\inv = (\tau(a_1), \tau(a_2), \ldots, \tau(a_{k-1})) (\tau(b_1), \ldots, \tau(b_{k_2})) \cdots (\tau(c_1), \ldots, \tau(c_{k_i})). \]
	
	A partire da ciò vale il seguente risultato:
	\begin{proposition}
		Due permutazioni $\sigma_1$, $\sigma_2$ sono \textit{coniugabili}
		(ossia appartengono alla stessa classe di coniugio) se e solo se
		hanno lo stesso tipo.
	\end{proposition}
	
	\begin{proof}
		Dalla seguente identità, se $\sigma_1$ è coniugata rispetto a
		$\sigma_2$, sicuramente le due permutazioni dovranno avere lo stesso
		tipo. Analogamente, se le due permutazioni hanno lo stesso tipo,
		si può costruire $\tau$ che associ ogni elemento di
		un ciclo di $\sigma_1$ a un elemento nella stessa posizione in un ciclo
		di $\sigma_2$ della stessa lunghezza in modo tale che $\tau$ rimanga
		una permutazione di $S_n$ e che valga $\sigma_2 = \tau \sigma_1 \tau\inv$.
	\end{proof}
	
	Come corollario di questo risultato, se $m_1$ rappresenta il numero di $1$-cicli di $\sigma$, $m_2$ quello dei suoi $2$-cicli, fino a $m_k$, vale il seguente risultato:
	\[ \abs{\Cl(\sigma)} = \frac{n!}{m_1! \, 1^{m_1} \, m_2! \, 2^{m_2} \cdots m_k! \, k^{m_k}}, \]
	e in particolare esistono tante classi di coniugio quante partizioni di $n$.
	Come conseguenza di questo risultato, per il Teorema orbita-stabilizzatore,
	vale che:
	\[ \abs{Z_{S_n}(\sigma)} = m_1! \, 1^{m_1} \, m_2! \, 2^{m_2} \cdots m_k! \, k^{m_k}, \]
	dove si ricorda\footnote{
		Infatti $Z_{S_n}(\sigma)$ è lo stabilizzatore di $\sigma$ nell'azione di coniugio.
	} che due permutazioni coniugano $\sigma$ nella stessa permutazione
	$\rho$ se queste due permutazioni fanno parte della stessa classe in $G \quot Z_{S_n}(\sigma)$. Infine,
	sempre come corollario dello stesso risultato,
	se $H \leq S_n$, $H$ è normale in $S_n$ se e solo se per ogni tipo di
	permutazione $H$ contiene
	tutte le permutazioni di quel tipo o nessuna. \medskip
	
	
	Per calcolare il centralizzatore di una permutazione $\sigma \in S_n$, la
	strategia generale si compone di due passi fondamentali: computare il
	numero di elementi del centralizzatore tramite il Teorema orbita-stabilizzatore
	(come visto precedentemente) e poi ``indovinare'' dei sottogruppi con
	cui $\sigma$ commuta che, combinati tramite il prodotto di sottogruppi,
	danno esattamente il numero calcolato inizialmente.
	
	\begin{example}
		Sia $\sigma = \overbrace{(1,2,3,4)}^{\sigma_1}\overbrace{(5,6,7)}^{\sigma_2}\overbrace{(8,9)}^{\sigma_3} \in S_9$. Si calcola $Z_{S_9}(\sigma)$.
		Tramite il Teorema orbita-stabilizzatore, vale che:
		\[ Z_{S_9}(\sigma) = 1! \cdot 4 \cdot 1! \cdot 3 \cdot 1! \cdot 2 = 4! = 24. \]
		Si osserva facilmente che $\sigma$ commuta con $\sigma_1$, $\sigma_2$ e
		$\sigma_3$, e quindi $\gen{\sigma_i} \leq Z_{S_9}(\sigma)$ $\forall i \in \{1,2,3\}$.
		In particolare $\gen{\sigma_i}$ commuta sempre con $\gen{\sigma_j}$ per $i \neq j$,
		dal momento che questi cicli sono tutti disgiunti. Si considera\footnote{
			Poiché $\sigma_i$ commuta con $\sigma_j$, questo sottogruppo è ben definito.
		} il
		sottogruppo $H = \gen{\sigma_1}\gen{\sigma_2}\gen{\sigma_3}$: ogni suo elemento
		è esprimibile in modo unico come prodotto di una potenza di $\sigma_1$, di
		$\sigma_2$ e di $\sigma_3$, e quindi $\abs{H} = \abs{\gen{\sigma_1}} \abs{\gen{\sigma_2}} \abs{\gen{\sigma_3}} = 4 \cdot 3 \cdot 2 = 24$; poiché
		allora $H \leq Z_{S_9}(\sigma)$ ha lo stesso numero di elementi del centralizzatore,
		$Z_{S_9}(\sigma) = H$. Infine, dal
		momento che $\gen{\sigma_i} \cap (\gen{\sigma_j} \gen{\sigma_k})$ per ogni
		$i$, $j$, $k$ distinti in $\{1, 2, 3\}$, $H \cong \gen{\sigma_1} \times \gen{\sigma_2} \times \gen{\sigma_3}$, e dunque:
		\[ Z_{S_9}(\sigma) \cong \ZZmod4 \times \ZZmod3 \times \ZZmod2 \cong \ZZmod{12} \times \ZZmod2. \]
	\end{example} \bigskip

	
	Si osserva adesso che $\An$ può scriversi come il sottogruppo generato dai
	$2-2$-cicli, infatti ogni permutazione pari è prodotto di un numero pari di trasposizioni,
	che possono dunque essere ridotte a $2-2$-cicli. Allo stesso tempo allora
	$\An$ è generato dai $3$-cicli se $n \geq 3$. Si consideri infatti $(i, j)(k, l)$. Se
	$\{i, j\} \cap \{k, l\} = 2$, $(i, j) = (k, l)$, e quindi $(i, j)(k, l) = e$;
	se $\{i, j\} \cap \{k, l\} = 1$, si può assumere senza perdita di generalità che
	$k = i$, da cui $(i, j)(i, l) = (i, l, j)$, un $3$-ciclo; se invece $\{i, j\} \cap \{k, l\} = 0$,
	$(i, j)(k, l) = (i, j)(j, k)(j, k)(k, l) = (i, j, k)(j, k, l)$, e quindi
	$(i, j)(k, l)$ è prodotto di due $3$-cicli. Pertanto si è dimostrato
	che $\An = \gen{(i, j)(k, l) \mid i, j, k, l \in X_n} \subseteq \gen{(i, j, k) \mid i, j, k \in X_n}$; allo stesso tempo ogni $3$-ciclo è una permutazione
	pari, e quindi vale anche l'inclusione inversa. \medskip
	
	
	
	Si consideri adesso $S_n'$, il sottogruppo derivato di $S_n$. Poiché $S_n$ è abeliano
	per $n \in \{ 1, 2 \}$, in tal caso $S_n' = \{e\}$; in tutti gli altri casi
	$S_n'$ non può essere uguale a $\{e\}$, altrimenti $S_n$ sarebbe abeliano. Si osserva
	che $[(i,j), (j,k)]$ con $i$, $j$ e $k$ distinti si scrive come:
	\[ [(i,j), (j,k)] = (i, j) (j, k) (i, j)\inv (j, k)\inv = (i, k) (j, k) = (i, k, j), \]
	e quindi si deduce che $\gen{(i, j, k) \mid \abs{\{i, j, k\}} = 3} = \An$ è un sottogruppo
	di $S_n'$. Inoltre\footnote{
		Alternativamente $[S_n : S_n']$ deve dividere $[S_n : \An] = 2$, e quindi,
		poiché $S_n \neq S_n'$, è necessario che $S_n'$ sia esattamente $\An$.
	} l'omomorfismo $\sgn$ ha come codominio un gruppo abeliano isomorfo
	a $\ZZmod2$, e quindi $S_n' \subseteq \Ker \sgn = \An$. Si conclude dunque che
	$S_n' = \An$ e che ${S_n}_{\,ab} = S_n \quot \An \cong \{\pm 1\} \cong \ZZmod2$ per $n \geq 3$. Pertanto adesso è immediato il seguente risultato:
	
	\begin{proposition}
		Sia $H$ un gruppo abeliano. Allora $\Hom(S_n, H) \bij \Hom(\ZZmod2, H)$.
	\end{proposition}

	In particolare, vi sono tanti omomorfismi non banali in $\Hom(S_n, H) \bij \Hom(\ZZmod2, H)$ quanti elementi di ordine $2$ vi sono in $H$. \bigskip
	
	
	Si ricercano adesso le classi di coniugio in $\An$. Si osserva innanzitutto che,
	se $\sigma \in \An$, $\Cl_{\An}(\sigma) \subseteq \Cl_{\Sn}(\sigma)$. Inoltre,
	per il Teorema orbita-stabilizzatore, vale che:
	\[ \abs{\Cl_{\An}(\sigma)}(\sigma) = \frac{\abs{\An}}{\abs{Z_{\An}(\sigma)}} =
	\frac{\abs{S_n}/2}{\abs{Z_{\Sn}(\sigma) \cap \An}}. \]
	Poiché\footnote{
		È sufficiente osservare che $Z_{\Sn}(\sigma) \cap \An = \Ker(\restr{\sgn}{\An})$,
		e dunque che $Z_{\Sn}(\sigma) \quot{(Z_{\Sn}(\sigma) \cap \An)}$ può essere
		isomorfo tramite il Primo teorema di isomorfismo soltanto a $\{1\}$ o
		a $\{\pm1\}$.
	} $Z_{\Sn}(\sigma) \cap \An$ in $Z_{\Sn}(\sigma)$ ha indice $1$ se
	$Z_{\Sn}(\sigma) \subseteq \An$ e $2$ altrimenti, vale che:
	
	\begin{itemize}
		\item $\abs{\Cl_{\An}(\sigma)}(\sigma) = \frac{1}{2} \abs{\Cl_{\Sn}(\sigma)}$, se $Z_{\Sn}(\sigma) \subseteq \An$,
		\item $\abs{\Cl_{\An}(\sigma)}(\sigma) = \abs{\Cl_{\Sn}(\sigma)}$, altrimenti.
	\end{itemize}
\end{document}