\documentclass[12pt]{scrartcl}
\usepackage{notes_2023}

\begin{document}
	\title{Commutatore e gruppo derivato}
	\maketitle
	
	\begin{note}
		Nel corso del documento con $G$ si indicherà un qualsiasi gruppo.
	\end{note}

	Siano $g$ e $h$ due elementi di $G$. Si definisce allora il loro \textbf{commutatore}
	come l'elemento $[g, h] = ghg\inv h\inv$. Tale elemento formalizza il concetto di
	``misura di commutatività'', ossia identifica formalmente quanto $g$ e $h$ commutano.
	Infatti vale che:
	\[ [g,h] = e \iff gh=hg. \]
	Si definisce allora il \textbf{gruppo derivato} di $G$, indicato con $G'$, come
	il sottogruppo di $G$ generato dai commutatori:
	\[ G' = \gen{[g, h] \mid g, h \in G}. \]
	Si osserva che $[g, h] [h, g] = ghg\inv h\inv hg h\inv g\inv = e$, e quindi
	che $[g, h]\inv = [h, g]$. In particolare valgono alcune proprietà particolari
	per $G'$, riassunte dalla:
	
	\begin{proposition}
		Sia $N$ un sottogruppo normale di $G$ e sia $H$ un gruppo abeliano. Allora:
		
		\begin{enumerate}[(i)]
			\item $G'$ è un gruppo caratteristico,
			\item $G \quot G'$ è un gruppo abeliano, ed è indicato come
			$G_{ab}$, \textbf{l'abelianizzato di $G$},
			\item Se $G \quot N$ è abeliano, $G' \leq N$\footnote{
				In un certo senso, questo punto dimostra che la scelta di definire
				$G_{ab}$ è tutt'altro che data al caso. $G_{ab}$ è infatti il ``più stretto
				parente'' abeliano di $G$. Si osservi anche che $G$ abeliano
				$\implies$ $G' = \{e\}$ $\implies$ $G_{ab} \cong G$. 
			},,
			\item Se $H$ è abeliano, ogni omomorfismo $\varphi \in \Hom(G, H)$ è
				tale per cui $G' \leq \Ker \varphi$, e quindi
				$\Hom(G, H)$ può identificarsi con $\Hom(G/G', H) =
				\Hom(G_{ab}, H)$.
		\end{enumerate}
	\end{proposition}
	
	\begin{proof}
		Si dimostrano le tesi punto per punto.
		
		\begin{enumerate}[(i)]
			\item Se si pone $S = \{ [x, y] \mid x, y \in G \}$ (ossia $S$ è l'insieme dei
				generatori di $G'$ dacché $S\inv = S$), è sufficiente mostrare che per $\varphi \in \Aut(G)$
				vale che $\varphi(S) = S$. Allora $\varphi([x, y]) = \varphi(x) \varphi(y)
				\varphi(x)\inv \varphi(y)\inv = [\varphi(x), \varphi(y)] \in S$, mostrando
				dunque che $G'$ è caratteristico.
			\item $G \quot G'$ è un gruppo perché $G'$, in quanto caratteristico, è
				normale. Siano $x$, $y \in X$, allora $xyG' = yxG'$ perché
				$xy(yx)\inv = xyx\inv y \inv = [x, y] \in G'$ per definizione, e quindi
				$G_{ab}$ è abeliano.
			\item Se $G \quot N$ è abeliano, $xyN = yxN \implies xy(yx)\inv \in N \implies
				[x, y] \in N$. Poiché allora $S \subseteq N$, vale che $G' = \gen{S} \leq N$.
			\item È sufficiente mostrare che $S \subseteq \Ker \varphi$. Si verifica dunque
				che:
				\[ \varphi([x, y]) = \varphi(x) \varphi(y) \varphi(x)\inv \varphi(y)\inv
				= e \implies [x, y] \in \Ker \varphi. \]
				Poiché allora $G' \subseteq \Ker \varphi$, per il Primo teorema di isomorfismo,
				ogni omomorfismo $\varphi \in \Hom(G, H)$ ammette un unico omomorfismo
				$\varphi' \in \Hom(G \quot G', H) = \Hom(G_{ab}, H)$ tale per cui il seguente
				diagramma commuti:
				\[\begin{tikzcd}[cramped]
					G && H \\
					\\
					{G_{ab}}
					\arrow["\varphi", from=1-1, to=1-3]
					\arrow["{\pi_{G_{ab}}}"', two heads, from=1-1, to=3-1]
					\arrow["{\varphi'}"', from=3-1, to=1-3]
				\end{tikzcd}\]
				Pertanto $\Hom(G, H) \bij \Hom(G_{ab}, H)$.
		\end{enumerate}
	\end{proof}
\end{document}