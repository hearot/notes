\documentclass[12pt]{scrartcl}
\usepackage{notes_2023}

\begin{document}
	\title{Il teorema di struttura per gruppi abeliani finiti
		e decomposizione di $\ZZmulmod{n}$}
	\maketitle
		
	\begin{note}
		Nel corso del documento con $G$ si indicherà un qualsiasi gruppo abeliano finito.
	\end{note}
	
	In questo documento si dimostra il celebre Teorema di
	struttura per gruppi abeliani finiti. In realtà questo
	teorema è un caso particolare del Teorema di struttura
	per gruppi abeliani finitamente generati (e quindi
	potenzialmente infiniti), a sua volta caso particolare
	del Teorema di struttura per moduli finitamente generati
	su un PID\footnote{
		Ho trattato e dimostrato questo teorema in un documento
		separato, reperibile su
		\url{https://git.phc.dm.unipi.it/g.videtta/scritti/src/branch/main/Geometria/Articoli/3.\%20Il\%20teorema\%20di\%20struttura\%20dei\%20moduli\%20finitamente\%20generati\%20su\%20un\%20PID/main.pdf}
	}. Per motivi didattici si riporta la dimostrazione semplificata
	per il semplice caso dei gruppi abeliani finiti. \medskip
	
	
	Il Teorema di struttura trova immediata applicazione nello
	studio dei gruppi abeliani finiti e, dato $n \in \NN^+$, permette di classificare tutti i gruppi abeliani di ordine $n$ (a meno
	di isomorfismo), come illustra il seguente enunciato:
	
	\begin{theorem}[di struttura per gruppi abeliani finiti,
		decomposizione in fattori invarianti]
		Sia $G$ un gruppo abeliano finito. Allora esistono unici
		$n_1$, \ldots, $n_s \in \NN^+ \setminus \{1\}$ tali per cui:
		\[ G \cong \ZZmod{n_1} \times \cdots \times \ZZmod{n_s}, \qquad n_s \mid n_{s-1} \mid \cdots \mid n_2 \mid n_1. \]
		Tale fattorizzazione di $G$ viene detta
		\textbf{decomposizione in fattori invarianti}, dove
		i fattori invarianti sono i vari $n_i$.
	\end{theorem}
	
	Equivalentemente si poteva enunciare il Teorema di struttura
	utilizzando la \textit{decomposizione in fattori elementari}
	mediante l'applicazione reiterata del Teorema cinese del
	resto, come illustra il:
	
	\begin{theorem}[di struttura per gruppi abeliani finiti,
		decomposizione primaria]
		Sia $G$ un gruppo abeliano finito. Allora esistono unici
		$p_1$, \ldots, $p_s$ numeri primi e unici $m_{i,1} \geq \cdots \geq m_{i,t_i}$ per ogni $1 \leq i \leq s$ tali per cui:
		\[ G \cong \ZZpmod{p_1^{m_{1,1}}} \times \cdots \times \ZZpmod{p_1^{m_{1,t_1}}} \times \ZZpmod{p_2^{m_{2,1}}} \times \cdots \times \ZZpmod{p_s^{m_{s,t_s}}}. \]
		Tale fattorizzazione di $G$ viene detta
		\textbf{decomposizione primaria} (o in fattori
		elementari), dove i fattori elementari sono i vari $p_i^{m_{i,j}}$.
	\end{theorem}
	
	
	Prima di dimostrare il Teorema di struttura, si definisce
	il concetto di $p$-componente relativa a un numero $p$
	primo.
	
	\begin{definition}[$p$-componente]
		Si definisce \textbf{$p$-componente} $G(p)$ (o $p$-torsione)
		di $G$ il sottogruppo di $G$ tale per cui:
		\[ G(p) = \{ x \in G \mid \ord(x) = p^k \text{ per qualche } k \}. \]
	\end{definition}
	
	\begin{remark}
		Si osserva facilmente che $G(p)$ è effettivamente
		un sottogruppo. Infatti vale chiaramente che
		$G(p) \subseteq G$; inoltre $e$ appartiene
		a $G(p)$. Dati allora $x$, $y \in G(p)$, allora
		$\ord(xy) \mid \mcm(\ord(x), \ord(y))$, e quindi
		$\ord(xy) = p^k$ per qualche $k$. Pertanto anche
		$xy \in G(p)$. Dal momento che $G(p)$ è finito,
		la chiusura sull'operazione di gruppo implica anche
		l'esistenza dell'inverso, e dunque
		$G(p)$ è un sottogruppo di $G$.
	\end{remark}
	
	\begin{remark}
		Si osserva che $G(p)$ ha ordine $p^n$, dove
		$p^n \exactdiv n = \abs{G}$ e che se $H \leq G$
		è un sottogruppo di ordine $p^i$, $H$ è chiaramente
		un sottogruppo di $G(p)$. Pertanto, si può definire
		equivalentemente $G(p)$ come il $p$-sottogruppo\footnote{
			Chiaramente $G(p)$ è un $p$-sottogruppo. Infatti,
			se $q$ fosse un primo diverso da $p$ che divide
			$\abs{G(p)}$, per il Teorema di Cauchy esisterebbe
			un elemento di ordine $q$ in $G(p)$, $\Lightning$.
		} massimo
		per inclusione di $G$.
	\end{remark}
	
	\begin{remark}
		La $p$-componente $G(p)$ è anche un sottogruppo caratteristico di $G$. Infatti
		$\varphi \in \Aut(G)$ lascia invariato l'ordine di
		un elemento di $G(p)$, e quindi $\varphi(G(p)) = G(p)$.
		Alternativamente si può utilizzare l'osservazione
		precedente e notare che $G(p)$ è l'unico sottogruppo
		del suo ordine\footnote{
			In ogni caso $G$ è abeliano e quindi, poiché
			tutti i $p$-Sylow sono coniugati, $G(p)$ è l'unico $p$-Sylow di $G$, e dunque
			è caratteristico perché unico del suo ordine.
			Più elementarmente, ogni $p$-sottogruppo di $G$
			è contenuto in $G(p)$, e quindi è l'unico del suo
			ordine.
		}.
	\end{remark}
	
	\begin{scheme}
		La dimostrazione del Teorema di struttura si fonda su
		due teoremi che verranno dimostrati nel seguito e che
		vengono ora enunciati:
	
		\begin{itemize}
			\item Se $G$ è abeliano con $\abs{G} = p_1^{e_1} \cdots p_r^{e_r}$, allora $G \cong G(p_1) \times \cdots \times G(p_r)$, ossia $G$ è isomorfo al prodotto diretto tra
			le sue $p$-componenti. Tale decomposizione di $G$
			come prodotto di $p$-gruppi di ordini tra loro
			coprimi è unica a meno di isomorfismi tra i vari $p$-gruppi.
			\item Se $G$ è un $p$-gruppo abeliano. Allora esistono
			e sono univocamente determinati degli interi
			positivi $r_1 \geq \cdots \geq r_s$ tali che
			$G \cong \ZZmod{p_1^{r_1}} \times \cdots
			\times \ZZmod{p_1^{r_s}}$.
		\end{itemize}
	\end{scheme}
	
	\begin{proof}[Dimostrazione a priori]
		Per il primo teorema, $G$ si può decomporre nelle sue
		$p$-componenti:
		\[ G \cong G(p_1) \times \cdots \times G(p_s). \]
		
		
		Allora, per il secondo teorema, ogni $G(p_i)$ può
		decomporsi come prodotto diretto di $\ZZmod{p_i^k}$,
		e quindi:
		\[ G \cong (\ZZmod{p_1}^{e_{1,1}} \times \cdots \times \ZZmod{p_1}^{e_{1,t_1}}) \times \cdots \times (\ZZmod{p_r}^{e_{r,1}} \times \cdots \times \ZZmod{p_1}^{e_{r,t_r}}). \]
		
		
		Sia $t = \max\{t_1, \ldots, t_r\}$. Posso allungare le
		fattorizzazioni di $G(p_i)$ fino ad ottenere $t$ fattori aggiungendo eventualmente dei gruppi banali nella
		fattorizzazione. \medskip
		
		
		Applicando allora il Teorema cinese del resto si ottiene
		l'esistenza della fattorizzazione secondo il Teorema
		di struttura per gruppi abeliani finiti. L'unicità segue
		riapplicando prima il primo teorema e poi il
		secondo teorema.
	\end{proof}
	
	\begin{example}[$\ZZmod{26} \times \ZZmod{169} \times \ZZmod{12}$]
		Si scrive il gruppo $G = \ZZmod{26} \times \ZZmod{169} \times \ZZmod{12}$ seguendo le regole del Teorema di struttura.
		Poiché $26 = 2 \cdot 13$, $169 = 13^2$ e $12 = 2^2 \cdot 3$,
		applicando il Teorema cinese del resto si può scrivere $G$
		come:
		\[ G \cong (\ZZmod{2} \times \ZZmod{13}) \times (\ZZmod{13^2}) \times (\ZZmod{2^2} \times \ZZmod{3}). \]
		Facendo commutare i fattori come nella dimostrazione
		del Teorema di struttura otteniamo che:
		\[ G \cong (\ZZmod{2^2} \times \ZZmod{13^2} \times \ZZmod{3}) \times (\ZZmod{2} \times \ZZmod{13}), \]
		e quindi vale la seguente decomposizione in fattori
		invarianti per $G$:
		\[ G \cong \ZZmod{2028} \times \ZZmod{26}, \]
		dove si osserva che $26 \mid 2028$.
	\end{example}
	
	Si dimostrano adesso i due teoremi impiegati nella dimostrazione
	del Teorema di struttura:
	
	\begin{theorem}
		Se $G$ è abeliano con $\abs{G} = p_1^{e_1} \cdots p_r^{e_r}$, allora $G \cong G(p_1) \times \cdots \times G(p_r)$, ossia $G$ è isomorfo al prodotto diretto tra
		le sue $p$-componenti. Tale decomposizione di $G$
		come prodotto di $p$-gruppi di ordini tra loro
		coprimi è unica a meno di isomorfismi tra i vari $p$-gruppi.
	\end{theorem}
	
	\begin{proof}
		Poiché $G$ è abeliano, esiste il sottogruppo $G(p_1) G(p_2) \cdots G(p_r)$.
		In particolare, dal momento che $\MCD(\abs{G(p_i)}, \abs{G(p_j)}) = 1$ e che
		$G(p_i) \cap G(p_j) = \{e\}$ per
		$i \neq j$, vale anche che $\abs{G(p_1) G(p_2) \cdots G(p_r)} = \abs{G}$.
		Allora deve valere in particolare che $G = G(p_1) G(p_2) \cdots G(p_r)$. Per
		il Teorema di decomposizione in prodotto diretto, si deduce che
		$G \cong G(p_1) \times \cdots \times G(p_r)$. \medskip
		
		
		Si mostra che la decomposizione di $G$ come prodotto di $p$-gruppi di ordini
		tra loro coprimi è unica. Sia $H_1 \times H_2 \times \cdots H_r$ un'altra
		decomposizione di $G$ in prodotto diretto in modo tale che $\abs{H_i} = p_i^{e_i}$.
		Sia $\varphi$ un isomorfismo da $H_1 \times H_2 \times \cdots \times H_r$ in
		$G(p_1) \times G(p_2) \times \cdots \times G(p_r)$. Allora $\varphi$ ristretto
		all'identificazione di $H_i$ in $H_1 \times H_2 \times \cdots H_r$ (ossia
		$\{e\} \times \cdots \times H_i \times \cdots \times \{e\}$) nell'identificazione
		di $G(p_i)$ è ancora un isomorfismo, dal momento che entrambi sono gli unici
		$p$-gruppi che compaiono nelle rispettive fattorizzazioni. Allora
		$H_i \cong G(p_i)$, concludendo la dimostrazione.
	\end{proof}
	
	\begin{theorem}
		Se $G$ è un $p$-gruppo abeliano. Allora esistono
		e sono univocamente determinati degli interi
		positivi $r_1 \geq \cdots \geq r_s$ tali che
		$G \cong \ZZmod{p_1^{r_1}} \times \cdots
		\times \ZZmod{p_1^{r_s}}$.
	\end{theorem}
	
	\begin{proof}
		%TODO
	\end{proof} \bigskip
	
	I gruppi moltiplicativi $\ZZmulmod{n}$ sono completamente classificati ed
	è noto l'algoritmo per dedurre le loro decomposizioni in fattori invarianti,
	come mostra il fondamentale
	
	\begin{theorem}
		Sia $p$ un numero primo dispari e $k \in \NN^+$. Allora,
		$\ZZmulmod{p^k}$ e $\ZZmulmod{2p^k}$ sono gruppi
		ciclici. Per $k > 2$, vale inoltre che
		$\ZZmulmod{2^k} \cong \ZZmod{2} \times \ZZmod{2^{k-2}}$,
		mentre $\ZZmulmod{2} \cong \{e\}$ e $\ZZmulmod{4} \cong
		\ZZmod{2}$. \medskip
		
		
		In particolare per $n \geq 1$, $\ZZmulmod{n}$ è ciclico se e solo se
		$n$ è $1$, $2$, $p^k$ o $2p^k$ con $p$ primo dispari.
	\end{theorem}
	
	\begin{proof}
		Chiaramente $\ZZmulmod{2} \cong \{e\}$ e
		$\ZZmulmod{4} \cong \ZZmod{2}$ dacché hanno uno
		ordine $1$ e l'altro ordine $2$. \medskip

		
		Sia ora $p$ un numero primo dispari. Allora l'ordine di
		$G = \ZZmulmod{p^k}$ è $\varphi(p^k) = p^k - p^{k-1} =
		p^{k-1}(p-1)$. È dunque sufficiente trovare in
		$\ZZmulmod{p^k}$ un elemento $x$ di ordine
		$p^{k-1}$ e uno $y$ di ordine $p-1$ per concludere
		che $\ZZmulmod{p^k}$ è ciclico. Infatti $\MCD(p^{k-1}, p-1) = 1$ e $x$ e $y$ commutano, dunque, in tal caso, si avrebbe
		che $\ord(xy) = \abs{G}$. \medskip
		
		
		Si mostra che $1+p$ ha ordine esattamente $p^{k-1}$ in
		$\ZZmulmod{p^k}$ mostrando per induzione che
		\[ (1+p)^{p^{i-2}} \equiv 1 + p^{i-1} \pod{p^i}, \]
		per $i \geq 2$. Per $i = 2$, la tesi è banale. Si
		assuma allora l'ipotesi induttiva. \medskip
		
		Per l'ipotesi induttiva vale allora che
		$(1+p)^{p^{i-3}} = 1 + p^{i-2} + \alpha p^{i-1}$ per qualche
		$\alpha \in \ZZ$, e quindi:
		\[ (1+p)^{p^{i-2}} \equiv ((1+p)^{p^{i-3}})^p \equiv
		(1 + p^{i-2}(1 + \alpha p))^p \pod{p^i}. \]
		Applicando allora il Teorema del binomio di Newton,
		vale che:
		\[ (1+p)^{p^{i-2}} \equiv 1 + p^{i-1}(1+\alpha p) + \sum_{j=2}^{p} \binom{p}{j} p^{j(i-2)}(1 + \alpha p)^j \pod{p^i}. \]
		Si osserva che $\binom{p}{j}$ è sempre divisibile per $p$
		con $2 \leq j \leq p$, e dunque ogni termine della
		somma è divisibile per $p^i$. Infatti $j(i-2)+1 \geq i$
		per $j \geq \lfloor 1 + \frac{1}{i-2} \rfloor = 1$. Allora
		si conclude che:
		\[  (1+p)^{p^{i-2}} \equiv 1 + p^{i-1}(1+\alpha p) \equiv 1  + p^{i-1} \pod{p^i}, \]
		completando l'induzione. \medskip
		
		
		Allora vale che $(1+p)^{p^{k-1}} \equiv 1 + p^k \pod{p^{k+1}}$,
		e quindi $(1+p)^{p^{k-1}} \equiv 1 \pod{p^k}$. Pertanto
		l'ordine di $(1+p)$ è della forma $p^i$ con $i \leq k-1$.
		Si mostra che $\ord(1+p) \nmid p^{k-2}$. Infatti vale
		che:
		\[ (1+p)^{p^{k-2}} \equiv 1 + p^{k-1} \not\equiv 1 \pod{p^k}. \]
		Si conclude dunque che $\ord(1+p) = p^{k-1}$. \bigskip
		
		
		Si consideri ora l'omomorfismo $\pi : \ZZmod{p^k} \to \ZZmod{p}$ tale per cui $[x]_{p^k} \xmapsto{\pi} [x]_p$.
		Si verifica facilmente che tale mappa è ben definita,
		infatti $a \equiv b \pod{p^k} \implies a \equiv b \pod{p}$. \medskip
		
		
		Sia $\pi^*$ la restrizione di $\pi$ a $\ZZmod{p^k} \to
		\ZZmod{p}$. Si osserva che tale restrizione è ben definita dacché
		$\MCD(p^k, a) = 1 \iff \MCD(p, a) = 1$. Allora anche
		$\pi^*$ è un omomorfismo e, come $\pi$, è surgettivo.
		Poiché $\ZZmulmod{p}$ è ciclico in quanto gruppo moltiplicativo finito del campo $\ZZmulmod{p}$, allora,
		dacché $\abs{\ZZmulmod{p}} = \varphi(p) = p-1$,
		esiste $x \in \ZZmulmod{p}$ tale per cui $\ord(x) = p-1$.
		\medskip
		
		
		Dal momento che $\pi^*$ è surgettivo, esiste $y \in \ZZmulmod{p^k}$ tale per cui $p-1 = \ord(x) \mid \ord(y)$.
		Allora esiste $z \in \gen{y}$ tale per cui $\ord(z) = p-1$.
		Pertanto $(1+p)z$ ha ordine $p^{k-1}(p-1)$, e dunque
		$\ZZmulmod{p^k}$ è ciclico. \medskip
		
		
		Dacché $p$ è dispari, vale che
		$\ZZmulmod{2p^k} \cong \ZZmulmod{2} \times
		\ZZmulmod{p^k} \cong \ZZmulmod{p^k}$, e quindi
		anche $\ZZmulmod{2p^k}$ è ciclico. \medskip
		
		
		Sia ora $k > 2$. Chiaramente $[5]_{2^n} \in
		\ZZmulmod{2^k}$ dal momento che $\MCD(5, 2^k) = 1$.
		Si mostra che $\ord([5]_{2^n})$ ha ordine
		$2^{n-2}$ in $\ZZmulmod{2^k}$. Analogamente a prima
		si dimostra per induzione che per $n \geq 3$:
		\[ (1 + 4)^{2^{n-3}} \equiv 1 + 2^{n-1} \pod{2^n}. \]
		Chiaramente per $n=3$, $(1 + 4) \equiv 1 + 4 \pod{8}$.
		Si assuma ora l'ipotesi induttiva. Allora
		$(1+4)^{2^{n-4}} = 1 + 2^{n-2} + \alpha 2^{n-1}$ per qualche
		$\alpha \in \ZZ$. Vale dunque che:
		\[ (1 + 4)^{2^{n-3}} \equiv (1 + 2^{n-2} + \alpha 2^{n-1})^2
		\pod{2^n}, \]
		e quindi:
		\[ (1 + 4)^{2^{n-3}} \equiv 1 + 2^{2(n-2)} + \alpha^2 2^{2(n-1)} + 2^{n-1} + \alpha 2^n + \alpha 2^{2n-3} \pod{2^n}. \]
		Pertanto vale che $(1 + 4)^{2^{n-3}} \equiv 1 + 2^{n-1} \pod{2^n}$, concludendo l'induzione. \medskip
		
		
		Allora $(1 + 4)^{2^{k-2}} \equiv 1 + 2^k \pod{2^{k+1}}$,
		e quindi $(1 + 4)^{2^{k-2}} \equiv 1 \pod{2^k}$. Pertanto
		$\ord([5]_{2^k}) = 2^i$ con $i \leq k-2$. Tuttavia
		$(1 + 4)^{2^{k-3}} \equiv 1 + 2^{k-1} \pod{2^k}$, e quindi
		$\ord([5]_{2^k})$ vale esattamente $2^{k-2}$. \medskip
		
		
		Per il Teorema di struttura, $\ZZmulmod{2^k}$ può
		dunque essere isomorfo solo a $\ZZmod{2} \times
		\ZZmod{2^{k-2}}$ o a $\ZZmod{2^{k-1}}$. È dunque
		sufficiente mostrare che $\ZZmulmod{2^k}$ non può
		essere ciclico. Si considerino i due sottogruppi
		$H_1 = \gen{5^{2^{k-3}}}$ e $H_2 = \gen{-5^{2^{k-3}}}$.
		Entrambi i sottogruppi sono di ordine $2$ e sono
		distinti. Infatti, $5^{2^{k-3}} \equiv -5^{2^{k-3}} \pod{2^k}$
		implicherebbe $2 \cdot 5^{2^{k-3}} \equiv 0 \pod{2^k}$, e
		quindi varrebbe $2^{k-1} \mid 5^{2^{k-3}}$, \Lightning.
		Se però $\ZZmulmod{2^k}$ fosse ciclico, esisterebbe un
		unico sottogruppo di ordine $2$. Pertanto
		$\ZZmulmod{2^k}$ è isomorfo a $\ZZmod{2} \times
		\ZZmod{2^{k-2}}$, come desiderato. \bigskip
		
		
		Si consideri ora $\ZZmulmod{n}$. Se $n$ è $1$, $2$,
		$4$, $p^k$ o $2p^k$ con $p$ dispari, $\ZZmulmod{n}$ è
		ciclico per quanto dimostrato. \medskip
		
		
		Si mostra ora per induzione su $n \geq 1$ che $\ZZmulmod{n}$ è ciclico
		se e solo se $n$ è $1$, $2$,
		$4$, $p^k$ o $2p^k$ con $p$ dispari. Per $(\ZZ \quot \ZZ)^*$,
		la tesi è banale. 
		Sia ora $\ZZmulmod{n}$
		ciclico. Allora, se $n$ fosse uguale ad $ab$ con
		$\MCD(a, b) = 1$ e $a$, $b > 1$, varrebbe $\ZZmulmod{n} \cong
		\ZZmulmod{a} \times \ZZmulmod{b}$. Dal momento che
		$\ZZmulmod{a} \cong \ZZmulmod{a} \times \{e\}$,
		che a sua volta si identifica come sottogruppo di
		$\ZZmulmod{n}$, e quindi ciclico, $\ZZmulmod{a}$ stesso
		è ciclico, e analogamente anche $\ZZmulmod{b}$. Pertanto
		deve valere $\MCD(\varphi(a), \varphi(b)) = 1$. \medskip
		
		
		Dal momento che $\varphi(a)$ può essere o $1$ o un
		numero pari, così come $\varphi(b)$, si può assumere
		senza perdita di generalità che $\varphi(a) = 1$,
		e dunque che $a$ sia $1$ o $2$. Poiché $\ZZmulmod{b}$
		è ciclico e $b$ è strettamente minore di
		$n$, per il passo induttivo $b$ può essere
		$1$, $2$, $p^k$ o $2p^k$. Dal momento che $\MCD(a, b) = 1$,
		$b$ può essere solo $1$ o $p^k$, e quindi $n = 2$ o
		$n = 2 p^k$. \medskip
		
		
		Se invece $n = ab$ con $\MCD(a, b) = 1$ implica che
		uno tra $a$ e $b$ sia $1$, allora $n$ è $1$ o una
		potenza di un primo, detto $p^k$. Se $p = 2$,
		allora, per $k \geq 3$, $\ZZmulmod{2^k}$
		conterrebbe una copia isomorfa di $\ZZmulmod{8} \cong
		\ZZmod{2} \times \ZZmod{2}$. In tal caso,
		non essendo $\ZZmulmod{8}$ ciclico, nemmeno $\ZZmulmod{2^k}$
		è ciclico, e quindi $2^k$ può essere solo $1$, $2$ o
		$4$. Si conclude così la dimostrazione del teorema.
	\end{proof}
	
	\begin{remark}[La funzione $\lambda(n)$ di Carmichael]
		Si definisce la funzione $\lambda : \NN^+ \to \NN^+$ di Carmichael in
		modo tale che $\lambda(n)$ sia il più piccolo intero positivo $m$ tale
		per cui $a^m \equiv 1 \pod{n}$ per ogni $a$ coprimo con $n$. \medskip
		
		
		Grazie al Teorema sulla decomposizione di $\ZZmulmod{n}$, calcolare
		$\lambda(n)$ risulta piuttosto semplice. Infatti, $\lambda(n)$ è
		esattamente il minimo comune multiplo di tutti gli ordini di
		$\ZZmulmod{n}$. In particolare, $\lambda(n)$ divide sempre $\varphi(n)$ e
		vale l'uguaglianza se e solo se esiste un elemento $x$ in $\ZZmulmod{n}$ di
		ordine $\varphi(n)$, ossia se e solo se $\ZZmulmod{n}$ è ciclico (e dunque se
		e solo se
		$n$ è $1$, $2$, $4$, $p^k$ o $2p^k$ per $p$ primo dispari).
	\end{remark}
	
	\begin{example}[$\lambda(1000)$]
		Si calcola $\lambda(1000)$. Dal momento che $1000 = 2^3 \cdot 5^3$, vale
		che $\ZZmulmod{1000} \cong \ZZmulmod{2^3} \times \ZZmulmod{5^3}$.
		Dacché $5$ è dispari e $\varphi(5^3) = 5^3 - 5^2 = 100$, $\ZZmulmod{5^3} \cong \ZZmod{100}$, mentre
		$\ZZmulmod{2^3} \cong \ZZmod{2} \times \ZZmod{2}$. Pertanto vale che:
		\[ \ZZmulmod{1000} \cong \ZZmod{2} \times \ZZmod{2} \times \ZZmod{100}, \]
		e quindi $\lambda(1000) = \mcm(2,2,100) = 100$.
	\end{example}
\end{document}