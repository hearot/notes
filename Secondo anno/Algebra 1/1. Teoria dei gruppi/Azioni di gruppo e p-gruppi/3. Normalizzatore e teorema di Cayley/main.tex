\documentclass[12pt]{scrartcl}
\usepackage{notes_2023}

\begin{document}
	\title{Normalizzatore e teorema di Cayley}
	\maketitle
	
	\begin{note}
		Nel corso del documento per $(G, \cdot)$ si intenderà un qualsiasi gruppo.
	\end{note}
	
	Sia $X = \{ H \subseteq G \mid H \leq G \}$ l'insieme dei sottogruppi di $G$.
	Allora si può costruire un'azione $\varphi : G \to S(X)$ in modo tale che:
	\[ g \xmapsto{\varphi} \left[ H \mapsto gHg\inv \right]. \]
	Si definisce \textbf{normalizzatore} lo stabilizzatore di un sottogruppo
	$H$ (e si indica con $N_G(H)$), mentre $\Orb(H)$ è l'insieme dei \textbf{coniugati}
	di $H$. In particolare $N_G(H)$ è il massimo sottogruppo per inclusione in cui $H$
	è normale. \medskip
	
	
	Si osserva ora in modo cruciale che $H \nsgeq G$ se e solo se
	$\Orb(H) = \{H\}$, e quindi se e solo se $N_G(H) = G$. Analogamente si
	osserva che $H$ è normale se e solo se:
	\[ H = \bigcup_{h \in H} \Cl(h). \]
	Tramite la stessa azione $\varphi$ possiamo illustrare un importante relazione
	tra gli stabilizzatori, dettata dalla:
	
	\begin{proposition}
		Sia $x \in X$ e sia $g \in G$. Allora vale che $\Stab(g \cdot x) = g \Stab(x) g\inv$,
		e i coniugati di $\Stab(x)$ sono esattamente altri stabilizzatori.
	\end{proposition}
	
	\begin{proof}
		Si osserva che se $ghg\inv \in g\Stab(x)g\inv$, allora:
		\[ (ghg\inv) \cdot (g \cdot x) = gh \cdot x = g \cdot x \implies ghg\inv \in \Stab(g \cdot x), \]
		e viceversa che se $h \in \Stab(g \cdot x)$:
		\[ (g\inv h g) \cdot x = g\inv \cdot (h \cdot (g \cdot x)) = (g\inv g) \cdot x = x \implies g\inv h g \in \Stab(x) \implies h \in g \Stab(x) g\inv, \]
		da cui si deduce che $\Stab(g \cdot x) = g \Stab(x) g\inv$.
	\end{proof}
	
	Da questa proposizione segue immediatamente il seguente:
	
	\begin{corollary}
		Sia $\varphi$ un'azione transitiva. Allora tutti gli stabilizzatori sono
		coniugati tra loro.
	\end{corollary}
	
	\begin{proof}
		Siano $x$ e $y \in X$. Poiché $\varphi$ è transitiva, esiste un'unica orbita
		e dunque esiste $g \in G$ tale per cui $g \cdot y = x$. Allora
		$\Stab(x) = \Stab(g \cdot y) = g \Stab(y) g\inv$.
	\end{proof}
	
	Infine, si verifica una proprietà dei sottogruppi coniugati:
	
	\begin{proposition}
		Se $H$ e $K$ sono coniugati, allora sono in particolare anche isomorfi.
	\end{proposition}
	
	\begin{proof}
		Poiché $H$ e $K$ sono coniugati, esiste un $g \in G$ tale per cui
		$K = gHg\inv$.
		Un isomorfismo tra i due gruppi è allora naturalmente dato dall'azione di
		coniugio tramite $g$, ossia dall'omomorfismo $\zeta : H \to K$
		tale per cui $h \xmapsto{\zeta} ghg\inv$. Tale mappa è sicuramente un omomorfismo;
		è ben definita e surgettiva perché i gruppi sono coniugati ed è iniettiva
		perché $ghg\inv = e \implies h = e$ (e quindi $\Ker \zeta = \{e\}$).
	\end{proof}
	
	\bigskip
	
	Si illustra adesso un risultato principale della teoria dei gruppi che mette in
	relazione ogni gruppo con il proprio gruppo di bigezioni, ed ogni gruppo finito con i
	sottogruppi dei gruppi simmetrici.
	
	\begin{theorem}[di Cayley]
		Ogni gruppo è isomorfo a un sottogruppo del suo gruppo di bigezioni.
		In particolare, ogni gruppo finito $G$ è isomorfo a un sottogruppo di un gruppo
		simmetrico.
	\end{theorem}
	
	\begin{proof}
		Si consideri l'azione\footnote{Tale azione prende il nome di \textbf{rappresentazione regolare a sinistra} o \textbf{\textit{embedding} di Cayley}.
		Si può definire un'azione analoga a destra ponendo $g \mapsto \left[ h \mapsto hg\inv \right]$,
		costruendo dunque una \textit{rappresentazione regolare a destra}.} $\varphi : G \to S(G)$ tale per cui:
		\[ g \xmapsto{\varphi} \left[ h \mapsto gh \right]. \]
		Si mostra che $\varphi$ è fedele\footnote{L'azione $\varphi$ è molto
		più che fedele; è infatti innanzitutto libera.}. Sia infatti $\varphi(g) = \Id$; allora
		vale che $ge = e \implies g = e$. Quindi $\Ker \varphi$ è banale, e per il
		Primo teorema di isomorfismo vale che:
		\[ G \cong \Im \varphi \leq S(G). \]
		Se $G$ è finito, $S(G)$ è isomorfo a $S_n$, dove $n := \abs{G}$, e quindi
		$\Im \varphi$ è a sua volta isomorfo a un sottogruppo di $S_n$, da cui
		la tesi.
	\end{proof}
	
	A partire dall'\textit{embedding} di Cayley si può dimostrare un risultato
	sui gruppi di ordine $2d$ con $d$ dispari:
	
	\begin{proposition}
		Sia $G$ un gruppo di ordine $2d$ con $d$ dispari. Allora $G$ ammette
		un sottogruppo $H$ di ordine $d$.
	\end{proposition}
	
	\begin{proof}
		Consideriamo l'\textit{embedding} di Cayley di $G$. In particolare,
		poiché $S(G) \cong S_{2d}$, possiamo identificare $S(G)$ con
		$S_{2d}$, studiando tale \textit{embedding} direttamente su quest'ultimo
		sottogruppo. \medskip
		
		
		Sia allora $\varphi : G \to S_{2d}$ la composizione $\xi \circ \lambda$ dove
		$\xi$ è un isomorfismo tra $S(G)$ e $S_{2d}$ e $\lambda : G \to S(G)$ è l'\textit{embedding} di Cayley associato a $G$.
		Si osserva che $\varphi\inv(\Ad{2d}) = \{ g \in G \mid \varphi(g) \in \Ad{2d} \} =
		\Ker (\pi_{\Ad{2d}} \circ \varphi)$. Per il Primo teorema di isomorfismo vale
		che:
		\[ G \quot \Ker (\pi_{\Ad{2d}} \circ \varphi) \cong \Im (\pi_{\Ad{2d}} \circ \varphi) \leq
		S_{2d} \quot {\Ad{2d}} \cong \{\pm 1\}, \]
		e quindi\footnote{
			Si può arrivare alla stessa conclusione mediante un ragionamento leggermente
			diverso. Se si considera $K = \varphi(G)$, $K \cap \Ad{2d} = \varphi(G) \cap \Ad{2d}$ è esattamente $\Ker(\restr{\sgn}{\varphi(G)})$, e quindi
			$[K : (K \cap \Ad{2d})] \in \{1, 2\}$. Pertanto, dal momento che $\varphi$
			è un isomorfismo tra $G$ e $\Im \varphi = \varphi(G)$, $\varphi\inv(\Ad{2d}) =
			\varphi\inv(\Ad{2d} \cap \varphi(G))$ può avere solo indice $1$ o $2$, ed
			ha indice $1$ se e solo se $\varphi(G) \subseteq \Ad{2d}$.
		} $[G : \Ker (\pi_{\Ad{2d}} \circ \varphi)]$ vale $1$ o $2$. \medskip
		
		
		Se
		$[G : \Ker (\pi_{\Ad{2d}} \circ \varphi)]$ fosse uguale a $1$,
		varrebbe che $G = \Ker (\pi_{\Ad{2d}} \circ \varphi) = \varphi\inv(\Ad{2d})$, e quindi che
		$\varphi(G) \subseteq \Ad{2d}$. Si mostra che ciò è impossibile esibendo un elemento
		$g \in G$ tale per cui $\varphi(g)$ sia dispari. Dacché $2 \mid \abs{G}$,
		esiste $g \in G$ con $\ord(g) = 2$ per il teorema di Cauchy. Allora la
		decomposizione in cicli di $\varphi(g)$ sarà la stessa di $\lambda(g)$, ossia\footnote{
			In generale, se $\ord(g) = k$, la sua decomposizione tramite $\lambda$ sarà:
			\[ (g_1, g g_1, \ldots, g^{k-1} g_1)  (g_2, g g_2, \ldots, g^{k-1} g_2) \cdots (g_s, g g_s, \ldots, g^{k-1} g_s), \]
			con $s = 2d/k$, ossia $\lambda(g)$ sarà prodotto di $2d/k$ $k$-cicli.
		}:
		\[ \lambda(g) = (g_1, g g_1) (g_2, g g_2) \cdots (g_d, g g_d). \]
		Poiché $\lambda(g)$ è allora prodotto di $d$ trasposizioni, $\lambda(g)$ è dispari,
		e così pure $\varphi(g)$. Pertanto $\varphi(g) \notin \Ad{2d} \implies
		[G : \Ker (\pi_{\Ad{2d}} \circ \varphi)] = 2$, e quindi
		$\abs{ \Ker (\pi_{\Ad{2d}} \circ \varphi)} = d$, concludendo la dimostrazione.
	\end{proof} \bigskip
	
	Si presentano adesso due risultati interessanti legati ai sottogruppi normali di
	un gruppo $G$.
	
	\begin{proposition}
		Sia\footnote{
			Si osserva che questa proposizione risulta superflua se si dimostra,
			come succede sul finire di questo documento, che per il più piccolo
			primo $p$ che divide $\abs{G}$, i sottogruppi corrispondenti di
			indice $p$ sono normali. Vista tuttavia la semplicità della dimostrazione,
			si è preferito lasciarla per motivi didattici.
		} $H \leq G$. Allora, se $[G : H] = 2$, $H$ è normale in $G$.
	\end{proposition}
	
	\begin{proof}
		Poiché $[G : H] = 2$, le uniche classi laterali sinistre rispetto ad $H$ in
		$G$ sono $H$ e $gH = G \setminus H$, dove $g \notin H$. Analogamente esistono
		due sole classi laterali destre, $H$ e $Hg = G \setminus H$. In particolare
		$gH$ deve obbligatoriamente essere uguale a $Hg$, e quindi $gHg\inv = H$, da
		cui la tesi.
	\end{proof}
	
	\begin{proposition}
		Siano $K \leq H \leq G$. Allora, se $H$ è normale in $G$ e $K$ è caratteristico
		in $H$, $K$ è normale in $G$.
	\end{proposition}
	
	\begin{proof}
		Sia $\varphi_g \in \Inn(G)$. Poiché $H$ è normale in $G$, $\varphi_g(H) = H$. Pertanto
		si può considerare la restrizione di $\varphi_g$ su $H$, $\restr{\varphi_g}{H}$.
		In particolare $\restr{\varphi_g}{H}$ è un automorfismo di $\Aut(H)$, e quindi,
		poiché $K$ è caratteristico in $H$, $\restr{\varphi_g}{H}(K) = K$, da cui si
		deduce che $gKg\inv = K$ per ogni $g \in G$.
	\end{proof}
	
	Si illustra adesso un risultato riguardante l'esistenza di sottogruppi normali in $G$:
	\begin{theorem}[di Poincaré]
		Sia $H$ un sottogruppo di $G$ di indice $n$. Allora esiste sempre un sottogruppo
		$N$ di $G$ tale per cui:
		\begin{enumerate}[(i)]
			\item $N$ è normale in $G$,
			\item $N$ è contenuto in $H$,
			\item $n \mid [G : N] \mid n!$.
		\end{enumerate}
	\end{theorem}
	
	\begin{proof}
		Si consideri l'azione $\varphi : G \to S(G \quot H)$ tale per cui
		$g \xmapsto{\varphi} [kH \mapsto gkK]$. Tale azione è sicuramente
		ben definita dal momento che $kH = k'H \implies gkH = gk'H$. Si
		studia $N := \Ker \varphi$. Chiaramente $N$ è normale in $G$, e si
		verifica facilmente che $N$ è contenuto anche in $H$, infatti, se
		$n \in N$, allora:
		\[ H = \varphi(n)(H) = nH \implies n \in H. \]
		Poiché $G \quot N$ è isomorfo a $\Im \varphi \leq S(G \quot H)$,
		$[G : N] \mid \abs{S(G \quot H)} = \abs{S_n} = n!$ considerando che
		$S(G \quot H) \cong S_n$. Dal momento allora che $N$ è un sottogruppo
		di $H$, vale che:
		\[ [G : N] = [G : H] [H : N] = n [H : N], \]
		e quindi $n \mid [G : N]$. Si è dunque esibito un sottogruppo $N$ con
		le proprietà indicate nella tesi.
	\end{proof}
	
	Dal precedente teorema sono immediati i seguenti due risultati:
	
	\begin{corollary}
		Sia $H$ un sottogruppo di $G$ con indice $n$. Se $n! < \abs{G}$ e
		$n>1$, allora $G$ non è semplice.
	\end{corollary}
	
	\begin{corollary}
		Sia $H$ un sottogruppo di $G$ con indice $p$, dove $p$ è il più piccolo
		primo che divide $n = \abs{G}$. Allora $H$ è normale.
	\end{corollary}
	
	\begin{proof}
		Per il Teorema di Poincaré, esiste un sottogruppo $N$ di $H$ tale per cui
		$N$ sia normale e $p \mid [G : N] \mid p!$ con $p = [G : H]$. In particolare
		$[G : N]$ deve dividere anche $n$, e quindi $[G : N]$ deve dunque
		dividere $\MCD(p!, n)$, che è, per ipotesi, $p$ stesso. Si conclude dunque
		che $[G : N] = p = [G : H]$, e quindi che $N = H$, ossia che $H$ stesso
		è normale.
	\end{proof}
	
	\begin{example} [Tutti i gruppi di ordine $15$ sono ciclici]
		Sia\footnote{
			In realtà $15$ è un numero molto speciale, in quanto è prodotto
			di due primi distinti ($3$ e $5$) tali per cui $3$ non divida
			$5-1 = 4$. In generale, ogni gruppo di ordine $pq$ con
			$p$ e $q$ primi tali per cui $p<q$ e $p \nmid q-1$ è ciclico.
		} $G$ un gruppo di ordine $15$. Per il teorema di Cauchy esistono
		due elementi $h$ ed $k$, uno di ordine $3$ e l'altro di ordine $5$.
		In particolare, si consideri $K = \gen{k}$; poiché $\abs{K} = 5$,
		$[G : K] = 3$, il più piccolo primo che divide $15$. Pertanto
		$K$ è normale per il corollario di sopra. \medskip
		
		
		Poiché $K$ è normale, si può considerare la restrizione $\iota :
		\Inn(G) \to \Aut(K)$ tale per cui $\varphi_g \xmapsto{\iota} \restr{\varphi_g}{K}$.
		Dal momento che $K$ è ciclico, $\Aut(K) \cong \Aut(\ZZ \quot 5 \ZZ) \cong
		(\ZZ \quot 5 \ZZ)^* \cong \ZZ \quot 4 \ZZ$. Quindi $[G : \Ker \iota]$ deve
		dividere sia $4$ che $15$; dal momento che $\MCD(4, 15) = 1$, $[G : \Ker \iota] = 1$,
		e quindi che $\iota$ è l'omomorfismo banale. Poiché $\iota$ è banale, $K$ è
		un sottogruppo di $Z(G)$. \medskip
		
		
		In particolare $[G : Z(G)] \mid [G : K] = 3$, e quindi in particolare
		$G \quot Z(G)$ è ciclico, da cui si deduce che $G$ è abeliano. Infine,
		dal momento che $\MCD(3, 5) = 1$ e $h$ e $k$ commutano,
		$hk$ è un elemento di ordine $15$, e dunque $G$ è ciclico.
	\end{example}
	
	Si illustrano infine due risultati interessanti sui coniugati di $G$:
	
	\begin{proposition}
		Sia $H \leq G$. Allora
		\[ \bigcup_{g \in G} gHg\inv = G \iff H = G. \]
	\end{proposition}
	
	\begin{proof}
		Se $H = G$, allora $gGg\inv = G$ e quindi l'identità è vera. Viceversa,
		$gHg\inv = kHk\inv \iff g N_G(H) = k N_G(H)$. Preso dunque un'insieme
		$\rotations$ di rappresentanti per ogni classe in $G \quot N_G(H)$,
		vale che:
		\[ \bigcup_{g \in \rotations} gHg\inv = G. \]
		In ogni $gHg\inv$ ci sono $\abs{H}$ elementi distinti, e quindi, poiché
		$\abs{\rotations} = \abs{G \quot N_G(H)}$, deve valere la seguente
		disuguaglianza:
		\[ \abs{\bigcup_{g \in \rotations} gHg\inv} \leq \abs{G \quot N_G(H)} \abs{H} \leq
			\frac{\abs{G}}{\abs{N_G(H)}} \abs{H} \leq \abs{G}, \]
		dove si è usato che $H \leq N_G(H)$.
		Se $\abs{G \quot N_G(H)}$ non valesse $1$, ci sarebbe più ripetizioni di $e$
		all'interno dell'unione, e quindi la prima disuguaglianza sarebbe stretta,
		\Lightning. Quindi $N_G(H) = G \implies H \nsgeq G$. Allora la disuguaglianza
		si riscrive come:
		\[ \abs{G} = \abs{\bigcup_{g \in \rotations} gHg\inv} \leq \abs{H} \leq \abs{G}, \]
		da cui si ricava che necessariamente $\abs{H} = \abs{G} \implies H = G$.
	\end{proof}
	
	\begin{proposition}
		Sia $\varphi$ un'azione transitiva di $G$ su $X$.
		Allora esiste sempre un $g \in G$ tale per cui $\Fix(g) = \emptyset$,
		se $\abs{X} \geq 2$.
	\end{proposition}
	
	\begin{proof}		
		Se $g$ non fissa alcun punto di $X$, allora $g \notin \bigcup_{x \in X} \Stab(x)$; pertanto tale $g$ esiste se e solo se $\bigcup_{x \in X} \Stab(x) \neq G$. Poiché tali sottogruppi sono tutti coniugati, scelto $u \in U$ vale
		che:
		\[ \bigcup_{x \in X} \Stab(x) = \bigcup_{g \in G} g \Stab(u) g\inv. \]
		Si conclude dunque che tale $g$ esiste se e solo se $\Stab(u) \neq G$.
		Se $\Stab(u)$ fosse uguale a $G$, allora, per il Teorema orbita-stabilizzatore,
		varrebbe che $\abs{\Orb(u)} = 1$; tuttavia $\varphi$ è transitiva e quindi
		$X = \Orb(u) \implies \abs{X} = \abs{\Orb(u)} = 1$, \Lightning. Pertanto
		$\Stab(u) \neq G$, e dunque l'unione non ricopre tutto $G$, concludendo
		la dimostrazione.
	\end{proof}
\end{document}