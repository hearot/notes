\documentclass[12pt]{scrartcl}
\usepackage{notes_2023}

\begin{document}
	\title{Il gruppo delle permutazioni}
	\maketitle
	
	\begin{note}
		Nel corso del documento con $X_n$ si indicherà l'insieme
		$\{1, \ldots, n\}$ e con $G$ un qualsiasi gruppo.
	\end{note}

	Si definisce brevemente il gruppo delle permutazioni $S_n$ come il gruppo
	delle bigezioni su $G$, ossia $S(X_n)$. Si deduce facilmente che
	$\abs{S_n} = n!$ dal momento che vi sono esattamente $n!$ scelte possibili
	per costruire una bigezione da $X_n$ in $X_n$ stesso. \medskip
	
	
	Come è noto, ogni $\sigma \in S_n$ può scriversi come prodotto di cicli
	disgiunti. Di seguito si introduce un modo formale per descrivere questi
	cicli. \medskip
	
	
	Si consideri l'azione di $\gen{\sigma}$ su $X_n$ univocamente determinata
	da $\sigma \cdot x = \sigma(x)$. Allora i cicli di $\sigma$ sono esattamente
	le orbite di $\sigma$ ordinate nel seguente modo:
	\[ \Orb(x) = \{ x, \sigma(x), \dots, \sigma^m(x) \}. \]

	
	Si osserva che in effetti tutti gli elementi di $X$ sono considerati nella
	scrittura delle orbite dal momento che tali orbite inducono una partizione
	di $X$ (infatti sono classi di equivalenza). Si definisce inoltre una
	permutazione \textit{ciclo} se esiste al più un'unica orbita di cardinalità diversa
	da $1$ e si dice \textit{lunghezza del ciclo} la cardinalità di tale orbita (o se non esiste, si dice che ha lunghezza unitaria). Due cicli si dicono disgiunti se almeno uno dei due è l'identità o se le loro uniche orbite non banali hanno intersezione nulla (e in entrambi i casi, commutano). Per ogni $k$-ciclo esistono esattamente $k$ scritture
	distinte (in funzione dell'elemento iniziale del ciclo). \medskip
	
	
	Pertanto si deduce facilmente che ogni permutazione $\sigma$ è prodotto
	di cicli disgiunti in modo unico (a meno della scelta del primo elemento
	dell'orbita). Poiché allora ogni $n$-ciclo è generato dalla composizione
	di $n-1$ trasposizioni ($2$-cicli) e ogni permutazione è prodotto di cicli,
	$S_n$ è generato dalle trasposizioni. Infatti:
	\[ (a_1, \dots, a_i) = (a_1, a_i) \circ (a_1, a_{i-1}) \circ \cdots \circ (a_1, a_2), \]
	o altrimenti:
	\[ (a_1, \dots, a_i) = (a_1, a_2) \circ (a_2, a_3) \circ \cdots \circ (a_{i-1}, a_i), \]
	da cui si deduce che la scrittura come prodotto di
	trasposizioni non è unica. Ciononostante viene sempre mantenuta la parità
	del numero di trasposizioni impiegate. \medskip
	

	Per questo motivo la mappa $\sgn : S_n \to \{\pm 1\}$ che vale $1$ sulle
	permutazioni con numero pari di trasposizioni impiegabili e $-1$ sul resto
	è ben definita. Inoltre questa mappa è un omomorfismo di gruppi, e si
	definisce $\An := \Ker \sgn$ come il sottogruppo di $S_n$ delle permutazioni
	pari, detto anche \textit{gruppo alterno}. La classe laterale $(1, 2) \An$
	rappresenta invece le permutazioni dispari. \medskip
	
	
	In particolare, se $\sigma_k$ è un $k$-ciclo, $\sgn(\sigma_k) = (-1)^{k-1}$ e $\ord(\sigma_k) = k$. Si osserva inoltre che vi sono esattamente $\binom{n}{k} \frac{k!}{k} =
	\binom{n}{k} (k-1)!$ $k$-cicli in $S_n$ e che in generale l'ordine
	di una permutazione è il minimo comune multiplo degli
	ordini dei suoi cicli. \medskip
	

	Si definisce \textit{tipo} di una permutazione $\sigma$ la sua decomposizione
	in cicli disgiunti a meno degli elementi presenti nei cicli. Sia $\sigma$
	tale per cui:
	\[ \sigma = (a_1, a_2, \ldots, a_{k_1}) (b_1, \ldots, b_{k_2}) \cdots (c_1, \ldots, c_{k_i}), \]
	allora vale la seguente relazione sul coniugio:
	\[ \tau \sigma \tau\inv = (\tau(a_1), \tau(a_2), \ldots, \tau(a_{k-1})) (\tau(b_1), \ldots, \tau(b_{k_2})) \cdots (\tau(c_1), \ldots, \tau(c_{k_i})). \]
	
	A partire da ciò vale il seguente risultato:
	\begin{proposition}
		Due permutazioni $\sigma_1$, $\sigma_2$ sono \textit{coniugabili}
		(ossia appartengono alla stessa classe di coniugio) se e solo se
		hanno lo stesso tipo.
	\end{proposition}
	
	\begin{proof}
		Dalla seguente identità, se $\sigma_1$ è coniugata rispetto a
		$\sigma_2$, sicuramente le due permutazioni dovranno avere lo stesso
		tipo. Analogamente, se le due permutazioni hanno lo stesso tipo,
		si può costruire $\tau$ che associ ogni elemento di
		un ciclo di $\sigma_1$ a un elemento nella stessa posizione in un ciclo
		di $\sigma_2$ della stessa lunghezza in modo tale che $\tau$ rimanga
		una permutazione di $S_n$ e che valga $\sigma_2 = \tau \sigma_1 \tau\inv$.
	\end{proof}
	
	Come corollario di questo risultato, se $m_1$ rappresenta il numero di $1$-cicli di $\sigma$, $m_2$ quello dei suoi $2$-cicli, fino a $m_k$, vale il seguente risultato:
	\[ \abs{\Cl(\sigma)} = \frac{n!}{m_1! \, 1^{m_1} \, m_2! \, 2^{m_2} \cdots m_k! \, k^{m_k}}, \]
	e in particolare esistono tante classi di coniugio quante partizioni di $n$. \medskip
	
	
	Si osserva infine che se $\tau_1 \sigma \tau_1\inv = \tau_2 \sigma \tau_2\inv = \rho$, allora:
	\[ \tau_1\inv (\tau_2 \sigma \tau_2\inv) \tau_1 = \tau_1\inv \rho \tau_1 = \sigma, \]
	per cui $\tau_1\inv \tau_2 \in Z_{S_n}(\sigma)$ dacché $(\tau_1\inv \tau_2) \sigma = \sigma (\tau_1\inv \tau_2)$. Allora $\tau_1 \in \tau_2 Z_{S_n}(\sigma)$. \medskip
	
	
	Infine, se $H \leq S_n$, $H$ è normale in $S_n$ se e solo se per ogni tipo $H$ contiene
	tutte le permutazioni di quel tipo o nessuna.
\end{document}