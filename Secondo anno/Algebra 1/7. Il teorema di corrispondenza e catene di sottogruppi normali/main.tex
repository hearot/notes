\documentclass[12pt]{scrartcl}
\usepackage{notes_2023}

\begin{document}
	\title{Il teorema di corrispondenza e catene di sottogruppi normali}
	\maketitle

	Si illustra adesso un teorema che mette in corrispondenza
	i sottogruppi di $G \quot H$ con i sottogruppi di $G$
	che contengono $H$. Benché questo teorema possa sembrare
	a prima vista di poca utilità, in realtà svela alcune
	proprietà che hanno portato allo sviluppo della celebre
	teoria di Galois. Non solo, guardando anche nelle piccole
	applicazioni, il teorema di corrispondenza permette di
	contare molto facilmente i sottogruppi di $G \quot H$,
	nonché di dimostrare l'esistenza di una catena di
	$p$-sottogruppi normali contenente tutti gli ordini
	possibili per un $p$-gruppo.
	
	\begin{theorem}[di corrispondenza]
		Sia $H$ un sottogruppo normale di $G$. Allora
		la proiezione al quoziente $\pi_H : G \to G \quot H$
		induce una bigezione tra l'insieme
		\[ X = \{ K \leq G \mid H \subseteq K \} \]
		dei sottogruppi di $G$ che contengono $H$ e l'insieme
		\[ Y = \{ K' \leq G \quot H \} \]
		dei sottogruppi di $G \quot H$. Tale bigezione preserva
		la normalità di un gruppo e il suo indice, ossia:
		\begin{itemize}
			\item $K \nsgeq G \iff K' \nsgeq G \quot H$,
			\item $\left[ G : K \right] = \left[ G \quot H : K' \right]$,
		\end{itemize}
		dove $K \in X$ e $K' \in Y$ sono in corrispondenza biunivoca
		mediante $\pi_H$.
	\end{theorem}

	\begin{proof}
		Sia $\alpha : X \to Y$ definita nel seguente modo:
		\[ K \xmapsto{\alpha} \pi_H(K), \]
		dove si osserva che $\pi_H(K) = \{ kH \mid k \in K \} = K \quot H \leq G \quot H$.
		Si definisce analogamente $\beta : Y \to X$ in modo tale che:
		\[ K' \xmapsto{\beta} \pi_H\inv(K'). \]
		Le due mappe sono entrambe ben definite (infatti $\pi_H\inv(K')$ è sempre un sottogruppo di $G$ e contiene
		sempre $H$, dacché $H \in K'$, essendo l'identità di $G \quot H$).
		È dunque sufficiente mostrare che vale $\beta \circ \alpha = \Id_X$ e che $\alpha \circ \beta = \Id_Y$. \bigskip
		
		
		Siano quindi $K \in X$ e $K' \in Y$. Chiaramente
		$\pi_H(\pi_H\inv(K')) = K'$, dal momento che $\pi_H$ è
		surgettiva; dunque $\alpha \circ \beta = \Id_Y$. Inoltre
		$\pi_H \inv (\pi_H(K)) = \pi_H\inv (K \quot H) = \{ g \in G \mid gH \in K \quot H \} = K$\footnote{
			Infatti se $gH=kH$ con $k \in K$, esiste un $h \in H$ tale per cui $g=kh$.
			Dal momento che $H \subseteq K$, $g$ è dunque un elemento di
		$K$.}, da cui $\beta \circ \alpha = \Id_X$. Quindi $X$ e $Y$ sono in corrispondenza biunivoca
		tramite $\alpha$ e $\beta$. \bigskip
		
		
		Rimane da dimostrare che $\alpha$ e $\beta$ preservano
		la normalità e l'indice di sottogruppo. Se $K \nsgeq G$,
		allora chiaramente $K' = K \quot H \nsgeq G \quot H$
		(infatti $gH \, kH \, g\inv H = (gkg\inv) H$, dove
		$gkg\inv \in K$ per ipotesi di normalità). Sia
		ora $K' \nsgeq G \quot H$. Allora, se $k \in K$,
		$gH \, kH \, g\inv H = (gkg\inv) H$, e per ipotesi
		di normalità deve esistere $k' \in K$ tale per cui
		$(gkg\inv) H = k'H$, e quindi deve esistere
		$h \in H$ tale per cui $gkg\inv = k'h$. Dal momento
		che $H \subseteq K$, $gkg\inv \in K$, e quindi
		$K \nsgeq G$. \bigskip
		
		
		Per mostrare che l'indice di sottogruppo si preserva
		si dimostra che esiste lo stesso numero di classi
		laterali in $G \quot K$ e $(G \quot H) \quot (K \quot H)$.
		Pertanto è sufficiente mostrare che:
		\[
			xK = yK \iff xH (K \quot H) = yH (K \quot H), \qquad x, y \in G.
		\]
		Infatti, in tal caso vi sarebbero esattamente $[G : K]$
		classi laterali in $(G \quot H) \quot (K \quot H)$.
		Si consideri ora la classe laterale $xH(K \quot H)$:
		\[
			xH(K \quot H) = \{ xHkH \mid k \in K \} = \{ (xk) H \mid k \in K \},
		\]
		dove nell'ultima uguaglianza si è impiegata la normalità
		di $H$ in $G$ (altrimenti il prodotto non sarebbe ben
		definito).
		Analogamente $yH(K \quot H) = \{ (yk)H \mid k \in K \}$. 
		Quindi, se $xH (K \quot H) = yH (K \quot H)$, allora $xH = (yk)H$, con $k \in K$.
		Allora $x = ykh$ con $h \in H$. Poiché $H \subseteq K$, si deduce
		quindi che $xK = yK$. Infine, se $xK = yK$, esiste
		$k \in K$ tale per cui $x=yk$. Allora:
		\[ xH(K \quot H) = yH \, kH (K \quot H) = yH(K \quot H), \]
		da cui la tesi.
	\end{proof}
\end{document}