\documentclass[12pt]{scrartcl}
\usepackage{notes_2023}

\begin{document}
	\title{Il teorema di Cauchy}
	\maketitle
	
	\begin{note}
		Nel corso del documento per $(G, \cdot)$ si intenderà un qualsiasi gruppo.
	\end{note}
	
	Si dimostra in questo documento, per ben due volte, un inverso parziale
	del teorema di Lagrange, il celebre teorema di Cauchy. Tale teorema
	asserisce che se $p$ è un numero primo che divide l'ordine di $G$,
	allora esiste un sottogruppo $H$ di $G$ di ordine $p$. \medskip
	
	
	Si mostra innanzitutto che il teorema vale per gruppi abeliani.
	
	\begin{theorem}[di Cauchy per gruppi abeliani]
		Sia $G$ un gruppo abeliano finito. Se $p$ divide $\abs{G}$, allora
		esiste $H \leq G$ tale per cui $\abs{H} = p$.
	\end{theorem}
	
	\begin{proof}
		Sia $\abs{G} = pn$ con $n \in \NN^+$. Si dimostra per induzione su $n$ la
		validità della tesi. Se $n = 1$, allora $G$ stesso è un sottogruppo di ordine $p$, completando il passo base. \medskip
		
		
		Sia allora $n > 1$ e si ipotizzi allora che tutti i gruppi tali che $\abs{G} = pk$ con $k < n$, $k \in \NN^+$ ammettano un sottogruppo di ordine $p$. Sia $h \in G$, $h \neq e$ (questo $h$ sicuramente esiste, dal momento che $p > 1$). Se $p \mid o(h)$, allora
		$\Cyc{h^{\nicefrac{o(h)}{p}}}$ è un sottogruppo di $G$ di ordine $p$. Altrimenti,
		si consideri $H = \Cyc{h}$. \medskip
		
		
		Dal momento che $G$ è abeliano, $H$ è normale, e dunque si può considerare il gruppo quoziente $G \quot H$. Poiché $p \nmid o(h) = \abs H$ e $p$ divide $\abs G$,
		$p$ divide anche $\abs{G \quot H}$ per il teorema di Lagrange. Inoltre, poiché
		$o(h) > 1$ (infatti $h \neq e$), $\abs{G \quot H} < \abs{G}$. Per l'ipotesi
		induttiva, allora, esiste un sottogruppo $T$ di ordine $p$ di $G \quot H$. Poiché
		$T$ è di ordine $p$, $T$ è ciclico, e quindi esiste $t \in G$, $t \neq e$ tale per cui
		$T = \Cyc{tH}$. \medskip
		
		
		Si mostra adesso che $p \mid o(t)$. Si consideri la proiezione al quoziente $\pi : G \to G \quot H$ tale per cui:
		\[ g \xmapsto{\pi} gH. \]
		Allora $p = o(tH) \mid o(t)$, dal momento che $eH=\pi(t^{o(t)})=(tH)^{o(t)}$.
		Pertanto, come prima, si può estrarre da $\Cyc{t}$ un sottogruppo di $G$
		di ordine $p$, concludendo il passo induttivo.
	\end{proof} \bigskip
	
	Di seguito si dimostra il teorema di Cauchy in generale.
	
	\begin{theorem}[di Cauchy]
		Sia $G$ un gruppo finito. Se $p \mid \abs{G}$, allora
		esiste $H \leq G$ tale per cui $\abs{H} = p$.
	\end{theorem}
	
	\begin{proof}
		Sia $\abs{G} = pn$ con $n \in \NN^+$. Si dimostra la tesi per induzione.
		Se $n = 1$, $G$ stesso è sottogruppo di ordine $p$, completando il passo
		base. Sia ora $n > 1$ e si assuma che ogni sottogruppo di ordine $pk$ con
		$k < n$ ammetta un sottogruppo di ordine $p$. \medskip
		
		
		Se esiste $H \lneq G$ tale per cui $p$ divide $\abs H$, allora $H$, e quindi
		anche $G$, ammette un sottogruppo di ordine $p$ per l'ipotesi induttiva.
		Si assuma dunque che non esiste alcun sottogruppo proprio $H < G$ tale
		per cui $p$ divide $\abs H$. Si consideri la formula delle classi
		di coniugio:
		\[ \abs G = \abs{Z(G)} + \sum_{g \in \mathcal{R} \setminus Z(G)} \frac{\abs G}{\abs{Z_G(g)}}, \]
		dove $\mathcal{R}$ è un insieme dei rappresentanti delle classi di coniugio
		di $G$. Se $g \in \mathcal{R} \setminus Z(G)$, allora $Z_G(g)$ è un sottogruppo
		proprio di $G$, e quindi, per ipotesi, $p$ non divide $\abs{Z_G(g)}$; e quindi
		$p$ divide ancora $\nicefrac{\abs{G}}{\abs{Z_G(g)}}$ (e quindi il secondo termine
		del secondo membro). Allora, prendendo
		l'identità modulo $p$, si deduce che:
		\[ \abs{Z(G)} \equiv 0 \pod p. \]		
		Poiché $Z(G)$ è un sottogruppo di $G$, se valesse $Z(G) < G$, si violerebbero
		le ipotesi iniziali. Pertanto deve necessariamente valere $Z(G) = G$, e quindi
		$G$ è abeliano. Pertanto $G$ ammette un sottogruppo di ordine $p$ per il Teorema
		di Cauchy per i gruppi abeliani; completando il passo induttivo.
	\end{proof} \smallskip
	
	
	Si mostra inoltre una dimostrazione alternativa del teorema di Cauchy (più immediata
	e facile da ricordare), basata su una particolare costruzione.
	
	\begin{proof}[Dimostrazione alternativa]
		Si consideri l'insieme $S$, dove:
		\[ S = \{ (a_1, \ldots, a_p) \in G^p \mid a_1 \cdots a_p = e \}. \]
		Dimostrando che esiste un elemento $h \in G$ diverso dall'identità tale
		per cui $(h, \ldots, h) \in S$, si mostra che $h^p = e$, e dunque che
		$o(h) = p$ (infatti $h \neq e$). Allora, in tal caso, $\Cyc{h}$ è
		un sottogruppo di $G$ di ordine $p$, e si dimostra la tesi. \medskip
		
		
		Si ipotizzi che tale elemento $h$ non esisti. Si consideri l'azione $\varphi$
		di $\ZZ \quot p\ZZ$ su $S$ univocamente determinata\footnote{$\ZZ \quot p\ZZ$ è infatti generato da $1$.} dalla relazione:
		\[ 1 \xmapsto{\varphi} \left[ (a_1, a_2, \ldots, a_p) \mapsto (a_2, \ldots, a_p, a_1) \right]. \]
		In particolare $m \cdot (a_1, \ldots, a_p)$ restituisce una $p$-upla ottenuta
		``ciclando a sinistra'' la $p$-upla iniziale di $m$ posizioni. Si consideri la
		somma data dal teorema orbita-stabilizzatore:
		\[ \abs{S} = \sum_{x \in S} \frac{p}{\abs{\Stab(x)}} = 1 + \sum_{x \in S \setminus \{(e,\ldots,e)\}} \frac{p}{\abs{\Stab(x)}}. \]
		Poiché $\Stab(x) \leq \ZZ \quot p\ZZ$, gli unici ordini di $\Stab(x)$ possono
		essere $1$ e $p$. Se tuttavia, per $x \in S \setminus \{(e,\ldots,e)\}$,
		valesse $\Stab(x) = \ZZ \quot p\ZZ$, $x$ avrebbe coordinate tutte uguali,
		e quindi, per ipotesi, $x = (e,\ldots,e)$, \Lightning. Quindi il secondo
		termine del secondo membro vale esattamente $pk$, dove $k = \abs{S \setminus \{(e,\ldots,e)\}}$. \medskip
		
		
		Si osserva adesso che $\abs S = n^{p-1}$, dove $n = \abs G$. Infatti è sufficiente
		determinare le prime $p-1$ coordinate, per le quali vi sono $n$ scelte, per determinare
		anche l'ultima coordinata tramite la relazione $a_1 \cdots a_n = e$. Prendendo
		allora la precedente identità modulo $p$, si ottiene:
		\[ 1 \equiv 0 \pod p, \]
		da cui l'assurdo ricercato, \Lightning.
	\end{proof}
\end{document}