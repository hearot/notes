\documentclass[12pt]{scrartcl}
\usepackage{notes_2023}

\begin{document}
	\title{Il gruppo degli automorfismi}
	\maketitle

	\begin{note}
		Nel corso del documento per $(G, \cdot)$ si intenderà un qualsiasi gruppo.
		Si scriverà $gh$ per indicare $g \cdot h$, omettendo il punto.
	\end{note}

	\begin{definition}[gruppo degli automorfismi]
		Si definisce \textbf{gruppo degli automorfismi} di un gruppo $G$ il
		gruppo $(\Aut(G), \circ)$ dotato dell'operazione di composizione.
	\end{definition} \smallskip

	
	Si può associare ad ogni elemento $g \in G$ un automorfismo particolare $\varphi_g$
	determinato dalla seguente associazione:
	\[ h \xmapsto{\varphi_g} ghg\inv. \]
	
	\begin{definition}[gruppo degli automorfismi interni] Si definisce \textbf{gruppo
		degli automorfismi interni} di un gruppo $G$ il gruppo $(\Inn(G), \circ)$
		dotato dell'operazione di composizione, dove:
		
		\[ \Inn(G) = \{ \varphi_g \mid g \in G \}. \]
	\end{definition}
	
	Gli automorfismi interni soddisfano alcune proprietà. Per esempio vale che:
	\[ \varphi_g \circ \varphi_h = \varphi_{gh}, \]
	così come vale anche che:
	\[ \varphi_g \inv = \varphi_{g\inv}. \] \smallskip


	Chiaramente $\Inn(G) \leq \Aut(G)$. Tuttavia vale anche che $\Inn(G)$ è un sottogruppo
	normale di $\Aut(G)$. Infatti, se $f \in \Aut(G)$, vale che:
	\[ f \circ \varphi_g \circ f\inv = \varphi_{f(g)} \in \Inn(G). \]
	
	Inoltre, se $G$ è abeliano, $\varphi_g$ coincide con la sola identità $\Id$
	(infatti, in tal caso, $\varphi_g(h) = ghg\inv = gg\inv h = h$). \bigskip
	

	Si dimostra adesso un teorema fondamentale che mette in relazione $\Inn(G)$
	con un gruppo quoziente particolare di $G$, $G \quot Z(G)$. Preliminarmente,
	si osserva che $Z(G)$ è un sottogruppo normale di $G$, e quindi
	$G \quot Z(G)$ è effettivamente un gruppo. Allora si può enunciare la:
	
	\begin{proposition}
		$\Inn(G) \cong G \quot Z(G)$.
	\end{proposition}
	
	\begin{proof}
		Sia $\zeta : G \to \Inn(G)$ la mappa che associa $g$ al proprio
		automorfismo interno associato $\varphi_g$. Si osserva che $\zeta$
		è un omomorfismo tra gruppi:
		\[ \zeta(gh) = \varphi_{gh} = \varphi_g \circ \varphi_h = \zeta(g) \circ \zeta(h). \]
		
		Chiaramente $\zeta$ è una mappa surgettiva, e quindi $\Im \zeta = \Inn(G)$.
		Si osserva inoltre che $\Ker \zeta$ è esattamente il centro di $G$, $Z(G)$. Infatti,
		se $g \in \Ker \zeta$, vale che $\zeta(g) = \Id$, e quindi che:
		\[ ghg\inv = h \implies gh=hg \quad \forall h \in G. \]
		
		Allora, per il Primo teorema di isomorfismo, $G \quot {\Ker \zeta} = G \quot Z(G) \cong \Inn(G)$.
	\end{proof} \bigskip


	Il gruppo $G \quot Z(G)$ risulta particolarmente utile nello studio della commutatività
	del gruppo. Infatti vale la:
	
	\begin{proposition}
		$G \quot Z(G)$ è ciclico se e solo se $G$ è abeliano (e quindi se e solo se $G \quot Z(G)$ è banale).
	\end{proposition}
	
	\begin{proof}
		Se $G$ è abeliano, $G \quot Z(G)$ contiene solo l'identità, ed è dunque ciclico.
		Viceversa, sia $g Z(G)$ un generatore di $G \quot Z(G)$.
		Se $h$, $k \in G$, vale in particolare che esistono $m$, $n \in \NN$ tali per cui
		$h Z(G) = g^m Z(G)$ e $k Z(G) = g^n Z(G)$. Allora esistono
		$z_1$, $z_2 \in Z(G)$ per cui $h = g^m z_1$ e $k = g^n z_2$. \bigskip


		Si conclude allora che:
		\[ hk = g^m z_1 g^n z_2 = g^n z_2 g^m z_1 = kh, \]
		e quindi $G$ è abeliano (da cui si deduce che $G \quot Z(G)$ è in realtà banale).
	\end{proof} \bigskip
	

	Allora, poiché $\Inn(G) \cong G \quot Z(G)$, $\Inn(G)$ è ciclico se e solo se
	$G$ è abeliano (e dunque se e solo se è banale). Inoltre, il gruppo $\Inn(G)$
	risulta utile per definire in modo alternativo (ma equivalente) la nozione
	di \textit{sottogruppo normale}. Infatti vale che:
	
	\begin{proposition}
		Sia $H \leq G$. Allora $H \nsgeq G$ se e solo se $H$ è $\varphi_g$-invariante
		per ogni $g \in G$ (ossia se $\varphi_g(H) \subseteq H$).
	\end{proposition}
	
	\begin{proof}
		Se $H$ è normale, allora $\varphi_g(h) = g h g\inv$ appartiene ad $H$ per
		definizione. Allo stesso modo dire che $H$ è $\varphi_g$-invariante
		equivale a dire che $gHg\inv \subseteq H$ per ogni $g \in G$.
	\end{proof} \bigskip


	In generale, se $H \nsgeq G$, vale che la restrizione $\restr{\varphi_g}{H}$ è
	ancora un omomorfismo ed è in particolare un elemento di $\Aut(H)$. Infatti
	$\restr{\varphi_g}{H}$ è ancora iniettiva, e per ogni $h \in H$ vale che:
	\[ \varphi_g(g\inv h g) = h, \]
	mostrando la surgettività di $\restr{\varphi_g}{H}$ (infatti $g\inv h g \in H$). \bigskip


	Si può estendere questa idea considerando i sottogruppi di $G$ che sono $f$-invarianti
	per ogni scelta di $f \in \Aut(G)$.
	
	\begin{definition}[sottogruppo caratteristico]
		$H \leq G$ si dice \textbf{sottogruppo caratteristico} di $G$ se $H$
		è $f$-invariante per ogni $f \in \Aut(G)$.
	\end{definition} \smallskip
	
	In particolare, $H \leq G$ è un sottogruppo caratteristico di $G$ se ogni
	automorfismo di $G$ si riduce, restringendolo su $H$, ad un automorfismo
	di $H$. Infatti, se $f(H) \subseteq H$, vale anche che $f\inv(H) \subseteq H \implies
	H \subseteq f(H)$, e quindi $f(H) = H$ (da cui la surgettività dell'omomorfismo
	in $H$). \bigskip
	

	Chiaramente ogni sottogruppo caratteristico è un sottogruppo normale (infatti è
	in particolare $\varphi_g$-invariante per ogni scelta di $g \in G$), ma non è
	vero il contrario. Per esempio, si definisca l'automorfismo $\eta$ per $(\QQ, +)$
	tale per cui:
	\[ x \xmapsto{\eta} \nicefrac{x}2. \]
	Si osserva facilmente che $\eta$ è un automorfismo. Dal momento che $(\QQ, +)$ è
	abeliano, ogni suo sottogruppo è normale. In particolare $(\ZZ, +) \nsg (\QQ, +)$.
	Tuttavia $\eta(\ZZ) \not\subseteq \ZZ$ (e quindi $\ZZ$ non è caratteristico in $\QQ$). \bigskip
	
	
	Esiste tuttavia, per qualsiasi scelta di gruppo $G$, un sottogruppo che è caratteristico,
	$Z(G)$ (oltre che $G$ stesso ed il sottogruppo banale). Infatti, se $z \in Z(G)$ e
	$g \in G$, vale che:
	\[ f(z)g = f(z)f(f\inv(g)) = f(z f\inv(g)) = f(f\inv(g) z) = g f(z) \quad \forall f \in \Aut(G), \]
	e quindi $f(Z(G)) \subseteq Z(G)$ per ogni scelta di $f \in \Aut(G)$. \bigskip
	
	
	Inoltre, se $H \leq G$ è l'unico sottogruppo di un certo ordine (o è comunque
	caratterizzato univocamente da una proprietà invariante per automorfismi),
	$H$ è anche caratteristico (infatti gli automorfismi preservano le cardinalità essendo
	bigezioni). \bigskip
	
	
	\begin{example}[$\Aut(S_3) \cong S_3$] %TODO: aggiungere Aut(Z_2 * Z_2) ~ S_3
		Si osserva che $Z(S_3)$ deve essere obbligatoriamente
		banale\footnote{
			In generale $Z(S_n)$ è banale per $n \geq 3$.
		}. Infatti, se non lo fosse, $Z(S_3)$ potrebbe
		avere come cardinalità gli unici divisori positivi di
		$\abs{S_3} = 6$, ossia $2$, $3$ e $6$ stesso. In tutti
		e tre i casi $S_3 \quot Z(S_3)$ sarebbe ciclico, e quindi
		$S_3$ sarebbe abeliano, \Lightning. \medskip
		
		
		Poiché allora $Z(S_3)$ è banale, $S_3$ è isomorfo a
		$\Inn(S_3) \leq \Aut(S_3)$. Pertanto $\abs{\Aut(S_3)} \geq \abs{S_3} = 6$. Ogni automorfismo è
		determinato dalle immagini dei propri generatori, e quindi
		ci sono al più $3 \cdot 2 = 6$ scelte dal momento che
		$S_3 = \gen{(1,2), (1,2,3)}$. Allora
		$\abs{\Aut(S_3)} \leq 6$, da cui si deduce che
		$\abs{\Aut(S_3)} = 6$. \medskip
		
		
		Dacché $\Aut(S_3)$ ha lo stesso numero di elementi
		del suo sottogruppo $\Inn(S_3)$, deve valere l'uguaglianza
		tra i due insiemi, e quindi $\Aut(S_3) = \Inn(S_3)$. Si
		conclude dunque che $\Aut(S_3) \cong S_3$.
	\end{example}
	
	
	Si illustrano adesso dei risultati molto interessanti sui gruppi di automorfismi
	dei prodotti diretti, a partire dalla:
	
	\begin{proposition}
		Siano $H$ e $K$ due gruppi finiti di cardinalità coprime tra loro. Allora
		$H \times \{e\}$ e $\{e\} \times K$ sono caratteristici in $H \times K$.
	\end{proposition}
	
	\begin{proof}
		Sia $\varphi \in \Aut(H \times K)$. Si deve dimostrare che se
		$\varphi(h, e) = (h', k')$, allora $k' = e$. Chiaramente
		$\ord(h, e) = \ord(h) \mid \abs{H}$. Allo stesso tempo
		$\ord(h', k') = \mcm(\ord(h'), \ord(k'))$. In particolare, dal momento
		che $\MCD(\abs{H}, \abs{K}) = 1$, $\ord(h', k') = \ord(h') \ord(k')$.
		Dacché $\varphi$ è un automorfismo, $\ord(h', k') = \ord(h, e) = \ord(h)$, e
		quindi $\ord(h') \ord(k') = \ord(h)$. Allora $\ord(k')$ deve dividere
		$\abs{H}$, e quindi può valere soltanto $1$, essendo $\abs{H}$ e
		$\abs{K}$ coprimi. Pertanto $k' = e$, e quindi $H \times \{e\}$ è caratteristico
		in $H \times K$. Analogamente si dimostra la tesi per $\{e\} \times K$.
	\end{proof}
	
	\begin{proposition}
		Siano $H$ e $K$ due gruppi con $H \times \{e\}$ e $\{e\} \times K$ caratteristici
		in $H \times K$. Allora $\Aut(H \times K) \cong \Aut(H) \times \Aut(K)$.
	\end{proposition}
	
	\begin{proof}
		Nel corso della dimostrazione, se $\varphi \in \Aut(H \times K)$, si
		denota con $\varphi_H = \iota_{H \xhookrightarrow{} H \times \{e\}}\inv \circ \restr{\varphi}{H \times \{e\}} \circ \iota_{H \xhookrightarrow{} H \times \{e\}}$ la proiezione di $\varphi$ su
		$H$ a partire da $H$, e analogamente si fa lo stesso con $\varphi_K$. Tale
		notazione è ben definita dal momento che $\varphi$ può sempre essere ristretta
		ad $H \times \{e\}$ (infatti è un sottogruppo caratteristico). \medskip
		
		
		Sia allora
		$\alpha : \Aut(H \times K) \to \Aut(H) \times \Aut(K)$ tale per cui
		$\varphi \xmapsto{\alpha} (\varphi_H, \varphi_K)$. La mappa è ben
		definita dal momento che $\varphi_H$ e $\varphi_K$ sono due automorfismi
		di $\Aut(H)$ e $\Aut(K)$. Analogamente si definisce la mappa
		$\beta : \Aut(H) \times \Aut(K) \to \Aut(H \times K)$ tale per cui
		$(\varphi_H, \varphi_K) \xmapsto{\beta} [(h, k) \mapsto (\varphi_H(h), \varphi_K(k))]$.
		\medskip
		
		
		Si verifica facilmente che $\alpha$ è un omomorfismo di gruppi, che
		$\alpha \circ \beta = \Id_{\Aut(H) \times \Aut(K)}$ e che
		$\beta \circ \alpha = \Id_{\Aut(H \times K)}$, da cui segue la tesi.
		%TODO: scrivere le verifiche
	\end{proof}
	
	
	Allo stesso modo si verifica che se $\alpha$ è un isomorfismo, allora
	$H \times \{e\}$ e $\{e\} \times K$ sono caratteristici in $H \times K$. \medskip
	
	
	A partire dal precedente risultato, si dimostra facilmente che se $\MCD(m, n) = 1$,
	allora:
	\[ \Aut(\ZZ \quot m \ZZ \times \ZZ \quot n \ZZ) \cong \Aut(\ZZ \quot m \ZZ) \times \Aut(\ZZ \quot n \ZZ), \]
	e quindi, ricordando che $\ZZ \quot m \ZZ \times \ZZ \quot n \ZZ \cong \ZZ \quot mn \ZZ$
	per il Teorema cinese del resto e che $\Aut(\ZZ \quot m \ZZ) \cong (\ZZ \quot m \ZZ)^*$,
	vale che:
	\[ (\ZZ \quot m \ZZ)^* \times (\ZZ \quot n \ZZ)^* \cong (\ZZ \quot mn \ZZ)^* \]
	%TODO: aggiungere dimostrazione Aut(Z/nZ) ~ (Z/nZ)*
\end{document}
