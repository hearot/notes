\documentclass[12pt]{scrartcl}
\usepackage{notes_2023}

\begin{document}
	\title{Normalizzatore e teorema di Cayley}
	\maketitle
	
	\begin{note}
		Nel corso del documento per $(G, \cdot)$ si intenderà un qualsiasi gruppo.
	\end{note}
	
	Sia $X = \{ H \subseteq G \mid H \leq G \}$ l'insieme dei sottogruppi di $G$.
	Allora si può costruire un'azione $\varphi : G \to S(X)$ in modo tale che:
	\[ g \xmapsto{\varphi} \left[ H \mapsto gHg\inv \right]. \]
	Si definisce \textbf{normalizzatore} lo stabilizzatore di un sottogruppo
	$H$ (e si indica con $N_G(H)$), mentre $\Orb(H)$ è l'insieme dei \textbf{coniugati}
	di $H$. In particolare $N_G(H)$ è il massimo sottogruppo per inclusione in cui $H$
	è normale. \medskip
	
	
	Si osserva ora in modo cruciale che $H \nsgeq G$ se e solo se
	$\Orb(H) = \{H\}$, e quindi se e solo se $N_G(H) = G$. Analogamente si
	osserva che $H$ è normale se e solo se:
	\[ H = \bigcup_{h \in H} \Cl(h). \] \bigskip
	
	
	Si illustra adesso un risultato principale della teoria dei gruppi che mette in
	relazione ogni gruppo con il proprio gruppo di bigezioni, ed ogni gruppo finito con i
	sottogruppi dei gruppi simmetrici.
	
	\begin{theorem}[di Cayley]
		Ogni gruppo è isomorfo a un sottogruppo del suo gruppo di bigezioni.
		In particolare, ogni gruppo finito $G$ è isomorfo a un sottogruppo di un gruppo
		simmetrico.
	\end{theorem}
	
	\begin{proof}
		Si consideri l'azione\footnote{Tale azione prende il nome di \textbf{rappresentazione regolare a sinistra}.
		Si può infatti definire un'azione analoga a destra ponendo $g \mapsto \left[ h \mapsto hg\inv \right]$,
		costruendo dunque una \textit{rappresentazione regolare a destra}.} $\varphi : G \to S(G)$ tale per cui:
		\[ g \xmapsto{\varphi} \left[ h \mapsto gh \right]. \]
		Si mostra che $\varphi$ è fedele\footnote{L'azione $\varphi$ è molto
		più che fedele; è infatti innanzitutto libera.}. Sia infatti $\varphi(g) = \Id$; allora
		vale che $ge = e \implies g = e$. Quindi $\Ker \varphi$ è banale, e per il
		Primo teorema di isomorfismo vale che:
		\[ G \cong \Im \varphi \leq S(G). \]
		Se $G$ è finito, $S(G)$ è isomorfo a $S_n$, dove $n := \abs{G}$, e quindi
		$\Im \varphi$ è a sua volta isomorfo a un sottogruppo di $S_n$, da cui
		la tesi.
	\end{proof}
	
	
	Si presentano adesso due risultati interessanti legati ai sottogruppi normali di
	un gruppo $G$.
	
	\begin{proposition}
		Sia $H \leq G$. Allora, se $[G : H] = 2$, $H$ è normale in $G$.
	\end{proposition}
	
	\begin{proof}
		Poiché $[G : H] = 2$, le uniche classi laterali sinistre rispetto ad $H$ in
		$G$ sono $H$ e $gH = G \setminus H$, dove $g \notin H$. Analogamente esistono
		due sole classi laterali destre, $H$ e $Hg = G \setminus H$. In particolare
		$gH$ deve obbligatoriamente essere uguale a $Hg$, e quindi $gHg\inv = H$, da
		cui la tesi.
	\end{proof}
	
	\begin{proposition}
		Siano $K \leq H \leq G$. Allora, se $H$ è normale in $G$ e $K$ è caratteristico
		in $H$, $K$ è normale in $G$.
	\end{proposition}
	
	\begin{proof}
		Sia $\varphi_g \in \Inn(G)$. Poiché $H$ è normale in $G$, $\varphi_g(H) = H$. Pertanto
		si può considerare la restrizione di $\varphi_g$ su $H$, $\restr{\varphi_g}{H}$.
		In particolare $\restr{\varphi_g}{H}$ è un automorfismo di $\Aut(H)$, e quindi,
		poiché $K$ è caratteristico in $H$, $\restr{\varphi_g}{H}(K) = K$, da cui si
		deduce che $gKg\inv = K$ per ogni $g \in G$.
	\end{proof}
\end{document}