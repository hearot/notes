\documentclass[12pt]{scrartcl}
\usepackage{notes_2023}

\begin{document}
	\title{Normalizzatore e teorema di Cayley}
	\maketitle
	
	\begin{note}
		Nel corso del documento per $(G, \cdot)$ si intenderà un qualsiasi gruppo.
	\end{note}
	
	Sia $X = \{ H \subseteq G \mid H \leq G \}$ l'insieme dei sottogruppi di $G$.
	Allora si può costruire un'azione $\varphi : G \to S(X)$ in modo tale che:
	\[ g \xmapsto{\varphi} \left[ H \mapsto gHg\inv \right]. \]
	Si definisce \textbf{normalizzatore} lo stabilizzatore di un sottogruppo
	$H$ (e si indica con $N_G(H)$), mentre $\Orb(H)$ è l'insieme dei \textbf{coniugati}
	di $H$. In particolare $N_G(H)$ è il massimo sottogruppo per inclusione in cui $H$
	è normale. \medskip
	
	
	Si osserva ora in modo cruciale che $H \nsgeq G$ se e solo se
	$\Orb(H) = \{H\}$, e quindi se e solo se $N_G(H) = G$. Analogamente si
	osserva che $H$ è normale se e solo se:
	\[ H = \bigcup_{h \in H} \Cl(h). \] \bigskip
	
	
	Si illustra adesso un risultato principale della teoria dei gruppi che mette in
	relazione ogni gruppo con il proprio gruppo di bigezioni, ed ogni gruppo finito con i
	sottogruppi dei gruppi simmetrici.
	
	\begin{theorem}[di Cayley]
		Ogni gruppo è isomorfo a un sottogruppo del suo gruppo di bigezioni.
		In particolare, ogni gruppo finito $G$ è isomorfo a un sottogruppo di un gruppo
		simmetrico.
	\end{theorem}
	
	\begin{proof}
		Si consideri l'azione\footnote{Tale azione prende il nome di \textbf{rappresentazione regolare a sinistra}.
		Si può infatti definire un'azione analoga a destra ponendo $g \mapsto \left[ h \mapsto hg\inv \right]$,
		costruendo dunque una \textit{rappresentazione regolare a destra}.} $\varphi : G \to S(G)$ tale per cui:
		\[ g \xmapsto{\varphi} \left[ h \mapsto gh \right]. \]
		Si mostra che $\varphi$ è fedele\footnote{L'azione $\varphi$ è molto
		più che fedele; è infatti innanzitutto libera.}. Sia infatti $\varphi(g) = \Id$; allora
		vale che $ge = e \implies g = e$. Quindi $\Ker \varphi$ è banale, e per il
		Primo teorema di isomorfismo vale che:
		\[ G \cong \Im \varphi \leq S(G). \]
		Se $G$ è finito, $S(G)$ è isomorfo a $S_n$, dove $n := \abs{G}$, e quindi
		$\Im \varphi$ è a sua volta isomorfo a un sottogruppo di $S_n$, da cui
		la tesi.
	\end{proof}
	
	
	Si presentano adesso due risultati interessanti legati ai sottogruppi normali di
	un gruppo $G$.
	
	\begin{proposition}
		Sia $H \leq G$. Allora, se $[G : H] = 2$, $H$ è normale in $G$.
	\end{proposition}
	
	\begin{proof}
		Poiché $[G : H] = 2$, le uniche classi laterali sinistre rispetto ad $H$ in
		$G$ sono $H$ e $gH = G \setminus H$, dove $g \notin H$. Analogamente esistono
		due sole classi laterali destre, $H$ e $Hg = G \setminus H$. In particolare
		$gH$ deve obbligatoriamente essere uguale a $Hg$, e quindi $gHg\inv = H$, da
		cui la tesi.
	\end{proof}
	
	\begin{proposition}
		Siano $K \leq H \leq G$. Allora, se $H$ è normale in $G$ e $K$ è caratteristico
		in $H$, $K$ è normale in $G$.
	\end{proposition}
	
	\begin{proof}
		Sia $\varphi_g \in \Inn(G)$. Poiché $H$ è normale in $G$, $\varphi_g(H) = H$. Pertanto
		si può considerare la restrizione di $\varphi_g$ su $H$, $\restr{\varphi_g}{H}$.
		In particolare $\restr{\varphi_g}{H}$ è un automorfismo di $\Aut(H)$, e quindi,
		poiché $K$ è caratteristico in $H$, $\restr{\varphi_g}{H}(K) = K$, da cui si
		deduce che $gKg\inv = K$ per ogni $g \in G$.
	\end{proof}
	
	Si illustra adesso un risultato riguardante l'esistenza di sottogruppi normali in $G$:
	\begin{theorem}[di Poincaré]
		Sia $H$ un sottogruppo di $G$ di indice $n$. Allora esiste sempre un sottogruppo
		$N$ di $G$ tale per cui:
		\begin{enumerate}[(i)]
			\item $N$ è normale in $G$,
			\item $N$ è contenuto in $H$,
			\item $n \mid [G : N] \mid n!$.
		\end{enumerate}
	\end{theorem}
	
	\begin{proof}
		Si consideri l'azione $\varphi : G \to S(G \quot H)$ tale per cui
		$g \xmapsto{\varphi} [kH \mapsto gkK]$. Tale azione è sicuramente
		ben definita dal momento che $kH = k'H \implies gkH = gk'H$. Si
		studia $N := \Ker \varphi$. Chiaramente $N$ è normale in $G$, e si
		verifica facilmente che $N$ è contenuto anche in $H$, infatti, se
		$n \in N$, allora:
		\[ H = \varphi(n)(H) = nH \implies n \in H. \]
		Poiché $G \quot N$ è isomorfo a $\Im \varphi \leq S(G \quot H)$,
		$[G : N] \mid \abs{S(G \quot H)} = \abs{S_n} = n!$ considerando che
		$S(G \quot H) \cong S_n$. Dal momento allora che $N$ è un sottogruppo
		di $H$, vale che:
		\[ [G : N] = [G : H] [H : N] = n [H : N], \]
		e quindi $n \mid [G : N]$. Si è dunque esibito un sottogruppo $N$ con
		le proprietà indicate nella tesi.
	\end{proof}
	
	Dal precedente teorema sono immediati i seguenti due risultati:
	
	\begin{corollary}
		Sia $H$ un sottogruppo di $G$ con indice $n$. Se $n! < \abs{G}$ e
		$n>1$, allora $G$ non è semplice.
	\end{corollary}
	
	\begin{corollary}
		Sia $H$ un sottogruppo di $G$ con indice $p$, dove $p$ è il più piccolo
		primo che divide $n = \abs{G}$. Allora $H$ è normale.
	\end{corollary}
	
	\begin{proof}
		Per il Teorema di Poincaré, esiste un sottogruppo $N$ di $H$ tale per cui
		$N$ sia normale e $p \mid [G : N] \mid p!$ con $p = [G : H]$. In particolare
		$[G : N]$ deve dividere anche $n$, e quindi $[G : N]$ deve dunque
		dividere $\MCD(p!, n)$, che è, per ipotesi, $p$ stesso. Si conclude dunque
		che $[G : N] = p = [G : H]$, e quindi che $N = H$, ossia che $H$ stesso
		è normale.
	\end{proof}
	
	\begin{example} [Tutti i gruppi di ordine $15$ sono ciclici]
		Sia $G$ un gruppo di ordine $15$. Per il teorema di Cauchy esistono
		due elementi $h$ ed $k$, uno di ordine $3$ e l'altro di ordine $5$.
		In particolare, si consideri $K = \gen{k}$; poiché $\abs{K} = 5$,
		$[G : K] = 3$, il più piccolo primo che divide $15$. Pertanto
		$K$ è normale per il corollario di sopra. \medskip
		
		
		Poiché $K$ è normale, si può considerare la restrizione $\iota :
		\Inn(G) \to \Aut(K)$ tale per cui $\varphi_g \xmapsto{\iota} \restr{\varphi_g}{K}$.
		Dal momento che $K$ è ciclico, $\Aut(K) \cong \Aut(\ZZ \quot 5 \ZZ) \cong
		(\ZZ \quot 5 \ZZ)^* \cong \ZZ \quot 4 \ZZ$. Quindi $[G : \Ker \iota]$ deve
		dividere sia $4$ che $15$; dal momento che $\MCD(4, 15) = 1$, $[G : \Ker \iota] = 1$,
		e quindi che $\iota$ è l'omomorfismo banale. Poiché $\iota$ è banale, $K$ è
		un sottogruppo di $Z(G)$. \medskip
		
		
		In particolare $[G : Z(G)] \mid [G : K] = 3$, e quindi in particolare
		$G \quot Z(G)$ è ciclico, da cui si deduce che $G$ è abeliano. Infine,
		dal momento che $\MCD(3, 5) = 1$ e $h$ e $k$ commutano,
		$hk$ è un elemento di ordine $15$, e dunque $G$ è ciclico.
	\end{example}
\end{document}