\documentclass[12pt]{scrartcl}
\usepackage{notes_2023}

\begin{document}
	\title{Prodotto di sottogruppi e ordini di gruppi abeliani}
	\maketitle

	\begin{note}
		Nel corso del documento per $(G, \cdot)$ si intenderà un qualsiasi gruppo.
	\end{note}

	Si introduce in questo documento la nozione di prodotto
	di sottogruppi, ripresa poi nella dimostrazione di un
	lemma fondamentale per lo studio dei gruppi abeliani.
	
	\begin{definition}[prodotto di sottogruppi]
		Siano $H$ e $K$ due sottogruppi di $G$. Si definisce
		il loro prodotto $HK$ come:
		\[ HK = \{ hk \mid h \in H, k \in K \} \subseteq G. \]
	\end{definition}

	In realtà, il concetto di ``prodotto di sottogruppi'' non
	è del tutto nuovo nello studio dell'Algebra per uno
	studente che ha già seguito con successo un corso di
	Algebra lineare. Infatti, la somma di due sottospazi
	vettoriali è un prodotto di sottogruppi, per quanto
	la scrittura $V+W$ possa trarre in inganno (infatti uno
	spazio vettoriale è in particolare un gruppo abeliano).
	L'unica, cruciale, differenza sta nel fatto che una
	somma di sottospazi è sempre un sottospazio, mentre
	$HK$ potrebbe non esserlo, come mostra la:
	
	\begin{proposition}
		Siano $H$ e $K$ due sottogruppi di $G$. Allora
		$HK$ è un sottogruppo di $G$ se e solo se $HK=KH$.
	\end{proposition}

	\begin{proof}
		Se $HK$ è un sottogruppo di $G$, si verifica
		facilmente che $HK=KH$. Infatti, se $k \in K$ e
		$h \in H$, $kh$, che appartiene chiaramente
		a $KH$, deve appartenere anche ad $HK$ dal momento
		che è l'inverso dell'elemento $h\inv k\inv \in HK$
		(infatti $HK$ è un gruppo); pertanto $KH \subseteq HK$.
		Analogamente, sia $x$ un elemento di $HK$. Allora
		$x$ ammette un inverso in $HK$, e quindi $x\inv = hk$,
		con $h \in H$, $k \in K$. Allora $x = k\inv h\inv \in KH$,
		da cui $HK \subseteq KH$ e quindi $HK=KH$. \bigskip
		
		
		Sia ora $HK=KH$. Chiaramente $e \in HK$. Siano $x=h_1 k_1$ e $y=h_2 k_2$ elementi
		di $HK$ con $h_1$, $h_2 \in H$ e $k_1$, $k_2 \in K$.
		Allora $xy = h_1 k_1 h_2 k_2$; tuttavia $k_1 h_2$ si può
		riscrivere per ipotesi (essendo $KH \subseteq HK$) come
		$hk$ con $h \in H$ e $k \in K$. Allora $xy = h_1 h k k_2 \in HK$, e quindi $HK$ è chiuso per l'operazione di gruppo di $G$.
		Inoltre $x\inv = k_1 \inv h_1 \inv \in KH$, e quindi,
		per ipotesi, $x\inv \in HK$, da cui la tesi.
	\end{proof}

	Quindi, se un gruppo è abeliano, vale sempre la relazione
	$HK=KH$, e dunque $HK$ è sempre un sottogruppo (e quindi
	si sostituisce con più tranquillità alla notazione $HK$ la più familiare $H+K$). In realtà, però, si può indebolire questa
	ipotesi richiedendo la normalità di $H$ o $K$ (come suggerisce
	la notazione $H = KHK\inv$), come mostra la:
	
	\begin{proposition}
		Siano $H$ e $K$ due sottogruppi di $G$. Allora, se $H \nsgeq G$, $HK = KH$.
	\end{proposition}

	\begin{proof}
		Siano $h \in H$ e $k \in K$. Si consideri l'elemento
		$hk \in HK$. Poiché $H$ è normale, $k\inv h k \in H$,
		e quindi $k\inv h k = h'$ con $h' \in H$, da cui
		$HK \subseteq KH$. Analogamente si mostra anche
		l'altra inclusione.
	\end{proof} \bigskip

	Come studiato nell'ambito dell'Algebra lineare, l'intersezione dei
	sottogruppi $H$ e $K$ gioca un ruolo fondamentale nel
	considerare l'insieme $HK$. In particolare, ci si chiede
	quando il prodotto $hk$ è univocamente rappresentato
	(ossia $hk=h'k' \implies h=h'$ e $k=k'$). Si può
	rispondere a questa domanda in due modi: mostrando
	sotto quali ipotesi si trova un isomorfismo tra $HK$ e
	$H \times K$ (che dunque codifica l'unicità tramite
	l'uguaglianza delle coordinate), o determinando
	la cardinalità di $HK$ per $G$ finito (e dunque l'unicità dipende
	dall'uguaglianza $\abs{HK} = \abs H \abs K$, dal momento
	che se le scritture sono uniche, tutti i prodotti tra elementi
	di $H$ e di $K$ sono distinti). In entrambi i casi si giungerà
	alla conclusione secondo cui $H \cap K$ deve essere
	banale\footnote{Mantenendo l'analogia con l'Algebra lineare,
	vale infatti che $V + W = V \oplus W$
	se e solo se $V \cap W = \zerovecset$. Si mostrerà che
	sotto le stesse ipotesi anche un prodotto di sottogruppi è
	un prodotto diretto (tramite isomorfismo).} \bigskip


	\begin{proposition}[cardinalità di $HK$]
		Sia $G$ un gruppo finito.
		Siano $H$ e $K$ due sottogruppi di $G$. Allora vale
		che $\abs{HK} = \frac{\abs{H} \abs{K}}{\abs{H \cap K}}$.
	\end{proposition}

	\begin{proof}
		Si costruisca la relazione di equivalenza $\sim$
		su $H \times K$ in modo tale che:
		\[ (h,k) \sim (h',k') \defiff hk=h'k'. \]
		Allora chiaramente $\abs{H \times K \quot \sim} = \abs{HK}$
		(infatti ad ogni classe di equivalenza corrisponde esattamente un unico elemento di $HK$). \bigskip
		
		Si esamini la classe di equivalenza di $(h,k) \in H \times K$. Si
		mostra che ogni elemento di tale classe è della forma
		$(ht,t\inv k)$ con $t \in H \cap K$. Sia infatti $(h_1,k_2) \in
		[(h,k)]_\sim$. Allora:
		\[
			h_1k_1=hk \implies h\inv h_1 = k k_1\inv \in H \cap K.
		\]
		Pertanto, se $h\inv h_1 = k k_1\inv = t$, vale che $h_1=ht$ e che $k_1 = t\inv k$.
		Quindi ogni classe di equivalenza contiene esattamente $\abs{H \cap K}$ elementi.
		Poiché $\sim$ induce una partizione di $H \times K$ in classi
		di equivalenza, vale dunque che:
		\[
			\abs{H} \abs{K} = \abs{H \times K} = \abs{H \times K \quot \sim} \abs{H \cap K} = \abs{HK} \abs{H \cap K},
		\]
		da cui la tesi. 
	\end{proof}
	
	\begin{proof}[Dimostrazione alternativa]
		Si osserva che vale la seguente identità:
		\[ HK = \bigcup_{h \in H} hK. \]
		Poiché gli $hK$ rappresentano delle classi laterali sinistre
		di $G$, se $h' \in H$, o $hK = h'K$ o $hK \cap h'K = \emptyset$. Se $hK = h'K$, allora $h h\inv \in K$, e quindi
		$h h\inv \in H \cap K$. Vi sono dunque esattamente
		$\abs{H \cap K}$ istanze della classe $hK$ nell'unione
		considerata all'inizio della dimostrazione. Allora:
		\[ \abs{HK} = \frac{\abs{H} \abs{K}}{\abs{H \cap K}}, \]
		dove $\abs{K}$ è il numero di elementi di ogni classe
		$hK$.
	\end{proof}

	Pertanto, se le scritture sono uniche, $H \cap K$ deve essere
	per forza banale (infatti deve valere $\abs{H \cap K} = 1$).
	Questo risultato può essere rafforzato dalla:
	
	\begin{proposition}
		Siano $H$ e $K$ due sottogruppi normali di $G$ tali che
		$H \cap K = \{e\}$. Allora $HK \cong H \times K$.
	\end{proposition}

	\begin{proof}
		Si costruisce la mappa $\rho : H \times K \to HK$
		in modo tale che:
		\[ (h,k) \xmapsto{\rho} hk. \]
		
		Si osserva che ogni elemento $h$ di $H$ commuta con
		ogni elemento $k$ di $K$. Se infatti si considera il
		commutatore $g = [h, k]$, vale che:
		\[
			g = \underbrace{(hkh\inv)}_{\in K} k \in K, \qquad g = h\inv \underbrace{(kh\inv k\inv)}_{\in H} \in H.
		\]
		Pertanto $g \in H \cap K \implies [h, k] = e \implies
		hk=kh$. Allora $\rho$
		è un omomorfismo, infatti:
		\[
			\rho((hh',kk')) = hh'kk' = hkh'k' = \rho((h,k)) \rho((h',k')).
		\]
		Chiaramente $\rho$ è surgettiva. Inoltre $\rho((h,k)) = e \implies h = k\inv \in H \cap K$,
		e dunque $h = k = e$, da cui l'iniettività di $\rho$
		e la tesi.
	\end{proof}
	
	
	Inoltre, se $G = G_1 \times G_2$ con $G_1$ e $G_2$ gruppi, si possono trovare
	facilmente due copie isomorfe di $G_1$ e $G_2$ in $G$, ossia $G_1' = G_1 \times \{e\}$ e $G_2' = \{e\} \times G_2$.
	Vale inoltre che $G_1'$, $G_2' \nsgeq G$ e dunque,
	per la proposizione precedente\footnote{Infatti $G_1' \cap G_2' = \{(e,e)\}$.}, che
	$G \cong G_1' \times G_2'$. \medskip
	
	
	In particolare vale il seguente risultato, considerando
	$\gen{x} \cap \gen{y} = \{e\}$:
	
	\begin{proposition}
		Siano\footnote{
			In generale, se $\MCD(\ord(x), \ord(y)) > 1$,
			non vale che $\ord(xy) = \mcm(\ord(x), \ord(y))$,
			benché sicuramente $\ord(xy) \mid \mcm(\ord(x), \ord(y))$,
			sempre a patto che $x$ e $y$ commutino.
			È sufficiente considerare in $\ZZmod6$ gli elementi
			$\cleq 1$ e $\cleq 2$: infatti $\ord(\cleq 1) = 6$ e
			$\ord(\cleq 2) = 3$, ma $\ord(\cleq 1 + \cleq 2) =
			\ord(\cleq 3) = 2 \neq 6$.
		}\footnote{
			A prescindere da quanto valga $\MCD(\ord(x), \ord(y))$,
			se $x$ e $y$ commutano, esiste sempre un elemento
			$g \in G$ tale per cui $\ord(g) = \mcm(\ord(x), \ord(y))$.
		} $x$, $y$ due elementi di $G$ che commutano con
		$\MCD(\ord(x), \ord(y)) = 1$. Allora $\ord(xy) = \ord(x) \ord(y)$. % TODO: aggiungere la dimostrazione
	\end{proposition}
	
	\begin{proof}
		Chiaramente $\ord(xy) \mid \ord(x) \ord(y)$, dal momento
		che $(xy)^{\ord(x) \ord(y)} = x^{\ord(x) \ord(y)} y^{\ord(x) \ord(y)} = e$, dove si è usato che $x$ e $y$ commutano.
		Sia allora $k = \ord(xy)$. Vale allora che
		$x^k y^k = e \implies x^k = y^{-k} \in \gen{x} \cap \gen{y}$.
		Tuttavia $\abs{\gen{x} \cap \gen{y}} \mid
		\MCD(\abs{\gen{x}}, \abs{\gen{y}}) = \MCD(\ord(x), \ord(y)) =
		1$, e quindi $\gen{x} \cap \gen{y} = \{e\}$. Allora
		deve valere che $x^k = y^{-k} = e \implies
		\ord(x), \ord(y) \mid k$, da cui si deduce che
		$\ord(x) \ord(y) \mid k = \ord(x) \ord(y)$. Si conclude dunque che
		$\ord(xy) = \ord(x) \ord(y)$.
	\end{proof} \vskip 0.2in


	Si può adesso dimostrare il seguente fondamentale
	teorema per i gruppi abeliani:
	
	\begin{theorem}
		Sia $G$ un gruppo abeliano finito di ordine $n$.
		Allora, se $m$ divide $n$, esiste un sottogruppo di
		$G$ di ordine $m$.
	\end{theorem}

	\begin{proof}
		Si dimostra preliminarmente che se $p^k$ divide $n$,
		dove $p$ è un numero primo e $k \in \NN^+$, allora
		$G$ ammette un sottogruppo di ordine $p^k$. Si
		mostra la tesi per induzione su $k$. \medskip
		
		
		Se $k=1$ la tesi è valida per il Teorema di Cauchy, completando il passo base. Si ipotizzi adesso
		che per ogni $t < k$ valga la tesi. Si consideri
		un sottogruppo $H$ di $G$ di ordine $p$
		(ancora una volta questo sottogruppo esiste per il
		Teorema di Cauchy). Poiché $G$ è abeliano, $H$ è
		normale in $G$, e quindi si può considerare il
		gruppo quoziente $G \quot H$. Per il Teorema di
		Lagrange, $p^{k-1}$ divide $\abs{G \quot H}$, e
		quindi, per l'ipotesi induttiva, esiste un sottogruppo
		$T$ di $G \quot H$ di ordine $p^{k-1}$. \medskip
		
		
		Si consideri la proiezione al quoziente $\pi_H : G \to G \quot H$. Poiché $\pi_H$ è un omomorfismo,
		$\pi_H\inv(T)$ è un sottogruppo. Inoltre, questo sottogruppo di $G$ ha ordine $p^k$, dal momento che $H$ ha ordine $p$
		(e quindi ogni elemento di $T$ corrisponde tramite
		la controimmagine a $p$ elementi), completando il passo
		induttivo. \medskip
		
		
		Sia ora $m$ scomposto nella sua fattorizzazione in primi
		$p_1^{k_1} \cdots p_s^{k_s}$. Per il risultato precedente,
		$G$ ammette dei sottogruppi $H_1$, ..., $H_s$ di ordine
		$p_1^{k_1}$, ..., $p_s^{k_s}$. Poiché $G$ è abeliano,
		tutti questi sottogruppi sono normali e si può dunque
		considerare il prodotto dei sottogruppi $H_1 \cdots H_s$
		(che è dunque un sottogruppo). Poiché
		$\MCD(p_1^{k_1}, p_2^{k_2})=1$, $H_1 \cap H_2$ è banale e vale
		che $\abs{H_1 H_2} = \abs{H_1} \abs{H_2} = p_1^{k_1} p_2^{k_2}$.
		Allora, poiché $\MCD(p_1^{k_1} p_2^{k_2}, p_3^{k_3}) = 1$, anche $(H_1H_2) \cap H_3$ è banale e dunque $\abs{H_1 H_2 H_3} =
		p_1^{k_1} p_2^{k_2} p_3^{k_3}$. Proseguendo induttivamente
		si mostra dunque che $H_1 \cdots H_s$ è un sottogruppo
		di $G$ di ordine $m$.
	\end{proof}
\end{document}