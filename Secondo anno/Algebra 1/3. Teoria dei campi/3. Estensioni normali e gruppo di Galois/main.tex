\documentclass[12pt]{scrartcl}
\usepackage{notes_2023}

\begin{document}
	\title{Estensioni normali e gruppo di Galois}
	\maketitle
	
	\begin{note}
		Per $K$, $L$ ed $F$ si intenderanno sempre dei campi.
		Se non espressamente detto, si sottintenderà anche
		che $K \subseteq L$, $F$, e che $L$ ed $F$ sono
		estensioni costruite su $K$. Per $[L : K]$ si
		intenderà $\dim_K L$, ossia la dimensione di $L$
		come $K$-spazio vettoriale. Per scopi didattici, si
		considerano solamente campi perfetti, e dunque estensioni che sono sempre separabili, purché
		non esplicitamente detto diversamente.
	\end{note} \bigskip

	Si introduce adesso il fondamentale concetto di
	\textit{estensione normale}, prerequisito per
	introdurre a sua volta la teoria di Galois.
	
	\begin{definition}[estensione normale]
		Un'estensione algebrica
		$\faktor{L}{K}$ si dice \textbf{normale}
		se per ogni $K$-immersione $\varphi : L \to \overline{K}$
		vale che $\varphi(L) = L$.
	\end{definition}

	Questa definizione viene immediatamente caratterizzata
	attraverso i coniugati dei suoi elementi, come mostra
	la:
	
	\begin{proposition}
		Sono equivalenti i seguenti fatti:
		
		\begin{enumerate}[(i)]
			\item $\faktor{L}{K}$ è un'estensione normale,
			\item Per ogni $\alpha \in \faktor{L}{K}$, ogni coniugato
			di $\alpha$ appartiene a $L$,
			\item $\faktor{L}{K}$ è il campo di spezzamento
			di una famiglia di polinomi di $K[x]$.
		\end{enumerate}
	\end{proposition}

	\begin{proof}
		Si mostra l'equivalenza delle proprietà:

		\begin{itemize}
			\item[$(i)\implies (ii)\;$] Sia $\varphi : L
			\to \overline{K}$ una $K$-immersione di $L$. Allora,
			poiché $L$ è normale su $K$, $\varphi(L) = L$.
			Sia $\alpha \in L \setminus K$.
			Dal momento che $K \subseteq K(\alpha) \subseteq L$,
			$\restr{\varphi}{K(\alpha)}$ è in particolare
			una $K$-immersione di $K(\alpha)$, e quindi
			deve associare ad $\alpha$ un suo coniugato.
			Dal momento però che $\varphi(\alpha) \in L$,
			questo significa che ogni coniugato di $\alpha$
			appartiene ad $L$.
			
			\item[$(ii)\implies (iii)\;$] Sia $\mathcal{F}$
			la famiglia dei polinomi minimi degli elementi
			di $\faktor{L}{K}$. Si dimostra che
			$L$ è il campo di spezzamento di $\mathcal{F}$ su
			$K$. Chiaramente $\mathcal{F} \subseteq L$,
			dal momento che $L$ contiene una radice per
			ipotesi di ogni polinomio minimo, e per
			(ii) contiene tutti i suoi coniugati (e dunque
			tutte le radici di ogni polinomio della famiglia
			$\mathcal{F}$). Inoltre vale anche
			che $L \subseteq \mathcal{F}$, dal momento che
			ogni elemento di $L$ è radice di un polinomio
			di $\mathcal{F}$, per costruzione. Pertanto
			$L = \mathcal{F}$.
			
			\item[$(iii)\implies (i)\;$] Sia $\varphi : L
			\to \overline{K}$ una $K$-immersione di $L$. Sia
			$\alpha \in L \setminus K$.Dal momento per che $L$
			è campo di spezzamento di una famiglia $\mathcal{F}$ di polinomi,
			$L$ è generato dalle radici di $\mathcal{F}$.
			Per ogni $\alpha$ generatore di $L$, allora,
			$\varphi$ deve mappare $\alpha$ ad un suo
			coniugato, ancora appartenente ad $L$ dacché
			$\mathcal{F}$ è campo di spezzamento. Pertanto
			$\varphi(\alpha) \in L$. Allora, dal momento
			che $L$ è generato dalle radici di $\mathcal{F}$,
			ogni suo elemento viene ancora mappato ad un
			elemento di $L$, e quindi $\faktor{L}{K}$ è
			un'estensione normale.
		\end{itemize}
	\end{proof}

	\begin{remark}
		Per esempio, $\faktor{\QQ(\sqrt[3]{2})}{\QQ}$ non
		è normale, dal momento che $\sqrt[3]{2} \zeta_3$,
		un coniugato di $\sqrt[3]{2}$, non appartiene
		a $\QQ(\sqrt[3]{2})$. Al contrario,
		$\faktor{\QQ(\zeta_3)}{\QQ}$ è normale, dal
		momento che l'unico coniugato di $\zeta_3$ è
		$\zeta_3^2$. 
	\end{remark}

	Dimostriamo inoltre che le estensioni di grado $2$
	sono sempre normali, come mostra la:
	
	\begin{proposition}
		Sia $\faktor{L}{K}$ un'estensione di grado $2$.
		Allora $L$ è normale su $K$, se $\Char K \neq 2$.
	\end{proposition}

	\begin{proof}
		Chiaramente $L$ è un'estensione algebrica di $K$,
		essendo finita. Sia\footnote{
			$L$ è di grado $2$ su $K$, e quindi $K$ deve
			essere un suo sottinsieme proprio.
		} allora $\alpha \in L \setminus K$. Dal momento che
		$\alpha \notin K$, $[K(\alpha) : K] = 2$, e quindi
		$L = K(\alpha)$. Inoltre $\deg_K \alpha = 2$, pertanto,
		poiché $\Char K \neq 2$,
		esiste un polinomio irriducibile
		$p(x) = x^2 + bx + c$ con $b$, $c \in K$
		di cui $\alpha$ è radice. In particolare,
		$\alpha$, $\overline{\alpha} = \frac{-b \pm \sqrt{\Delta}}{2}$, dove
		$\overline{\alpha}$ il coniugato di $\alpha$.
		Allora $\alpha$, $\overline{\alpha} \in K(\sqrt{\Delta})$. Dal momento allora che
		$L = K(\alpha) = K(\sqrt{\Delta})$, $L$ è
		campo di spezzamento di $p \in K[x]$, e dunque,
		per la proposizione precedente, è normale su $K$.
	\end{proof}

	Infine, si esplora la normalità su un diagramma di
	estensioni.
	
	\begin{proposition}[normalità nel composto e nell'intersezione]
		Siano $\faktor{L}{K}$ e $\faktor{M}{K}$ estensioni
		normali. Allora $\faktor{LM}{K}$ e
		$\faktor{L \cap M}{K}$ sono a loro volta normali.
		
		\[\begin{tikzcd}[column sep=small]
			&& LM \\
			\\
			L &&&& M \\
			\\
			&& {L \cap M} \\
			\\
			&& K
			\arrow[no head, from=5-3, to=7-3]
			\arrow[no head, from=1-3, to=3-5]
			\arrow[no head, from=1-3, to=3-1]
			\arrow[no head, from=3-1, to=5-3]
			\arrow[no head, from=5-3, to=3-5]
			\arrow[no head, from=1-3, to=5-3]
		\end{tikzcd}\]
	\end{proposition}

	\begin{proof}
		Chiaramente $LM$ e $L \cap M$ sono estensioni
		algebriche di $K$, in quanto sia $L$ che $M$ lo sono.
		Sia $\varphi : LM \to \overline{K}$ una $K$-immersione
		di $LM$. Allora $\varphi(LM) = \varphi(L(M)) =
		L(\varphi(M)) = L(M) = LM$, e quindi $LM$ è normale
		su $K$. Analogamente, se $\varphi : L \cap M \to \overline{K}$ è una $K$-immersione di $L \cap M$,
		$\varphi(L \cap M) = \varphi(L) \cap \varphi(M) = L \cap M$, e quindi $L \cap M$ è normale su $K$.
	\end{proof}

	\begin{proposition}
		Sia $K \subseteq F \subseteq L$ una torre di campi. Allora
		$\faktor{L}{K}$ normale $\implies$ $\faktor{L}{F}$
		normale.
	\end{proposition}

	\begin{proof}
		Poiché $L$ è normale su $K$, $L$ è un campo di
		spezzamento di una famiglia $\mathcal{F}$ di polinomi
		di $K[x]$. A maggior ragione, allora,
		$L$ è campo di spezzamento di $\mathcal{F}$ come
		polinomi di $F[x]$, e quindi è normale anche su $F$.
	\end{proof} \medskip

	Si può adesso introdurre la teoria di Galois introducendo
	prima l'insieme $\Aut_K L$ e poi il gruppo $\Gal(\faktor{L}{K})$.
	
	\begin{definition}
		Si definisce l'insieme $\Aut_K L$ come l'insieme
		delle $K$-immersioni di $L$, ossia delle immersioni
		$\varphi : L \to \overline{K}$ tali per cui
		$\restr{\varphi}{K} = \Id_K$.
	\end{definition}

	Se $L$ è normale su $K$, le immersioni di
	$\Aut_K L$ possono essere ristrette al codominio su
	$L$ (infatti $\varphi(L) = L$ per definizione) e sono
	tali per cui mandano gli elementi di $L$ nei loro
	coniugati su $K$. Inoltre, se $L$ è un'estensione finita
	di $K$, la separabilità di $L$ garantisce che\footnote{
		In generale, se $L$ è un'estensione finita e normale di $K$,
		$\abs{\Aut_K L} = [L : K]$ se e solo se $L$
		separabile su $K$.
	}
	$\abs{\Aut_K L} = [L : K]$. Pertanto, riducendoci a
	considerare le estensioni normali e separabili di $K$,
	ogni immersione, ristretta opportunamente sul codominio,
	ammette un inverso, e quindi si può considerare
	$\Aut_K L$ come gruppo sulla composizione, denotato
	come $\Gal(\faktor{L}{K})$. Tali estensioni sono
	speciali, e vengono pertanto dette \textit{di Galois}.
	
	\begin{definition}[estensioni di Galois]
		Si dice che $\faktor{L}{K}$ è un'\textbf{estensione
		di Galois} se $L$ è sia normale che separabile su $K$.
	\end{definition}

	\begin{definition}[gruppo di Galois di $\faktor{L}{K}$]
		Si definisce il gruppo di Galois di $\faktor{L}{K}$,
		denotato come $\Gal(\faktor{L}{K})$, il gruppo
		rispetto alla composizione
		delle immersioni di $\Aut_K L$ ristrette sul codominio
		a $L$.
	\end{definition}

	La maggior parte dei teoremi della teoria di Galois si
	fondano particolarmente sul fatto che il gruppo di Galois
	di un campo di spezzamento di un irriducibile $f$
	agisce sulle radici di $f$, come mostra la:
	
	\begin{proposition}
		Sia $f(x) \in K[x]$ un irriducibile. Allora,
		se $L$ è il suo campo di spezzamento,
		$\Gal(\faktor{L}{K})$ agisce fedelmente e transitivamente sulle
		radici di $L$. Pertanto $\Gal(\faktor{L}{K}) \mono S_n$,
		dove $n = [L : K] = \deg f(x)$, e quindi
		$n \mid [L : K] \mid n!$. 
	\end{proposition}

	\begin{proof}
		Si consideri l'azione $\Xi : \Gal(\faktor{L}{K}) \to
		S(\{ \alpha_1, \ldots, \alpha_n \})$ tale per cui
		$\varphi \xmapsto{\Xi} [\alpha_i \mapsto \varphi(\alpha_i)]$, dove\footnote{
			Si ricorda l'ipotesi di $K$ campo perfetto;
			pertanto $f(x)$ è separabile.
		}
		le $\alpha_i$ sono le radici distinte di $f(x)$.
		Allora chiaramente $n \mid [L : K]$, dal momento
		che $[K(\alpha_1) : K] = n$ e $K(\alpha_1) \subseteq L$. \medskip
		
		
		Inoltre $\Xi$ è un'azione fedele dacché $\Ker \Xi$ è banale. Infatti
		l'unica $K$-immersione che fissa ogni radice è
		necessariamente
		l'identità. Allora $\Xi$ è un'immersione di
		$\Gal(\faktor{L}{K})$ in $S(\{ \alpha_1, \ldots, \alpha_n \}) \cong S_n$, e quindi
		$[L : K] = \abs{\Gal(\faktor{L}{K})} \mid n!$. Infine, esiste sempre una $K$-immersione di
		$L$ che mappa un qualsiasi $\alpha_i$ ad un altro
		$\alpha_j$, purché $i \neq j$. Pertanto $\Gal(\faktor{L}{K})$ agisce transitivamente sulle
		radici di $f(x)$.
	\end{proof}
\end{document}