\documentclass[12pt]{scrartcl}
\usepackage{notes_2023}

\begin{document}
	\title{Chiusura algebrica di un campo e campi di spezzamento}
	\maketitle
	
	\begin{note}
		Per $K$, $L$ ed $F$ si intenderanno sempre dei campi.
		Se non espressamente detto, si sottintenderà anche
		che $K \subseteq L$, $F$, e che $L$ ed $F$ sono
		estensioni costruite su $K$. Per $[L : K]$ si
		intenderà $\dim_K L$, ossia la dimensione di $L$
		come $K$-spazio vettoriale.
	\end{note} \bigskip


	Questo documento si propone di illustrare le principali
	proprietà e caratteristiche dei campi algebricamente
	chiusi, delle chiusure algebriche e dei campi di
	spezzamento, col proposito di dare i mezzi necessari
	per approcciarsi alla teoria di Galois. Per questo
	motivo si presentano le seguenti definizioni:
	
	\begin{definition}[campo algebricamente chiuso]
		Un campo $K$ si dice \textbf{algebricamente chiuso}
		se ogni polinomio a coefficienti in $K$ ammette una
		radice in $K$. Equivalentemente, $K$ è algebricamente
		chiuso se ogni polinomio $p \in K[x]$ ha tutte le proprie
		radici in $K$, e quindi se gli irriducibili di $K$ sono
		tutti e soli i polinomi di grado unitario.
	\end{definition}
	
	\begin{definition}[chiusura algebrica]
		Un estensione $\faktor{\Omega}{K}$ si dice
		\textbf{chiusura algebrica} di $K$, e si
		indica usualmente con $\overline{K}$, se $\Omega$
		è un campo algebricamente chiuso e se
		$\Omega$ è un'estensione algebrica su $K$.
	\end{definition}
	
	\begin{remark}
		Per esempio, una chiusura algebrica di $\RR$ è $\CC$,
		per il Teorema fondamentale dell'algebra.
	\end{remark}
	
	\begin{proposition}
		Sia $\Omega$ un campo algebricamente chiuso. Se allora
		$K$ è un sottocampo di $\Omega$, vale che
		$K'$, il campo degli elementi algebrici
		su $K$, è una chiusura algebrica di $K$.
	\end{proposition}
	
	\begin{proof}
		Chiaramente
		$K'$ è un'estensione algebrica su
		$K$. Si verifica allora che $K'$ è
		algebricamente chiuso. Sia $p \in K'[x]$.
		Dal momento che $K$ è algebricamente chiuso, e
		che $p$ appartiene anche a $K[x]$, allora
		$p$ ammette una radice $\alpha \in \Omega$. Si mostra che
		$\alpha$ è algebrico su $K$. Poiché allora
		$\faktor{K'(\alpha)}{K'}$ è
		un'estensione algebrica (infatti $p$ annulla $\alpha$
		per ipotesi) e $\faktor{K'}{K}$ è algebrica
		per ipotesi, allora $K'(\alpha)$ è algebrica
		su $K$, e dunque $\alpha$ è algebrico su $K$, pertanto
		$\alpha \in K'$, da cui la tesi.
	\end{proof}
	
	\begin{remark}
		Poiché $\QQ$ è un sottocampo di $\CC$ e $\CC$ è
		un campo algebricamente chiuso, il campo degli
		elementi algebrici di $\QQ$ è una chiusura algebrica di
		$\QQ$ per la proposizione precedente.
	\end{remark}
	
	Adesso si enuncia, senza dimostrarlo, un teorema su cui si baserà buona parte della prossima teoria:
	
	\begin{theorem}[esistenza ed unicità della chiusura algebrica]
		Esiste ed è unica, a meno di $K$-isomorfismo\footnote{
			Un $K$-isomorfismo è un isomorfismo tra estensioni
			di $K$ che fissa $K$, ossia che ristretto a $K$ è
			l'identità di $K$.
		},
		la chiusura algebrica di $K$.
	\end{theorem}
	
	\begin{remark}
		Poiché il campo degli elementi algebrici di $\QQ$ è una chiusura algebrica di
		$\QQ$ ed è un insieme numerabile, $\CC$ non può
		essere una chiusura algebrica di $\QQ$ dacché
		$\CC$ ha la cardinalità del continuo (e dunque non possono
		esistere bigezioni tra $\CC$ e $\overline{\QQ}$). Poiché
		$\CC$ è però algebricamente chiuso, può solamente
		verificarsi che $\CC$ non sia un'estensione algebrica
		di $\QQ$. Più facilmente, $\pi \in \RR$ non è algebrico su $\QQ$,
		e così né $\RR$ né $\CC$ sono estensioni algebriche su $\QQ$.
	\end{remark}
	
	\begin{definition}[campo di spezzamento]
		Sia $\mathcal{F}$ una famiglia di polinomi di $K[x]$.
		Si definisce allora \textbf{campo di spezzamento} di
		$\mathcal{F}$ una estensione $F$ di $K$ tale per cui:
		
		\begin{itemize}
			\item ogni $p \in \mathcal{F}$ si decompone in fattori lineari in
				$F[x]$,
			\item se $L$ è un'estensione su $K$ tale per cui
				$L \subsetneq F$, allora esiste $p \in \mathcal{F}$
				non si decompone in fattori lineari in $L[x]$.
		\end{itemize}
		
		Equivalentemente $F$ è un'estensione minimale in cui
		ogni polinomio di $\mathcal{F}$ si decompone in fattori
		lineari.
	\end{definition}
	
	Come per le chiusure algebriche, si enuncia il seguente
	teorema senza dimostrazione\footnote{
		L'esistenza di un campo di spezzamento è piuttosto
		facile da dimostrare, è sufficiente considerare
		l'estensione di $K$ a cui si aggiungono tutte le
		radici del polinomio.
	}:
	
	\begin{theorem}[esistenza ed unicità del campo di spezzamento]
		Esiste ed è unico, a meno di $K$-isomorfismo, il
		campo di spezzamento di $\mathcal{F}$ su $K$.
	\end{theorem}
	
	\begin{definition}[coniugati di $\alpha$]
		Se $\alpha \in \faktor{L}{K}$ è algebrico su $K$, si definiscono \textbf{coniugati} di $\alpha$ su $K$ le
		radici di $\mu_\alpha$ su $K$.
	\end{definition}
	
	I coniugati di $\alpha$ sono speciali in quanto
	permettono di studiare
	le $K$-immersioni\footnote{
		Una $K$-immersione è un monomorfismo tra estensioni di $K$
		che fissa $K$.
	} di $K(\alpha)$ in $\overline{K}$, ossia
	di studiare i campi $K$-isomorfi a $K(\alpha)$ presenti in
	$\overline{K}$, come dimostra il:
	
	\begin{theorem}[$K$-immersioni da $K(\alpha)$ in $\overline{K}$]
		Sia $\alpha \in \faktor{L}{K}$ algebrico su $K$. Allora,
		se $d$ è il numero di coniugati distinti di $\alpha$,
		esistono esattamente $d$ $K$-immersioni di $K(\alpha)$
		in $\overline{K}$ e sono tali da mandare $\alpha$ in
		un suo altro coniugato.
	\end{theorem}
	
	\begin{proof}
		Per considerare le $K$-immersioni di $K(\alpha)$ in
		$K$, si considera prima l'isomorfismo:
		\[ K(\alpha) \cong K[x] \quot{(\mu_\alpha)}. \]
		Per il Primo teorema di isomorfismo, esistono
		allora tanti omomorfismi da $K(\alpha)$ in $\overline{K}$
		quanti sono gli omomorfismi da $K[x]$ in $\overline{K}$ che
		annullano $(\mu_\alpha)$. Un omomorfismo $\varphi$
		da $K[x]$ a $\overline{K}$ che fissa $K$ è completamente determinato da
		$\beta = \varphi(x)$ ed in particolare mappa $p \in K[x]$
		a $p(\beta)$. Affinché allora $(\mu_\alpha)$
		appartenga a $\Ker \varphi$, $\mu_\alpha(\beta) = 0$, e quindi
		$\beta$ deve essere un coniugato di $\alpha$. Pertanto
		gli omomorfismi da $K(\alpha)$ a $\overline{K}$ sono
		tali per cui $\alpha$ venga mandato in $\beta$. Questi
		omomorfismi
		sono $K$-immersioni dal momento che l'unità viene preservata,
		da cui la tesi.
	\end{proof}
	
	\hr
	
	\begin{definition}[polinomio separabile]
		Un polinomio $p \in K[x]$ si dice \textbf{separabile}
		se $p$ ha radici distinte in un suo campo di
		spezzamento.
	\end{definition}
	
	\begin{definition}[estensione separabile]
		Un'estensione $\faktor{L}{K}$ si dice \textbf{separabile}
		se per ogni $\alpha \in L$, $\mu_{\alpha,K}$ è
		un polinomio separabile.
	\end{definition}
	
	\begin{definition}[campo perfetto]
		Un campo si dice \textbf{perfetto} se ogni suo
		polinomio irriducibile è separabile.
	\end{definition}
	
	\begin{remark}
		Le estensioni di un campo perfetto sono sempre separabili.
		Infatti il polinomio minimo su $K$ è in particolare
		un irriducibile, e quindi ha radici distinte.
	\end{remark}
	
	\begin{note}
		Si assumerà d'ora in poi che \underline{\textit{$K$
		è un campo perfetto}}, in modo tale da semplificare l'introduzione
		alla teoria di Galois.
	\end{note}
	
	\begin{remark}
		Poiché $K$ è perfetto, le $K$-immersioni di $K(\alpha)$
		sono esattamente $[K(\alpha) : K] = \deg_K \alpha$.
	\end{remark}
	
	\begin{remark}
		Se $\varphi_i : K(\alpha) \mono \overline{K}$ è un'estensione di $\varphi : K \mono \overline{K}$, allora
		$\varphi_i(K(\alpha)) = K(\varphi_i(\alpha))$.
	\end{remark}

	Poiché i campi considerati sono perfetti, si possono
	studiare in generale le estensioni di tutte le immersioni
	di $K$ in $\overline{K}$, e quindi non solo le estensioni
	dell'identità, come dimostra il:
	
	\begin{theorem}[estensioni di $\varphi$ da $K(\alpha)$ in $\overline{K}$]
		Sia $\alpha \in \faktor{L}{K}$ algebrico su $K$. Allora
		per ogni $\varphi : K \mono \overline{K}$ esistono
		esattamente $\deg_K \alpha$ estensioni $\varphi_i : K(\alpha) \mono K$ di $\varphi$, ossia monomorfismi per cui $\restr{\varphi_i}{K} = \varphi$. Tali estensioni sono tali da mappare $\alpha$
		nelle radici di $\varphi(\mu_\alpha)$.
	\end{theorem}
	
	\begin{proof}
		Per considerare le estensioni di $\varphi$ da $K(\alpha)$ in
		$K$, si considera prima l'isomorfismo:
		\[ K(\alpha) \cong K[x] \quot{(\mu_\alpha)}. \]
		Per il Primo teorema di isomorfismo, esistono
		allora tanti omomorfismi da $K(\alpha)$ in $\overline{K}$
		quanti sono gli omomorfismi da $K[x]$ in $\overline{K}$ che
		annullano $(\mu_\alpha)$. Un omomorfismo $\varphi_i$
		da $K[x]$ a $\overline{K}$ tale per cui $K$ viene mappato
		tramite $\varphi$ è completamente determinato da
		$\beta = \varphi_i(x)$ ed in particolare mappa $p \in K[x]$
		alla valutazione del polinomio $q$, ottenuto mappando i
		coefficienti di $p$ tramite $\varphi$, in $\beta$, detto
		$\varphi(p)(\beta)$. Affinché allora $(\mu_\alpha)$
		appartenga a $\Ker \varphi$, deve valere $\varphi(\mu_\alpha)(\beta) = 0$, e quindi
		$\beta$ deve essere una radice di $\varphi(\mu_\alpha)$. 
		Pertanto gli omomorfismi da $K(\alpha)$ a $\overline{K}$ sono
		tali per cui $\alpha$ venga mandato nelle radici di
		$\varphi(\mu_\alpha)$. Questi omomorfismi sono
		ancora immersioni dal momento che l'unità viene
		preservata da $\varphi_i$. Dal momento che $\varphi$ è
		a sua volta un'immersione, $\varphi(\mu_\alpha)$ è
		irriducibile dacché $\mu_\alpha$ lo è, ed inoltre
		$\deg \varphi(\mu_\alpha) = \deg \mu_\alpha$. Pertanto,
		poiché $K$ è un campo perfetto,
		le radici di $\varphi(\mu_\alpha)$ sono $\deg_K \alpha$,
		e quindi le estensioni di $\varphi$ sono esattamente
		$\deg_K \alpha$.
	\end{proof}
	
	A partire da questa proposizione, si può dimostrare un
	risultato più generale sulle estensioni finite di $K$,
	come mostra il fondamentale:
	
	\begin{theorem}[estensioni di $\varphi$ da $\faktor{L}{K}$ in $\overline{K}$]
		Sia $[L : K] = n$. Allora per ogni $\varphi : K \mono \overline{K}$ immersione esistono esattamente $n$
		estensioni $\varphi_i : L \to \overline{K}$ di $\varphi$,
		ossia tali per cui $\restr{\varphi_i}{K} = \varphi$.	
	\end{theorem}
	
	\begin{proof}
		Se $n = 1$, la tesi è del tutto ovvia.
		Si dimostra facilmente il teorema per $n \geq 2$ applicando il principio di induzione ed il teorema precedente.
		Se $n = 2$, $L$ è un'estensione semplice di $K$ e quindi
		esiste $\alpha \in L \setminus K$
		tale per cui $L = K(\alpha)$.
		La tesi allora segue applicando il teorema precedente. \medskip
		

		Se $n > 2$, sia $\alpha \in L \setminus K$.
		Sia $[K(\alpha) : K] = m$. Se
		$m = n$, allora $L = K(\alpha)$ e la tesi
		segue ancora applicando il teorema precedente. Se invece
		$m < n$, sia $[L : K(\alpha)] = d$. Per il teorema precedente
		esistono esattamente $m$ estensioni $\varphi_i$ di $\varphi$ da $K(\alpha)$ in $K$. Invece, per il teorema delle torri algebriche,
		$n = md$, e quindi $d < n$. Applicando allora l'ipotesi
		induttiva, ogni $\varphi_i$ può essere unicamente
		esteso in $d$ modi da $K(\alpha)$ a $L$. Pertanto esistono
		solamente $n = md$ estensioni di $\varphi$, concludendo
		il passo induttivo. 
	\end{proof}
\end{document}