\documentclass[12pt]{scrartcl}
\usepackage{notes_2023}

\begin{document}
	\title{Il gruppo diedrale e i suoi sottogruppi}
	\maketitle

	In questo documento si definisce il gruppo diedrale e si illustrano
	le sue proprietà principali, a partire da come sono costruiti i suoi
	sottogruppi. \medskip
	
	
	Sia $n \geq 3$. Si definisce \textbf{gruppo diedrale}, denotato\footnote{
		Alcuni testi denotano il gruppo diedrale come $D_{2n}$, dal
		momento che vale $\abs{D_n} = 2n$.
	} come $D_n$, il gruppo delle isometrie del piano $\RR^2$ che mappano i vertici di
	un poligono regolare centrato nell'origine con $n$ lati in sé stessi. \medskip
	
	
	Si verifica facilmente che $D_n$ è un gruppo:
	\begin{itemize}
		\item Ammette un'identità, che coincide con l'identità delle isometrie,
		\item La composizione di due isometrie che mappano i vertici del poligono in
			sé stessi è ancora un'isometria che lascia fissi i vertici del poligono,
		\item Ogni isometria per cui i vertici del poligono rimangono fissi ammette
			un'inversa con la stessa proprietà\footnote{
				Si ricorda che ogni isometria è invertibile a prescindere.
			}\footnote{
				Dal momento che $D_n$ ha cardinalità $2n$, come mostrato dopo, questa
				condizione è automaticamente verificata come conseguenza della finitezza
				di $D_n$.
			}.
	\end{itemize}
	
	In particolare, se $\sigma \in D_n$, $\sigma$ permuta i vertici del poligono (pertanto
	si può visualizzare $D_n$ come un sottogruppo naturale di $S_n$). Denotando con
	$r$ la rotazione primitiva del gruppo (ossia una rotazione di $\frac{2\pi}{n}$ gradi in
	senso antiorario) e con $s$ la simmetria rispetto all'asse $y$, si osserva che
	ogni elemento della forma $s r^k$ con $k \in \ZZ$ è ancora una simmetria, benché non
	per forza rispetto all'asse $y$\footnote{
		La matrice associata di $s$ nella base canonica è $-1 E_{11} + E_{22}$, e quindi deve valere $\det(s) = -1$. Al contrario $r \in \SOO(2)$, e quindi $\det(r) = 1$. Si conclude pertanto che
		$\det(s r^k) = \det(s) \det(r)^k = -1$, e dunque che $s r^k$ deve obbligatoriamente
		appartenere alla classe laterale $s \SOO(2)$ delle riflessioni.
	}. In particolare, per $n$ pari, le riflessioni di $D_n$ sono esattamente le riflessioni
	rispetto alle rette passanti per i vertici o per i punti medi del poligono. \medskip
	
	
	Dal momento che $\sigma \in D_n$ è in particolare una isometria, e quindi
	un'applicazione lineare, $\sigma$ è completamente determinata da
	$\sigma(V_1)$ e $\sigma(V_2)$, dove $V_i$ sono i vertici del poligono numerati
	in senso antiorario. In particolare, se $\sigma(V_1) = V_k$, allora
	$\sigma(V_2)$, affinché venga preservata la distanza, può valere\footnote{
		Per semplicità si pone $V_0 := V_n$ e $V_{n+1} := V_1$.
	} o
	$V_{k-1}$ o $V_{k+1}$. Pertanto vi sono al più $2n$ scelte possibili di
	$\sigma(V_1)$ e $\sigma(V_2)$ (e quindi $\abs{D_n} \leq 2n$). D'altra parte
	si osserva che tutti gli elementi $1$, $r$, ..., $r^{n-1}$, $s$, $sr$, ..., $s r^{n-1}$
	sono distinti:
	\begin{itemize}
		\item Gli $r^k$ con $0 \leq k \leq \ord(r) - 1$ sono tutti distinti e $\ord(r)$ vale
			esattamente\footnote{
				Infatti $r$ è rappresentato in $\SOO(2)$ dalla matrice $\SMatrix{
					\cos(\frac{2\pi}{n}) & -\sin(\frac{2\pi}{n}) \\
					\sin(\frac{2\pi}{n}) & \cos(\frac{2\pi}{n})
				}$, che ha ordine esattamente $n$.
			} $n$,
		\item Gli $sr^k$ con $0 \leq k \leq n - 1$ sono tutti distinti, altrimenti
			la precedente osservazione sarebbe contraddetta,
		\item Nessun $r^i$ coincide con un $s r^j$, dal momento che i loro determinanti
			sono diversi ($\det(r^i) = 1$, mentre $\det(s r^j) = -1$). In particolare
			$r^i \in \SOO(2)$, mentre $s r^j \in s \SOO(2)$.
	\end{itemize}
	
	Pertanto $\abs{D_n} \geq 2n$, e quindi $\abs{D_n} = 2n$. Si conclude inoltre
	che $D_n$ è generato da $r$ e da $s$, e quindi che $D_n = \gen{r, s}$. Esistono
	dunque due sottogruppi naturali di $D_n$:
	\[ \rotations := \gen{r} \cong \ZZ \quot n\ZZ, \quad \gen{s} \cong \ZZ \quot 2\ZZ. \]
	
	
	\begin{proposition}
		Vale l'identità $s r s\inv = r\inv$.
	\end{proposition}

	\begin{proof}
		Si sviluppa $s r s\inv$ in termini matriciali, considerando
		$s = \SMatrix{
			-1 & 0 \\
			0 & 1
		}$ e $r = \SMatrix{
			\cos(\frac{2\pi}{n}) & -\sin(\frac{2\pi}{n}) \\
			\sin(\frac{2\pi}{n}) & \cos(\frac{2\pi}{n})
		}$:
		\[ s r s\inv = \Matrix{
			\cos(\frac{2\pi}{n}) & \sin(\frac{2\pi}{n}) \\
			-\sin(\frac{2\pi}{n}) & \cos(\frac{2\pi}{n})
		}, \]
		ottenendo la matrice associata a $r\inv$ nella base canonica.
	\end{proof}
	
	
	In generale vale dunque che $s r^k s\inv = r^{-k}$. Si deduce allora la presentazione del gruppo $D_n$:
	\[ D_n = \gen{r,s \mid r^n = 1, s^2 = 1, s r s\inv = r\inv}. \] \smallskip
	
	
	Si descrivono adesso tutti i sottogruppi di $D_n$. Innanzitutto, in $\rotations$
	per ogni $d \mid n$ esiste un unico sottogruppo di ordine $d$ dal momento che
	$\rotations$ è ciclico. Pertanto ogni tale sottogruppo assume la forma
	$\gen{r^{\frac{n}{d}}}$. Inoltre, dal momento che\footnote{
		Infatti ogni elemento di $D_n$, come visto prima, è della forma $r^k$ o $s r^k$.
	} $[D_n : \rotations] = 2$, $\rotations$ è un sottogruppo normale di
	$D_n$. Allora, poiché $\rotations$ è normale in $D_n$ e ogni sottogruppo
	$H \leq \rotations$ è caratteristico\footnote{
		Per ogni ordine di $\rotations$ esiste un unico sottogruppo $H \leq \rotations$,
		e quindi tale sottogruppo deve essere caratteristico.
	} in $D_n$, ogni sottogruppo di $\rotations$ è normale anche in $D_n$. \medskip
	
	
	Sia ora $H$ un sottogruppo di $D_n$ con $H \not\subseteq \rotations$. Si consideri
	la proiezione al quoziente mediante $\rotations$, ossia $\pi_\rotations : D_n \to D_n /
	\rotations$. Chiaramente deve valere che $\pi_\rotations(H) = D_n / \rotations$: l'unica
	altra possibilità è che $\pi_\rotations(H)$ sia $\{\rotations\}$, e quindi che
	$H \subseteq \Ker \pi_\rotations = \rotations$, \Lightning. \medskip
	
	
	Si consideri adesso
	la restrizione di $\pi_\rotations$ ad $H$, $\restr{\pi_\rotations}{H} : H \to D_n / \rotations$. Vale in particolare che $\Ker \restr{\pi_\rotations}{H} = H \cap \Ker \pi_\rotations = H \cap \rotations$ e che $\Im \restr{\pi_\rotations}{H} = D_n / \rotations$
	(da prima vale infatti che $\pi_\rotations(H) = D_n / \rotations$). Allora, per il Primo
	teorema di isomorfismo, vale che:
	\[ \frac{H}{H \cap \rotations} \cong D_n \quot \rotations, \]
	da cui si deduce che $\abs{H} = 2 \abs{H \cap \rotations}$. In particolare $H \cap \rotations$ è un sottogruppo di $\rotations$, e quindi esiste $d \mid n$ tale per cui
	$H \cap \rotations = \gen{r^d}$, con $\abs{H \cap \rotations} = \frac{n}{d}$. \medskip
	
	
	Sia ora $s r^k$ una simmetria di $H$.
	Innanzitutto si osserva che $\gen{r^d}$ è normale in $D_n$ e quindi
	$\gen{ r^d }\gen{ s r^k }$ è effettivamente un sottogruppo di $D_n$. Dal momento che\footnote{
		Infatti l'unica rotazione che è anche una simmetria è l'identità.
	}
	$\gen{ r^d } \cap \gen{ s r^k } = \{ e \}$, allora $\abs{\gen{ r^d } \gen{ s r^k }} = \abs{\gen{r^d}} \abs{\gen{s r^k}} = \frac{2n}{d}$. Anche $\abs{H} = \frac{2n}{d}$ e
	quindi, per questioni di cardinalità, $H = \gen{ r^d } \gen{ s r^k } = \gen{ r^d, s r^k }$.
	\medskip
	
	
	In conclusione, ogni sottogruppo di $D_n$ è della forma $\gen{r^d}$ o della forma
	$\gen{r^d, sr^k}$. Si mostra adesso che per $0 \leq k < d < n$ e $d \mid n$ la
	classificazione è unica e completa. Chiaramente è completa, dal momento che
	$d$ si può sempre ridurre in modulo $n$ e che per $k>d$ si può moltiplicare
	$sr^k$ per riottenere una riflessione con esponente minore di $d$. \medskip
	
	
	Si verifica adesso che è vi è un solo modo di esprimere un sottogruppo in queste
	condizioni. Se $H_1 := \gen{r^{d_1}, s r^{k_1}} = \gen{r^{d_1}, s r^{k_1}} =: H_2$, allora
	in particolare $H_1 \cap \rotations = \gen{r^{d_1}} = \gen{r^{d_2}} = H_2 \cap \rotations$,
	da cui si deduce facilmente che $d_1 = d_2$. Sia ora $s r^{k_1} r^{t d_1} =
	s r^{k_2} r^{t' d_2}$ con $t$, $t' \in \ZZ$. Allora deve valere che:
	\[ r^{k_1 + t d_1} =r^{k_2 + t' d_2} \iff k_1 + t d_1 \equiv k_2 + t' d_2 \pod n, \]
	e quindi, se $d := d_1 = d_2$, questo implica che:
	\[ k_1 \equiv k_2 \pod d \implies k_1 = k_2 \impliedby 0 \leq k_1, k_2 < d. \]
	Pertanto esistono esattamente\footnote{
		La scrittura $d(n)$ indica il numero di divisori di $n$, mentre $\sigma(n)$ indica la 
		somma dei divisori di $n$. Infatti per ogni divisore $d$ di $n$ si conta un
		sottogruppo ciclico $\gen{r^d}$ e un sottogruppo della forma $\gen{r^d, sr^h}$
		con $0 \leq h < d$ (e quindi $d$ sottogruppi di questo tipo).
	} $d(n) + \sigma(n)$ sottogruppi di $D_n$. \bigskip
	
	
	
	Si studiano adesso i sottogruppi normali di $D_n$. Come già visto, i sottogruppi di
	$\rotations$ sono tutti normali in $D_n$. Si studiano dunque soltanto i gruppi
	della forma $\gen{r^d, s r^h}$. Si ricorda che il sottogruppo
	$H = \gen{r^d, s r^h}$ è normale se e solo se $N_G(H) = G$, ossia se il suo
	normalizzatore è tutto $G$. In particolare questo è vero se i generatori di $G$
	appartengono a $N_G(H) = G$ e quindi se $r H r\inv = H$ e se $s H s\inv = H$.
	In particolare\footnote{
		Si è utilizzata la relazione $g \gen{g_1, \ldots, g_i} g\inv = \gen{g g_1 g\inv, \ldots, g g_i g\inv}$.
	} si deve studiare quando valgono le seguenti identità:
	\[
		\gen{r^d, s r^h} = \underbrace{\gen{r^d, s r^{h-2}}}_{r H r\inv},
		\qquad \gen{r^d, s r^h} = \underbrace{\gen{r^{-d}, r^h s\inv}}_{s H s\inv} =
		\gen{r^d, s r^{-h}}.
	\]
	La prima identità è vera se e solo se $h \equiv h-2 \pod d$, e quindi se e solo
	se $d \mid 2$. Pertanto $d$ può valere solo $1$ o $2$: se $d = 1$, $H$ è
	esattamente $\gen{r, s} = D_n$; se invece $d = 2$, $H$ può essere soltanto
	$\gen{r^2, s}$ o $\gen{r^2, s r}$. Ciononostante, nel caso $d = 2$, dal momento
	che $d \mid n$, $n$ deve essere pari (e quindi se $n$ è dispari, $D_n$ non ammette
	sottogruppi normali non banali, ed è in particolare semplice). \medskip
	
	
	Si consideri dunque $n$ pari. È sufficiente controllare che per $d = 2$ valga
	anche la seconda identità, ossia che valga $h \equiv -h \pod 2$, sempre verificata.
	Si conclude dunque con la seguente classificazione:
	
	\begin{itemize}
		\item se $n$ è dispari, $D_n$ ammette come sottogruppi normali soltanto
		$D_n$, $\{ e \}$ e i sottogruppi di $\rotations$,
		\item se $n$ è pari, $D_n$ ammette come sottogruppi normali tutti quelli
			del caso dispari insieme a $\gen{r^2, s}$ e $\gen{r^2, s r}$.
	\end{itemize}
\end{document}