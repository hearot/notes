\documentclass[12pt]{scrartcl}
\usepackage{notes_2023}

\begin{document}
	\title{Il discriminante polinomiale e la formula di Cardano}
	\maketitle
	
	\begin{note}
		Per $K$, $L$ ed $F$ si intenderanno sempre dei campi.
		Se non espressamente detto, si sottintenderà anche
		che $K \subseteq L$, $F$, e che $L$ ed $F$ sono
		estensioni costruite su $K$. Per $[L : K]$ si
		intenderà $\dim_K L$, ossia la dimensione di $L$
		come $K$-spazio vettoriale. Per scopi didattici, si
		considerano solamente campi perfetti, e dunque estensioni che sono sempre separabili, purché
		non esplicitamente detto diversamente.
	\end{note} \bigskip
	
	In questo documento si illustra il \textit{discriminante polinomiale} e
	le sue principali applicazioni nella teoria di Galois.
	
	\begin{definition}[discriminante polinomiale]
		Sia $p \in K[x]$. Se $\deg p = n$ e $a_1$, ..., $a_n \in \overline{K}$ sono le radici
		di $p$, si definisce il \textbf{discriminante polinomiale} $\disc p$ in modo
		tale che:
		\[ \disc p = \prod_{i < j} (a_i - a_j)^2 \in K[a_1, \ldots, a_n]. \]
	\end{definition}
	
	\begin{remark}[radici multiple di $p$ e formule di Viète]
		Si verifica facilmente che $p$ ha radice multiple se e solo se $\disc p = 0$.
		Altrettanto semplicemente si verifica che $\disc p$ è un polinomio simmetrico
		in $a_1$, ..., $a_n$. Pertanto, per il Teorema fondamentale dei polinomi simmetrici,
		$\disc p$ può esprimersi\footnote{
			Un algoritmo per calcolare efficacemente un'espressione di $\disc p$ in
			questo senso è reperibile su \url{https://git.phc.dm.unipi.it/g.videtta/scritti/src/branch/main/Algebra/Notebook/1.\%20Algoritmo\%20di\%20rappresentazione\%20dei\%20polinomi\%20simmetrici}.
		} come elemento di $K[e_1, \ldots, e_n]$,
		dove $e_i := e_i(a_1, \ldots, a_n)$ è il polinomio simmetrico elementare
		negli $a_i$. Per le formule di Viète, i vari $e_i$ possono esprimersi tramite
		i coefficienti $c_i$ di $p(x)$ secondo la seguente relazione:
		\[ c_i = (-1)^{n-i} a e_{n-i}, \]
		dove si pone $e_0 := 1$. Inoltre, l'annullamento di $\disc p$ è indipendente
		dal coefficiente di testa del polinomio, dal momento che polinomi associati
		condividono le stesse radici. \medskip
		

		Per esempio, per $n = 2$, se $p(x) = a x^2 + b x + c$ con $a \neq 0$, vale
		che:
		\[ \disc p(x) = \prod_{i < j} (a_i - a_j)^2 = a_1^2 + a_2^2 - 2 a_i a_j = (a_1 + a_2)^2 - 4 a_i a_j = e_1^2 - 4 e_2, \]
		e dunque, poiché $b = c_1 = (-1)^{2-1} a e_{2-1} = - a e_1$ e
		$c = c_0 = (-1)^2 a e_2 = a e_2$, vale che\footnote{
			In generale, compare sempre un termine $a^{2n-2}$ al denominatore
			di $\disc p(x)$. Pertanto, in letteratura si definisce $\disc p(x)$
			anche come il prodotto tra $a^{2n-2}$ e il discriminante qui definito.
			In tal caso, il discriminante di un polinomio di secondo grado è
			esattamente $\Delta$. 
		}:
		\[ \disc p(x) = \left( -\frac{b}a \right)^2 - 4 \frac{c}{a} = \frac{b^2}{a^2} - 4 \frac{ac}{a^2} = \frac{b^2 - 4ac}{a^2} = \frac{\Delta}{a^2}, \]
		dove $\Delta$ è l'usuale discriminante delle equazioni di secondo grado. Pertanto,
		poiché\footnote{
			Per quanto detto prima, $a$ non svolge alcun ruolo nel determinare se $p$
			ha radici multiple.
		} $a \neq 0$, $p$ ha radici multiple se e solo se $\disc p = 0$, e quindi
		se e solo se $\Delta = 0$.
	\end{remark}
	
	\begin{remark}[utilizzo della matrice di Vandermonde]
		Un'espressione di $\disc p$ può anche essere calcolata attraverso le matrici
		di Vandermonde. Infatti, se $M$ è la matrice di Vandermonde di $a_1$, ..., $a_n$
		radici di $p$, vale che:
		\[ M = \Matrix{1 & a_1 & a_1^2 & \cdots & a_1^{n-1} \\ 1 & a_2 & a_2^2 & \cdots & a_2^{n-1} \\ \vdots & \vdots & \vdots & \vdots & \vdots \\ 1 & a_n & a_n^2 & \cdots & a_n^{n-1}}, \]
		e quindi:
		\[ \det(M) = \prod_{i < j} (a_i - a_j). \]
		Pertanto vale che:
		\[ \det(M^2) = \det(M M^T) = \prod_{i < j} (a_i - a_j)^2 = \disc p(x). \]
	\end{remark}
	
	\begin{remark}[invarianza di $\disc p$ e trasformazione di Tschirnhaus]
		Si osserva facilmente che $\disc p$ è invariante per traslazioni. Infatti,
		se si considera $p(x+a)$ con $a \in K$ e $a_1$, ..., $a_n \in \overline{K}$ sono
		radici di $p(x)$, le radici di $p(x+a)$ sono $a_1 - a$, ..., $a_n - a$. Pertanto
		vale che:
		\[ \disc p(x+a) = \prod_{i<j} (a_1 - a - a_2 + a)^2 = \prod_{i<j} (a_1 - a_2)^2 = \disc p(x). \]
		Sia ora $p(x) = c_n x^n + c_{n-1} x^{n-1} + \ldots + c_1 x + c_0$. Dalle formule
		di Viète, se $a_1$, ..., $a_n$ sono le radici di $p(x)$, vale che:
		\[ c_{n-1} = -c_n (a_1 + \ldots + a_n) \implies \left(a_1 + \frac{c_{n-1}}{n c_n}\right) + \ldots +  \left(a_n + \frac{c_{n-1}}{n c_n}\right) = 0. \]
		Si può allora considerare la \textit{trasformazione di Tschirnhaus} $\tau : K[x] \to K[x]$ tale per cui $p(x) \xmapsto{\tau} p(x +  \frac{c_{n-1}}{n c_n})$,
		$\tau(p)$ ha come radici esattamente i vari $a_i + \frac{c_{n-1}}{n c_n}$, e
		quindi, sempre per le formule di Viète, è tale per cui il coefficiente di $x^{n-1}$
		è nullo. Dal momento che $\disc p$ è invariante per traslazione,
		calcolare $\disc \tau(p)$ può risultare più semplice e dunque più efficiente.
		In letteratura, applicare la trasformazione di Tschirnhaus su un polinomio
		si dice ``deprimerlo'', e un polinomio tale per cui $c_{n-1} = 0$ è
		detto \textit{polinomio depresso}. Si osserva facilmente che deprimere un
		polinomio depresso non ha alcun effetto e la mappa restituisce il polinomio
		depresso di partenza. In particolare vale che $\tau^2 = \tau$.
	\end{remark}

	\begin{remark}
		Sia $p \in K[x]$ di grado $n$ e siano $a_1$, \ldots, $a_n$ le sue radici.
		Se allora $\sigma \in S(\{a_1, \ldots, a_n\})$, vale che:
		\[ \prod_{i < j} (\sigma(a_i) - \sigma(a_j)) = \prod_{i < j} \frac{\sigma(a_i) - \sigma(a_j)}{a_i - a_j} \prod_{i < j} (a_i - a_j) = \sgn(\sigma) \prod_{i < j} (a_i - a_j),  \]
		dove si identifica tale segno come $\sgn(\varphi(\sigma))$, dove
		$\varphi$ è l'azione di del gruppo di Galois di $p$ sulle radici di $p$.
		Pertanto, se $p$ è irriducibile e separabile, e $\sigma \in G := \Gal(L / K)$ dove
		$L = K(a_1, \ldots, a_n)$, vale che:
		\[ \sigma(\disc p) = \disc p, \]
		e quindi $\disc p \in L^G = K$.
	\end{remark}
	
	L'utilità del discriminante polinomiale per la teoria di Galois è sancita dalla
	seguente proposizione:
	
	\begin{proposition}
		Sia $p$ un polinomio irriducibile e separabile di grado $n$. Allora, se $L$ è il suo
		campo di spezzamento su $K$, $\Gal\left(\faktor{L}{K}\right) \mono A_n$ se e
		solo se $\disc p$ è un quadrato\footnote{
			Questa proposizione è ancora vera utilizzando il discriminante moltiplicato
			per $a^{2n-2}$, e quindi vale ancora per la definizione alternativa di
			discriminante.
		} in $K$.
	\end{proposition}
	
	\begin{proof}
		Sia $G := \Gal\left(\faktor{L}{K}\right)$ e sia $\sigma \in G$. Allora $G \mono A_n$ se e solo se
		$\sgn(\sigma) = 1$. Si consideri la seguente identità:
		\[ \sigma\left(\prod_{i < j} (a_i - a_j)\right) =  \sgn(\sigma) \prod_{i < j} (a_i - a_j). \]
		Poiché gli elementi fissati da tutte le $\sigma \in G$ sono esattamente gli
		elementi di $K$, se $G \mono A_n$, $\sgn(\sigma)$ è $1$, e quindi
		$\left(\prod_{i < j} (a_i - a_j)\right) \in K$. Pertanto, $\disc p$ è un
		quadrato in $K$, essendo $\left(\prod_{i < j} (a_i - a_j)\right)$ una sua
		radice quadrata. Analogamente, se $\disc p$ è un quadrato in $K$,
		$\left(\prod_{i < j} (a_i - a_j)\right) \in K$, e quindi
		$\sgn(\sigma) = 1$, da cui la tesi.
	\end{proof}

	\begin{remark}[discriminanti polinomiali per $n \leq 3$]
		Si illustrano i discriminanti polinomiali per alcune specie di polinomio:
		\begin{itemize}
			\item se $p(x) = x-a$, $\disc p(x) = 1$, essendo il prodotto vuoto;
			\item se $p(x) = ax^2 + bx + c$, $\disc p = \frac{\Delta}{a^2}$;
			\item se $p(x) = x^3 + px + q$, $\disc p = -4p^3 - 27 q^2$.
		\end{itemize}
	\end{remark}

	\begin{remark}[classificazione dei gruppi di Galois per $\deg p = 3$]
		Sia $p \in K[x]$ un polinomio di grado $3$ irriducibile e separabile. Sia
		$L$ il campo di spezzamento di $p$ su $K$ e $G := \Gal\left(\faktor{L}{K}\right)$. Si osserva che $3$ divide $\abs{G}$ dal momento che, se $\alpha$ è una radice di $p$,
		$[K(\alpha) : K] = 3 \mid [L : K] = \abs{G}$. Allora,
		se $\disc p$ è un quadrato in $K$, $G \mono A_3$, e dunque, per cardinalità,
		$G \cong A_3$. Se invece $\disc p$ non è quadrato in $K$, $G$ ha
		cardinalità $6$, e dunque
		$G$ è obbligatoriamente isomorfo a $S_3$ stesso. \medskip
		
		
		Pertanto vale che:
		\[  \Gal\left(\faktor{L}{K}\right) \cong \begin{cases}
			A_3 & \se \disc p \text{ è quadrato in $K$}, \\
			S_3 & \altrimenti.
		\end{cases} \]
	\end{remark} \bigskip
	
	
	Si illustra adesso il metodo risolutivo delle equazioni di terzo grado,
	tramite la cosiddetta \textit{formula di Cardano}. Innanzitutto, si
	assume che $p(x)$ sia un polinomio \textit{depresso} di terzo grado
	della forma $x^3 + px + q$ (altrimenti è sufficiente applicare la
	trasformazione di Tschirnhaus a $p(x)$, ricavare le soluzioni e poi
	tornare indietro). \medskip
	
	
	Sia $x = u + v$. Allora $p(u + v) = u^3 + v^3 + 3 u^2 v + 3 u v^2 + p(u + v) + q =
	(u^3 + v^3 + q) + (3uv + p)(u + v)$. Si pone allora il seguente sistema di
	equazioni:
	\[ \begin{cases}
		u^3 + v^3 = -q, \\
		uv = -\frac{p}{3} \implies u^3 v^3 = - \frac{p^3}{27}.
	\end{cases} \]
	Infatti, se il precedente sistema ammette soluzione, $p(x) = p(u + v)$ si annulla e
	$u + v$ è soluzione. \medskip
	
	
	Dal momento che abbiamo sia la somma che il prodotto di $u^3$ e $v^3$, è possibile
	ricavare queste due quantità risolvendo l'equazione di secondo grado
	associata:
	\[ 0 = y^2 - (u^3 + v^3) y + u^3 v^3 = y^2 + q y - \frac{p^3}{27}. \]
	Una volta ottenuti sia $u^3$ che $v^3$, prendendone la radice cubica, si
	otterrà dunque una radice di $p(x)$. In particolare varrà che:
	\[ y_{1,2} = \frac{-q \pm \sqrt{q^2 + \frac{4p^3}{27}}}{2} = -\frac{q}{2} \pm \sqrt{\frac{q^2}{4} + \frac{p^3}{27}}, \]
	e quindi:
	\[ x = u + v = \sqrt[3]{-\frac{q}{2} + \sqrt{\frac{q^2}{4} + \frac{p^3}{27}}} + \sqrt[3]{-\frac{q}{2} - \sqrt{\frac{q^2}{4} + \frac{p^3}{27}}}. \]
	Le altre due soluzioni di $p(x)$ si possono poi computare facilmente riducendosi
	a considerare il polinomio $p(x) / (x-\alpha)$, dove $\alpha$ è la soluzione
	ottenuta.
\end{document}