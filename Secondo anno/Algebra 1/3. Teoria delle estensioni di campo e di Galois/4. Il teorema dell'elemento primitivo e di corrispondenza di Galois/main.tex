\documentclass[12pt]{scrartcl}
\usepackage{notes_2023}

\begin{document}
	\title{Il teorema dell'elemento primitivo e di corrispondenza di Galois}
	\maketitle
	
	\begin{note}
		Per $K$, $L$ ed $F$ si intenderanno sempre dei campi.
		Se non espressamente detto, si sottintenderà anche
		che $K \subseteq L$, $F$, e che $L$ ed $F$ sono
		estensioni costruite su $K$. Per $[L : K]$ si
		intenderà $\dim_K L$, ossia la dimensione di $L$
		come $K$-spazio vettoriale. Per scopi didattici, si
		considerano solamente campi perfetti, e dunque estensioni che sono sempre separabili, purché
		non esplicitamente detto diversamente.
	\end{note} \bigskip

	Si dimostrano in questo documento i due teoremi più
	importanti della teoria elementare delle estensioni di campo
	e di Galois, il \textit{teorema dell'elemento primitivo} ed
	il \textit{teorema di corrispondenza di Galois}.
	
	\begin{theorem}[dell'elemento primitivo]
		Sia $\faktor{L}{K}$ un'estensione separabile e
		finita. Allora $\faktor{L}{K}$ è semplice.
	\end{theorem}
	
	\begin{proof}
		Si distinguono i casi in cui $K$ è un campo finito
		o infinito.
		
		\begin{itemize}
			\item[($K$ finito)\;] Poiché $K$ è finito e
			$L$ è un'estensione finita su $K$, a sua volta
			$L$ è un campo finito. Pertanto $L^*$ è un
			sottogruppo moltiplicativo finito di un campo, ed
			è pertant ciclico. Se $\alpha \in L^*$ è allora
			un generatore di $L^*$, vale che $L$ è uguale a
			$K(\alpha)$. Pertanto $\faktor{L}{K}$ è un'estensione
			semplice.

			\item[($K$ infinito)\;] 
			Si fornisce una dimostrazione costruttiva del
			teorema, che permette di trovare algoritmicamente
			un elemento primitivo per $L$.
			Poiché $L$ è un'estensione
			finita di $K$, $L$ è finitamente generato da
			elementi algebrici su $K$. \medskip
			
			
			Sia allora
			$L = K(\alpha_1, \ldots, \alpha_n)$, dove
			$\{ \alpha_i \}$ è una base di
			$\faktor{L}{K}$ come $K$-spazio. È sufficiente
			che $K(\alpha_1, \alpha_2)$ sia semplice affinché
			anche $L$ lo sia. Infatti
			si dimostrerebbe che $K(\alpha_1, \alpha_2) =
			K(\gamma)$ per qualche $\gamma \in K(\alpha_1, \alpha_2)$,
			e quindi $K(\alpha_1, \ldots, \alpha_n) =
			K(\gamma, \alpha_3, \ldots, \alpha_n)$. Reiterando
			allora il processo su $K(\gamma, \alpha_3)$ si
			troverà un elemento primitivo, e così, induttivamente,
			si dimostra che in particolare $L$ è semplice. Se
			invece $n = 1$, la tesi è ovvia. \medskip
			
			
			Sia allora, senza perdita di generalità, $L = K(\alpha, \beta)$. Sia $[L : K] = n$. Allora, poiché $L$
			è un'estensione separabile su $K$, esistono
			esattamente $n$ distinte $K$-immersioni di $L$,
			dette $\varphi_i$.
			Si definisca allora $p(x) \in \overline{K}[x]$ tale per cui:
			\[ p(x) = \prod_{1 \leq i < j \leq n} (x \varphi_i(\alpha) + \varphi_i(\beta) - x \varphi_j(\alpha) - \varphi_j(\beta)). \]
			Si dimostra che $p(x)$ non è nullo. Infatti,
			se lo fosse, almeno uno dei fattori della produttoria
			dovrebbe essere nullo. In tal caso si avrebbe
			$\varphi_i(\alpha) = \varphi_j(\alpha)$ e
			$\varphi_i(\beta) = \varphi_j(\beta)$, e dunque
			$\varphi_i \equiv \varphi_j$, benché $i \neq j$,
			\Lightning. Allora $\deg p = \binom{n}{2} > 0$.
			Dal momento che $K$ è infinito, esiste\footnote{
				A livello algoritmico è sufficiente valutare
				$p(x)$ in al più $n+1$ valori distinti in $K$
				per ottenere un $x$ funzionale per la tesi.
			} $t \in K$
			tale per cui $p(t) \neq 0$. \medskip
			
			
			Detto $\gamma = \alpha t + \beta$, $\gamma$
			ha esattamente $n$ coniugati. Infatti
			$\varphi_i(\gamma) \neq \varphi_j(\gamma)$ $\forall i < j$, altrimenti $\gamma$ annullerebbe $p(x)$. Pertanto
			$[K(\gamma) : K] = n = [K(\alpha, \beta) : K]$,
			da cui $K(\alpha, \beta) = K(\gamma)$, ossia la tesi.
		\end{itemize}
	\end{proof}
	
	%TODO: aggiungere corrispondenza di Galois
\end{document}