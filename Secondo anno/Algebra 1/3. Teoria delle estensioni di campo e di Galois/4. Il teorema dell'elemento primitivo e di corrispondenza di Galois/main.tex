\documentclass[12pt]{scrartcl}
\usepackage{notes_2023}

\begin{document}
	\title{Il teorema dell'elemento primitivo e di corrispondenza di Galois}
	\maketitle
	
	\begin{note}
		Per $K$, $L$ ed $F$ si intenderanno sempre dei campi.
		Se non espressamente detto, si sottintenderà anche
		che $K \subseteq L$, $F$, e che $L$ ed $F$ sono
		estensioni costruite su $K$. Per $[L : K]$ si
		intenderà $\dim_K L$, ossia la dimensione di $L$
		come $K$-spazio vettoriale. Per scopi didattici, si
		considerano solamente campi perfetti, e dunque estensioni che sono sempre separabili, purché
		non esplicitamente detto diversamente.
	\end{note} \bigskip

	Si dimostrano in questo documento i due teoremi più
	importanti della teoria elementare delle estensioni di campo
	e di Galois, il \textit{teorema dell'elemento primitivo} ed
	il \textit{teorema di corrispondenza di Galois}.
	
	\begin{theorem}[dell'elemento primitivo]
		Sia $\faktor{L}{K}$ un'estensione separabile e
		finita. Allora $\faktor{L}{K}$ è semplice.
	\end{theorem}
	
	\begin{proof}
		Si distinguono i casi in cui $K$ è un campo finito
		o infinito.
		
		\begin{itemize}
			\item[($K$ finito)\;] Poiché $K$ è finito e
			$L$ è un'estensione finita su $K$, a sua volta
			$L$ è un campo finito. Pertanto $L^*$ è un
			sottogruppo moltiplicativo finito di un campo, ed
			è pertant ciclico. Se $\alpha \in L^*$ è allora
			un generatore di $L^*$, vale che $L$ è uguale a
			$K(\alpha)$. Pertanto $\faktor{L}{K}$ è un'estensione
			semplice.

			\item[($K$ infinito)\;] 
			Si fornisce una dimostrazione costruttiva del
			teorema, che permette di trovare algoritmicamente
			un elemento primitivo per $L$.
			Poiché $L$ è un'estensione
			finita di $K$, $L$ è finitamente generato da
			elementi algebrici su $K$. \medskip
			
			
			Sia allora
			$L = K(\alpha_1, \ldots, \alpha_n)$, dove
			$\{ \alpha_i \}$ è una base di
			$\faktor{L}{K}$ come $K$-spazio. È sufficiente
			che $K(\alpha_1, \alpha_2)$ sia semplice affinché
			anche $L$ lo sia. Infatti
			si dimostrerebbe che $K(\alpha_1, \alpha_2) =
			K(\gamma)$ per qualche $\gamma \in K(\alpha_1, \alpha_2)$,
			e quindi $K(\alpha_1, \ldots, \alpha_n) =
			K(\gamma, \alpha_3, \ldots, \alpha_n)$. Reiterando
			allora il processo su $K(\gamma, \alpha_3)$ si
			troverà un elemento primitivo, e così, induttivamente,
			si dimostra che in particolare $L$ è semplice. Se
			invece $n = 1$, la tesi è ovvia. \medskip
			
			
			Sia allora, senza perdita di generalità, $L = K(\alpha, \beta)$. Sia $[L : K] = n$. Allora, poiché $L$
			è un'estensione separabile su $K$, esistono
			esattamente $n$ distinte $K$-immersioni di $L$,
			dette $\varphi_i$.
			Si definisca allora $p(x) \in \overline{K}[x]$ tale per cui:
			\[ p(x) = \prod_{1 \leq i < j \leq n} (x \varphi_i(\alpha) + \varphi_i(\beta) - x \varphi_j(\alpha) - \varphi_j(\beta)). \]
			Si dimostra che $p(x)$ non è nullo. Infatti,
			se lo fosse, almeno uno dei fattori della produttoria
			dovrebbe essere nullo. In tal caso si avrebbe
			$\varphi_i(\alpha) = \varphi_j(\alpha)$ e
			$\varphi_i(\beta) = \varphi_j(\beta)$, e dunque
			$\varphi_i \equiv \varphi_j$, benché $i \neq j$,
			\Lightning. Allora $\deg p = \binom{n}{2} > 0$.
			Dal momento che $K$ è infinito, esiste\footnote{
				A livello algoritmico è sufficiente valutare
				$p(x)$ in al più $n+1$ valori distinti in $K$
				per ottenere un $x$ funzionale per la tesi.
			} $t \in K$
			tale per cui $p(t) \neq 0$. \medskip
			
			
			Detto $\gamma = \alpha t + \beta$, $\gamma$
			ha esattamente $n$ coniugati. Infatti
			$\varphi_i(\gamma) \neq \varphi_j(\gamma)$ $\forall i < j$, altrimenti $\gamma$ annullerebbe $p(x)$. Pertanto
			$[K(\gamma) : K] = n = [K(\alpha, \beta) : K]$,
			da cui $K(\alpha, \beta) = K(\gamma)$, ossia la tesi.
		\end{itemize}
	\end{proof} \medskip


	Si illustrano adesso i prerequisiti per dimostrare
	il Teorema di corrispondenza di Galois:
	
	\begin{definition}
		Sia $\faktor{L}{K}$ un'estensione di Galois. Allora,
		se $H \leq \Gal(\faktor{L}{K})$, si definisce
		$L^H = \Fix(H)$ come la sottoestensione di $L$ su
		$K$ degli elementi fissati da ogni $\varphi \in H$,
		ossia:
		\[ L^H = \{ \alpha \in L \mid \varphi(\alpha) = \alpha \, \forall \varphi \in H \}. \]
	\end{definition}

	\begin{nlemma}
		Sia $\faktor{L}{K}$ un'estensione di Galois. Allora,
		se $H \leq \Gal(\faktor{L}{K})$ vale che:
		\[ L^H = K \iff H = \Gal(\faktor{L}{K}). \]
	\end{nlemma}

	\begin{proof}
		Sia $H = \Gal(\faktor{L}{K})$. Allora sicuramente
		$K \subseteq L^H$. Si mostra che non può valere
		$K \subsetneq L^H$. Se infatti $K \subsetneq L^H$,
		varrebbe che $[L^H : K] > 1$, e quindi esisterebbe
		una $K$-immersione non banale di $L^H$, detta
		$\varphi : L^H \to \overline{K}$. In particolare
		$\varphi$ può estendersi a una $K$-immersione di
		$L$, detta $\tilde{\varphi}$. In particolare
		$\tilde{\varphi} \in \Gal(\faktor{L}{K})$, e
		quindi $\tilde{\varphi}$ deve fissare $L^H$ per
		ipotesi. Tuttavia $\tilde{\varphi}$ ristretta
		a $L^H$ non fissa $L^H$ per ipotesi, \Lightning.
		Pertanto $L^H = K$. \medskip
		
		
		Sia adesso $L^H = K$. Per il Teorema dell'elemento
		primitivo, $\exists \alpha \in L^H$ tale per cui
		$L = K(\alpha)$. Si consideri allora il
		polinomio $p$ a coefficienti in $\overline{K}$ tale
		per cui:
		\[ p(x) = \prod_{\varphi \in H} (x - \varphi(\alpha)). \]
		Poiché l'identità di $\Gal(\faktor{L}{K})$ appartiene
		ad $H$, $(x-\alpha) \mid p(x)$, e quindi
		$p(\alpha) = 0$. Inoltre $p$ è in realtà un polinomio
		a coefficienti in $L^H$. Se infatti $\rho \in H$,
		\[ \rho(p(x)) = \prod_{\varphi \in H} (x - \rho(\varphi(\alpha))) = p(x), \]
		dove l'uguaglianza è dovuta al fatto\footnote{
			In particolare è stato applicato l'\textit{embedding} di Cayley su $H$
			attraverso l'elemento $\rho \in H$, e
			quest'azione si è rivelata essere transitiva.
		} che le
		mappe $\{ \rho \circ \varphi \}$ sono esattamente le
		mappe $\{ \varphi \}$. Pertanto $\abs{\Gal(\faktor{L}{K})} = [L : K] =
		[K(\alpha) : K] \leq \deg p(x) = \abs{H}$ dal
		momento che $\alpha$ è radice di $p(x)$. Dal momento
		che vale anche che $\abs{\Gal(\faktor{L}{K})} \geq \abs{H}$, allora
		$H = \Gal(\faktor{L}{K})$, da cui la tesi.
	\end{proof}

	\begin{proposition}
		Sia $\sigma \in \Gal{\faktor{L}{K}}$. Allora,
		se $H \leq \faktor{L}{K}$, vale che
		$\sigma(L^H) = L^{\sigma H \sigma\inv}$.
	\end{proposition}

	\begin{proof}
		Si osserva che:
		\[ \sigma(L^H) = \{ \sigma(\alpha) \mid \alpha \in L, \, \varphi(\alpha) = \alpha \, \forall \varphi \in H \} = \{ \beta \in L \mid \varphi(\sigma\inv(\beta)) = \sigma\inv(\beta) \, \forall \varphi \in H \}, \]
		dove si è sfruttato in modo cruciale il fatto che $\varphi \in H$ è bigettiva. Si
		conclude allora che:
		\[ \varphi(L^H) = \{ \beta \in L \mid \sigma(\varphi(\sigma\inv(\beta))) = \beta \, \forall \varphi \in H \} = L^{\sigma H \sigma\inv}. \]
	\end{proof}

	Si può adesso dimostrare il Teorema di corrispondenza di
	Galois:
	
	\begin{theorem}[di corrispondenza di Galois]
		Sia $\mathcal{E}$ l'insieme delle sottoestensioni
		di $\faktor{L}{K}$ estensione di Galois. Sia
		$\mathcal{G}$ l'insieme dei sottogruppi di
		$\Gal(\faktor{L}{K})$. Allora $\mathcal{E}$ è
		in bigezione con $\mathcal{G}$ attraverso
		la mappa $\alpha : \mathcal{E} \to \mathcal{G}$
		tale per cui:
		\[ F \xmapsto{\alpha} \Gal(\faktor{L}{F}) \leq
			\Gal(\faktor{L}{K}), \]
		la cui inversa $\beta : \mathcal{G} \to \mathcal{E}$
		è tale per cui:
		\[ H \xmapsto{\beta} L^H \subseteq L. \]
		Inoltre, una sottoestensione $\faktor{F}{K}$ di
		$\faktor{L}{K}$ è normale su $K$ se e solo se
		il corrispondente sottogruppo di $\Gal(\faktor{L}{K})$
		è normale. Infine, se $\faktor{F}{K}$ è normale,
		$F$ è in particolare di Galois\footnote{
			Si ricorda che si considera $K$ un campo perfetto.
		} e vale che:
		\[ \Gal(\faktor{F}{K}) \cong \faktor{\Gal(\faktor{L}{K})}{\Gal(\faktor{L}{F})}. \]
	\end{theorem}

	\begin{proof}
		Le mappe $\alpha$ e $\beta$ sono ovviamente ben definite. Si mostra direttamente che sono l'una
		l'inversa dell'altra. Sia $H \leq \Gal(\faktor{L}{K})$.
		Si osserva che:
		\[ \alpha(\beta(H)) = \alpha(L^H) = \Gal(\faktor{L}{L^H}). \]
		Sia $L^H = M$. Se si pone $K = \Gal(\faktor{L}{L^H})$, vale chiaramente che $H \leq K$ dal momento che $H$
		fissa per definizione tutti gli elementi di 
		$L^H$. Dacché allora $L^H = M$, per il lemma precedente
		$H = K$, e quindi $\alpha(\beta(H)) = H$. \medskip
		
		
		Analogamente si osserva che per $K \subseteq F \subseteq K$ vale che:
		\[ \beta(\alpha(F)) = \beta(\Gal(\faktor{L}{F})) =
			L^{\Gal(\faktor{L}{F})}. \]
		Pertanto, detto $H = \Gal(\faktor{L}{F})$,
		per il lemma precedente vale che $L^H = F$,
		e quindi $\beta(\alpha(F)) = F$, dimostrando
		la prima parte del teorema. \medskip
		
		
		Sia ora $\faktor{F}{K}$ una sottoestensione normale
		di $\faktor{L}{K}$. Allora, se $\varphi \in \Gal(\faktor{L}{F})$ e $\sigma \in \Gal(\faktor{L}{K})$,
		$\tau = \sigma \circ \varphi \circ \sigma\inv$ è ancora
		un elemento di $\faktor{L}{K}$. Pertanto,
		$\tau$ si può restringere ad una $K$-immersione di $F$.
		Poiché allora $F$ è normale su $K$, $\tau(F) = F$, e
		quindi $\tau \in \Gal(\faktor{L}{F})$, e dunque
		$\Gal(\faktor{L}{F}) \nsgeq \Gal(\faktor{L}{K})$. \medskip
		
		
		Sia adesso $\Gal(\faktor{L}{F}) \nsgeq \Gal(\faktor{L}{K})$. Sia $\varphi$ una $K$-immersione
		di $F$ su $\overline{K}$. Allora $\varphi$ può essere
		estesa ad un elemento $\tilde{\varphi} \in \Gal(\faktor{L}{K})$. In particolare,
		se $H = \Gal(\faktor{L}{F})$, $\varphi(F) =
		\tilde{\varphi}(F) = L^{\varphi H \varphi\inv} = L^H = F$, dove si è sfruttata la normalità di $H$ in
		$\Gal(\faktor{L}{K})$. Pertanto $F$ è normale su $K$,
		e dunque, in quanto separabile per ipotesi, di Galois. \medskip
		
		
		Si consideri adesso l'omomorfismo $\tau : \Gal(\faktor{L}{K}) \to \Gal(\faktor{F}{K})$ dato
		dalla restrizione delle immersioni di $\Gal(\faktor{L}{K})$ su $F$. Chiaramente $\tau$
		è una mappa surgettiva, dal momento che ogni $K$-immersione di $\Gal(\faktor{F}{K})$ può estendersi
		a $K$-immersione di $\Gal(\faktor{L}{K})$. Inoltre
		vale che $\Ker \tau$ è esattamente il sottogruppo
		di $\Gal(\faktor{L}{K})$ che fissa $F$, ossia
		$\Gal(\faktor{L}{F})$. Applicando allora il Primo
		teorema di isomorfismo vale che:
		\[ \Gal(\faktor{F}{K}) \cong \faktor{\Gal(\faktor{L}{K})}{\Gal(\faktor{L}{F})}, \]
		da cui la tesi.
	\end{proof}

	\begin{example}[studio dei sottocampi di $\QQ(\sqrt{2}, \sqrt{3}) \quot \QQ$]
		Dal momento che $L := \QQ(\sqrt{2}, \sqrt{3})$ è
		il campo di spezzamento dei polinomi $x^2 - 2$
		e $x^2 - 3$, tale estensione è normale su $\QQ$,
		e quindi di Galois. Inoltre, dal momento che
		$\sqrt{3} \notin \QQ(\sqrt{2})$,
		$[L : \QQ] = 2 \cdot 2 = 4$, dal Teorema delle
		torri algebriche. Pertanto $\Gal(\faktor{L}{\QQ})$
		è un gruppo di ordine $4$. \medskip
		
		
		Si definisce $\varphi_{ij}$ con $i$, $j \in \{0, 1\}$
		come le $\QQ$-immersioni di $L$
		tali per cui $\sqrt{2} \xmapsto{\varphi_{ij}} (-1)^i \sqrt{2}$ e analogamente $\sqrt{3} \xmapsto{\varphi_{ij}} (-1)^j \sqrt{3}$. Dal momento
		che le varie $\varphi_{ij}$ sono distinte,
		che ogni $\varphi_{ij}$ ha ordine $2$ e che ogni
		gruppo di ordine $4$ è abeliano (o, più semplicemente,
		le varie $\varphi_{ij}$ commutano tra loro),
		vale che $\Gal(\faktor{L}{\QQ}) \cong \ZZmod2 \times \ZZmod2$. \medskip
		
		
		Ogni sottoestensione di $L$ ha
		grado su $\QQ$ divisore di $[L : \QQ]$, e quindi
		ha grado $1$, $2$ o $4$. Se il grado è $4$,
		la sottoestensione considerata è proprio $L$, mentre
		se il grado è $1$ la sottoestensione è $\QQ$ stesso.
		Si studiano ora le sottoestensioni di grado $2$.
		Tali sottoestensioni corrispondono ai sottogruppi
		di $\Gal(\faktor{L}{\QQ})$ di ordine $4/2 = 2$. Inoltre,
		a priori, essendo $\Gal(\faktor{L}{\QQ})$ abeliano, tutte
		le sottoestensioni sono normali su $\QQ$. \medskip
		
		
		Ogni sottogruppo di ordine $2$ è ciclico e generato
		da elementi di ordine $2$, e quindi, mantenendo
		la corrispondenza con $\ZZmod2 \times \ZZmod2$,
		da $(1,0)$, $(0,1)$ o $(1,1)$. Pertanto esistono
		esattamente $3$ sottoestensioni distinte di grado $2$
		su $\QQ$. \medskip
		
		
		In particolare queste sottoestensioni corrispondono
		ai sottocampi di $L$ fissati da $\varphi_{10}$,
		$\varphi_{01}$ e $\varphi_{11}$, ossia
		$\QQ(\sqrt{3})$, $\QQ(\sqrt{2})$ e $\QQ(\sqrt{6})$. \medskip
		
		
		Inoltre $\alpha := \sqrt{2} + \sqrt{3}$ è un elemento primitivo
		di $L$, dal momento che non può appartenere né a
		$\QQ(\sqrt{3})$ né a $\QQ(\sqrt{2})$ (altrimenti
		tali sottoestensioni coinciderebbero con $L$, \Lightning), e così nemmeno a $\QQ(\sqrt{6})$ (altrimenti $\alpha$ si scriverebbe
		come combinazione lineare di $1$ e $\sqrt{6}$,
		\Lightning). Alternativamente $\alpha$
		ha esattamente $4$ coniugati tramite
		le varie\footnote{
			Tali $4$ coniugati sono distinti dal momento
			che $\{1, \sqrt{2}, \sqrt{3}, \sqrt{6}\}$ è
			una base di $L$ come $\QQ$-spazio.
		} $\varphi_{ij}$, e quindi ha grado $4$ su
		$\QQ$. In particolare vale che:
		\[ \mu_\alpha(x) = \prod_{i=0}^1 \prod_{j=0}^1 (x + (-1)^i \sqrt{2} + (-1)^j \sqrt{3}) = x^4 - 10x^2 + 1. \]
		
		In modo analogo si ottengono i polinomi minimi
		di $\sqrt{2} + \sqrt{3}$ su $\QQ(\sqrt2)$,
		$\QQ(\sqrt3)$ e $\QQ(\sqrt6)$, rispettivamente
		$x^2-2\sqrt{2}x-1 = (x-\sqrt{2})^2 - 3$,
		$x^2-2\sqrt{3}x-1 = (x-\sqrt{3})^2 - 2$ e
		$x^2-(\sqrt{2} + \sqrt{3})^2 = x^2 - 2\sqrt{6} - 5$.
		Tutte le informazioni sono infine raccolte nel seguente
		diagramma di estensioni:
		\[\begin{tikzcd}[column sep=2.25em]
			&& {\overbrace{\mathbb{Q}(\sqrt{2}, \sqrt{3})}^{\mathbb{Q}(\sqrt{2} + \sqrt3)}} \\
			\\
			{\mathbb{Q}(\sqrt{2})} && {\mathbb{Q}(\sqrt6)} && {\mathbb{Q}(\sqrt3)} \\
			\\
			&& {\mathbb{Q}}
			\arrow["2"{pos=0.3}, no head, from=1-3, to=3-1]
			\arrow["2"', no head, from=1-3, to=3-3]
			\arrow["2"', no head, from=1-3, to=3-5]
			\arrow["2"', no head, from=3-3, to=5-3]
			\arrow["2"{description}, no head, from=3-5, to=5-3]
			\arrow["2"{description}, no head, from=3-1, to=5-3]
			\arrow["{\small x^4 - 10x^2 + 1}", curve={height=-24pt}, no head, from=1-3, to=5-3]
			\arrow["{x^2-3}", curve={height=-12pt}, no head, from=3-5, to=5-3]
			\arrow["{x^2-2}"', curve={height=12pt}, no head, from=3-1, to=5-3]
			\arrow["{x^2-6}"'{pos=0.3}, shift left, curve={height=12pt}, no head, from=3-3, to=5-3]
			\arrow["{x^2-2\sqrt3\,x+1}", curve={height=-18pt}, no head, from=1-3, to=3-5]
			\arrow["{x^2-2\sqrt2\,x-1}"', curve={height=18pt}, no head, from=1-3, to=3-1]
			\arrow["{\small x^2-2\sqrt6-5}"'{pos=0.8}, curve={height=12pt}, no head, from=1-3, to=3-3]
		\end{tikzcd}\]
		Tramite la corrispondenza di Galois abbiamo fatto
		corrispondere questo diagramma al seguente diagramma
		di gruppi:
		\[\begin{tikzcd}
			& {\{ \text{Id}_L \equiv \varphi_{00} \}} \\
			\\
			{\{ \varphi_{00}, \varphi_{01} \}} & {\{ \varphi_{00}, \varphi_{11} \}} & {\{\varphi_{00}, \varphi_{10}\}} \\
			\\
			& {\{\varphi_{01}, \varphi_{10}, \varphi_{01}, \varphi_{11}\}}
			\arrow[no head, from=1-2, to=3-2]
			\arrow[no head, from=1-2, to=3-1]
			\arrow[no head, from=1-2, to=3-3]
			\arrow[no head, from=3-2, to=5-2]
			\arrow[no head, from=3-3, to=5-2]
			\arrow[no head, from=3-1, to=5-2]
		\end{tikzcd}\]
	\end{example}
\end{document}