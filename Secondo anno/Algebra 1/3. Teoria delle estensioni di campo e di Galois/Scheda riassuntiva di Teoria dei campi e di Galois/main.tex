\documentclass[10pt,landscape]{article}
\usepackage{amssymb,amsmath,amsthm,amsfonts}
\usepackage{multicol,multirow}
\usepackage{marvosym}
\usepackage{calc}
\usepackage{ifthen}
\usepackage[landscape]{geometry}
\usepackage[colorlinks=true,citecolor=blue,linkcolor=blue]{hyperref}
\usepackage{notes_2023}

\setlength{\extrarowheight}{0pt}

\ifthenelse{\lengthtest { \paperwidth = 11in}}
{ \geometry{top=.5in,left=.5in,right=.5in,bottom=.5in} }
{\ifthenelse{ \lengthtest{ \paperwidth = 297mm}}
	{\geometry{top=1cm,left=1cm,right=1cm,bottom=1cm} }
	{\geometry{top=1cm,left=1cm,right=1cm,bottom=1cm} }
}
%\pagestyle{empty}
\makeatletter
\renewcommand{\section}{\@startsection{section}{1}{0mm}%
	{-1ex plus -.5ex minus -.2ex}%
	{0.5ex plus .2ex}%x
	{\normalfont\large\bfseries}}
\renewcommand{\subsection}{\@startsection{subsection}{2}{0mm}%
	{-1explus -.5ex minus -.2ex}%
	{0.5ex plus .2ex}%
	{\normalfont\normalsize\bfseries}}
\renewcommand{\subsubsection}{\@startsection{subsubsection}{3}{0mm}%
	{-1ex plus -.5ex minus -.2ex}%
	{1ex plus .2ex}%
	{\normalfont\small\bfseries}}
\makeatother
\setcounter{secnumdepth}{0}
\setlength{\parindent}{0pt}
\setlength{\parskip}{0pt plus 0.5ex}
% -----------------------------------------------------------------------

\title{Scheda riassuntiva di Teoria dei campi e di Galois}

\begin{document}
	
	\parskip=0.7ex
	
	\raggedright
	\footnotesize
	
	\begin{center}
		\Large{\textbf{Scheda riassuntiva di Teoria dei campi e di Galois}} \\
	\end{center}
	\begin{multicols}{3}
		\setlength{\premulticols}{1pt}
		\setlength{\postmulticols}{1pt}
		\setlength{\multicolsep}{1pt}
		\setlength{\columnsep}{2pt}
		
		\section{Campi e omomorfismi}
		
		Si dice \textbf{campo} un anello commutativo non banale
		$K$ che è
		contemporaneamente anche un corpo. Si dice
		\textbf{omomorfismo di campo} tra due campi $K$ ed $L$
		un omomorfismo di anelli. Dal momento che un omomorfismo
		$\varphi$ è tale per cui $\Ker \varphi$ è un ideale
		di $K$ con $1 \notin \Ker \varphi$, deve per forza
		valere $\Ker \varphi = \{0\}$, e quindi ogni omomorfismo
		di campi è un'immersione. \medskip
		
		
		\section{Caratteristica di un campo}
		
		Dato l'omomorfismo $\zeta : \ZZ \to K$ completamente
		determinato dalla relazione $1 \xmapsto{\zeta} 1_K$,
		si definisce \textbf{caratteristica di $K$}, detta
		$\Char K$, il
		generatore non negativo di $\Ker \zeta$. In particolare
		$\Char K$ è $0$ o un numero primo. Se $\Char K$ è zero,
		$\zeta$ è un'immersione, e quindi $K$ è un campo infinito,
		e in particolare vi si immerge anche $\QQ$. \medskip
		
		
		Tuttavia non è detto che $\Char K = p$ implichi che $K$ è
		finito. In particolare $\ZZ_p(x)$, il campo delle funzioni
		razionali a coefficienti in $\ZZ_p$, è un campo infinito
		a caratteristica $p$.
		
		
		\subsection{Proprietà dei campi a caratteristica $p$}
		
		Se $\Char K = p$, per il Primo
		teorema di isomorfismo per anelli, $\ZZmod{p}$ si immerge
		su $K$ tramite la proiezione di $\zeta$; pertanto
		$K$ contiene una copia isomorfa di $\ZZmod{p}$. Per
		campi di caratteristica $p$, vale il Teorema del
		binomio ingenuo, ossia:
		\[ (a + b)^p = a^p + b^p, \]
		estendibile anche a più addendi.
		In particolare, per un campo $K$ di caratteristica $p$,
		la mappa $\Frob : K \to K$ tale per cui $a \xmapsto{\Frob} a^p$
		è un omomorfismo di campi, ed in particolare è un'immersione
		di $K$ in $K$, detta \textbf{endomorfismo di Frobenius}. Se $K$ è un campo finito, $\Frob$ è anche
		un isomorfismo. Si osserva che per gli elementi della
		copia $K \supseteq \FF_p \cong \ZZmod{p}$ vale
		$\restr{\Frob}{\FF_p} = \Id_{\FF_p}$, e quindi
		$\Frob$ è un elemento di $\Gal(K / \FF_p)$. 
		
		
		\section{Campi finiti}
		
		
		Per ogni $p$ primo e $n \in \NN^+$ esiste un campo finito
		di ordine $p^n$. In particolare, tutti i campi finiti di
		ordine $p^n$ sono isomorfi tra loro, possono essere visti
		come spazi vettoriali di dimensione $n$ sull'immersione di $\ZZmod{p}$ che contengono,
		e come campi di spezzamento di $x^{p^n}-x$
		su tale immersione. Tali campi hanno obbligatoriamente
		caratteristica $p$, dove $\abs{K} = p^n$. Esiste
		sempre un isomorfismo tra due campi finiti che manda la copia isomorfa di $\ZZmod{p}$ di uno nell'altra. \medskip
		
		
		Poiché i campi finiti di medesima cardinalità sono isomorfi,
		si indicano con $\FF_p$ e $\FF_{p^n}$ le strutture
		algebriche di tali campi. In particolare con
		$\FF_{p^n} \subseteq \FF_{p^m}$ si intende che
		esiste un'immersione di un campo con $p^n$ elementi in
		uno con $p^m$ elementi, e analogamente si farà con
		altre relazioni (come l'estensione di campi)
		tenendo bene in mente di star
		considerando tutti i campi di tale ordine. \medskip
		
		
		Vale la relazione $\FF_{p^n} \subseteq \FF_{q^m}$
		se e solo se $p=q$ e $n \mid m$. Conseguentemente,
		l'estensione minimale per inclusione comune a
		$\FF_{p^{n_1}}$, ..., $\FF_{p^{n_i}}$ è
		$\FF_{p^m}$ dove $m := \mcm(n_1, \ldots, n_i)$. Pertanto
		se $p \in \FF_{p^n}[x]$ si decompone in fattori irriducibili
		di grado $n_1$, \ldots, $n_i$, il suo campo di spezzamento
		è $\FF_{p^m}$. Inoltre, $x^{p^n}-x$ è in $\FF_p$ il
		prodotto di tutti gli irriducibili di grado divisore
		di $n$.
		
		\section{Proprietà dei polinomi di $K[x]$}
		
		Per il Teorema di Lagrange sui campi, ogni polinomio
		di $K[x]$ ammette al più tante radici quante il suo grado.
		Come conseguenza pratica di questo teorema, ogni sottogruppo
		moltiplicativo finito di $K$ è ciclico. Pertanto
		$\FF_{p^n}^* = \gen{\alpha}$ per $\alpha \in \FF_{p^n}$,
		e quindi $\FF_{p^n} = \FF_p(\alpha)$, ossia
		$\FF_{p^n}$ è sempre un'estensione semplice su $\FF_p$. Si dice
		\textbf{campo di spezzamento} di una famiglia $\mathcal{F}$
		di polinomi di $K[x]$ un sovracampo minimale per
		inclusione di $K$ che fa sì che ogni polinomio di $\mathcal{F}$ si decomponga in fattori lineari. I campi
		di spezzamento di $\mathcal{F}$ sono sempre
		$K$-isomorfi tra loro. Per il criterio della derivata,
		$p \in K[x]$ ammette radici multiple se e solo se
		$\MCD(p, p')$ non è invertibile, dove $p'$ è la derivata
		formale di $p$. \medskip
		
		
		Se $p$ è irriducibile in $K[x]$, $(p)$ è un ideale
		massimale, e $K[x] / (p)$ è un campo che
		ne contiene una radice, ossia $[x]$. In
		particolare $K$ si immerge in $K[x] / (p)$,
		e quindi tale campo può essere identificato come
		un'estensione di $K$ che aggiunge una radice di $p$.
		Se $K$ è finito, detta $\alpha$ la radice aggiunta
		all'estensione, $L := K[x] / (p) \cong K(\alpha)$ contiene
		tutte le radici di $p$ (ed è dunque il suo campo
		di spezzamento). Infatti detto $[L : \FF_p] = n$,
		$[x]$ annulla $x^{p^n}-x$ per il Teorema di Lagrange
		sui gruppi, e quindi $p$ deve dividere $x^{p^n}-x$;
		in tal modo $p$ deve spezzarsi in fattori lineari,
		e quindi ogni radice deve già appartenere ad $L$.
		In particolare, ogni estensione finita e semplice
		di un campo finito è normale, e quindi di Galois. \medskip
		
		\section{Estensioni di campo}
		
		
		Si dice che $L$ è un'estensione di $K$, e si indica
		con $L / K$, se $L$ è un sovracampo di $K$,
		ossia se $K \subseteq L$. Si indica con $[L : K] = \dim_K L$ la
		dimensione di $L$ come $K$-spazio vettoriale. Si
		dice che $L$ è un'estensione finita di $K$ se $[L : K]$
		è finito, e infinita altrimenti. Un'\textbf{estensione finita}
		di un campo finito è ancora un campo finito. Un'estensione
		è finita se e solo se è finitamente generata da elementi algebrici. Una $K$-immersione è un omomorfismo di campi
		iniettivo da un'estensione di $K$ in un'altra estensione di $K$ che
		agisce come l'identità su $K$. Un $K$-isomorfismo è
		una $K$-immersione che è isomorfismo. \medskip
		
		
		Date estensioni $L$ e $M$ su $K$, si definisce
		$LM = L(M) = M(L)$ come il \textbf{composto} di $L$
		ed $M$, ossia come la più piccola estensione di $K$ che
		contiene sia $L$ che $M$. In particolare, $LM$
		può essere visto come $L$-spazio vettoriale con
		vettori in $M$, o analogamente come $M$-spazio con
		vettori in $L$.
		
		
		\subsection{Omomorfismo di valutazioni, elementi algebrici e trascendenti e polinomio minimo}
		
		
		Dato $\alpha$, si definisce $K(\alpha)$ il più piccolo
		sovracampo di $K$ che contiene $\alpha$. Si definisce l'\textbf{omomorfismo di
		valutazione} $\varphi_{\alpha, K} : K[x] \to K[\alpha]$, detto
		$\varphi_\alpha$ se $K$ è noto, l'omomorfismo completamente
		determinato dalla relazione $p \xmapsto{\varphi_\alpha} p(\alpha)$. Si verifica che $\varphi_\alpha$ è
		surgettivo. Se $\varphi_\alpha$ è iniettivo,
		si dice che $\alpha$ è \textbf{trascendentale} su $K$ e
		$K[x] \cong K[\alpha]$, da cui $[K[\alpha] : K] =
		[K[x] : K] = \infty$. Se invece $\varphi_\alpha$ non
		è iniettivo, si dice che $\alpha$ è \textbf{algebrico}
		su $K$. Si definisce $\mu_\alpha$, detto il \textbf{polinomio
		minimo} di $\alpha$ su $K$, il generatore monico
		di $\Ker \varphi_\alpha$. IDal momento che $K$ è
		in particolare un dominio di integrità, $\mu_\alpha$ è sempre irriducibile. \medskip
		
		
		Si definisce
		$\deg_K \alpha := \deg \mu_\alpha$. Se $\alpha$ è
		algebrico su $K$, $K[x] / (\mu_\alpha) \cong
		K[\alpha]$, e quindi $K[\alpha]$ è un campo. Dacché
		$K[\alpha] \subseteq K(\alpha)$, vale allora
		$K[\alpha] = K(\alpha)$. Inoltre, poiché $\dim_K K[x] / (\mu_\alpha) = \deg_K \alpha$, vale
		anche che $[K(\alpha) : K] = \deg_K \alpha$.
		Infine, si verifica che $\alpha$ è algebrico se e solo se
		$[K(\alpha) : K]$ è finito. \medskip


		\subsection{Estensioni semplici, algebriche}

		
		Si dice che $L$ è un'\textbf{estensione semplice} di
		$K$ se $\exists \alpha \in L$ tale per cui $L = K(\alpha)$.
		In tal caso si dice che $\alpha$ è un \textbf{elemento primitivo} di $K$. Si dice che $L$ è un'\textbf{estensione
		algebrica} di $K$ se ogni suo elemento è algebrico su $K$.
		Ogni estensione finita è algebrica. Non tutte le
		estensioni algebriche sono finite (e.g.~$\overline{\QQ}$ su $\QQ$). \medskip
		
		
		L'insieme degli elementi algebrici di un'estensione
		di $K$ su $K$ è un estensione algebrica di $K$.
		Pertanto se $\alpha$ e $\beta$ sono algebrici,
		$\alpha \pm \beta$, $\alpha \beta$, $\alpha \beta\inv$
		e $\alpha\inv \beta$ (a patto che o $\alpha \neq 0$ o
		$\beta \neq 0$) sono algebrici.  
		
		
		\subsection{Campi perfetti, estensioni separabili e coniugati}
		
		
		Si dice che un'estensione algebrica $L$ è un'\textbf{estensione separabile} di
		$K$ se per ogni elemento $\alpha \in L$,
		$\mu_\alpha$ ammette radici distinte. Si dice
		che $K$ è un \textbf{campo perfetto} se ogni
		polinomio irriducibile ammette radici distinte.
		In un campo perfetto, ogni estensione algebrica
		è separabile. Si definiscono i coniugati di
		$\alpha$ algebrico su $K$ come le radici
		di $\mu_\alpha$. Se $K(\alpha)$ è separabile su $K$,
		$\alpha$ ha esattamente $\deg_K \alpha$ coniugati,
		altrimenti esistono al più $\deg_K \alpha$ coniugati. \medskip
		
		
		Un campo è perfetto se e solo se ha caratteristica
		$0$ o altrimenti se l'endomorfismo di
		Frobenius è un automorfismo. Equivalentemente,
		un campo è perfetto se le derivate dei polinomi
		irriducibili sono sempre non nulle. Esempi di
		campi perfetti sono allora tutti i campi di
		caratteristica $0$ e tutti i campi finiti. 
		
		
		\subsection{Campi algebricamente chiusi e chiusura algebrica di $K$}
		
		Un campo $K$ si dice \textbf{algebricamente chiuso} se
		ogni $p \in K[x]$ ammette una radice in $K$. Equivalentemente $K$ è algebricamente chiuso se
		ogni $p \in K[x]$ ammette tutte le sue radici in $K$.
		Si dice \textbf{chiusura algebrica} di $K$
		una sua estensione algebrica e algebricamente
		chiusa. Le chiusure algebriche di $K$ sono
		$K$-isomorfe tra loro, e quindi si identifica
		con $\overline{K}$ la struttura algebrica della
		chiusura algebrica di $K$. \medskip
		
		
		Se $L$ è una sottoestensione di $K$ algebricamente
		chiuso, allora $\overline{L}$ è il campo degli
		elementi algebrici di $K$ su $L$. Infatti se
		$p \in L[x]$, $p$ ammette una radice $\alpha$ in $K$, essendo
		algebricamente chiuso. Allora $\alpha$ è un elemento
		di $K$ algebrico su $L$, e quindi $\alpha \in \overline{L}$. Per il Teorema fondamentale dell'algebra,
		$\overline{\RR} = \CC$.
		
		
		\subsection{Estensioni normali e $K$-immersioni di un'estensione finita di $K$}
		
		Sia $\alpha$ un elemento algebrico su $K$. Allora
		$[K(\alpha) : K] = \deg_K \alpha$. Le
		$K$-immersioni da $K(\alpha)$ in $\overline{K}$
		sono esattamente tante quanti sono i coniugati di
		$\alpha$ e sono tali da mappare $\alpha$ ad un suo coniugato. Se $K$ è perfetto, esistono esattamente
		$\deg_K \alpha$ $K$-immersioni da $K(\alpha)$
		in $\overline{K}$. \medskip
		
		
		Se $L / K$ è un'estensione finita su $K$, allora
		esistono esattamente $[L : K]$ $K$-immersioni
		da $L$ in $\overline{K}$. Per quanto detto prima,
		tali immersioni mappano gli elementi $L$ nei
		loro coniugati. \medskip
		
		
		Se $L$ è un'estensione separabile finita, allora per
		ogni $\varphi : K \to \overline{K}$ esistono
		esattamente $[L : K]$ estensioni $\varphi_i : L \to \overline{K}$ di $\varphi$, ossia omomorfismi
		tali per cui $\restr{\varphi_i}{K} = \varphi$. \medskip
		
		
		Si dice che un'estensione algebrica $L / K$ è un'\textbf{estensione normale}
		se per ogni $K$-immersione $\varphi$ da $L$ in $\overline{K}$
		vale che $\varphi(L) = L$. Equivalentemente
		un'estensione è normale se è il campo di spezzamento
		di una famiglia di polinomi (in particolare è il campo
		di spezzamento di tutti i polinomi irriducibili che
		hanno una radice in $L$). Ancora, un'estensione $L$
		è normale se e solo se per ogni $\alpha \in L$,
		i coniugati di $L$ appartengono ancora ad $L$.
		Per un'estensione normale, per ogni $K$-immersione
		$\varphi : L \to \overline{K}$ si può restringere
		il codominio ad un campo isomorfo a $L \subseteq \overline{K}$, e quindi considerare $\varphi$ come
		un automorfismo di $L$ che fissa $K$. \medskip
		
		
		Si indica con $\Aut_K(L) = \Aut(L / K)$ l'insieme
		degli automorfismi di $L$ che fissano $K$. Se
		$L$ è normale e separabile, si dice
		\textbf{estensione di Galois}, e si definisce
		$\Gal(L / K) := (\Aut_K L, \circ)$, ossia come
		il gruppo $\Aut_K L$ con l'operazione di
		composizione.
		
		
		\vfill
		\hrule
		~\\
		Ad opera di Gabriel Antonio Videtta, \url{https://poisson.phc.dm.unipi.it/~videtta/}.
		~\\Reperibile su
		\url{https://notes.hearot.it}, nella sezione \textit{Secondo anno $\to$ Algebra 1 $\to$ 3. Teoria delle estensioni di campo e di Galois $\to$ Scheda riassuntiva di Teoria dei campi e di Galois}.
	\end{multicols}
	
\end{document}